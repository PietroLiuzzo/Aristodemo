\small{\subsection*{Salamina}
FGrHist 104\index[n]{FGrHist 104} non si interessa minimamente alla
battaglia di Salamina come scontro navale, si occupa di tutto il suo
contorno, dei personaggi e delle azioni legate al \textit{racconto} di
questa battaglia. Come è noto, Erodoto\index[n]{Erodoto} (\href{http://data.perseus.org/citations/urn:cts:greekLit:tlg0016.tlg001.perseus-grc1:8.56}{8.56}-98\indexp{Erodoto!8!00560000 @56-98|ca}) ed
Eschilo (\textit{Pers.}
302-514\indexp{Eschilo!\textit{Persiani}!0302 @302-514|ca}) sono considerati le fonti principali a
riguardo, ma la narrazione  di Diodoro\index[n]{Diodoro} (11.15-19\indexp{Diodoro!11!00150000 @15-19|ca}) fornisce i dettagli più
completi per la battaglia. \cite[67-73]{Roux1974}; \cite[13s]{Parker2007}.
Sono importanti anche alcuni dati riportati da Plutarco\index[n]{Plutarco}, soprattutto nella\textit{Vita di Temistocle} (10-17\indexp{Plutarco!\textit{Temistocle}!00100000 @10-17|ca}) e nella \textit{Vita di Aristide} (8-9\indexp{Plutarco!\textit{Aristide}!00080000 @8-9|ca}). 
Elementi di queste ed altre narrazioni vengono enfatizzati anche in opere come quella di
Timoteo di Mileto\index[n]{Timoteo} che nel suo nomo lirico
sulla battaglia,  fornisce interessanti particolari come (vv.
101-104\indexp{Timoteo!\textit  {Persiani}!101 @101-4|ca}): \textgreek{χειρῶν δ'ἔγβαλλον ὀρεί/ους πόδας ναός, στόματος / δ'ἐξήλλοντο μαρμαροφε / γεῖς παῖδες συγκρουόμενοι.} \cite[60]{Janssen1984}; \cite[155]{ Kuch1995} (trad.); \cite[48s]{Gambetti2001}.
%%%%%%%%%
\subsubsection{\textgreek{αἰτησάμενος γὰρ μίαν ἡμέραν μόνην... εἰς τὸ μένειν αὐτούς}}\label{bkm:RefHeading696981501267828}
La lacuna iniziale, prima di \textgreek{αἰτησάμενος γάρ}, è ipotizzata sulla base della
mancanza di un soggetto. Nel codice la frase ha
una sua struttura logica compiuta anche se il soggetto implicito nel
participio medio con valore temporale, Temistocle\index[n]{Temistocle} (si veda
l'identica frase a 5.2.), non è
nominato direttamente. Per maggiori dettagli
sugli aspetti paleografici Cfr. p.\pageref{bkm:RefHeading707961501267828}.
L'espressione \textgreek{μίαν ἡμέραν μόνην}, non pare
avere alcuna attestazione classica e si trova solo in autori dal II
secolo d.C. in poi. Jacoby (\emph{Komm. ad loc.}) sosteneva che questo
''tratto personale di Aristodemo'' tenterebbe
di ''\textit{erklären,
warum die Hellenen am anderen morgen noch da
sind, was nach den Voraussetzungen der
vulgaten Tradition tatsächlich
unbegreiflich ist''}
Già Eschilo\index[n]{Eschilo} nei \textit{Persiani} (355-360\indexp{Eschilo!\textit{Persiani}!0355 @355-360!|ca}) riportava l'episodio dove un: \textgreek{ἀνὴρ γὰρ Ἕλλην ἐξ Ἀθηναίων στρατοῦ / ἐλθὼν ἔλεξε παιδὶ σῷ Ξέρξῃ τάδε·  / ὡς εἰ μελαίνης νυκτὸς ἵξεται κνέφας / Ἕλληνες οὐ μενοῖεν, ἀλλὰ σέλμασιν / ναῶν ἐπανθορόντες ἄλλος ἄλλοσε / δρασμῷ κρυφαίῳ βίοτον ἐκσωσοίατο} [un uomo greco, giunto dall'esercito ateniese, disse a tuo figlio Serse\index[n]{Serse} che, come fosse giunta l'oscurità della nera notte, i Greci non sarebbero rimasti fermi, ma sui ponti delle navi si sarebbero guardati gli uni gli altri e ciascuno per sé avrebbe cercato di salvarsi la vita con una fuga segreta.] 
Erodoto ci dà come indicazione solo l'avvento dell'alba, dopo la turbolenta discussione (\href{http://data.perseus.org/citations/urn:cts:greekLit:tlg0016.tlg001.perseus-grc1:8.83}{8.83}\indexp{Erodoto!8!00830000 @83|ca} \textgreek{ἡώς τε διέφαινε}, Cfr. Aesch. \emph{Pers.} 386\indexp{Eschilo!\textit{Persiani}!0386 @386|ca}) e ci dice che i Persiani non dormirono per passare a Psittalea ed accerchiare i Greci (\href{http://data.perseus.org/citations/urn:cts:greekLit:tlg0016.tlg001.perseus-grc1:8.76}{8.76}.3\indexp{Erodoto!8!00760003 @76.3|ca}). \cite[188-9]{Garvie2009}. Non penso quindi che \textgreek{Αἰτησάμενος γὰρ μίαν ἡμέραν μόνην} sia in contraddizione con il resto della tradizione. La frase può invece essere letta come riassunto di una delle possibili argomentazioni di Temistocle\index[n]{Temistocle} e ne sintetizza il senso decisivo: prendere tempo, come racconta anche Erodoto, per poter agire secondo il suo piano di riserva. La sintesi informativa di FGrHist 104 è basata su un criterio razionalizzante che fa sempre quadrare i conti. Durante la notte infatti (insieme all'arrivo di Aristide\index[n]{Aristide} e alla conferma dei Teni secondo Erodoto), si colloca anche il momento in cui Temistocle\index[n]{Temistocle}, \textgreek{λαθών}, se ne va dal Sinedrio per inviare Sicinno\index[n]{Sicinno} a parlare con i Persiani.  Diodoro\index[n]{Diodoro} non dice nulla al riguardo, mentre ritroviamo in  Nepote\index[n]{Cornelio Nepote} e Polieno\index[n]{Polieno} che Sicinno\index[n]{Sicinno}\index[n]{Sicinno\index[n]{Sicinno}} sarebbe stato inviato di notte (Nep. \emph{Them.} 4.3\indexp{Cornelio Nepote!\textit{Temistocle}!00040003 @4.3|ca} noctu de; Poliaen. \emph{Strat}. 1.30.3\indexp{Polieno!1.30.3|ca} \textgreek{νύκτωρ}). Jacoby leggeva in Plutarco\index[n]{Plutarco} (\emph{Them.} 12.2\indexp{Plutarco!\textit{Temistocle}!00120002 @12.2|ca}) una spiegazione che dovrebbe essere simile a quella di FGrHist 104, ma penso che ''\textgreek{ἐδόκει δὴ τῆς νυκτὸς ἀποχωρεῖν, καὶ παρηγγέλλετο πλοῦς τοῖς κυβερνήταις}'' [decisero di partire quella notte, e fu annunciato ai piloti di salpare] non si riferisca alla ''notte seguente'', come intende anche Carena nella sua traduzione per la fondazione Valla (1992). Per le discussioni sul momento esatto della spedizione di Sicinno\index[n]{Sicinno}, si veda \cite[2 n. 6]{Pelling1997}, \cite[170]{Lazenby1988}. Pelling stesso colloca la missione all'imbrunire, ''\emph{the appropriate time for dark and dusty deeds of derring-do}'' (\cite[3]{Pelling1997}). Jacoby considerava la cosa ''\emph{possibile}'': FGrHist 104 Komm, 321. Poco oltre troviamo parole molto simili a quelle di FGrHist 104 che potrebbero giustificare la posizione di Jacoby: \textgreek{ὃν ἐκπέμπει πρὸς τὸν Ξέρξην κρύφα} (Plut. \emph{Them.}   12.4\indexp{Plutarco!\textit{Temistocle}!00120004 @12.4|ca}). Solo in FGrHist 104 e Plutarco, poiché solo in essi \textgreek{κρύφα} è riferito a Sicinno\index[n]{Sicinno}, e nella tradizione che va da Eschilo\index[n]{Eschilo} a Plutarco\index[n]{Plutarco} e Polieno\index[n]{Polieno}, l'elemento della notte permane mentre quello del nascondimento per assimilazione alla ''notte che nasconde'', viene ad alternarsi nei testi in riferimento alla missione di Sicinno\index[n]{Sicinno}, laddove in origine era legato alla flotta greca (Eschilo) o a Temistocle \index[n]{Temistocle} (Erodoto \href{http://data.perseus.org/citations/urn:cts:greekLit:tlg0016.tlg001.perseus-grc1:8.75}{8.75}.2\indexp{Erodoto!8!00750002 @75.2|ca}). \cite[58]{Roux1974}. Si è quasi tentati di vedere in Sicinno\index[n]{Sicinno} un Argilio\index[n]{Argilio}  portalettere parallelo a quello di Pausania\index[n]{Pausania (reggente)}, a motivare il pretesto lacedemone che \textgreek{ἐν ταῖς Παυσανίου ἐπιστολαῖς κοινωνὸν εὑρηκέναι τῆς προδοσίας Θεμιστοκλέα} (10.1) identificando in questo pretesto la genesi del racconto su Sicinno\index[n]{Sicinno}. Per un'introduzione al rapporto tra Erodoto e i tragici, si veda \cite{Griffin2006}.
%%%%%
\subsubsection{\textgreek{νομίσας τὸν Θεμιστοκλέα μηδίζοντα}}\label{temistoclemedizza}
Mentre Eschilo\index[n]{Eschilo} in tre poderosi versi sottolinea la frettolosità e l'avventatezza della credulità di Serse\index[n]{Serse} (\textgreek{ὁ δ’ εὐθὺς ὡς ἤκουσεν, οὐ ξυνεὶς δόλον / Ἕλληνος ἀνδρὸς οὐδὲ τὸν θεῶν φθόνον, / πᾶσιν προφωνεῖ τόνδε ναυάρχοις λόγον·} [appena ebbe sentito, senza accorgersi dell'inganno del greco, né dell'invidia degli dei, rese noto il  messaggio a tutti i comandanti] Aesch. \emph{Pers.} 361-363\indexp{Eschilo!\textit{Persiani}!0361 @361-363|ca}), Erodoto\index[n]{Erodoto} si limita a registrare che i generali credettero a Sicinno\index[n]{Sicinno} (\href{http://data.perseus.org/citations/urn:cts:greekLit:tlg0016.tlg001.perseus-grc1:8.76}{8.76}.1\indexp{Erodoto!8!00760001 @76.1|ca}),  Diodoro\index[n]{Diodoro} segue quasi alla lettera Erodoto, ma enfatizza la credibilità del messaggio e così anche Plutarco\index[n]{Plutarco} (\emph{Them.}  12.4\indexp{Plutarco!\textit{Temistocle}!00120004 @12.4|ca}). Serse\index[n]{Serse} non ha motivo per dubitare di Temistocle. La concordanza generica delle fonti spinge a chiedersi perché FGrHist 104 scelga proprio \textgreek{μηδίζω}: probabilmente all'epoca della stesura del testo il termine era abbastanza diffuso e familiare, nonché molto adatto al contesto ed al periodo da un punto di vista retorico-stilistico. \cite[15-17 e 19]{Graf1984}. 
%%%%%%%%%%
\subsubsection{\textgreek{ὁ Ξέρξης ζεῦγμα κατασκευάσας}}\label{bkm:RefHeading696921501267828}
\cite[61]{Roux1974} stimava 900 m di distanza tra Cinosura e Psittalea, 1200 tra l'Attica e Psittalea. \cite[294]{Wallace1969} ricorda invece che il livello del mare era almeno tre metri più basso, portando così la larghezza della superficie d'acqua a dimensioni inferiori a quelle stimate da Roux. Dubito comunque che all'epoca di Strabone fosse di due stadi. Il ponte o terrapieno è paragonato a quello più famoso sull'Ellesponto, descritto da Erodoto\index[n]{Erodoto} (\href{http://data.perseus.org/citations/urn:cts:greekLit:tlg0016.tlg001.perseus-grc1:7.35}{7.35}-37\indexp{Erodoto!7!00350000 @35-37|ca}) con una dettagliata descrizione tecnica della costruzione. Il passaggio di Serse\index[n]{Serse} in Europa tramite l'aggiogamento dell'Ellesponto è un evento di portata colossale: viene formulato per la prima volta in Aesch. \emph{Pers.} 176-200\indexp{Eschilo!\textit{Persiani}!0176 @176-200|ca}, in un sogno dove la Regina vede le due sorelle, Grecia e  Oriente e nell'oracolo di Bacide\index[n]{Bacide} in Hdt. \href{http://data.perseus.org/citations/urn:cts:greekLit:tlg0016.tlg001.perseus-grc1:7.20}{7.20}\indexp{Erodoto!7!00200000 @20|ca}. Pare che Cherilo di Samo\index[n]{Cherilo di Samo} vi dedicasse un intero poema. La rilevanza di questo evento perdurerà, non solo come data cardine per la storia greca, ma anche per quella romana in Polibio\index[n]{Polibio}  (3.22\indexp{Polibio!003022 @3.22|ca}). \cite[90-91]{Mazzarino1947}; \cite[7]{Pelling1997}.
Si nota come, in nessuna delle restanti fonti, venga mai esplicitamente riferito l'esito dell' ''operazione-ponte'' a Salamina, assunta come fallimentare (lasciato intendere da Erodoto\index[n]{Erodoto} e Strabone con \textgreek{ἐπειρᾶτο}, del tutto abbandonato da Plutarco): solo qui l'impossibilità di tale impresa è espressamente dichiarata. Essa non era sicuramente dovuta a motivi tecnici, perché nonostante la battaglia fosse vinta, ancora l'esercito di terra di Serse\index[n]{Serse} era in loco e la guerra non si sarebbe conclusa in tempi brevi. Ogni ipotesi purtroppo rischia di essere pura illazione; basti quindi segnalare che questo è un dettaglio in più fornito da FGrHist 104. 
%%%%%%%%%
\subsubsection{\textgreek{καθεζόμενος … ἑώρα τὴν ναυμαχίαν}}\label{tronoserse}
Jacoby si soffermava invece sul problema dell'Erakleion e sulla posizione del ''trono di Serse\index[n]{Serse}'' a cui sono legati numerosi aspetti della ricostruzione corografica della battaglia di Salamina. 
Il fatto che Serse\index[n]{Serse} fosse seduto è più importante di quanto non sembri. Demostene \emph{In Timocratem} (24.129\indexp{Demostene!00240129 @24.129}) nomina un \textgreek{δίφρον τὸν ἀργυρόποδα} rubato insieme alla spada di Mardonio\index[n]{Mardonio} (Cfr. anche Arpocrazione s.v. A 226), dalle offerte dell'acropoli e ricordato anche da Plutarco\index[n]{Plutarco} (\emph{Them.}   13.1\indexp{Plutarco!\textit{Temistocle}!00130001 @13.1|ca} \textgreek{χρυσοῦν δίφρον θέμενος}; \emph{Them.}   16.1\indexp{Plutarco!\textit{Temistocle}!00160001 @16.1|ca} \textgreek{καθήμενος ὑπὸ σκιάδι χρυσῇ θεάσεται}; cfr. anche \emph{Arist. }9.5\indexp{Plutarco!\textit{Aristide}!00090005 @9.5|ca}), e da Filostrato (\emph{Imagines} 2.31\indexp{Filostrato (il vecchio)!Immagini!2.31|ca} \textgreek{ἐπὶ χρυσοῦ θρόνου}). La scena del re in trono non faceva solo parte dell'immaginario, ma trovava, nel dipinto di Mandrocle\index[n]{Mandrocle} all'Heraion di Samo (Hdt. \href{http://data.perseus.org/citations/urn:cts:greekLit:tlg0016.tlg001.perseus-grc1:4.88}{4.88}\indexp{Erodoto!4!00880000 @88|ca}), una fonte plausibile e non ancora sbiadita alla fine del V secolo. Esso fu dedicato con i doni offerti da Dario\index[n]{Dario} in cambio della costruzione del ponte sul Bosforo e portava un'iscrizione. Questo dipinto, non certo originale nella tradizione vicino-orientale, ispira probabilmente la scena di Hdt. \href{http://data.perseus.org/citations/urn:cts:greekLit:tlg0016.tlg001.perseus-grc1:7.44}{7.44}s.\indexp{Erodoto!7!00440000 @44s.|ca} \cite[267]{Asheri2003}; \cite[234 e 939-940]{Briant1996}. \cite[50-51]{Taboada1999}, mette in evidenza come questo motivo avrà grande fortuna (tuttavia, credo, non tramite il canale greco), riportando ad esempio un passo di Tabari\index[n]{Tabari}: ''davanti agli Arabi che avanzano, il generale Rustum\index[n]{Rustum} posiziona il suo stato maggiore su una collina e da un trono, collocato sotto un baldacchino, si prepara a contemplare la vittoria, però quel che vede sono gli Arabi che mettono in fuga il suo esercito sassanide, si leva un forte vento che porta via il padiglione dei generali e un guerriero arabo, Hilal Ibn 'Ufalla si avventa sul baldacchino e ammazza Rustum che vorrebbe scappare'' (Tarij al rusul wa al muluk 1.2335/7\index[t]{Tarij al rusul wa al muluk 1.2335/7}); e  Ibn Jaldun\index[n]{Ibn Jaldun}, sulla battaglia di Qadisiyya (637d.C.): ''Rustum stava seduto su un trono, rialzato fin che non si sbandarono le file persiane e gli Arabi avanzarono fino a quel trono. Tentò di fuggire verso l'Eufrate ma lo ammazzarono'' (\emph{Muqaddima} 3.37\index[t]{Muqaddima 3.37}). 
%%%%%%%%
\subsubsection{\textgreek{ἤρξαντο δὲ τοῦ ναυμαχεῖν … οἱ Ἀθηναῖοι.}}\label{bkm:RefHeading3610919231068}
L'inizio della battaglia è improvviso, ma tutte le fonti concordano nell'attribuirlo ai Greci. Il primo a muoversi è identificato in Aminia\index[n]{Aminia}\label{ref:Aminia} da Erodoto\index[n]{Erodoto} (\href{http://data.perseus.org/citations/urn:cts:greekLit:tlg0016.tlg001.perseus-grc1:8.84}{8.84}.1\indexp{Erodoto!8!00840001 @84.1|ca}) e  Diodoro\index[n]{Diodoro} (11.27.2\indexp{Diodoro!11!00270002 @27.2|ca}). 
Cfr. p.\pageref{aminiacinegiro} ed Eliano VH 5.19\indexp{Eliano!\textit{Varia Historia}!5.19|ca}, dove Aminia\index[n]{Aminia} è presentato con una mano mozzata, chiaramente ricostruito sull'immagine delle gesta del fratello Cinegiro a Maratona (Culasso Gastaldi 1990, 69), dipinte nella \emph{Stoà Poikile}. \cite{Pownall2011}, ad loc. 
Per l'uso di  \textgreek{ἐκπρεπέστερον} con anteposizione dell'informazione, si veda anche Grethlein 2009, 156-9; in senso avverbiale pare essere peculiare di FGrHist 104, che lo riutilizza anche per riprendere la narrazione a 1.6. Si trova anche in Thuc. 3.55.2\indexp{Tucidide!3!00550002 @55.2|ca} con un significato leggermente diverso, glossato anche dallo scolio, che dice \textgreek{ἔξω τοῦ πρέποντος καὶ τοῦ ἁρμόζοντος}.  In questo passo e in 1.6, abbiamo i primi due evidenti segnali della predilezione ateniese di FGrHist 104. Il resto della tradizione, da Erodoto\index[n]{Erodoto} (\href{http://data.perseus.org/citations/urn:cts:greekLit:tlg0016.tlg001.perseus-grc1:8.93}{8.93}.1\indexp{Erodoto!8!00930001 @93.1|ca}) a Eliano (VH 12.10)\indexp{Eliano!\textit{Varia Historia}!12.10|ca}, assegna agli Egineti il primato e agli Ateniesi  premi individuali. Sulle celebrazioni dopo una vittoria e i loro rituali nei principali storici greci si vedano le osservazioni di \cite{Hau2013}. Secondo  Diodoro\index[n]{Diodoro} 11.27.2\indexp{Diodoro!11!00270002 @27.2|ca}, questo episodio è all'origine del timore dei Lacedemoni riguardo il dominio marittimo di Atene: \textgreek{Προορώμενοι τὸ μέλλον ἐφιλοτιμοῦντο ταπεινοῦν τὸ φρόνημα τῶν Ἀθηναίων}, espressione che pare voler ''ostracizzare la città'' per Carcopino 1935, 38 (sulla base di Diod. 11.87.2\indexp{Diodoro!11!00870002 @87.2|ca}, Filocoro FGrHist 328 F 30\indexp{Filocoro FGrHist 328!F!30|ca} e Teop. FGrHist 115 F 88\indexp{Teopompo FGrHist 115!F!88|ca} = \emph{Schol. Aristeid.} p. 528, 4) se confrontato con Plut. \emph{Arist.} 7.2\indexp{Plutarco!\textit{Aristide}!00070002 @7.2|ca} \textgreek{βαρυτέρας ταπείνωσις}.
Legata a questa è anche una lettura particolare (e molto più ragionevole, bisogna ammettere) della votazione all'Istmo, tramite la quale i Lacedemoni avrebbero in qualche modo indotto Atene a rimuovere Temistocle\index[n]{Temistocle} dall'incarico (sostituito da Santippo\index[n]{Santippo}). Hdt. \href{http://data.perseus.org/citations/urn:cts:greekLit:tlg0016.tlg001.perseus-grc1:8.123}{8.123}\indexp{Erodoto!8!01230000 @123|ca}, Aristid. 218. Quasi sicuramente questo è anche l'evento rievocato da Timocreonte\index[n]{Timocreonte} coi versi \textgreek{ἀργυρίων δ'ὑπόπλεως Ἰσθμοῖ γελοίως πανδόκευε ψυχρὰ \Ladd{τὰ} κρεῖα παρίσκων · οἱ δ'ἤσθιον κηὔχοντο μὴ ὤραν Θεμιστοκλέος γενέσθαι }(Plut. \emph{Them.}  21.4\indexp{Plutarco!\textit{Temistocle}!00210004 @21.4|ca} = fr. 727, 10-12 PMG) \cite[279s]{Zadorojnyi2006}; \cite[88]{Funaioli2007};\cite[42]{Fornara1971}) nomina entrambi alla strategia del 480/79.
    %%QUESTI ULTIMI DUE PUNTI SONO DA CHIARIRE RILEGGENDO I TESTI... E DA COLLEGARE MEGLIO!!
\subsubsection{\textgreek{συνεστηκυίας δὲ …  ναυάγια τῶν βαρβάρων ἀνασώζεσθαι}}
L'identificazione di quest'isola è molto discussa. Il lavoro più completo resta Wallace che, a partire da una accuratissima lettura di Strabone (9.1.13-14\indexp{Strabone!000900010013 @9.1.13-14|ca}), ricostruisce tutta la toponomastica possibile, considerando un innalzamento del livello del mare osservabile  e mettendo in campo diverse fonti, tra le quali FGrHist 104, il cui  \textgreek{πλησίον} ''\emph{speaking of the island with respect to its nearness to the Persian camp at Phaleron}'' aiuta l'autore nella discussione per scartare Agios Georgios (\cite[295]{Wallace1969}). Altre fonti utilizzate dall'autore sono \emph{Schol. Pers.} 413\indexp{Schol. Persae!413|ca}, Aristotele \emph{Retorica} 1411 a 15\indexp{Aristotele!\textit{Retorica}!1411a|ca}, IG II1 476\index[pap]{IG!II,1!476|ca}. Si veda anche il dettagliatissimo \cite[692-705 (spec. 700-703)]{Muller1987} e \cite[275]{Asheri2003}. 
%%%%%%%%%
\subsubsection{\textgreek{Ἀριστείδης δὲ Ἀθηναῖος …   ὑπὲρ τῶν Ἑλλήνων}}
Il titolo di \textgreek{δίκαιος}, arcinoto alla tradizione, probabilmente ha origine letteraria in Erodoto\index[n]{Erodoto} \href{http://data.perseus.org/citations/urn:cts:greekLit:tlg0016.tlg001.perseus-grc1:8.79}{8.79}\indexp{Erodoto!8!00790000 @79|ca}, dove Aristide\index[n]{Aristide} viene detto \textgreek{ἄριστον ἄνδρα γενέσθαι ἐν  Ἀθήνῃσι καὶ δικαιότατον}. È un soprannome formulare al punto da generare fenomeni come quello studiato da Citti (\cite*[137]{Citti2004}) in \emph{Arist. }3.5\indexp{Plutarco!\textit{Aristide}!00003005 @3.5|ca}, dove la citazione da Aesch. \emph{Sept.} 592-94\indexp{Eschilo!\textit{Sette contro Tebe}!592-94|ca} invece di \textgreek{ἀριστὸς} mette \textgreek{δίκαιος}. Si veda anche \cite[88]{Funaioli2007}. Eupoli\index[n]{Eupoli}, nei \emph{Demi}, dopo aver fatto condannare da Aristide\index[n]{Aristide} un povero Sicofante, ci presenta il Giusto che scherza proprio su questo titolo dicendo, da \textgreek{προστατῆς τοῦ δεμοῦ}  (P.Cairo 43227\index[pap]{P.Cairo!inv. 43227}: fr. 99 K.-A. vv. 100-119 ): \textgreek{ἔγω δὲ πάσῃ προαγορεύω τῇ πόλ[ει / εἶναι δικαίους, ὡς ὃς ἂν δίκαιος ᾖ.} \cite[275s]{Telo2006}.
    Nel racconto di Erodoto\index[n]{Erodoto} Aristide\index[n]{Aristide} arriva da Egina (Plut. \emph{Arist. }8.2\indexp{Plutarco!\textit{Aristide}!00008002 @8.2|ca}), 
    %Demostene, In Aristogitonem 2, (26) 6\indexp{Demostene!In Aristogitonem!2, (26) 6}), 
    durante la notte, ed ha una conversazione con Temistocle\index[n]{Temistocle} che tuttavia non ha un effetto decisivo sul consiglio, paragonabile all'arrivo della nave dei Tenii (Hdt. \href{http://data.perseus.org/citations/urn:cts:greekLit:tlg0016.tlg001.perseus-grc1:8.79}{8.79}-81\indexp{Erodoto!8!00790000 @79-81|ca}). 
    Secondo Ctesia\index[n]{Ctesia di Cnido} (FGrHist 688 F13.30\indexp{Ctesia FGrHist 688!F!13.30}) e Aristotele (\emph{Costituzione degli Ateniesi} 23.5\indexp{Aristotele!\textit{Costituzione degli Ateniesi}!23.5|ca}) Aristide\index[n]{Aristide} e Temistocle\index[n]{Temistocle} condussero insieme la battaglia, ma questo pare poco probabile. Aristide era stato ostracizzato nel 483/2, secondo Plutarco\index[n]{Plutarco} per volontà di Temistocle, ma in FGrHist 104 la motivazione non è data. Hdt. \href{http://data.perseus.org/citations/urn:cts:greekLit:tlg0016.tlg001.perseus-grc1:8.79}{8.79}\indexp{Erodoto!8!00790000 @79|ca}, Plut. \emph{Arist.} 7,\indexp{Plutarco!\textit{Aristide}!000070000 @7|ca} Arist. \emph{Costituzione degli Ateniesi} 22.7\indexp{Aristotele!\textit{Costituzione degli Ateniesi}!22.7|ca}, \cite[37s]{Carcopino1935}. La clausola sulle distanze da mantenere, evidentemente votata successivamente, è emendata nel testo di Aristotele, e il \textgreek{μὴ} del P. Lond. 131\index[pap]{P. Lond.!131} viene espunto, sulla base del confronto con Filocoro FGrHist 328 F 30\indexp{Filocoro FGrHist 328!F!00300000 @30|ca} \textgreek{μὴ ἐκβαίνοντα ἐντὸς Γεραιστοῦ τοῦ Εὐβοίας ἀκρωτηρίου} citato in \index[pap]{P.Berol.!5008}, dopo il titolo \textgreek{ΟΤΙ ΘΕΜΙΣΤΟΚΛΗΣ ΩΣΤΡΑΚΙΣΘΗ}. I veri motivi per la cacciata di Aristide\index[n]{Aristide}, mano a mano che le scoperte e gli scavi procedono, sembrano sempre più simili a quelle che porteranno poi alla fine di Temistocle\index[n]{Temistocle} e alla caduta di Pausania\index[n]{Pausania (reggente)}. \cite[88 n.157]{Blosel2004} ricorda infatti alcuni ''\emph{Aristides ostrakon von der Agora, die diesen als ''bruder des Datis'', somit als Perserfreund diffamiert}''. e.g. SEG XIX 36a\index[pap]{SEG!XIX 36a|ca}, \cite[38 n.56]{Lang1990}: \textgreek{Ἀριστ[είδεν] / τὸν Δά[τιδος] / ἀδελφ[όν]} e ''Aristeides' enger'' come in SEG XIX 36b\index[pap]{SEG!XIX 36b|ca}, \textgreek{[Ἀριστείδες / }h\textgreek{ο Λυσιμ]άχο / [}h\textgreek{ος τὸ]ς} h\textgreek{ικέτας / [ἀπέοσ]εν.} Il fatto che FGrHist 104 parli di Aristide\index[n]{Aristide}, Arconte ad Egina a quel tempo è probabilmente una semplificazione 
    \begin{itemize}
    \item delle diverse missioni avanti e indietro dall'isola (con e senza Eacidi) di quel manipolo di navi
    \item di una tradizione, di cui anche Erodoto\index[n]{Erodoto} fa parte, desiderosa di redimere Aristide\index[n]{Aristide} e dargli un ruolo di spicco anche a Salamina, non solo tramite l'impresa di Psittalea. 
    \end{itemize}
    In \emph{Them. Ep.}  11.3\indexp{Lettere di Temistocle!00110003 @11.3|ca} si elencano fra gli accusatori di Temistocle\index[n]{Temistocle} un ''Aristide di Egina'' di cui Culasso Gastaldi (\cite*[126 e 130-132]{CulassoGastaldi1990}) discute identità, ruolo e posizione nei confronti di Temistocle\index[n]{Temistocle}. Sono interessanti a questo proposito anche \emph{Them. Ep.}  1.7\indexp{Lettere di Temistocle!00010007 @1.7|ca} e  \emph{Them. Ep.}  2.2\indexp{Lettere di Temistocle!00020002 @2.2|ca} invece Temistocle\index[n]{Temistocle} viene insistentemente richiesto di assumere ad Argo la massima carica disponibile (\cite*[260]{CulassoGastaldi1990}). Aristide\index[n]{Aristide} si presenta da lui come ad un re per chiedere un esercito. Questo può essere frutto di una mala sintesi della discussione tra i due in Erodoto, ma non v'è accenno a Psittalea in essa (Hdt. \href{http://data.perseus.org/citations/urn:cts:greekLit:tlg0016.tlg001.perseus-grc1:8.79}{8.79}-81\indexp{Erodoto!8!00790000 @79-81|ca}). Ha ragione Jacoby notando che ''\emph{begnügt Aristodemos sich mit dem vagen \textgreek{παρεγένετο}}''. Lo \textgreek{στρατὸν} chiesto da Aristide era probabilmente costituito di quei \textgreek{πολλοὺς  τῶν  ὁπλιτέων […] γένος  ἐόντες  Ἀθηναῖοι} rispetto ai quali Erodoto\index[n]{Erodoto} dice Aristide \textgreek{παραλαβὼν} (\href{http://data.perseus.org/citations/urn:cts:greekLit:tlg0016.tlg001.perseus-grc1:8.95}{8.95}\indexp{Erodoto!8!00950000 @95|ca}). Cfr. anche \textgreek{ὅπλοισι} in Aesch. \emph{Pers.} 457\indexp{Eschilo!\textit{Persiani}!0457 @457|ca}; \cite[210-11]{Garvie2009}.  
Questo esercito poteva contenere anche arcieri, come parte delle fonti sostiene: Eschilo\index[n]{Eschilo} in primis, \textgreek{πέτροισιν ἠράσσοντο, τοξικῆς τ'ἀπὸ / θώμιγγος ἰοὶ προσπίτνοντες ὤλλυσαν} (460-1\indexp{Eschilo!\textit{Persiani}!0460 @460-1|ca}, 458\indexp{Eschilo!\textit{Persiani}!0458 @458|ca}, 418\indexp{Eschilo!\textit{Persiani}!0418 @418|ca}, Pelling 1997,9 e Garvie 2009, 211), ma anche Timoteo, che secondo \cite[47]{Gambetti2001} (che critica \cite[24-30]{Janssen1984}), si riferirebbe ad arcieri ed opliti in piedi sul bordo della nave con l'espressione (vv.4-5\indexp{Timoteo!\textit{Persiani}!004 @4-5|ca}) \textgreek{πο[σ]ὶ δὲ γε[ισό]λογχο[ν ὄγ- / κωμ(α)] ἀμφέθεντ(ο) ὀδόντων} [coi piedi sul bordo di ferro rigonfio cinto di denti]. Il testo più importante sulla presenza di questi arcieri resta Ctesia FGrHist 688 F 13.30\indexp{Ctesia FGrHist 688!F!00130030 @13.30|ca}, dove entrano in campo, praticamente soli, i \textgreek{τοξόται μὲν ἀπὸ Κρήτης προσκαλοῦνται}: gli arcieri chiamati da Creta. In un interessante articolo, Silvana Cagnazzi ha dimostrato efficacemente come sia ingenuo escluderli dalla narrazione della battaglia, solo perché apparentemente non ve n'è traccia in Erodoto. In Paus. 1.29.6\indexp{Pausania!000100290006 @1.29.6|ca} si parla infatti di un monumento per gli arcieri cretesi ''che potrebbe risalire proprio al 479''; nello stesso autore a 4.8.3\indexp{Pausania!000400080003 @4.8.3|ca} arcieri cretesi mercenari si trovano nelle file dei Messeni e in  Thuc. 6.25.2\indexp{Tucidide!6!00250002 @25.2|ca} Nicia vuole arcieri ateniesi; in Platone \emph{Leggi} 4. 707b\indexp{Platone!\textit{Leggi}!040707 @4.707b|ca} si trova  \textgreek{ἡμεῖς γε οἱ Κρῆτες … τὴν Ἑλλάδα φαμὲν σῶσαι} (\cite[31-33]{Cagnazzi2003}) e tutto ciò porta a rivalutare la risposta data dai Cretesi all'ambasciata che ne chiedeva l'aiuto. Punto di riferimento è sempre Plut. \emph{Them.} 14.2\indexp{Plutarco!\textit{Temistocle}!00014002 @14.2|ca} dove si dice che ogni barca conteneva 18 uomini di cui 4 arcieri. 
    Data la situazione alla richiesta segue un momento di riconciliazione in cui Temistocle\index[n]{Temistocle} sorvola sulla precedenti discordie. Questo elemento della narrazione risale direttamente a Erodoto\index[n]{Erodoto} (\href{http://data.perseus.org/citations/urn:cts:greekLit:tlg0016.tlg001.perseus-grc1:8.79}{8.79}\indexp{Erodoto!8!00790000 @79|ca}). Un altro parallelo si trova però in Arist. \emph{Costituzione degli Ateniesi} 23.4\indexp{Aristotele!\textit{Costituzione degli Ateniesi}!23.4|ca} dove con altre parole si dice che Temistocle\index[n]{Temistocle} ed Aristide\index[n]{Aristide} costruirono insieme le mura nonostante non andassero d'accordo (\textgreek{καίπερ διαφερόμενοι πρὸς ἀλλήλους}). L'epistolografo delle \emph{Lettere di Temistocle}\index[n]{Temistocle} non manca di utilizzare questo saporitissimo dato in \emph{Them. Ep.}  3.5\indexp{Lettere di Temistocle!00030005 @3.5|ca} (\textgreek{καίπερ ἐχθρός}). \cite[132]{CulassoGastaldi1990}. \textgreek{ ὅμως ἔδωκε} è un'espressione semplicissima, ma pare fin troppo, dato che si trova solo qui (precedenti solo due passi in Aristotele HA, 601b13\indexp{Aristotele!\textit{Historia Animalium}!601b|ca} e fr. 117 l.21\indexp{Aristotele!\textit{Historia Animalium}!fr. 117|ca}) e dell'ampio uso fattone in epoca tarda. 
    Aristide con l'esercito gentilmente concesso compie la grande impresa. Il testo ricorda gli \textgreek{ἔργα ἀπεδέξαντο λόγου ἄξια} degli Egineti agli stretti di Hdt. \href{http://data.perseus.org/citations/urn:cts:greekLit:tlg0016.tlg001.perseus-grc1:8.91}{8.91}\indexp{Erodoto!8!00910000 @91|ca}. 
    Davvero peculiare di FGrHist 104 è invece la grandezza dell'impresa, mai sottolineata dalle altre fonti. Abbiamo già visto come FGrHist 104 sia affezionato ad Aristide\index[n]{Aristide}. Questa specificazione si dovrà intendere nell'ottica encomiastica plutarchea e non è in contraddizione con le ''numerose migliaia'' di truppe che si è appena detto essere sbarcate a Psittalea. Il momento peraltro è importante, perché compiere una tale impresa, durante la battaglia, è davvero un \textgreek{μέγιστον ἔργον} con cui farsi notare dai Greci.
%%%%%%%%%
\subsubsection{\textgreek{τῶν δὲ βαρβάρων γυνὴ Ἁλικαρνασὶς τὸ γένος, ὄνομα δὲ Ἀρτεμισία}} 
Mentre il premio ad Aminia\index[n]{Aminia} è noto (Hdt. \href{http://data.perseus.org/citations/urn:cts:greekLit:tlg0016.tlg001.perseus-grc1:8.93}{8.93}\indexp{Erodoto!8!00930000 @93|ca}, Diod. 11.27.2\indexp{Diodoro!11!00270002 @27.2|ca}), Artemisia\index[n]{Artemisia di Alicarnasso} è premiata solo da FGrHist 104, forse sulla base di una memoria generale degli ultimi tre libri di Erodoto\index[n]{Erodoto} e specialmente di Hdt. \href{http://data.perseus.org/citations/urn:cts:greekLit:tlg0016.tlg001.perseus-grc1:7.99}{7.99}.1\indexp{Erodoto!7!00990001 @99.1|ca} e \href{http://data.perseus.org/citations/urn:cts:greekLit:tlg0016.tlg001.perseus-grc1:8.87}{8.87}.1\indexp{Erodoto!8!00870001 @87.1|ca}. Hdt. \href{http://data.perseus.org/citations/urn:cts:greekLit:tlg0016.tlg001.perseus-grc1:8.87}{8.87}-88\indexp{Erodoto!8!00870000 @87-88|ca}, \href{http://data.perseus.org/citations/urn:cts:greekLit:tlg0016.tlg001.perseus-grc1:8.93}{93}\indexp{Erodoto!8!00930000 @93|ca}. Figlia di Ligdami, di Alicarnasso per parte paterna, di stirpe cretese per parte materna, la signora di Alicarnasso portò le navi più belle che la flotta persiana potesse vantare, dopo quelle dei Sidonii. Plut. \emph{Them.} 14.4\indexp{Plutarco!\textit{Temistocle}!00014004 @14.4|ca} ricorda la prima delle azioni che si rammentano di lei. Durante la battaglia Erodoto\index[n]{Erodoto} si concentra talmente su Artemisia\index[n]{Artemisia di Alicarnasso} da  dare adito ad una facile critica in Plut. \emph{De Malignitate Herodoti} 873E\indexp{Plutarco!\textit{De Malignitate Herodoti}!00873 @873E|ca}. Fozio (cod. 190, 153b\indexp{Fozio!0190 @Cod. 190|ca}) racconta che, in età matura, si innamorò di un giovane Abideno, Dardano, che tuttavia non la corrispose: Artemisia\index[n]{Artemisia di Alicarnasso} accecata dall'odio cavò gli occhi al giovane nel sonno, in una storia di ambientazione e tematica simile a quella di Cleonice\index[n]{Cleonice} (8.1). Secondo l'acuta indagine di Rosaria \cite{MunsonVignolo1988} Artemisia\index[n]{Artemisia di Alicarnasso} è l'alterego di Temistocle\index[n]{Temistocle} e il suo doppio. Pausania\index[n]{Pausania il periegeta} (1.11.3\indexp{Pausania!000100110003 @1.11.3|ca}) ne sigla l'epitaffio nelle nostre fonti, ricordandone la presenza nella \emph{Stoà Persiké} di Sparta, accanto a Mardonio\index[n]{Mardonio}, \cite[289]{Asheri2003}. Secondo il Periegeta è lì che resterà anche dopo il restauro sotto Augusto, come simbolo dell'ideologia antipersiana, che contrapponeva la Grecia virile all'effeminata Persia, secondo l'affermazione di Serse\index[n]{Serse}. Ulteriori informazioni si possono trovare alla voce ''Artemisa di Alicarnasso'' sull'Enciclopedia delle Donne (www.enciclopediadelledonne.it). Il racconto più dettagliato a proposito della manovra di Artemisia\index[n]{Artemisia di Alicarnasso} è in Polieno\index[n]{Polieno} (8. 53. 1-2\indexp{Polieno!8.53.1-2|ca}) che precisa e arricchisce il racconto di Erodoto. Polieno\index[n]{Polieno} è l'unica altra fonte che ci parli di Artemisia\index[n]{Artemisia di Alicarnasso} di Alicarnasso, ed è difficile dire, data la genericità dei termini, se FGrHist 104 gli abbia fatto da tramite nella ripresa diretta di \textgreek{διωκομένης} ed \textgreek{ἔμπροσθεν} da Erodoto. Ma nello storico le navi dei Calindi sono \textgreek{νέες φίλιαι}, non \textgreek{ἰδίαν}, come invece apprendiamo da FGrHist 104. Mentre in \href{http://data.perseus.org/citations/urn:cts:greekLit:tlg0016.tlg001.perseus-grc1:8.87}{8.87}\indexp{Erodoto!8!00870000 @87|ca} l'inseguimento è lasciato ad un \textgreek{τριήραρχος} un po' superficiale, in \href{http://data.perseus.org/citations/urn:cts:greekLit:tlg0016.tlg001.perseus-grc1:8.93}{8.93}\indexp{Erodoto!8!00930000 @93|ca} ritroviamo Aminia\index[n]{Aminia} di Pallene. Il verbo \textgreek{ἐβύθισεν} è interessante poiché, sconosciuto alla storiografia di V e IV secolo, si trova solo in  Diodoro\index[n]{Diodoro} (5.4.2\indexp{Diodoro!5!00040002 @4.2|ca} = Timeo FGrHist 566 F164\indexp{Timeo FGrHist 566!F!01640000 @164|ca}; l.59; 20.93.2; 23.16.1; 26.18.1). All'episodio di Artemisia\index[n]{Artemisia di Alicarnasso} e Aminia\index[n]{Aminia} segue la celebre chiosa di Serse\index[n]{Serse} ''\textgreek{οἱ μὲν ἄνδρες μοι γυναῖκες γεγόνασιν, αἱ δὲ γυναῖκες ἄνδρες}''. Di nuovo il tentativo razionalizzante del racconto rende la famosa frase quasi fraintesa per il gran desiderio di inserirla, soprattutto collocandola nel contesto alternativo dell'impresa di Aminia\index[n]{Aminia}. La ritroviamo in\emph{ Anonymi Paradoxographi, Tractus de mulieribus} 13. 6\indexp{Anon. Paradoxographi!\textit{Tractus de mulieribus}!13.6|ca}, in Suda e nella sua più esplicita formulazione latina in Giustino:  \emph{inter primos duces bellum acerrime ciebat [scil. Artemisia\index[n]{Artemisia di Alicarnasso}], quippe ut in viro muliebrem timorem, ita in muliere virilem audaciam cerneres} (2.12.24\indexp{Giustino!000200120024 @2.12.24|ca}). Non è difficile che questo, se non era già una facile sentenza, sia divenuto un detto proverbiale tramite Erodoto.  Giustino\index[n]{Giustino} probabilmente riferisce una forma più elaborata secondo il suo tempo, sullo stesso tema.  La conservazione delle massime proverbiali è una caratteristica di FGrHist 104 (Cfr. 2.5 e 16.4), e così la riduzione di episodi o dialoghi in sentenze. L'inversione di \textgreek{μοι γυναῖκες} con \textgreek{γεγόνασιν}, rispetto all'originale erodoteo, non è significativa, e potrebbe essere avvenuta in uno qualsiasi dei passaggi della tradizione senza danno.
%%%%%    
\subsubsection{\textgreek{οἱ Ἕλληνες ἐβούλοντο λύειν τὸ ἐπὶ τοῦ Ἑλλησπόντου ζεῦγμα}}
Tutte le fonti concordano sul fatto che vi fu dibattito riguardo all'eventualità di tagliare la ritirata a Serse\index[n]{Serse}. La proposta è attribuita a Temistocle, da Erodoto\index[n]{Erodoto} (\href{http://data.perseus.org/citations/urn:cts:greekLit:tlg0016.tlg001.perseus-grc1:8.108}{8.108}\indexp{Erodoto!8!01080000 @108|ca}) e Plutarco; essa invece spetta ai Greci oltre che per FGrHist 104, per  Giustino\index[n]{Giustino} (2.13.5\indexp{Giustino!000200130005 @2.13.5|ca}) e Polieno\index[n]{Polieno} (1.30.4\indexp{Polieno!1.30.4|ca}) entrambi molto simili ad FGrHist 104. Anche perché questo è uno di quei casi in cui un nucleo di informazione si presta effettivamente, senza controindicazioni, ad essere modificato per ricoprire diverse funzioni nella narrazione. In Erodoto\index[n]{Erodoto} i Persiani fuggono terrorizzati e i Greci li inseguono subito, fermandosi per questo dibattito ad Andro, perché non vedono le navi (e non sanno probabilmente da che parte dirigersi per l'inseguimento) ed è Temistocle\index[n]{Temistocle} che propone di andare a tagliare i ponti (Hdt. \href{http://data.perseus.org/citations/urn:cts:greekLit:tlg0016.tlg001.perseus-grc1:8.108}{8.108}.2\indexp{Erodoto!8!01080002 @108.2|ca}). In  Diodoro\index[n]{Diodoro} non si trova esattamente questo episodio e tutta l'azione del capitolo è affidata a Temistocle, ma nelle parole del messaggio riportato in forma indiretta si nota come sia proprio questo messaggio ciò che la tradizione preserva: \textgreek{μέλλουσιν οἱ Ἕλληνες πλεύσαντες ἐπὶ τὸ ζεῦγμα λύειν τὴν γέφυραν} [i Greci erano in procinto di navigare verso il giogo per sciogliere il ponte] (11.19.5\indexp{Diodoro!11!00190005 @19.5|ca}). \cite{Tuci2007}; \cite[289s]{Baragwanath2008}.
    Soltanto  Diodoro\index[n]{Diodoro} considera che i Greci non conoscessero il luogo in cui i ponti erano stati installati: non troviamo in questo capitolo alcuna parola sull'Ellesponto e l'espressione  utilizzata pare ridondante (\textgreek{ζεῦγμα/γέφυραν}). Plutarco\index[n]{Plutarco} ambienta la discussione a Salamina e la immagina tra Aristide\index[n]{Aristide} e Temistocle\index[n]{Temistocle} e in quel capitolo ricorrono diversi elementi del nostro testo come il molo di cui si è già parlato (Plut. \emph{Them.}  16\indexp{Plutarco!\textit{Temistocle}!000160000 @16|ca}), ma quello che mi pare più interessante è il piccolo frammento dell'argomentazione diretta di Temistocle\index[n]{Temistocle}, che avrebbe proposto la missione \textgreek{ὅπως τὴν Ἀσίαν ἐν τῇ Εὐρώπῃ λάβωμεν}, ripreso in forma indiretta in \emph{Arist. }9\indexp{Plutarco!\textit{Aristide}!000090000 @9|ca} come \textgreek{κρεῖττον δὲ λείπεσθαι τὸ λαβεῖν ἐν τῇ Εὐρώπῃ τὴν Ἀσίαν}.  L'argomentazione di Temistocle\index[n]{Temistocle} non c'è in Erodoto\index[n]{Erodoto} e nemmeno in  Diodoro\index[n]{Diodoro} come abbiamo visto. Ma doveva esserci nella fonte di Plutarco. L'uso del verbo \textgreek{λαμβάνω} è l'unico pallido indizio, insieme al banalissimo \textgreek{ἐν} col dativo, del fatto che FGrHist 104 si rifacesse alla medesima fonte di Plutarco, diversa da quella di  Diodoro\index[n]{Diodoro} e  Nepote\index[n]{Cornelio Nepote} che, nel passo parallelo della \emph{Vita di Aristide}, risulta arricchita di elementi erodotei come per esempio \textgreek{τὴν  ταχίστην} per \textgreek{ἰθέως} che non si trovano nell'altro luogo plutarcheo. 
    A seconda delle fonti il ruolo raziocinante di chi si oppone a tale iniziativa è attribuito all'uno o all'altro soggetto in questione, così come la proposta di quest'azione sconsiderata viene messa in bocca agli uni o agli altri a seconda dell'intento di chi scrive: Erodoto\index[n]{Erodoto} fa di  Euribiade\index[n]{Euribiade}  il saggio consigliere, Plutarco\index[n]{Plutarco} dà il ruolo ad Aristide\index[n]{Aristide} (\textgreek{Οὐκ ἀσφαλὲς} riassume bene le argomentazioni dell' Euribiade\index[n]{Euribiade}  erodoteo e dell'Aristide\index[n]{Aristide} plutarcheo.); FGrHist 104, Giustino, Polieno\index[n]{Polieno} e l'epistolografo di Temistocle\index[n]{Temistocle} (\emph{Them. Ep.}  20.34 \indexp{Lettere di Temistocle!00200034 @20.34|ca}) a Temistocle, come in modo del tutto diverso  Diodoro\index[n]{Diodoro} e Nepote. La preoccupazione è dei Greci per  Diodoro\index[n]{Diodoro} (11.19.5\indexp{Diodoro!11!00190005 @19.5|ca}), mentre la attribuiscono a Temistocle\index[n]{Temistocle} sia  Nepote\index[n]{Cornelio Nepote} (\emph{Them.}   5,1\indexp{Cornelio Nepote!\textit{Temistocle}!5.1|ca}) che  Giustino\index[n]{Giustino} (2.13.6\indexp{Giustino!000200130006 @2.13.6|ca}). In Plutarco\index[n]{Plutarco} le argomentazioni riprendono un po' più da vicino quelle di FGrHist 104: \textgreek{ἀλλὰ πάντα τολμῶν καὶ πᾶσιν αὐτὸς παρὼν διὰ τὸν κίνδυνον}  [ma oserà tutto per tutto egli di persona, per via del pericolo]  (\emph{Them.}  16\indexp{Plutarco!\textit{Temistocle}!000160000 @16|ca}) questo scioglie il significato del \textgreek{φιλοκινδυνώτερον} del nostro testo, ma a FGrHist 104 si accostano ancora più decisamente Polieno\index[n]{Polieno} e Giustino. In quest'ultimo troviamo, dopo la decisione di lasciare indietro Mardonio\index[n]{Mardonio}: \emph{ut intercluso reditu aut cum exercitu deleretur aut desperatione rerum pacem victus petere cogeretur} (2.13.5\indexp{Giustino!000200130005 @2.13.5|ca}) ma il più interessante è il confronto con \emph{Stratagemata} dove si dice che \textgreek{Θεμιστοκλῆς ἀντιβουλεύεται λέγων ‘βασιλεὺς ἀποληφθεὶς ἀναμαχεῖται τάχα· πολλάκις δὲ ἀπόνοια δίδωσιν, ὅσα μὴ ἔδωκεν ἀνδρεία’}. [Temistocle diede un'opinione opposta dicendo ''Il Re, ripresosi, ritornerà a combattere: spesso offrono  temerarietà quanti non hanno dato in coraggio''] (1.30.4\indexp{Polieno!1.30.4|ca}). Il verbo non è presente in nessun altro testo parallelo. Ciò che è più interessante tuttavia è il breve discorso diretto con sentenza gnomica che Polieno\index[n]{Polieno} riporta nel suo testo: questo è tipico di FGrHist 104, Cfr.\ref{anteprasse}. \textgreek{ἀπόνοια} corrisponderebbe al nostro \textgreek{φιλοκινδυνώτερον}, mentre \textgreek{ἀποληφθεὶς ἀναμαχεῖται} a \textgreek{ἐξ ὑποστροφῆς}. Dalla stessa fonte, il compilatore del testo del codice parigino potrebbe in questo punto aver omesso il discorso diretto di Temistocle\index[n]{Temistocle}, laddove Polieno\index[n]{Polieno} non l'ha fatto. Con questo e con il frammento di proposta in Plutarco\index[n]{Plutarco} avremmo due pezzetti di discorso diretto provenienti dalla fonte comune. Se ammettiamo che Polieno\index[n]{Polieno} ne riporti la versione più completa, a Plutarco\index[n]{Plutarco} si potrebbe attribuire una riscrittura della sentenza nella frase sopra citata, e conservare FGrHist 104 come riscrittura in forma indiretta. 
%%%%
    \subsubsection{\textgreek{ἀντέπρασσε}}\label{anteprasse}
    Questo ruolo del ''saggio consigliere'' dopo la proposta dei ponti, è anche il caso in cui si può osservare una effettiva divergenza tra Plutarco\index[n]{Plutarco} e FGrHist 104.  Su questo si ritrovano d'accordo anche  Giustino\index[n]{Giustino} e Polieno\index[n]{Polieno} (che attribuiscono il ruolo, significativamente, a Temistocle\index[n]{Temistocle}), ma soprattutto l'Epistolografo di Temistocle. La tipologia di azione si ritrova anche nell'autore degli Stratagemata che in 1.30.4\indexp{Polieno!1.30.4|ca} dopo aver detto, come il nostro, che \textgreek{Θεμιστοκλῆς ἀντιβουλεύεται}, riporta una sentenza in un frammento di discorso diretto, metodo tipico di FGrHist 104, che compie la stessa scelta in due casi su tre di riduzione del discorso in apoftegma (1.5 e 16.4).
%%%%%%    
    \subsubsection{\textgreek{ἔπεμψε κρύφα Ξέρξῃ}}
    In Erodoto\index[n]{Erodoto} (\href{http://data.perseus.org/citations/urn:cts:greekLit:tlg0016.tlg001.perseus-grc1:8.110}{8.110}.2\indexp{Erodoto!8!01100002 @110.2|ca}) un secondo intervento di Temistocle\index[n]{Temistocle} guadagna la fiducia degli Ateniesi e procrastina alla successiva primavera la spedizione all'Ellesponto per poi mandare Sicinno\index[n]{Sicinno} e altri fedelissimi ad informare Serse\index[n]{Serse} di questa benemerenza  nei suoi confronti. Narrazione e contesto sono diversi, o meglio, parzialmente omessi da  Diodoro\index[n]{Diodoro} che pure riprende con precisione Erodoto. L'unica soluzione sarebbe pensare che  Diodoro\index[n]{Diodoro} utilizzi una fonte omologa a FGrHist 104, depennando il dialogo precedente e mantenendo soltanto l'inganno, che risulterebbe conforme a tutte le fonti tranne Erodoto, l'unico a considerare l'aspetto egoistico del comportamento di Temistocle\index[n]{Temistocle}.\emph{ Schol. Arist.} III 615. \cite[260 n.27]{Blosel2004} dice ''\emph{diese Konstellation hat die spätere Tradition so massiv irritiert, daß sie sie umkehrte}''  ma anche qui, come per gli arcieri cretesi studiati da Cagnazzi (\cite*{Cagnazzi2003}) non credo dovremmo ''ingenuamente'' servirci del solo Erodoto, considerando tutto il resto uno stravolgimento. Blösel ha ragione quando sottolinea che ''\emph{lässt ihn diese radikale Kehrtwendung vollziehen. Entscheidend ist dabei der Bezugspunkt des Sinneswandels: Themistokles´ Haltung gegenüber seines athenischen Landsleuten}'' (\textgreek{μεταβαλὼν πρὸς τοὺς Ἀθηναίους}). Come FGrHist 104 riprende la formula utilizzata per Sicinno\index[n]{Sicinno},  Diodoro\index[n]{Diodoro} riutilizza \textgreek{Διόπερ ὁ βασιλεὺς πιστεύσας τοῖς λόγοις διὰ τὴν πιθανότητα}, la stessa frase che aveva utilizzato per la reazione del Re a quel primo messaggio. 
    FGrHist 104 si pone in una posizione ancor più vicina a Temistocle\index[n]{Temistocle} di  Diodoro\index[n]{Diodoro} e  Nepote\index[n]{Cornelio Nepote} ( Diodoro\index[n]{Diodoro} 11.19.6\indexp{Diodoro!11!00190006 @19.6|ca},  Nepote\index[n]{Cornelio Nepote} \emph{Them.}  5.3.\indexp{Cornelio Nepote!\textit{Temistocle}!00050003 @5.3|ca}), che rientra nell'ottica del frammento 189 di Eforo\index[n]{Eforo}\indexp{Eforo FGrHist 70!F!01890000 @189|ca}\label{ref:EforoF189plutdemalign855f} (Plut. \emph{De Herodoti Malignitate} 855 F\indexp{Plutarco!\textit{De Malignitate Herodoti}!00855 @855F|ca}). Sullo scopo di questo secondo messaggio segreto, generalmente si pensa, con  Euribiade\index[n]{Euribiade}  e l'Aristide\index[n]{Aristide} di Plutarco\index[n]{Plutarco} (\emph{Them.}  16\indexp{Plutarco!\textit{Temistocle}!000160000 @16|ca} che proponeva addirittura un secondo ponte) all'intento di  velocizzare l'evacuazione di terre e mari greci dai Persiani. Blösel (\cite[270s]{Blosel2004}) nel suo studio sul Temistocle\index[n]{Temistocle} di Erodoto, nega la necessità di tale intervento e pensa piuttosto che esso prova la completa dissolutezza del racconto erodoteo che, senza questo episodio, ''reggerebbe''. Credo che, nell'ottica dell'inizio degli anni 20 del V secolo a.C., la diffusione di una tradizione anti-Temistoclea, o meglio, che ne considerava il cambiamento in itinere, come sottolineato dallo stesso Blösel, fosse mal compresa e male accetta, ma nonostante tutto abbia trovato una sua via tramite la riabilitazione probabilmente operata nel IV secolo. 
%%%%%%%%%%%    
    \subsubsection{\textgreek{Δίκαιος γὰρ ὁ Θεοκύδους}}
    \textgreek{ὡς δισμυρίων ἀνδρῶν} è problematico rispetto al \textgreek{τρισμυρίων} di Hdt. \href{http://data.perseus.org/citations/urn:cts:greekLit:tlg0016.tlg001.perseus-grc1:8.65}{8.65}\indexp{Erodoto!8!00650000 @65|ca}. L'indicazione è legata ad un numero indicativo, ed il dato non è importante. Tuttavia, volendo ipotizzare un'origine per questa divergenza, essa potrebbe essere trovata con facilità in uno scambio grafico nel sistema numerale acrofonico, che potrebbe aver riportato solo \textgreek{MM} leggendo \textgreek{MMM} come in tanti casi di confusione da traslitterazione dei numerali. \cite[147s]{Ronconi2003}. Il codice usa il sistema milesio prevalentemente, scrivendo per esteso l'unità di misura (e.g. \textgreek{εἰ λάβοι στρατοῦ μυριάδας λ.} 2.1; \textgreek{Πλαταιῶν στάδιά ἐστιν π} 2.3; \textgreek{Βοιωτῶν μυριάδες δ} 2.3; 5.4; \textgreek{ἔχοντες ς ναῦς ἐπολέμησαν ἐπὶ ἔτη ἓξ} 11.3) o scrivendo in lettere il numero (\textgreek{δώσειν αὐτοῖς μύρια τάλαντα}  2.2; \textgreek{ἑκατόν τε ναῦς ἑλόντες } 11.2; \textgreek{οἱ μὲν Λακεδαιμόνιοι ἦσαν τὸν ἀριθμὸν μύριοι τρισχίλιοι, οἱ δὲ Ἀθηναῖοι μύριοι ἑξακισχίλιοι·}  12). Il principio in base al quale la prima o la seconda modalità vengono usate è variabile e si direbbe casuale.
%%%%%%%%%%    
\subsection*{Platea}
Anche il racconto della battaglia di Platea raccoglie e dispone una considerevole serie di tradizioni provenienti da diverse fonti. Se ne ricava comunque un'informazione completa ma senza racconto, senza critica. Elencare tre, quattro cambi di posizione è antieconomico, se l'evento distintivo e da ricordare non è che il semplice scambio in sé e il motivo di questo.
%%%%%%%    
\subsubsection{\textgreek{φεύγοντος δὲ τοῦ Ξέρξου Μαρδόνιος … στρατοῦ μυριάδας λ}} 
Questa specifica introduzione riguardo al ruolo di Mardonio\index[n]{Mardonio} - cui probabilmente si riferiscono anche i versi di Timoteo  \textgreek{πέδιος ἀνήρ, ἁμερο - / δρόμοιο χώρας ἄναξ}, piuttosto che a Serse\index[n]{Serse}, poiché l'uomo della battaglia campale per antonomasia, è proprio il figlio di Gobria, coetaneo probabilmente del Re (\textgreek{νεωτέρων ἔργων ἐπιθυμητὴς ἐων}, Hdt. \href{http://data.perseus.org/citations/urn:cts:greekLit:tlg0016.tlg001.perseus-grc1:7.5}{7.5}\indexp{Erodoto!7!00050000 @5|ca}) - non implica necessariamente, come sosteneva Jacoby, che il personaggio fosse precedentemente nominato, è invece interessante perché parrebbe riprendere direttamente i primi capitoli del settimo libro delle \emph{Storie} (\href{http://data.perseus.org/citations/urn:cts:greekLit:tlg0016.tlg001.perseus-grc1:7.5}{7.5}-8\indexp{Erodoto!7!00050000 @5-8|ca}) dove Mardonio\index[n]{Mardonio} comincia a svolgere un ruolo di primo piano e viene presentato come personaggio fortemente caratterizzato dai temi di vendetta che usa per convincere Serse\index[n]{Serse} a compiere la missione. Il riferimento ai Magi è interessante, riportandoci ad Erodoto\index[n]{Erodoto} che racconta dettagliatamente delle Magofonie istituite da Gobria (Hdt. \href{http://data.perseus.org/citations/urn:cts:greekLit:tlg0016.tlg001.perseus-grc1:3.79}{3.79}\indexp{Erodoto!3!00790000 @79|ca} e Ctesia FGrHist 688 F13\indexp{Ctesia FGrHist 688!F!00130000 @13|ca}). \cite[295]{Asheri1990}. In Erodoto\index[n]{Erodoto} Mardonio\index[n]{Mardonio} viene lasciato in Grecia perché figlio di Gobria\index[n]{Gobria} (\href{http://data.perseus.org/citations/urn:cts:greekLit:tlg0016.tlg001.perseus-grc1:8.100}{8.100}\indexp{Erodoto!8!01000000 @100|ca}) su consiglio di Artemisia\index[n]{Artemisia di Alicarnasso} di Alicarnasso (\href{http://data.perseus.org/citations/urn:cts:greekLit:tlg0016.tlg001.perseus-grc1:8.101}{8.101}-103\indexp{Erodoto!8!01000000 @100-103|ca}), che nonostante si opponga a lui decisamente, in qualche modo lo salva, almeno per un altro anno. In  Diodoro\index[n]{Diodoro} 11.19.5-6\indexp{Diodoro!11!00190005 @19.5-6|ca}, molto più sintetico, non vengono richiamati il ruolo di Mardonio\index[n]{Mardonio}, né la sua richiesta, né tanto meno i suoi motivi e si dice soltanto che fu lasciato indietro con cavalieri e fanti (Diod. 11.19.6\indexp{Diodoro!11!00190006 @19.6|ca}). La spiegazione tattica di Mardonio\index[n]{Mardonio} (\textgreek{πολὺ πλήθος}) è un altro di quei dettagli di completamento che FGrHist 104 inserisce per chiarire in modo semplice la logica della serie degli eventi.  Giustino\index[n]{Giustino} fornisce una sintesi più completa delle motivazioni che emergono dalle posizioni di Mardonio\index[n]{Mardonio} e Artemisia\index[n]{Artemisia di Alicarnasso} in Erodoto\index[n]{Erodoto} (\href{http://data.perseus.org/citations/urn:cts:greekLit:tlg0016.tlg001.perseus-grc1:2.13}{2.13}.2\indexp{Erodoto!2!00130002 @13.2|ca}). Plutarco\index[n]{Plutarco} (\emph{Arist. }10.1\indexp{Plutarco!\textit{Aristide}!00010001 @10.1|ca}) parla invece delle truppe di Mardonio\index[n]{Mardonio} e di una lettera di minacce altrimenti sconosciuta. La tradizione associava Persiani, Medi e Magi confondendoli, \cite[15-17]{Graf1984}. In Teopompo\index[n]{Teopompo}, FGrHist 115 F 64a\indexp{Teopompo FGrHist 115!F!00640000 @64a|ca} (= Diog. Laert. I 8-9\indexp{Diogene Laerzio!1.8-9|ca}), dove i Magi, di cui si parla nell'ottavo libro dei \emph{Philippika}, nella digressione sugli \textgreek{Θαυμάσια}, sono così definiti: \textgreek{ὃς καὶ ἀναβιώσεσθαι κατὰ τοὺς Μάγους φησὶ  τοὺς ἀνθρώπους καὶ ἔσεσθαι ἀθανάτους} [dice sui Magi che gli uomini tornano in vita e sono immortali] e  F 65\indexp{Teopompo FGrHist 115!F!00650000 @65|ca} ( = Plut. \emph{De Is. et Os.} 370 B\indexp{Plutarco!\textit{De Iside et Osiride}|ca}) \textgreek{Θεόπομπος δέ φησι κατὰ τοὺς Μάγους ἀνὰ μέρος τρισχίλια ἔτη τὸν μὲν κρατεῖν τὸν δὲ κρατεῖσθαι τῶν θεῶν, ἄλλα δὲ τρισχίλια μάχεσθαι καὶ πολεμεῖν καὶ ἀναλύειν τὰ τοῦ ἑτέρου τὸν ἕτερον·} potrebbe far pensare al numero di truppe rimaste con Mardonio\index[n]{Mardonio} per combattere a Platea.
    Come in 1.1 ed 1.4 sembra che il verbo \textgreek{αἰτέω} venga usato all'aoristo per riassumere le discussioni presenti in Erodoto. \cite[104]{Pelling2006}. In questo caso chiaramente si tratta del dialogo che troviamo in Hdt. 8.100-103\label{ref:riduzionedialogo}. Mentre Mardonio\index[n]{Mardonio} cerca di calmarlo assumendosi in qualche modo la responsabilità (8.100.5), è Artemisia\index[n]{Artemisia di Alicarnasso} che fa definitivamente pendere l'ago della bilancia per la scelta che verrà in effetti praticata. Il lessico utilizzato, con il Re autore della decisione, è quello di  Diodoro\index[n]{Diodoro} (\textgreek{καταλιπῶν} Diod. 11.19.6\indexp{Diodoro!11!00190006 @19.6|ca} / \textgreek{ὑπελείπετο} \emph{Arist. }10.1\indexp{Plutarco!\textit{Aristide}!00010001 @10.1|ca}) che risale, nella sua semplicità, probabilmente a Eschilo\index[n]{Eschilo} (\emph{Pers.} 803\indexp{Eschilo!\textit{Persiani}!0803 @803|ca}), ma il cui più diretto referente sono le parole di Artemisia\index[n]{Artemisia di Alicarnasso} (\href{http://data.perseus.org/citations/urn:cts:greekLit:tlg0016.tlg001.perseus-grc1:8.102}{8.102}\indexp{Erodoto!8!01020000 @102|ca}) dove \textgreek{σὺν τοῖσι} potrebbero sì essere le truppe, ma anche le \textgreek{κεναῖσιν ἐλπίσιν}. Il numero delle truppe della vulgata, che abbiamo appena visto in Hdt. \href{http://data.perseus.org/citations/urn:cts:greekLit:tlg0016.tlg001.perseus-grc1:8.100}{8.100}.5\indexp{Erodoto!8!01000005 @100.5|ca} è molto variabile.
La tradizione delle cifre è ancora più flessibile e incerta di quella delle parole, e soltanto le esagerazioni estreme possono essere considerate significative. Eschilo\index[n]{Eschilo} (\emph{Pers.} 803\indexp{Eschilo!\textit{Persiani}!0803 @803|ca}) non fornisce comunque informazioni quantitative.  Giustino\index[n]{Giustino} (22.12.3\indexp{Giustino!002200120003 @22.12.3|ca}), Strabone (9.2.31\indexp{Strabone!000900020031 @9.2.31|ca}) e Plutarco\index[n]{Plutarco} (\emph{Arist. }10\indexp{Plutarco!\textit{Aristide}!000100000 @10|ca}; 19\indexp{Plutarco!\textit{Aristide}!000190000 @19|ca}) seguono la  tradizione dei 300.000.  Diodoro\index[n]{Diodoro} (11.19.6\indexp{Diodoro!11!00190006 @19.6|ca}) ha \textgreek{μ' μυριάδων} (400.000) che per Jacoby è una ''\emph{korruptel der hss}'' (Komm. 324); diverge  Nepote\index[n]{Cornelio Nepote} (\emph{cum CC milibus peditum, quos viritim legerat, et XX equitum}; in \emph{Paus}. 1,1\indexp{Cornelio Nepote!\textit{Pausania}!1.1|ca}).
%%%%%%%%    
\subsubsection{\textgreek{Μαρδόνιος ἔπεμψε ... τὸν Φιλίππου πρόγονον}} 
Se si esclude l'ipotesi che sia un cattivo riassunto di Hdt. \href{http://data.perseus.org/citations/urn:cts:greekLit:tlg0016.tlg001.perseus-grc1:8.139}{8.139}\indexp{Erodoto!8!01390000 @139|ca} e che si siano confusi il Filippo nonno di Argeo con il famoso Filippo II, questo passo stabilisce, nella vita di Filippo (c.382-336 a.C.), il \emph{terminus post quem} per la composizione dell'opera. Secondo Jacoby questa notazione ''non dimostra la dipendenza di FGrHist 104 da uno ''\emph{historicus Philippo coaevo}'' (Jacoby FGrHist 104 Komm. 324). Anche Pownall (\cite*{Pownall2011}, \emph{ad loc.}) critica la possibilità di assegnare ad un autore di quarto secolo l'intero testo in ragione di questo passo. C'era un legame politico con Alessandro\index[n]{Alessandro I il filelleno} il Filelleno che motiva probabilmente il far riferimento proprio a Filippo tra i tanti successori intercorsi, se, come sostiene Mari (\cite[46]{Mari2000}), ''la politica di Filippo II a Delfi e Olimpia fu in qualche modo la piena realizzazione della basi gettate dal Filelleno, riuscendo a imporre sia pure in un contesto di rapporti di forza totalmente diversi, la medesima combinazione di rispetto delle regole e spirito rivoluzionario'', ma è sicuramente più rilevante di questa connessione a posteriori il legame che, come Filippo al suo tempo, Alessandro\index[n]{Alessandro I il filelleno} molto prima aveva intrattenuto con gli Ateniesi. Alessandro\index[n]{Alessandro I il filelleno} I nacque non prima del 524/3 perché aveva 65 anni quando morì (Plut. \emph{Them.}  31.5\indexp{Plutarco!\textit{Temistocle}!00031005 @31.5|ca}), molto probabilmente nel 459. e doveva averne più di 16 all'epoca della spedizione scitica di Dario, quando si colloca l'episodio che per primo ne segna la non completa devozione nei confronti dei Persiani: Hdt. \href{http://data.perseus.org/citations/urn:cts:greekLit:tlg0016.tlg001.perseus-grc1:5.17}{5.17}-21\indexp{Erodoto!5!00170000 @17-21|ca}, dove lo troviamo a far uccidere l'intera ambasceria persiana (\cite[38]{Cole1978}). Per altri elementi \cite[131]{Scaife1989}; \cite[38]{Mari2000}; \cite[107]{Badian2006}. Il rapporto di prossenia, che in Erodoto\index[n]{Erodoto} è motivazione del suo invio come messaggero (\href{http://data.perseus.org/citations/urn:cts:greekLit:tlg0016.tlg001.perseus-grc1:8.136}{8.136}.1\indexp{Erodoto!8!01360001 @136.1|ca}) da parte di Mardonio\index[n]{Mardonio}, era probabilmente un diretto legame con Temistocle\index[n]{Temistocle}. Badian (\cite[117s]{Badian2006}.), sulla base Hdt. \href{http://data.perseus.org/citations/urn:cts:greekLit:tlg0016.tlg001.perseus-grc1:6.44}{6.44}.1\indexp{Erodoto!6!00440001 @44.1|ca} mostra che, quando Mardonio\index[n]{Mardonio} giunse la prima volta nel 483 per la fallimentare spedizione in Tracia, la Macedonia era libera, e l'ostilità di Alessandro\index[n]{Alessandro I il filelleno} ben nota. La definizione di Alessandro\index[n]{Alessandro I il filelleno} come \textgreek{πρόγονος} di Filippo II era abbastanza diffusa nel IV secolo e  Demostene (6.11\indexp{Demostene!00060011 @6.11|ca}) la attesta nel medesimo contesto di riferimento; si trova anche in Arpocrazione (\emph{Lessico sui dieci oratori} s.v. \textgreek{ Ἀλέξανδρος}\indexp{Arpocrazione!Lessico sui dieci oratori|ca}) e ancor più dettagliatamente in Elio Aristide (I 233\indexp{Elio Aristide!I 233|ca} e II 286\indexp{Elio Aristide!II 286}). Troviamo un sintagma simile anche in P. Med. Inv 71.76, 71.78, 71.79\index[pap]{P.Med.!Inv. 71.76, 71.78, 71.79} Fr. 5+6, ma riferito dall'autore dell'\emph{Encomio} agli Ateniesi stessi (\textgreek{ἐν Πλαταιὰς Ἀλέξανδρον δὲ τὸν Μακεδόναπρεσβευτὴν καὶ τοὺς ἡμετεροὺς …αντος προγόνους}). Nel 504-500 ''\emph{there is a distinct possibility that Themistocles was present at the Olympiad in which Alexander was judged Greek and competed in the stadion. Occasion in which they first met}'' (\cite[39]{Cole1978}). Attestato anche da Licurgo\index[n]{Licurgo} \indexp{Licurgo!71|ca} \textgreek{τὸν παρὰ Ξέρξου πρεσβευτὴν Ἀλέξανδρον, φίλον ὄντα} [Alessandro il messaggero da Serse\index[n]{Serse}, in quanto era amico...] (71). Alessandro\index[n]{Alessandro I il filelleno} il Macedone compare anche in Thuc. 1.137.1\indexp{Tucidide!1!01370001 @137.1|ca} e nell'Epistolografo di Temistocle (\emph{Them. Ep.}  5.6\indexp{Lettere di Temistocle!00050006 @5.6|ca} e 20.15\indexp{Lettere di Temistocle!00200015 @20.15|ca}) ad ulteriore comprova dei legami personali che lo stratega intratteneva con quelli che poi sarebbero stati i suoi ospiti nel momento dell'esilio, \cite[66]{CulassoGastaldi1990}. Sull'eventualità di una conoscenza meno mediata della corte macedone da parte di Teopompo che fu forse, come Aristotele, a Pella, \cite[46]{Ferretto1984}.
Le missioni di Alessandro\index[n]{Alessandro I il filelleno} sono due in Erodoto, ma la prima (\href{http://data.perseus.org/citations/urn:cts:greekLit:tlg0016.tlg001.perseus-grc1:8.136}{8.136}\indexp{Erodoto!8!01360000 @136|ca}) è stata considerata da alcuni studiosi come una costruzione a posteriori di intenti filomacedoni (\cite[18]{Badian2006} su Bozra), perché nel riportare un messaggio da parte di Mardonio\index[n]{Mardonio} che, a sua volta Erodoto\index[n]{Erodoto} fa tramite di Serse\index[n]{Serse} (\href{http://data.perseus.org/citations/urn:cts:greekLit:tlg0016.tlg001.perseus-grc1:8.140}{8.140}\textgreek{α}1\indexp{Erodoto!8!01400000 @140a1|ca}), Alessandro\index[n]{Alessandro I il filelleno} aggiunge il proprio consiglio benevolo agli Ateniesi. FGrHist 104 tralascia invece il secondo intervento, la vera benemerenza che si trova in Hdt. \href{http://data.perseus.org/citations/urn:cts:greekLit:tlg0016.tlg001.perseus-grc1:9.44}{9.44}-46\indexp{Erodoto!9!00440000 @44-46|ca}, dove è proprio Alessandro, a suo rischio e pericolo, ad informare i Greci della decisione di Mardonio\index[n]{Mardonio} di attaccare. Anche questa non è certo una scelta filomacedone, ma certamente filoateniese, come conferma anche la successiva violenta cacciata. Alessandro\index[n]{Alessandro I il filelleno} non è presente in Diodoro, dove i responsabili dell'ambasciata sono semplicemente \textgreek{οἱ μὲν οὖν ὑπὸ τῶν Περσῶν ἀποσταλέντες}.
%%%%%%%%%    
\subsubsection{\textgreek{ὑπισχνούμενος δώσειν … ὑβρίσαντές τε τὸν Ἀλέξανδρον ἀπεπέμψαντο}} 
Le offerte di Serse\index[n]{Serse} / Mardonio\index[n]{Mardonio}  tramite Alessandro\index[n]{Alessandro I il filelleno} sono uno dei punti che ha dato più problemi a Jacoby, che dichiarava ''\emph{die quellenverhältnisse liegen nicht einfach, weil die paar grunddarstellungen zahllose male kombiniert und veriiert sind}'' (Jacoby FGrHist 104 Komm, 324). Ciò è indubbiamente vero, ma forse il contesto delle altre tradizioni può orientare almeno un po'. Anche non inserire  Diodoro\index[n]{Diodoro} e Plutarco\index[n]{Plutarco} nello stesso filone di tradizioni per la comunanza nel ''non menzionare'' Alessandro, rende la situazione più chiara. Paradossalmente, quella di Hdt. \href{http://data.perseus.org/citations/urn:cts:greekLit:tlg0016.tlg001.perseus-grc1:8.140}{8.140}\textgreek{α}2 \indexp{Erodoto!8!140a2|ca} è l'offerta più scarsa. Serse\index[n]{Serse} dice a Mardonio\index[n]{Mardonio}: \textgreek{τοῦτο μὲν τὴν γὴν σφι ἀπόδος, τοῦτο δὲ ἄλλην πρὸς ταύτῃ ἑλέσθων αὐτοί, ἥντινα ἂν ἐθέλωσι, ἐόντες αὐτόνομοι. Ἱρά τε πάντα σφι, ἢν δὴ βούλωνταί γε ἐμοὶ ὁμολογέειν, ἀνόρθωσον, ὅσα ἐγὼ ἐνέπρησα.} [Da loro indietro la terra, e a quella loro ne aggiungano un'altra, quella che vogliono, essendo autonomi. Tutti i loro templi che ho distrutto, se vorranno andar d'accordo con me, ricostruiscili.]  Diodoro\index[n]{Diodoro} 11.28.1\indexp{Diodoro!11!00280001 @28.1|ca} (seguito da  Giustino\index[n]{Giustino} 2.14.1\indexp{Giustino!000200140001 @2.14.1|ca} e Aristid. I 293\indexp{Elio Aristide!I 293|ca}) ci dice invece: \textgreek{ἐὰν τὰ Περσῶν προέλωνται, δώσειν χώραν ἣν ἂν βούλωνται τῆς Ἑλλάδος καὶ τὰ τείχη καὶ τοὺς ναοὺς πάλιν ἀνοικοδομήσειν, καὶ τὴν πόλιν ἐάσειν αὐτόνομον} [se sceglieranno la parte dei Persiani, di dare la terra che vorranno della Grecia, e ricostruiranno mura e navi e la città sarà lasciata autonoma].
    Sono le promesse che si trovano anche in Licurgo\index[n]{Licurgo}  (71\indexp{Licurgo!71|ca}) e Demostene (6.11\indexp{Demostene!00060011 @6.11|ca}) dove viene concesso ad Atene, come nel nostro testo, il governo della Grecia. Proposte molto simili si trovano anche in P.Med. 71.76, 71.78, 71.79\index[pap]{P.Med.!Inv. 71.76, 71.78, 71.79}: \textgreek{τῆς μὲν τῶν Ἑλλήνων ὁμαιχμίας ἀποστῆναι πείθοντος προτεινομένου δ’ἐλευθερίαν αὐτόνομον ἀσφάλειαν καὶ πρὸς  τούτοις  χρήματα πάμπολλα διδόντος}  [chiedendo di lasciare l'alleanza dei Greci, offriva loro libertà, autonomia e l'immunità nonché di dare loro tantissime ricchezze] (Fr. 5+6 col II, 5-11)
    L'ultima testimonianza si trova in Plutarco\index[n]{Plutarco} ed è estremamente simile per forma alla sopracitata dell'\emph{Encomio}: \textgreek{παρὰ βασιλέως τήν τε πόλιν αὐτοῖς ἀναστήσειν ἐπαγγελλομένου καὶ χρήματα πολλά δώσειν καὶ τῶν Ἑλλήνων κυρίους καταστήσειν, ἐκποδὼν τοῦ πολέμου γενομένους} [proponendo da parte del Re, di ricostruire per loro la città, di dare molte ricchezze e di renderli padroni della Grecia, se si fossero ritirati dalla guerra] (\emph{Arist. }10.2).\indexp{Plutarco!\textit{Aristide}!00010002 @10.2|ca} 
    Il primo elemento notevole è che FGrHist 104 è sicuramente il più completo riguardo alle promesse riportate da Alessandro\index[n]{Alessandro I il filelleno} (\textgreek{ὑπισχνούμενος}). Se si trattasse di un accumulo di elementi trovati nelle diverse tradizioni, dovremmo avere l'indicazione dei templi, di Erodoto, le mura di Diodoro, o almeno il governo sulla Grecia promesso in Demostene (\cite[78]{Schepens2007}).
    Per prima cosa la condizione posta agli Ateniesi è di passare dalla parte del Re (Erodoto, Diodoro) o di starsene fuori dalla guerra (FGrHist 104, Plutarco); l'autonomia (\cite[14s]{Ostwald1982}.) è una promessa che Serse\index[n]{Serse} fa soltanto in Erodoto\index[n]{Erodoto} e  Diodoro\index[n]{Diodoro} oltre che in  FGrHist 104, e nell'\emph{Encomio}. In Plutarco\index[n]{Plutarco} invece andranno viste modifiche della promessa di terra fatta sin da Erodoto, più o meno esagerata a fini retorici. La promessa di ricostruzione dei templi sembra subire un processo di laicizzazione in Diodoro, ma in qualche modo paradossale, promettendo ciò che Atene effettivamente cercherà appena dopo Platea, cioè le mura e il dominio della flotta comune. Questa ricostruzione diventa generica rifondazione della città nella \emph{Vita di Aristide}, dove, oltre alla citazione dell'autonomia si perde anche il verbo che la caratterizza (\textgreek{ἀπόδος / δώσειν}) rispetto alla libertà. FGrHist 104 condivide infine con Plutarco\index[n]{Plutarco} e l'\emph{Encomio} l'informazione circa le ricchezze, insieme alla terra e alla ricostruzione. I precisi \textgreek{ μύρια τάλαντα} del nostro corrispondono alle \textgreek{χρήματα πολλά} (ingenti ricchezze) della \emph{Vita di Aristide} ed alle \textgreek{χρήματα πάμπολλα} dell'\emph{Encomio}. L'\emph{Encomio} a sua volta è legato ad Erodoto\index[n]{Erodoto} (\textgreek{ὁμαιχμίας} \href{http://data.perseus.org/citations/urn:cts:greekLit:tlg0016.tlg001.perseus-grc1:8.140}{8.140}\textgreek{α}\indexp{Erodoto!8!01400000 @140a|ca}), anche nel ricordare la concomitante ambasceria spartana. La presenza del passo erodoteo è innegabile, ma anche i passaggi che l'hanno tramandata sono interessanti. Essa viene riletta con il senno di poi da  Diodoro\index[n]{Diodoro} ma sostanzialmente mantenuta tale, modificata in Demostene (per quel che ne resta), nell'\emph{Encomio} ed in FGrHist 104, che ritroviamo in Plutarco. La perdita del verbo della concessione è motivata in FGrHist 104 dalla promessa di libertà e la presenza del più preciso elenco di donazioni e del nome di Alessandro\index[n]{Alessandro I il filelleno} come emissario fa sicuramente optare per una precedenza di questa rispetto a Plutarco, ma direi anche a Diodoro, collocandosi nei dintorni dell'arco cronologico di vita di Demostene come l'\emph{Encomio}. Gli Ateniesi di quest'ultimo testo rispondono allo stesso modo sia al Macedone che agli Spartani, FGrHist 104 invece opera una selezione diversa, tralascia l'interlocuzione con Sparta e non approva del tutto (\textgreek{ὑβρισάντες}) il netto rifiuto di Atene. In Erodoto\index[n]{Erodoto} questo viene posto prima dell'intervento di Sparta, che naturalmente non vuole perdere Atene e dovrà essere avvisata e rammentata delle offerte di Mardonio\index[n]{Mardonio} della precedente ambasciata, per farla accelerare sulla via di Platea (\href{http://data.perseus.org/citations/urn:cts:greekLit:tlg0016.tlg001.perseus-grc1:9.6}{9.6}\indexp{Erodoto!9!00060000 @6|ca}); in  Diodoro\index[n]{Diodoro} invece, il racconto è esposto dopo la scaramuccia, vista però nella questione dei premi. (\textgreek{Διαβοηθείσης δὲ τῆς  τῶν Ἕλληνας ἀλλοτριότητος} 11.28.1\indexp{Diodoro!11!00280001 @28.1|ca})
    %%%%%%%%%%
    \subsubsection{\textgreek{παραγενόμενός τε εἰς τὰς Θήβας … Βοιωτῶν μυριάδες δ}}  
    Il solo FGrHist 104 specifica il numero dei Beoti (40.000) e in P.Oxy. 2469\index[pap]{P.Oxy.!00272469 @27.2469} li accosta ai Greci medizzanti. In  Diodoro\index[n]{Diodoro} troviamo una frase sulle forze alleate ai Persiani complessive (\textgreek{ἐκ τῆς Θράικης καὶ Μακεδονίας καὶ τῶν ἄλλων τῶν συμμαχίδων πόλεων.} 11.28.4\indexp{Diodoro!11!00280004 @28.4|ca}; cfr. 30.1\indexp{Diodoro!11!00300001 @30.1|ca}) stimata in più di 200.000. Tra gli ultimi dell'elenco sicuramente erano considerati anche i Beoti. FGrHist 104 ha un'attenzione particolare per i dati numerici e le altre misure e questo coincide probabilmente con uno dei criteri di selezione delle informazioni (alcuni passi paralleli in \cite{Pownall2011} ad loc.) assieme alla coerenza e alla razionalità che richiedono un contingente per motivare la punizione di 3.4. È notevole l'uso di \textgreek{συνπαρατάσσω }che troviamo per esempio in Xen. Hell. 3.5.22\indexp{Senofonte!\textit{Elleniche}!3.5.22|ca}, D. 18.216\indexp{Demostene!00180216 @18.216|ca} e 229\indexp{Demostene!00180229 @18.229|ca}; Isocr. 12.180\indexp{Isocrate!12180 @12.180|ca}, Licurgo\index[n]{Licurgo}  144\indexp{Licurgo!144|ca}, a confermare ulteriormente le origini del nostro testo nel IV secolo a.C.  
   %%%%%%%%% 
    \subsubsection{\textgreek{μετέστησαν δὲ αὐτούς …  καὶ ἐνίκησαν..}}
    La complessità delle manovre che precedono la battaglia in Erodoto\index[n]{Erodoto} non ha reso la vita semplice ai successivi lettori. Le truppe si trovano disposte a rovescio rispetto alle posizioni iniziali in Hdt. \href{http://data.perseus.org/citations/urn:cts:greekLit:tlg0016.tlg001.perseus-grc1:9.28}{9.28}s.\indexp{Erodoto!9!00280000 @28s.|ca}: gli Spartani alla destra greca, e i Persiani alla sinistra  del loro schieramento, opposti ai Persiani i primi, ai Greci medizzanti i secondi. Queste sono le posizioni di \href{http://data.perseus.org/citations/urn:cts:greekLit:tlg0016.tlg001.perseus-grc1:9.47}{9.47}\indexp{Erodoto!9!00470000 @47|ca}, che è il momento dell'inizio della battaglia, e quindi ancora con l'esercito persiano schierato con i propri alla destra, cioè di fronte agli Ateniesi in questo momento. La semplificazione degli scambi da tre a due evita l'intercorrere di tempo e le alterne decisioni. Di nuovo siamo di fronte ad una riduzione della materia ai ''fatti'' che in questo caso conserva genericamente gli scambi, senza andar troppo per il sottile sulla loro natura, e sottolineando invece alcuni, precisi elementi. Il posto d'onore spetta di nuovo agli Ateniesi e si arriva a dire che gli Spartani combatterono \textgreek{ἀκουσίως} [controvoglia], mentre dei restanti contingenti non si fa parola. Inoltre è posto l'accento sul timore dei Persiani, quando, come Serse\index[n]{Serse} poco prima a Salamina, Mardonio\index[n]{Mardonio} è definito \textgreek{δεδοικώς}.   
  %%%%%%%%%%%  
    \subsubsection{\textgreek{ἐνταῦθα Μαρδόνιος ἔπεσεν γυμνῇ τῇ κεφαλῇ μαχόμενος}}
    La morte di Mardonio\index[n]{Mardonio} sul campo è un evento determinante per lo svolgersi della battaglia, come la morte di Ariamene\index[n]{Ariamene}, comandante della flotta a Salamina, sotto lo sprone di Aminia\index[n]{Aminia}. Non per tutte le fonti, come per Erodoto, la morte avviene in battaglia (Hdt. \href{http://data.perseus.org/citations/urn:cts:greekLit:tlg0016.tlg001.perseus-grc1:9.63}{9.63}\indexp{Erodoto!9!00630000 @63|ca}, \href{http://data.perseus.org/citations/urn:cts:greekLit:tlg0016.tlg001.perseus-grc1:9.64}{9.64}.2\indexp{Erodoto!9!00640002 @64.2|ca}; Diod. 11.31.4\indexp{Diodoro!11!00310004 @31.4|ca}). Ctesia (F13.29\indexp{Ctesia FGrHist 688!F!00130029 @13.29|ca}) e  Giustino\index[n]{Giustino} (2.14.5\indexp{Giustino!000200140005 @2.14.5|ca}) dicono infatti che Mardonio\index[n]{Mardonio} fuggì ignominiosamente  e,  Giustino\index[n]{Giustino} aggiunge, ''morendo poco dopo''. Ctesia (FGrHist 688 F 13.29\indexp{Ctesia FGrHist 688!F!00130029 @13.29|ca}) carica ulteriormente la tradizione, probabilmente di origine persiana ma con chiare influenze greche, contraria alla politica di Mardonio\index[n]{Mardonio}, raccontando di come morì, avendo cercato di depredare il santuario di Apollo a Delfi. Forse queste alternative enfatizzano la minaccia di Artabano\index[n]{Artabano} in Hdt. \href{http://data.perseus.org/citations/urn:cts:greekLit:tlg0016.tlg001.perseus-grc1:7.10}{7.10}\indexp{Erodoto!7!00100000 @10|ca} (Cfr. p.\pageref{ArtyVSMardoc}) e il racconto erodoteo nel passo sopracitato. Cfr. la sezione \ref{104Plut}. 
   %%%%%%%%%%  
    \subsubsection{\textgreek{περὶ τὴν λειποταξίαν γνώμης εἶναι, τὸ τελευταῖον δὲ περὶ τὴν ἀριστείαν τύχης}}
    \label{bkm:RefHeading690041501267828}
    Questo è il paragrafo a cui fa riferimento la nota in cima al foglio 83v del codice parigino (Cfr. p.\pageref{aristodemoincodparsupplgr607}). Aristodemo\index[n]{Aristodemo (il sopravvissuto)} (\cite[187-91]{Lombardo2005} e \cite[62]{Dewald2011}) era stato rimandato a Sparta dalle Termopili per un'infezione all'occhio, per quanto sappiamo da Erodoto\index[n]{Erodoto} \href{http://data.perseus.org/citations/urn:cts:greekLit:tlg0016.tlg001.perseus-grc1:7.229}{7.229}-231\indexp{Erodoto!7!02290000 @229-231|ca} (\cite[75]{Clarke2002}), ma poi, diversamente dall'altro che era con lui, non era tornato sul campo di battaglia, né si era dignitosamente suicidato. Il racconto di FGrHist 104 pare seguire Erodoto, ma in verità se ne scosta notevolmente riguardo al motivo per cui il premio non venne attribuito ad Aristodemo. In Erodoto\index[n]{Erodoto} l'accento per la discriminazione è posto sulla volontà o meno di morire di chi si dimostra valoroso: 
    \textgreek{ἔγνωσαν οἱ παραγενόμενοι Σπαρτιητέων Ἀριστόδημον μὲν βουλόμενον φανερῶς ἀποθανεῖν ἐκ τῆς παρεούσης οἱ αἰτίης} [gli Spartiati presenti ritennero che Aristodemo volesse evidentemente morire per l'accusa che gravava su di lui] (\href{http://data.perseus.org/citations/urn:cts:greekLit:tlg0016.tlg001.perseus-grc1:9.71}{9.71}\indexp{Erodoto!9!00710000 @71|ca}): Posidonio\index[n]{Posidonio (spartano)} al contrario avrebbe lottato senza voler morire (\textgreek{οὐ βουλόμενον ἀποθνῄσκειν}). In FGrHist 104 invece, il valore di Aristodemo\index[n]{Aristodemo (il sopravvissuto)} è posto in dubbio sulla base di un'opposizione tra caso e scelta, e in fondo sulla base del pregiudizio gravante su di lui. La formula utilizzata è interessante per due motivi. Il primo è l'uso del termine  \textgreek{λειποταξίαν}, nome ateniese di un'accusa militare di cui parla ampiamente Lisia\index[n]{Lisia}, ma che usa, in una forma lessicalmente variata, anche Teopompo\index[n]{Teopompo}, in un passo della digressione sui demagoghi (FGrHist 115 F 93\indexp{Teopompo FGrHist 115!F!00930000 @93|ca} = \emph{Schol. Aristoph. Eq.} 226\indexp{Scolia ad Aristoph.!\textit{Equites} 226|ca}: \textgreek{κατηγόρησε γὰρ αὐτῶν ὡς λειποστρατούντων}) e nella stele di Acarne\label{ref:stelediacarne} (GHI n°88\index[pap]{GHI n°88} Cfr. Teopompo\index[n]{Teopompo}  FGrHist 115 F153\indexp{Teopompo FGrHist 115!F!01530000 @153|ca}), laddove gli Ateniesi giurano \textgreek{καὶ οὐκ ἀπολείψω τὸν ταξίαρχων} (Cfr. anche Licurgo\index[n]{Licurgo}  81\indexp{Licurgo!81|ca} \textgreek{ουδ’ἐγκαταλείψω τοὺς ἡγεμόνας}). In secondo luogo è rilevante la struttura apoftegmatica con parallelismo sintattico. Chiaramente FGrHist 104 sceglie di schematizzare l'informazione tradita, non direttamente da Erodoto\index[n]{Erodoto} e la legge in termini ''tecnici'' ateniesi proponendo per tutto il racconto una sintesi netta e memorabile quanto l'episodio, come in diversi altri casi. Il tratto è peculiare del nostro autore, amante delle sentenze brevi e incisive e si notano ancora l'interferenza dell'ottica ateniese e le modalità della selezione e sintesi del racconto storico.
 %%%%%%%%%%%%   
    \subsection*{Dal Filelleno a Micale}
    \subsubsection{\textgreek{οἱ Πέρσαι ἔφυγον … ἐφόνευσαν}} 
    Per Erodoto\index[n]{Erodoto} (\href{http://data.perseus.org/citations/urn:cts:greekLit:tlg0016.tlg001.perseus-grc1:9.70}{9.70}\indexp{Erodoto!9!00700000 @70|ca}) venne distrutto  l'intero esercito di Mardonio\index[n]{Mardonio} salvo 3.000 e tutti quelli che fuggirono con Artabazo (40.000 in Hdt. \href{http://data.perseus.org/citations/urn:cts:greekLit:tlg0016.tlg001.perseus-grc1:9.66}{9.66}\indexp{Erodoto!9!00660000 @66|ca}; \href{http://data.perseus.org/citations/urn:cts:greekLit:tlg0016.tlg001.perseus-grc1:9.70}{70}\indexp{Erodoto!9!00700000 @70|ca}; Diod. 11.31.3\indexp{Diodoro!11!00310003 @31.3|ca}; 33.1\indexp{Diodoro!11!00330001 @33.1|ca}). Sarebbero dunque caduti circa 240.000 uomini. Per Ctesia F13.28\indexp{Ctesia FGrHist 688!F!00130028 @13.28|ca} (= Phot. Bibl. 39B 27), le perdite persiane in tutta la guerra sarebbero state di 120.000 effettivi. Cfr. \cite[XCIII-XCV]{Lenfant2004}.  Diodoro\index[n]{Diodoro} parla di più di 100.000 morti (11.32.5\indexp{Diodoro!11!00320005 @32.5|ca}) a Platea anche se non è chiaro se solo in battaglia o se si considerino anche quelli nell'accampamento. \cite{Pownall2011} \emph{ad loc.}
  %%%%%%%%%%%%  
    \subsubsection{\textgreek{Ἀλέξανδρος ὁ Μακεδών … ἀπολογούμενος ὅτι ἄκων ἐμήδισεν}} 
    Se si parli del contingente fuggito con Artabazo come in  Diodoro\index[n]{Diodoro} (11.33.1\indexp{Diodoro!11!00330001 @33.1|ca}) e in Erodoto\index[n]{Erodoto} (\href{http://data.perseus.org/citations/urn:cts:greekLit:tlg0016.tlg001.perseus-grc1:9.89}{9.89}\indexp{Erodoto!9!00890000 @89|ca}) che racconta come furono decimati dai Traci (\cite[117]{Badian2006}), non è sicuro, ma plausibile. È molto più interessante l'appunto riguardo al destino di questo schieramento, poiché quest'informazione sulla battaglia sostenuta da Alessandro\index[n]{Alessandro I il filelleno} I e sul suo aver medizzato contro voglia (\textgreek{ἄκων ἐμήδισεν}) è unica in tutta la tradizione ed implica un significato di ''medizzare'' non comportamentale ma già politico. \cite[18]{Graf1984}. Erodoto\index[n]{Erodoto} (\href{http://data.perseus.org/citations/urn:cts:greekLit:tlg0016.tlg001.perseus-grc1:8.121}{8.121}.2\indexp{Erodoto!8!01210002 @121.2|ca}), collegandola con una delle dediche offerte dopo Salamina, di cui parla Pausania\index[n]{Pausania il periegeta} (10.14.5-6\indexp{Pausania!001000140005 @10.14.5-6|ca}), menzionava una statua, dedicata da Alessandro, che difficilmente si spiegherebbe altrimenti che in riferimento a questo episodio, avendo egli assiduamente combattuto dalla parte dei Persiani. Mari (\cite*[40]{Mari2000}) ricorda che i Greci avevano giurato ad Apollo Pizio  di \textgreek{δεκατεῦσαι} quanti, essendo Greci, si fossero schierati coi Persiani senza esservi costretti (Hdt. \href{http://data.perseus.org/citations/urn:cts:greekLit:tlg0016.tlg001.perseus-grc1:7.132}{7.132}\indexp{Erodoto!7!01320000 @132|ca}, Diod. 11.3.3\indexp{Diodoro!11!00030003 @3.3|ca}; Licurgo\index[n]{Licurgo}  81\indexp{Licurgo!81|ca}. Plut. \emph{Them.}   20,3-4\indexp{Plutarco!\textit{Temistocle}!00020003 @20.3-4|ca}; Paus 10. 19.1\indexp{Pausania!001000190001 @10.19.1|ca}), e come siano attestati altri gesti di riconciliazione e ricomposizione, di  \textgreek{μὴ μνησικακεῖν}  come la tavola di bronzo dei \textgreek{μαστροὶ} elei (M-L 27\index[pap]{M-L!27|ca}). Alessandro\index[n]{Alessandro I il filelleno} non sarebbe stato sottoposto a questo trattamento comunque, ma avrebbe avuto ogni interesse a promuovere la propria posizione. Soprattutto nel IV secolo poi, sarebbe stato necessario per una parte degli Ateniesi elaborare la posizione del filelleno con tutti gli elementi a disposizione. Demostene riporta questa vittoria macedone sui barbari in due punti delle sue orazioni (23.200\indexp{Demostene!00230200 @23.200|ca} e 13.24\indexp{Demostene!00130024 @13.24|ca}) e in un'orazione del \emph{corpus} demostenico ([Dem.] 12.21\indexp{Demostene!00120021 @12.21|ca} = Anaximenes FGrHist 72 F 31\indexp{Anassimene FGrHist 72!F!00310000 @31|ca}\label{bkm:dem1221}), ormai quasi unanimemente attribuita ad Anassimene di Lampsaco\index[n]{Anassimene di Lampsaco}, sulla base del commento di Didimo\index[n]{Didimo} al \emph{corpus}, si trova una dichiarazione esplicita della vittoria militare legata alla dedica delle statue: \textgreek{Ἀλεξάνδρου τοῦ προγόνου πρώτου κατασχόντος τὸν τόπον, ὅθεν καὶ τῶν αἰχμαλώτων Μήδων ἀπαρχὴν ἀνδριάντα χρυσοῦν ἀνέστησεν εἰς Δελφούς.} [il mio progenitore Alessandro\index[n]{Alessandro I il filelleno} occupò il luogo, quando pose anche la statua umana dorata a Delfi per la vittoria come primizia presa dai prigionieri Medi.] Nessun'altra fonte, fatto salvo per FGrHist 104 parla dell'apologia del Filelleno ma \textgreek{Ἀπολογούμενος} è probabilmente traccia di un discorso presente in una delle fonti di FGrHist 104 in cui si impostava la difesa del macedone.
 %%%%%%%%   
    \subsubsection{\textgreek{σταδίους τέσσαρας τοὺς ἀπὸ Σαλαμῖνος εἰς Μίλητον}}
    I Persiani a Micale optarono per la terraferma a causa della testata \textgreek{Ἐμπειρία} ateniese (Hdt. \href{http://data.perseus.org/citations/urn:cts:greekLit:tlg0016.tlg001.perseus-grc1:9.96}{9.96}.2\indexp{Erodoto!9!00960002 @96.2|ca}; Diod. 11.34.3\indexp{Diodoro!11!00340003 @34.3|ca}). I passi paralleli sulle contemporaneità tra le maggiori battaglie delle guerre persiane sono Diod. 11.24\indexp{Diodoro!11!00240000 @24|ca} per Imera-Termopili ed Hdt. \href{http://data.perseus.org/citations/urn:cts:greekLit:tlg0016.tlg001.perseus-grc1:7.166}{7.166}\indexp{Erodoto!7!01660000 @166|ca} per Imera-Salamina; insieme a Diod. 11.35.1\indexp{Diodoro!11!00350001 @35.1|ca},   Giustino\index[n]{Giustino} 2.14.8-9\indexp{Giustino!000200140008 @2.14.8-9|ca},  Plut. \emph{Aem}. 25.1\indexp{Plutarco!\textit{Emilio Paolo}!000250001 @25.1|ca}. 
    Anche Leotichida\index[n]{Leotichida} nel 476 o nel 469 fu accusato di aver accettato denaro dai Tebani, come molti suoi colleghi in quegli anni (\cite[204]{Flower2002}, CAH5 97-8).  Diodoro\index[n]{Diodoro} (11.27.2\indexp{Diodoro!11!00270002 @27.2|ca}) ci dice come Temistocle fu rimosso dalla strategia. Lo stesso racconto si trova nelle \emph{Lettere di Temistocle} (11. 3\indexp{Lettere di Temistocle!00110003 @11.3|ca})
   %%%%%%%%%%%%% 
    \subsubsection{\textgreek{τρόπαια, καὶ ἑορτὴν ἤγαγον Ἐλευθέρια}}
    L'Epistolografo di Temistocle (\indexp{Lettere di Temistocle!00120002 @12.2|ca}\emph{Them. Ep.}  12.2) parla del trofeo in un simpatico inciso in cui si serve acutamente di un'informazione trovata nella sua fonte riutilizzandola a fini letterari. Fa scrivere da Temistocle\index[n]{Temistocle} ad Aristide\index[n]{Aristide} \textgreek{ἐπιπεσεῖταί σοι τὸ ἐν Σαλαμῖνι τρόπαιον· ἔστιν δὲ λίθινον, οἶσθα, καὶ νεανικῶς εὐμέγεθες} [Ti cadesse addosso il trofeo di Salamina: è di pietra, sai, e bello grosso!]\label{bkm:themep122} \cite[53-56]{CulassoGastaldi1990}. Sul trofeo di Salamina \cite{Beschi2002}. Sui trofei nei principali storici greci si vedano le osservazioni di \cite{Hau2013}.
    L'istituzione delle feste è ricordata per la prima volta da Tucidide, per bocca di Archidamo\index[n]{Archidamo} (2.71.2\indexp{Tucidide!2!00710002 @71.2|ca}), ma le troviamo anche in  Diodoro\index[n]{Diodoro} (11.29.1\indexp{Diodoro!11!00290001 @29.1|ca}), Strabone (9.2.31\indexp{Strabone!000900020031 @9.2.31|ca}) e Plutarco\index[n]{Plutarco} (\emph{Arist. }21\indexp{Plutarco!\textit{Aristide}!000210000 @21|ca}), che parla dell'altare a Zeus Eleuterio e di un'iscrizione su di esso insieme all'istituzione dei giochi (\textgreek{πενταετηρικὸν ἀγῶνα τῶν Ἐλευθερίων}), infine anche in Pausania\index[n]{Pausania il periegeta} (9.2.6\indexp{Pausania!000900020006 @9.2.6|ca}) in una descrizione che ricorda molto i termini utilizzati da Erodoto\index[n]{Erodoto} per la posizione del tripode di Platea. \cite{Liuzzo2012}. Sull'attenzione alle questioni religiose legate alla guerra, si veda \cite{Goodman1986}.
    %%%%%%%%%%%
    \subsubsection{\textgreek{Θηβαίους τε, καθὼς ὤμοσαν, ἐδεκάτευσαν}}
    Il giuramento e l'applicazione della decima sono riportati separatamente da Polibio\index[n]{Polibio}, in una formulazione che tenta di rivalutare parzialmente il ruolo dei Tebani: \textgreek{οἵ γε Θηβαίους τοὺς κατ’ἀνάγκην ἡσυχίαν ἄγειν βουλευσαμένους μόνους τῶν Ἑλλήνων κατὰ τὴν τῶν Περσῶν ἔφοδον ἐψηφίσαντο δεκατεύσειν τοῖς θεοῖς κρατήσαντες τῷ πολέμῳ τῶν βαρβάρων.} [quelli che avevano vinto nella guerra contro i Persiani votarono che i Tebani che, soli, avevano condotto il passaggio dei Persiani per essere lasciati in pace, pagassero una decima agli Dei] (Polyb 9.39.5\indexp{Polibio!09039 @9.39.5|ca}). Erodoto\index[n]{Erodoto} ci ricorda un giuramento simile nel settimo libro, prima della battaglia delle Termopili: \textgreek{οἱ Ἕλληνες ἔταμον ὅρκιον οἱ τῷ βαρβάρῳ πόλεμον ἀειρόμενοι· τὸ δὲ ὅρκιον ὧδε εἶχε· ὅσοι τῷ Πέρσῃ ἔδοσαν σφέας αὐτοὺς Ἕλληνες ἐόντες, μὴ ἀναγκασθέντες, καταστάντων σφι εὖ τῶν πρηγμάτων, τούτους δεκατεῦσαι τῷ ἐν Δελφοῖσι θεῷ} [i Greci che intrapresero la guerra contro il Persiano stabilirono un giuramento. Il giuramento era questo: quanti dei Greci si fossero dati ai Persiani, senza esservi costretti, se fossero andate bene le cose per loro, costoro avrebbero pagato una decima agli dei] (\href{http://data.perseus.org/citations/urn:cts:greekLit:tlg0016.tlg001.perseus-grc1:7.132}{7.132}\indexp{Erodoto!7!01320000 @132|ca}). L'\textgreek{ἀνάγκην} lega questi due testi lasciando poco margine di dubbio rispetto al riferimento di Polibio\index[n]{Polibio}  che pare essere direttamente ad Erodoto. I Tebani saranno infatti oggetto in Erodoto\index[n]{Erodoto} di un episodio di alcuni capitoli in cui i Greci attaccano la città per chiedere risarcimento (\href{http://data.perseus.org/citations/urn:cts:greekLit:tlg0016.tlg001.perseus-grc1:9.86}{9.86}-88\indexp{Erodoto!9!00860000 @86-88|ca}).  Evidentemente in Polibio\index[n]{Polibio} i due fatti sono associati, uno come conseguenza dell'altro, ma il giuramento è ricordato come precedente l'invasione e non si era giurato di colpire Tebe, restando sul generico, non essendo ancora avvenuto lo schieramento dei Beoti con Mardonio\index[n]{Mardonio} e Serse\index[n]{Serse}. In  Diodoro\index[n]{Diodoro} troviamo un formulario più diretto che, al posto dell'\textgreek{ἀνάγκην} e dell'eccezione, mette in primo piano la scelta compiuta: \textgreek{ἐψηφίσαντο τοὺς ἐθελοντὶ τῶν Ἑλλήνων ἑλομένους τὰ Περσῶν δεκατεῦσαι τοῖς θεοῖς, ἐπὰν τῷ πολέμῳ κρατήσωσι.} [Votarono che avrebbero dedicato una decima agli dei dai Greci che avevano scelto la parte dei Persiani, se avessero vinto la guerra.] (11.3.3\indexp{Diodoro!11!00030003 @3.3|ca}) Alla decima che poi, secondo questo giuramento, i Tebani dovettero pagare,  Diodoro\index[n]{Diodoro} fa probabilmente riferimento invece quando dice, che: \textgreek{Θηβαῖοι ταπεινωμένοι διὰ τὴν πρὸς Ξέρξην γενομένην συμμαχίαν} [I Tebani essendo stati umiliati per l'alleanza intrattenuta con Serse] (11.81.1\indexp{Diodoro!11!00810001 @81.1|ca}). Come in Erodoto\index[n]{Erodoto} quindi, nel racconto di  Diodoro\index[n]{Diodoro} questa decisione viene presa prima della battaglia delle Termopili, mentre, prima di Platea (11.29\indexp{Diodoro!11!00290000 @29|ca}) è sottoscritto un altro giuramento formale che include un formulario molto simile a quello ricordato da Licurgo\index[n]{Licurgo}  come proprio della città di Atene e  precedente Platea, dove la decima da prendere dai medizzanti è inclusa nel giuramento stesso. Osserviamo in questo testo la fusione in una delle due diverse decisioni, una distorsione comprensibile, soprattutto nel contesto del discorso retorico ateniese nel IV secolo: \textgreek{οὐ ποιήσομαι περὶ πλείονος τὸ ζῆν τῆς ἐλευθερίας , οὐδ’ἐγκαταλείψω τοὺς ἡγεμόνας οὔτε ζῶντας οὔτε ἀποθανόντας, ἀλλὰ τοὺς ἐν τῇ μάχῃ τελευτήσαντας τῶν συμμάχων ἅπαντας θάψω. καὶ κρατήσας τῷ πολέμῳ τοὺς βαρβάρους, τῶν μὲν μαχεσαμένων ὑπὲρ τῆς Ἑλλάδος πόλεων οὑδεμίαν ἀνάστατον ποιήσω, τὰς δὲ τὰ τοῦ βαρβάρου προελομένας ἁπάσας δεκατεύσω. καὶ τῶν ἱερῶν τῶν ἐμπρησθέντων καὶ καταβληθέντων ὑπὸ τῶν βαρβάρων οὐδὲν ἀνοικοδομήσω παντάπασιν, ἀλλ’ὑπόμνημα τοῖς ἐπιγιγνομένοις ἐάσω καταλείπεσθαι τῆς τῶν βαρβάρων ἀσεβείας} [Non considererò più la vita della libertà, non abbandonerò i capi né vivi né morenti, ma seppellirò tutti quelli tra gli alleati che dovessero cadere in battaglia. E se avrò la meglio nella guerra contro i Persiani, non distruggerò alcuna delle città che hanno combattuto per la Grecia, ma prenderò una decima da tutte quelle che hanno scelto la parte dei Persiani. E non ricostruirò proprio nessuno dei templi distrutti e bruciati dai Persiani, ma lascerò che restino come memoriale dell'empietà dei Persiani per le generazioni a venire] (Licurgo\index[n]{Licurgo} 81 \label{bkm:lycleocr81}\indexp{Licurgo!81|ca}). La fonte di  Diodoro\index[n]{Diodoro} è molto vicina alla formulazione  di Licurgo\index[n]{Licurgo}, sia per il lessico della condizione di vittoria, sia per la scelta diretta. C'è un'altra importante fonte che lega il giuramento esplicitamente a Tebe, la famosa stele di Acarne, testo datato alla metà del IV secolo. \cite[n° 88]{Rhodes2004}\index[pap]{GHI n°88}. Si veda anche \cite{Pownall2011}, \emph{ad loc.} e  \cite[731s]{Krentz2007}. Non riporto tutto il testo, ma, tra le altre cose, i due passi seguenti sono di grande interesse per arrivare alla sintetica formulazione del nostro: \textgreek{καὶ ν/ικήσας μαχόμενος τοὺς βαρβάρους δεκατεύσω / τὴν Θηβαίων πόλιν} [e se vincerò combattendo i Barbari farò  pagare alla città dei Tebani una decima]  (ll.31s.) e poco oltre: \textgreek{καὶ εἰ μὲν ἐμπεδορκοίην τὰ ἐν τῶι ὅ/ρκῶι γεγραμμένα, ἡ πόλις ἡμὴ ἄνοσος εἴ/η, εἰ δὲ μή, νοσοίη·} [e se sarò fedele alle cose scritte nel giuramento, la mia città sia sana, se no, appestata] (ll.39s.). L'iscrizione prevede un giuramento direttamente contro i Tebani e vi troviamo \textgreek{ἄνοσος} detto della città, che ricordiamo in Tucidide\index[n]{Tucidide} 2.49\indexp{Tucidide!2!00490000 @49|ca}, e in Suda (\textgreek{νοσησάσης δὲ τῆς πόλεως}) nel passo in cui si tramandano le conseguenze della morte di Pausania\index[n]{Pausania (reggente)}  che in FGrHist 104 è descritta come \textgreek{λοιμός} (8.5). Il primo dato è quello che qui maggiormente importa. Il testo dell'iscrizione è infatti stato inizialmente riconosciuto come precedente o appena successivo alla battaglia di Platea, ma con alcuni punti critici, che hanno portato Krentz a proporre che esso si riferisse alla battaglia di Maratona. Nessun dubbio sulla datazione del testo, sicuramente della metà del IV secolo. È probabilmente questo il tipo di documento criticato da Teopompo\index[n]{Teopompo}  secondo \indexp{Teone!\textit{Progymn.}!2 II 67, 22|ca}Teone (Progym. 2 II 67, 22 = FGrHist 115 F 153\indexp{Teopompo FGrHist 115!F!01530000 @153|ca}): \textgreek{παρὰ δὲ Θεοπόμπου ἐκ τῆς πέμπτης καὶ εἰκοστῆς τῶν Φιλιππικῶν, ὅτι \Ladd{ὁ} Ἑλληνικὸς ὅρκος καταψεύδεται, ὃν Ἀθηναῖοί φασιν ὀμόσαι τοὺς Ἕλληνας πρὸ τῆς μάχης τῆς ἐν Πλαταιαῖς πρὸς τοὺς βαρβάρους} [nel venticinquesimo libro dei \emph{Philippika}, Teopompo\index[n]{Teopompo} dice che il giuramento dei Greci, che gli Ateniesi dicono essere stato giurato prima della battaglia di Platea contro i Persiani, è falso.] Vedi \cite[9, 80 e n.25]{Shrimpton1991}. Il testo si riferisce all'iscrizione, che al suo tempo poteva essere stata re-iscritta con i caratteri ionici scelti da Archino\index[n]{Archino} (Teopompo FGrHist 115 F 155\indexp{Teopompo FGrHist 115!F!01550000 @155|ca}) e quindi era un ''falso'' l'iscrizione con il giuramento in termini di riproduzione, ma forse anche il giuramento in sé che così direttamente rivolgeva il suo intento di vendetta a Tebe, mescolando tra l'altro due momenti della storia. Non credo ci sia motivo di credere, con Jacoby, che FGrHist 104 segua volontariamente una tradizione respinta da Teopompo\index[n]{Teopompo}, anche perché non sappiamo a quale momento faccia riferimento \textgreek{ καθὼς ὤμοσαν}. La collocazione è coerente con il racconto di Erodoto\index[n]{Erodoto} ma il giuramento di riferimento parrebbe proprio essere ricordato come è tramandato sulla stele, asciugato della complessità del processo storico, e riscritto \emph{post eventum}. Il testo di FGrHist 104 si colloca quindi nella tradizione argomentata da Teopompo\index[n]{Teopompo}  nei \emph{Philippika}, in quel filone di retorica atticista che elaborava il racconto del passato come vediamo nel testo di Licurgo\index[n]{Licurgo}  e nell'iscrizione. Non è questo l'unico esempio, come abbiamo visto per la stele di Trezene e il papiro P. Med. Inv. 71.76, 71.78. 71.79\index[pap]{P.Med.!Inv. 71.76, 71.78, 71.79|ca} (cfr. p.\pageref{ref:decretoditemistocle}). \cite{Siewert1972}.
     %%%%%%%%%%%%   
        \subsection*{Pausania traditore}
        %%%%%%%%%
        \subsubsection{\textgreek{Ἀπὸ δὲ τῆς Περσικῆς … ἐπράχθη τάδε}}\label{bkm:RefHeading3610719231068}
        La divisione che il compilatore del codice o la sua fonte operano in questo punto ha un sapore molto tucidideo (1.118\indexp{Tucidide!1!01180000 @118|ca}), ma non possiamo dire, dati i contenuti divergenti, e spesso più completi, sebbene meno dettagliati rispetto a Tucidide\index[n]{Tucidide} (Cfr. Plutarco, \emph{De Gloria Atheniensium} 347A\indexp{Plutarco!\textit{De Gloria Atheniensium}!00347 @347A|ca}), che il nostro autore si rifaccia in ultima analisi all'Ateniese da questo punto in poi. Dobbiamo comunque fare riferimento in generale a Thuc. 1.89-95\indexp{Tucidide!1!00890000 @89-95|ca}, 1.128-138\indexp{Tucidide!1!01280000 @128-38|ca} per questa sezione relativa a Pausania\index[n]{Pausania (reggente)}, Temistocle\index[n]{Temistocle} e Cilone\index[n]{Cilone}. L'ordine delle vicende esposte è diverso: in Tucidide\index[n]{Tucidide} troviamo innanzi tutto l'episodio delle mura di Atene, poi il periodo di Pausania\index[n]{Pausania (reggente)}  a Bisanzio e la fondazione della Lega e le prime missioni di Cimone\index[n]{Cimone}: Itome, la spedizione in Egitto, Tanagra, Cheronea e tutti gli altri episodi della cosiddetta pentecontaetia (\cite[444s]{Meiggs1972}), che si concludono con la narrazione della morte di Pausania\index[n]{Pausania (reggente)}  e della conseguente persecuzione e morte di Temistocle\index[n]{Temistocle} in Asia. FGrHist 104 mostra in generale e nei dettagli, come vedremo, una riorganizzazione ''logica'' della narrazione, ma una sintesi più ridotta rispetto allo stesso Tucidide\index[n]{Tucidide} (per esempio conserva la favola della figlia di Coronide\index[n]{Coronide} 8.1, e il capitoletto sulla lunghezza delle mura 5.4). Non può trattarsi, quindi di un'elaborazione diretta del solo Tucidide. Nella prima parte, parla di Pausania\index[n]{Pausania (reggente)}  a Bisanzio (4) seguendo il discorso lasciato con Leotichida e Santippo; nel capitolo successivo si osservano i movimenti di Temistocle\index[n]{Temistocle} riguardo l'episodio delle mura ad Atene (5), fornendo dettagli sulle misure che Tucidide\index[n]{Tucidide} dà in un contesto diverso (2.13.7\indexp{Tucidide!2!00130007 @13.7|ca}). Troviamo Pausania\index[n]{Pausania (reggente)}  e gli Ioni, Temistocle\index[n]{Temistocle} e i Peloponnesiaci, con un breve paragrafo che lega i due personaggi nel partire dell'uno e tornare dell'altro (6), e prelude ad un paragrafo dedicato al rompersi dell'alleanza e all'istituzione della lega navale Ateniese, con il probabile riferimento allo spostamento del tesoro da Delo ad Atene (7). Alla collezione degli episodi che preludono e portano alla fine di Pausania\index[n]{Pausania (reggente)}  (8) segue quella che parrebbe una digressione o un recupero di un appunto sul trofeo delle guerre persiane (9) estranea a Tucidide\index[n]{Tucidide} ed al resto della tradizione. La conseguenza dei destini di Pausania\index[n]{Pausania (reggente)}, come in Tucidide\index[n]{Tucidide} è la fuga di Temistocle\index[n]{Temistocle} in Persia con tutte le sue peripezie (10). 
 %%%%%%%%%%%%       
        \subsubsection{\textgreek{κατὰ φιλοτιμίαν τὴν ὑπὲρ τῶν Ἑλλήνων, ἅμα διὰ προδοσίαν}}
        FGrHist 104 è indeciso (Hdt. \href{http://data.perseus.org/citations/urn:cts:greekLit:tlg0016.tlg001.perseus-grc1:8.3}{8.3}\indexp{Erodoto!8!00030000 @3|ca} e Thuc. 1.95\indexp{Tucidide!1!00950000 @95|ca}), quasi contraddittorio. Si dedica subito a Pausania\index[n]{Pausania (reggente)}, ma non prende parte, sorvola sul giudizio e accumula motivazioni e dati, sia personali che politici. Nonostante anche Erodoto\index[n]{Erodoto} nomini per inciso la vicenda (\href{http://data.perseus.org/citations/urn:cts:greekLit:tlg0016.tlg001.perseus-grc1:5.32}{5.32}\indexp{Erodoto!5!00320000 @32|ca}), è la linea di Tucidide\index[n]{Tucidide} e  Diodoro\index[n]{Diodoro} che FGrHist 104 conserva nella richiesta della mano della figlia di Serse\index[n]{Serse} \label{ref:figliadiSerse} (Thuc. 1.128.7\indexp{Tucidide!1!01280007 @128.7|ca}, Diod. 11.44\indexp{Diodoro!11!00440000 @44|ca}, \emph{Them. Ep.}  14.4\indexp{Lettere di Temistocle!00140004 @14.4|ca}). Fortunatamente di questo pezzetto di tradizione possiamo notare il passaggio anche tramite Eforo\index[n]{Eforo} e Teopompo\index[n]{Teopompo}. Plutarco\index[n]{Plutarco} (\emph{De Herod. Mal.} 5 p. 855 F\indexp{Plutarco!\textit{De Malignitate Herodoti}!00855 @855F|ca} = Eforo\index[n]{Eforo} FGrHist 70 F 189\indexp{Eforo FGrHist 70!F!01890000 @189|ca}) ricorda che Eforo\index[n]{Eforo}, parlando di Temistocle\index[n]{Temistocle}, criticava l'omissione tucididea favorendo invece la ''versione spartana'' degli inviati ad Atene. \cite[402-3]{Parmeggiani2011}. Non ci sono lettere però citate in Tucidide, mentre le abbiamo in FGrHist 104 10.1 su Pausania\index[n]{Pausania (reggente)}, come prova del fatto che lo stratego ateniese fosse in effetti a conoscenza dell'accaduto. Eforo\index[n]{Eforo}, nel passo citato da Plutarco\index[n]{Plutarco} dice proprio \textgreek{παρακαλοῦντος αὐτὸν ἐπὶ τὰς \Ladd{αὐτὰς} ἐλπίδας}, come troviamo poco oltre in questo passo di FGrHist 104 rispetto a Pausania\index[n]{Pausania (reggente)}: \textgreek{ὃς ἐπηρμένος τε τῇ ἐλπίδι ταύτῃ.} 
            I banchetti di Pausania\index[n]{Pausania (reggente)}  e i suoi nuovi costumi sono il simbolo esteriore che fa da segnale evidente nonché da prova per i suoi detrattori contemporanei e posteri. In quanto elemento di ''evidenza'' ne troviamo memoria in tutte le fonti, dalle tavole imbandite di Erodoto\index[n]{Erodoto} (\href{http://data.perseus.org/citations/urn:cts:greekLit:tlg0016.tlg001.perseus-grc1:9.82}{9.82}\indexp{Erodoto!9!00820000 @82|ca}), a Tucidide\index[n]{Tucidide} (1.95.1\indexp{Tucidide!1!00950001 @95.1|ca}, 1.130\indexp{Tucidide!1!01300000 @130|ca}),  Diodoro\index[n]{Diodoro} (11.44.5\indexp{Diodoro!11!00440005 @44.5|ca}; 11.46.1-3\indexp{Diodoro!11!00460001 @46.1-3|ca}),  Nepote\index[n]{Cornelio Nepote} (\emph{Paus.} 3\indexp{Cornelio Nepote!\textit{Pausania}!3|ca}), Plutarco\index[n]{Plutarco} (\emph{Cim.} 6\indexp{Plutarco!\textit{Cimone}!000060000 @6|ca}; \emph{Arist.} 16\indexp{Plutarco!\textit{Aristide}!000160000 @16|ca}) e Suda. La semplificazione dell'episodio Erodoteo è già in parte in atto in Tucidide.
            Pausania\index[n]{Pausania (reggente)}  \textgreek{ἐμετριοπάθει} ricorda molte altre notazioni sullo stato d'animo dei personaggi in FGrHist 104 (1.7, 5, 10.3) che possono essere ricondotti alla descrizione dello stile di Teopompo\index[n]{Teopompo}  di Dionigi di Alicarnasso (D.H. Ad Pomp. 6\indexp{Dionigi di Alicarnasso!\emph{Ad Pompeium}!6|ca}). 
      %%%%%%%%%      
            \subsection*{Temistocle e le mura}\label{bkm:RefHeading697021501267828}
            \subsubsection{\textgreek{φθονοῦντες καὶ μὴ βουλόμενοι πάλιν αὐξηθῆναι}}
           Nell'avverbio \textgreek{πάλιν} c'è una prospettiva diversa, dimentica per un attimo del contesto della narrazione e forse coinvolta in altre ricostruzioni delle mura, come quella di Conone\index[n]{Conone}.  L'episodio è celeberrimo e celebrato. La sua posizione all'interno del racconto, speculare all'episodio del tradimento di Pausania\index[n]{Pausania (reggente)}, delinea la struttura parallela delle vite dei due illustri personaggi che si svolgerà nei seguenti capitoli. Rassegna delle fonti e analisi si trovano in \cite[82 e n.15]{CulassoGastaldi1990}. La centralità del personaggio è quasi biografica, tanto da far sparire gli altri ambasciatori, ma l'unità narrativa con il racconto delle vicende del generale spartano è una caratteristica ancora più pregnante. Quella dei tre sacrilegi di Cilone\index[n]{Cilone}, Temistocle\index[n]{Temistocle} e Pausania\index[n]{Pausania (reggente)}, a partire da Pericle\index[n]{Pericle}  testimoniata da Tucidide\index[n]{Tucidide} ha avuto ampia fortuna, fino ad oggi. Sull'\emph{auxesis} ateniese: \cite[447]{Parmeggiani2011}.
  %%%%%%%%%%%%%%          
            \subsubsection{\textgreek{οἱ Λακεδαιμόνιοι αἰσθόμενοι … δεδοικότες περὶ τῶν ἰδίων}}
            Il carattere tucidideo di questo appunto, necessario alla narrazione altrimenti spogliata di ogni dettaglio che renda l'inganno effettivamente comprensibile, è evidente: \textgreek{Οἱ δὲ Λακεδαιμόνιοι ἀκούσαντες ὀργὴν μὲν φανερὰν οὐκ ἐποιοῦντο τοῖς Ἀθηναίοις οὐδὲ γὰρ ἐπὶ κωλύμῃ, ἀλλὰ γνώμης παραινέσει δῆθεν τῷ κοινῷ ἐπρεσβεύσαντο […] τῆς μέντοι βουλήσεως ἁμαρτάνοντες ἀδήλως ἤχθοντο.} [Dopo aver ascoltato, gli Spartani non resero nota la loro ira agli Ateniesi, infatti non avevano mandato l'ambasceria alla loro assemblea per impedire ma per dare una raccomandazione rispetto ad un'opinione […] e perciò essendosi sbagliati nell'intento sotto sotto erano arrabbiati.] (Thuc. 1.92.1\indexp{Tucidide!1!00920001 @92.1|ca}). Questa è una traccia dell'importanza storica dell'inciso più che del racconto. Il commento tucidideo deve restare, seppur banalizzato e razionalizzato e la \textgreek{βούλησις} spartana è definitivamente fraintesa e così le sue conseguenze: \textgreek{οἱ δὲ Λακεδαιμόνιοι οὐκ ἐπέτρεπον αὐτοῖς}. Se anche in questo filtro c'è traccia del secolo successivo,  è da dire che c'è forse più fortuna per Tucidide\index[n]{Tucidide} qui, che in qualsiasi altro passo citato alla lettera. La necessaria spiegazione della posizione spartana viene riletta alla luce della più scontata delle caratteristiche tucididee degli Spartani, il timore dell'ascesa di Atene. \cite{Vattuone2007} in LHG\&L s.v. \textgreek{Ἀφανής}.
  %%%%%%%%%%%          
            \subsubsection{\textgreek{ἐτειχίσθησαν αἱ Ἀθῆναι τὸν τρόπον τοῦτον}}
            Ha ragione Pownall quando dice che è il desiderio di completezza che ha preservato in FGrHist 104  queste misure. Dopo il racconto sulla costruzione delle mura temistoclee, FGrHist 104 5.4 dà tutte le misure delle mura di Atene costruite \textgreek{ἐν δὲ τῷ μεταξὺ χρόνῳ}, lunghe mura comprese, nonostante siano state iniziate da  Cimone\index[n]{Cimone} e finite da Pericle\index[n]{Pericle}. Per le cifre il nostro testo si allontana parzialmente da quelle date da Tucidide\index[n]{Tucidide} (2.13.7)\indexp{Tucidide!2!00130007 @13.7|ca}, oltre a fornire diverse informazioni topografiche. È probabilmente un'altra cinta muraria quella che ha in mente o rispetto alla quale le misure utilizzate sono state calcolate. Conwell (\cite*[37-54]{Conwell2008}) discute ampiamente le fonti (Thuc. 1.105.1-106.2\indexp{Tucidide!1!01050001 @105.1-106.2|ca}; Aeschin. 2.172-3\indexp{Eschine!0175 @2.172.3|ca}; Plut. \emph{Cim.} 13.5-7\indexp{Plutarco!\textit{Cimone}!13.5-7|ca}) relative a questa prima fase della costruzione delle mura e propone come punto di svolta per l'inizio della costruzione gli sviluppi dopo Itome (\cite*[52]{Conwell2008}). Che i due testi siano in stretta relazione, è confermato dalla misura della larghezza delle mura, che in FGrHist 104 è definita come \textgreek{πλατὺ δὲ ὥστε δύο ἅρματα ἀλλήλοις συναντᾶν} proprio come in Tucidide, dove troviamo (1.93.5\indexp{Tucidide!1!00930005 @93.5|ca}) \textgreek{δύο γὰρ ἅμαξαι ἐναντίαι ἀλλήλαις τοὺς λίθους ἐπῆγον}. La tradizione sul muro non risale però a Tucidide: si trova già in Aristofane\index[n]{Aristofane}, \emph{Eq.} 814-6\indexp{Scolia ad Aristoph.!\textit{Equites} 814-6|ca}, dove il riferimento esplicito sembrerebbe essere proprio ai lavori delle mura ai tempi di Temistocle\index[n]{Temistocle}. \cite[272-3]{Montana2002}. I resti delle mura di Thuc. 1.89\indexp{Tucidide!1!00890000 @89|ca} (\textgreek{περιβόλου βραχέα εἱστήκει}) potrebbero corrispondere all'\textgreek{ἀρχαίων} di cui Aristofane\index[n]{Aristofane}  dice che \textgreek{ἀφελών τ’ οὐδέν}. Si parla poi anche di un allargamento del muro con un nuovo tratto (\textgreek{καινοὺς παρέθηκεν} [ne pose di nuovi]), e di conseguenza la cifra che troviamo in Tucidide\index[n]{Tucidide} parrebbe essere stata presa da una fonte scritta giunta probabilmente fino a Dione Crisostomo (75.4\indexp{Dione Crisostomo!75.4} che da ''più di 90'') e precedente al completamento del nuovo tratto. \cite[203]{Totaro2004} prende in considerazione anche \emph{Pericle}\index[n]{Pericle}  13.8\indexp{Plutarco!\textit{Pericle}!00013008 @13.8|ca} sulla lentezza nella costruzione del muro ''\emph{dia mesou}''  confrontando con Plut. \emph{De Gloria Athen} 351A\indexp{Plutarco!\textit{De Gloria Atheniensium}!00351 @351A|ca}; Pl. \emph{Gorgia}, 455E\indexp{Platone!\textit{Gorgia}!0455 @455e|ca} (= Cratino fr. 326 K.-A.) "da tempo, a parole Pericle\index[n]{Pericle}  lo porta avanti; nei fatti, però, tutto è fermo". Il legame delle vicende di Temistocle e Pausania\index[n]{Pausania (reggente)}  con Pericle\index[n]{Pericle}  sembra inscindibile fin proprio dalla commedia. 
            La ripetizione della formula finale, simile a quella che ritroveremo in 13.4 è riconducibile più alla tradizione scoliografica che agli stilemi di Erodoto. 
  %%%%%%%%%          
          \subsection*{Dalla tirannia di Pausania allo spostamento del tesoro}\label{bkm:RefHeading690901501267828}
            In questa sezione il testo mette in parallelo l'espulsione di Temistocle\index[n]{Temistocle} da Atene, probabilmente equivalente all'ostracismo di cui parlano le altre fonti, e il richiamo di Pausania\index[n]{Pausania (reggente)}  con la \emph{scitale}. Fatto che lo accosta al secondo richiamo nel racconto di Tucidide\index[n]{Tucidide} (1.131.1\indexp{Tucidide!1!01310001 @131.1|ca}), quando Pausania\index[n]{Pausania (reggente)}  si trovava per sua volontà e contro l'opinione degli Spartiati in Troade a curare i propri affari. La \textgreek{σκυτάλη} è descritta nello scolio (\emph{Schol. in Thuc.} 1.131.1\indexp{Schol. in Thuc.!1.131.1|ca}). Anche in Nep. \emph{Paus.} 2.6\indexp{Cornelio Nepote!\textit{Pausania}!2.6|ca}. Kelly 1985, 143s. È strano che Pausania\index[n]{Pausania (reggente)}, partito per conto suo  rientri al secondo richiamo tramite questa missiva.
            In Tucidide, l'ostracismo di Temistocle\index[n]{Temistocle} è ricordato per motivare la fuga da Argo al momento in cui gli Ateniesi mandano i Lacedemoni con alcuni dei loro a cercarlo, dopo la morte di Pausania\index[n]{Pausania (reggente)}  (Thuc. 1.135.3\indexp{Tucidide!1!01350003 @135.3|ca}). \cite[55]{Konishi1970}; \cite[174]{Ellis1994}; \cite[107-109]{Westlake1977}. In  Diodoro\index[n]{Diodoro} (11.55\indexp{Diodoro!11!00550000 @55|ca}), invece, l'ostracismo è conseguenza di un'accusa lacedemone, generata dall'umiliazione per il tradimento di Pausania\index[n]{Pausania (reggente)}  (Arist. \emph{Costituzione degli Ateniesi} 23.4\indexp{Aristotele!\textit{Costituzione degli Ateniesi}!23.4|ca}) ed è associabile alla seconda accusa, che infatti segue immediatamente. In Plutarco\index[n]{Plutarco} (\emph{Them.}   22.1\indexp{Plutarco!\textit{Temistocle}!00022001 @22.1|ca}), come in FGrHist 104, la causa esterna  dell'ostracismo è l'invidia. Segue anche l'accenno alle grandi gesta, implicate dall'\textgreek{ὑπερβάλλουσαν σύνεσιν καὶ ἀρετὴν} di FGrHist 104. Sarebbe intrigante ipotizzare che questi eventi lasciati intendere includessero l'operazione condotta con Efialte\index[n]{Efialte} di cui ci informa la sola \emph{Costituzione degli Ateniesi} (25.3\indexp{Aristotele!\textit{Costituzione degli Ateniesi}!25.3|ca}), testo nel quale mi pare si possa ravvisare un'indicazione cronologica che riporta a molto prima dell'arcontato di Conone\index[n]{Conone}  nell'inciso \textgreek{ἔμελλε δὲ κρίνεσθαι μηδισμοῦ}. L'azione di Temistocle\index[n]{Temistocle}, nota Santoni (\cite*[188]{Santoni1999}), è come quella di chi ha paura di essere punito (\emph{Pol.} 5.1302b21\indexp{Aristotele!\textit{Politica}!1302b21|ca}) e dunque si confà alla fuga verso Argo. Se l'accusa è di medismo, l'unica fonte associabile ad Aristotele è Diodoro. La morte del reggente di Sparta stabilisce il momento, a seguito del quale gli Spartani decidono di pretendere vendetta anche contro Temistocle\index[n]{Temistocle}. In nessun caso comunque l'ostracismo è associato con un momento così preciso e probabilmente collocato più indietro nel tempo. Tucidide\index[n]{Tucidide} forse si riferiva a questo momento, ma non c'è modo di provarlo. Resta invece la consonanza con Plutarco.  Diodoro\index[n]{Diodoro} pone questo evento nel 471/0 e anche per White (\cite[146]{White1964}) di tutti gli eventi questo è quello più probabilmente databile in quell'anno. \cite[46]{Cole1978}; \cite[222]{CulassoGastaldi1990}. Tenendo conto di Giustino\emph{ haec urbs [Byzantium] condita primo a Pausania, rege Spartanorum, et per septem annos possessa fuit} (9.1.3\indexp{Giustino!000900010003 @9.1.3|ca}) il periodo di Pausania\index[n]{Pausania (reggente)}  a Bisanzio andò dal 478/7 di Bisanzio (Thuc. 1.94\indexp{Tucidide!1!00940000 @94|ca}) fino al 472/1. \cite[267]{Fornara1966}. Maggiori dettagli e bibliografia in \cite{Liuzzo2010}. Considerando questo dato si potrebbe presumere che il computo sia considerato dal ritorno di Pausania\index[n]{Pausania (reggente)}  dopo il primo processo, che, accettando il punto fermo del 471/0 offerto da  Diodoro\index[n]{Diodoro} (\cite[206]{Giorgini2004}), risalirebbe per sette anni al 477/6, che, mese più mese meno, coincide con le argomentazioni che portano a datare alla tarda primavera 477 la rottura tra la lega e Pausania\index[n]{Pausania (reggente)}, in perfetto accordo con Diodoro, che riporta appunto sotto questo anno il ''momento'' di Aristide\index[n]{Aristide} (\textgreek{τῷ καιρῷ χρώμενος ἐμφρόνως}). Dando un po' di tempo anche a Pausania\index[n]{Pausania (reggente)}, per ricevere la scitale e ritornare a Sparta tramite il Tanaro, anche le date del successivo periodo risultano adeguate. La morte di Pausania\index[n]{Pausania (reggente)} nel 468/7 lascerebbe tempo alla fuga di Temistocle\index[n]{Temistocle}, all'inseguimento e al passaggio per Nasso, senza troppe difficoltà.
            La seconda lettera di Temistocle\index[n]{Temistocle} (\emph{Them. Ep.}  2.3-5\indexp{Lettere di Temistocle!00020003 @2.3-5|ca}) dimostra nuovamente di utilizzare FGrHist 104 o la sua fonte, implicando la contemporaneità del dominio in Troade con la permanenza di Temistocle\index[n]{Temistocle} ad Argo. \cite[222]{CulassoGastaldi1990}. 
            % Nel papiro PBerol 5008 è riportata parte del commento di Didimo a Demostene (ἐκεῖνοι Θεμιστοκλέα λαβόντες μεῖζον ἑαυτῶν ἀξιοῦντα φρονεῖν ἐξήλασαν ἐκ τῆς πόλεως καὶ μηδισμὸν κατέγνωσαν 23.204-5), che si pensa usasse Eforo\index[n]{Eforo} come sua fonte.
            FGrHist 104 riporta l'interpretazione tucididea della vicenda. \cite[130-132]{Badian1993}; \cite[42-67]{Meiggs1972}. La cifra del tributo non è conservata ma difficilmente, anche se lo fosse, potrebbe servire a dirimere il problema dei 460 Talenti. L'unica opzione possibile a riguardo pare essere quella dell'errore paleografico già in Tucidide, poi seguito dalle altre fonti,  Nepote\index[n]{Cornelio Nepote} (\emph{Arist.} 2.3-3.1\indexp{Cornelio Nepote!\textit{Aristide}!2.3-3.1|ca}) e  Diodoro\index[n]{Diodoro} (11.47.3\indexp{Diodoro!11!00470003 @47.3|ca}), che riprendono anche la notizia del trasferimento non presente nell'ateniese. \cite[130 n240 e 150 n302]{Green2006}. \cite[324-339]{Meiggs1972}.  Nepote\index[n]{Cornelio Nepote} soprattutto, pare ricalcare FGrHist 104, seppur con qualche modifica. Il ruolo di Aristide\index[n]{Aristide} non è rimasto tuttavia nel nostro testo. Vedi \cite[275]{Telo2006}, Pap Cair. 43227\index[pap]{Pap Cair. 43227}, fr. 99 K.-A, scena in cui Aristide\index[n]{Aristide} è un altro dei quattro \textgreek{προστάται τοῦ δήμου} titolo utilizzato dall'\emph{Athenaion Politeia} per Aristide, anche se per \cite[53]{Fornara1966} si tratta di un Aristide diverso. Qualche problema si presenta per la collocazione già a questo punto dello spostamento del tesoro da Delo ad Atene che è generalmente posto a conseguenza dalla sconfitta in Egitto e prima della vittoria di  Cimone\index[n]{Cimone} a Cipro. Plutarco\index[n]{Plutarco} ne parla contestualmente all'ostracismo di  Cimone\index[n]{Cimone} (Plut. \emph{Per.} 12\indexp{Plutarco!\textit{Pericle}!000120000 @12|ca}), lasciando pensare ad una datazione nella prima metà del V a.C. non meglio precisabile, nonostante la collocazione narrativa. Le parole di Pericle\index[n]{Pericle}  sembrano avere già in mente il consiglio del giovane Alcibiade ma, facendo riferimento a qualcosa che è già avvenuto, cioè lo spostamento stesso, implicano che dopo l'ostracismo di  Cimone\index[n]{Cimone} ciò venisse discusso, in relazione alla politica edilizia, ma non che il tesoro fosse stato allora spostato, e nemmeno che ciò fosse avvenuto dopo la spedizione in Egitto. Questo è uno dei dettagli omessi da Tucidide\index[n]{Tucidide} che però descrive il contributo a 1.99.3\indexp{Tucidide!1!00990003 @99.3|ca}.  Diodoro\index[n]{Diodoro} 12.38\indexp{Diodoro!12!00380000 @38|ca} può aiutare a ricollocare l'evento e a valutare la collocazione data da FGrHist 104. \cite[422 n.123]{Parmeggiani2011} calcola che lo spostamento del tesoro sia avvenuto nel 461. Sotto l'anno 431, iniziando la sua narrazione con un flashback, dice che: quando stavano ottenendo l'egemonia sul mare spostarono il tesoro, e lo diedero a Pericle\index[n]{Pericle}  da custodire. Ma Pericle\index[n]{Pericle}  arriva al potere quando l'egemonia c'era da un pezzo: \textgreek{καὶ} implica un lasso di tempo tra il trasferimento e l'affidamento come c'è un lasso di tempo tra l'egemonia e lo spostamento, sottolineato dai modi verbali. \textgreek{Μετὰ δὲ τινα χρόνον} vengono richiesti i rendiconti e qui infatti si ritrova in  Diodoro\index[n]{Diodoro} l'episodio di Alcibiade cui ''segue'' il discorso che ricorda Plutarco. Se la datazione al 454 sulla base di ATL (ML 39\index[pap]{M-L!39}) non è utilizzabile perché basata appunto su queste fonti, anche quelle relative alla spedizione in Egitto, che abbiamo visto è possibile ricollocare tra il 463 e il 458, sono logiche ma non possono essere confermate. Per il trasferimento (\cite{Pritchett1969}) qualsiasi data a cavallo tra gli anni '60 e '50 può andare bene, anche prima dell'Eurimedonte. Certo, la spiegazione data da Pericle\index[n]{Pericle} in Plutarco\index[n]{Plutarco} ha una collocazione cronologica diversa, che è quella della querelle sull'edilizia ateniese, rispetto al momento del trasporto \textgreek{Ἀθηναῖοι τῆς κατὰ θάλατταν ἡγεμονίας ἀντεχόμενοι}, che fa invece proprio pensare ad un momento precedente persino all'Eurimedonte, sebbene di necessità successivo (ma non possiamo sapere di quanto) alla costituzione della lega. Credo che le lamentele sorte intorno a Pericle\index[n]{Pericle} e al suo uso delle ricchezze, nonché i malumori degli alleati non avrebbero consentito un tale gesto, che si colloca invece meglio laddove gli alleati credevano ad Atene come nuova egemone sul mare invece di Sparta. Ed è  a questo punto peraltro che Plutarco\index[n]{Plutarco} colloca il sinodo di Samo (\emph{Arist. }25.3\indexp{Plutarco!\textit{Aristide}!00025003 @25.3|ca}). Non è in conflitto con le altri fonti, ed è anzi del tutto coerente e logica (sebbene non abbia pretese di verità in senso assoluto) la cronologia di FGrHist 104, per cui lo spostamento sarebbe stato di poco successivo alla fondazione della lega. Non sappiamo in che rapporto con Taso, sebbene si possa pensare che dopo una tale dimostrazione nessuno sarebbe stato ben disposto, ma sicuramente prima di Nasso, velocemente nominata poco oltre. 
     %%%%%%%%%%       
            \subsection*{Pausania,  Coronide, Arigilio e la mamma}
            Il principio di completezza proprio della narrazione presentata conserva sia il racconto di Cleonice\index[n]{Cleonice} che quello di Argilio\index[n]{Argilio}  che il ruolo della madre di Pausania\index[n]{Pausania (reggente)}, pur non essendo testimone privilegiato di alcuna tradizione in particolare. \cite[121]{Ogden2002} ''\emph{Aristodemos is the exception that confirms the rule}''.
  %%%%%%%%          
            \subsubsection{\textgreek{διεπράξατο δέ τι καὶ τοιοῦτον}}
            È un breve flash back che rincara la dose su Pausania\index[n]{Pausania (reggente)}  e ne riprende la storia prima del \textgreek{πικρῶς τυραννεῖσθαι} del capitolo precedente, quasi a giustificazione. La legge con la quale Licurgo\index[n]{Licurgo}   \textgreek{οὑδ'ἀποδημεῖν ἔδωκε τοῖς βουλομένοις καὶ πλανᾶσθαι, ξενικὰ συνάγοντας ἤθη καὶ μιμήματα βίων ἀπαιδεύτων καὶ πολιτευμάτων διαφόρων} [non permise di viaggiare a chi volesse e anche di esplorare, imparando costumi stranieri e imitando vite senza istruzione e differenti politiche] (Plut. \emph{Lyc.} 27.6\indexp{Plutarco!\textit{Licurgo}!00027006 @27.6|ca})  è violata punto per punto da Pausania\index[n]{Pausania (reggente)}. Ma non perché vi fosse una legge da violare, quanto perché probabilmente in questo episodio la legge trovò la causa per la quale fu emanata (\cite[203]{Flower2002}): è la violazione che porta alla legislazione. Si potrebbe quasi dire che, nella ricostruzione degli eventi legati al secondo ritorno di Pausania\index[n]{Pausania (reggente)}, si cela in realtà la decostruzione di un mito fondativo, rimodellato su ogni successiva necessità di giustificazione. Un ragionamento giuridico non è adatto a questo testo, se posto in termini di medismo e tradimento e porta meno frutto della considerazione delle leggi sociali e religiose. Da tempo l'accento sulle vicende di Pausania\index[n]{Pausania (reggente)}  è posto su queste, piuttosto che sui primi aspetti. Già Erodoto\index[n]{Erodoto} e Tucidide\index[n]{Tucidide} non erano in accordo sul modo di definire il comportamento di Pausania\index[n]{Pausania (reggente)}  (Thuc. 1.130\indexp{Tucidide!1!01300000 @130|ca} e 138.3\indexp{Tucidide!1!01380003 @138.3|ca} \textgreek{ἡγεμῶν}; Hdt. \href{http://data.perseus.org/citations/urn:cts:greekLit:tlg0016.tlg001.perseus-grc1:5.32}{5.32}\indexp{Erodoto!5!00320000 @32|ca} \textgreek{τύραννος}), ed il nostro autore varia rispetto ad essi, utilizzando il titolo di \textgreek{ὑπάρχων} (già usato per Aristide\index[n]{Aristide} ad Egina in 1.4.) descrivendo poi il comportamento di Pausania\index[n]{Pausania (reggente)}, di cui già aveva parlato in 4.2 senza diretti paralleli nella tradizione, soprattutto per l'insistenza sulla  \textgreek{δίαιταν} (4.3), come contenuto del suo comportamento \textgreek{πικρῶς [...] καὶ τυραννικῶς}. \cite[153]{Nafissi2004}. Proprio quest'ultima notazione porta, tramite il \textgreek{ζηλώσαντος γὰρ αὐτοῦ τὴν Περσικὴν τρυφὴν καὶ τυραννικῶς} di  Diodoro\index[n]{Diodoro} (11.44.5\indexp{Diodoro!11!00440005 @44.5|ca}), senza necessità di interpretare, al passo da cui siamo partiti. 
    %%%%%%%%%%%%%        
            \subsubsection{\textgreek{ἦν ἐπιχωρίου τινὸς θυγάτηρ Κορωνίδου ὄνομα}}\label{bkm:RefHeading3656419231068}
            La storia della figlia di Coronide\index[n]{Coronide} non si trova in Tudicide ma è raccontata da  Plutarco\index[n]{Plutarco} (\emph{Cim.} 6.4-7\indexp{Plutarco!\textit{Cimone}!00006004 @6.4-7|ca}) come \textgreek{ὑπὸ πολλῶν ἱστόρηται} e da Pausania\index[n]{Pausania il periegeta} il periegeta (3.17.8-9\indexp{Pausania!000300170008 @3.17.8-9|ca}) in cui ritroviamo il maggior numero di dettagli concordi con FGrHist 104. Il nome Cleonice\index[n]{Cleonice} è caduto in fase di trascrizione, ma FGrHist 104 è l'unica fonte a conservare il nome dell' \textgreek{ἐπιφανῶν γονέων} (\emph{Cim.} 6.4\indexp{Plutarco!\textit{Cimone}!00006004 @6.4|ca}), Coronide. Se il nome della figlia poteva parere ''parlante'' nel contesto della vicenda di Pausania\index[n]{Pausania il periegeta}, il nome del padre può riferirsi alla punizione che FGrHist 104 infligge a Pausania\index[n]{Pausania il periegeta}. Pur non potendo identificare quale dei molti storici indicati da Plutarco, sia alla base di FGrHist 104, resta il fatto che deve essere uno tra quelli che in modo diverso ha usato anche Pausania, e che si conclude per FGrHist 104 con una \textgreek{μανία} che avrebbe richiesto tempo per guarire; similmente leggiamo in Plutarco\index[n]{Plutarco} che la giovane  \textgreek{οὐκ ἐᾶν τὸν Παυσανίαν ἡσυχάζειν}, mentre Pausania\index[n]{Pausania il periegeta} punta direttamente all'\textgreek{ἄγος} da cui non è possibile scampare. La tradizione che ha aggiunto questo elemento è sicuramente una tradizione interessata agli aspetti intimi e personali dell'affaire Pausania\index[n]{Pausania il periegeta}  ma lo scagiona anche, per via sacrale, dalle accuse a lui rivolte: un'attenzione ai \textgreek{πάθη τῆς ψυχῆς} (DH \emph{Ad Pomp.} 6\indexp{Dionigi di Alicarnasso!\emph{Ad Pompeium}!6|ca}) ''psicologica'' nota per Teopompo\index[n]{Teopompo}  e che troviamo qui, sulla scia di Tucidide, incentivata per la descrizione dei comportamenti degli Ateniesi, degli Spartani, ma anche di individui come Pausania\index[n]{Pausania il periegeta}  qui, Temistocle\index[n]{Temistocle} poco sopra e oltre, Aristide\index[n]{Aristide}, Alessandro\index[n]{Alessandro I il filelleno} I, etc. \cite[115]{Rebuffat1993}. Plutarco, nella \emph{Vita di Cimone}, inserisce l'episodio come causa dell'inasprirsi dei rapporti con gli Ioni (quindi nel 478/7) e conclude la persecuzione, con una visita al \textgreek{νεκυμαντεῖον} di Eraclea, dove gli viene profetizzato dallo ''spettro'' della giovane che \textgreek{παύσεσθαι τῶν κακῶν αὐτὸν ἐν Σπάρτῃ γενόμενον} [avrebbero avuto fine i suoi mali una volta tornato a Sparta], aggiunge Plutarco\index[n]{Plutarco} \textgreek{αἰνιττομένη τὴν μέλλουσαν ὡς ἔοικεν αὐτῷ τελευτήν} [riferendosi, sembra, alla sua prossima fine]. 
            %Cfr. anche SEG 9.72.111-121\index[pap]{SEG 9, 72 l.111-121}, una legge della Cirenaica che ha gettato nuova luce sul significato di ''Supplice'' (Casalla 1997, 333s. Con bibliografia). 
            Il termine avrebbe un valore doppio, indicando sia il demone persecutore che il supplice. La \emph{Legge Sacra} di Selinunte (\cite[68s]{Giuliani1998}) identifica inoltre, alcune norme su come un individuo che deve purificarsi dall'influsso negativo di entità demoniche designate come \textgreek{ἐλάστεροι} (\textgreek{ἀλαστῶρ}) debba comportarsi. \cite[114]{Ogden2002}. In FGrHist 104 la specificazione degli effetti del \emph{daimon} della ragazza, specifico del nostro autore, penso vada riferita direttamente alla mania, razionalizzando o almeno completando, rispetto all'\textgreek{εἴδωλον} persecutore e profeta di Plutarco. Se \textgreek{Δαίμονας} davvero mostrasse come FGrHist 104 sia ''more familiar with latin'' perché usa questo termine nel senso di \emph{Di Manes} (Raphael Sealey, citato in \cite[259 n.7]{Frost2005}), l'ipotesi di Ogden (\cite*[121s]{Ogden2002}) ne trarrebbe grande giovamento. Non credo sia così ma, dal confronto dei testi, si direbbe che  la versione di Plutarco\index[n]{Plutarco} semplifichi rispetto a questa.  
            Il suggerimento che Jacoby dà ad Aristodemo (FGrHist 104), che \emph{hätte besser getan, die geschichte 4,2 einzulegen} (Komm. 328) è ben giustificato dall'inizio del successivo periodo \textgreek{τῆς δὲ προδοσίας οὐκ ἐπαύετο}, che di certo non richiama la storia di Cleonice\index[n]{Cleonice}, ma appunto si rifà al capitolo indicato da Jacoby (Diod. 11.45.1\indexp{Diodoro!11!00450001 @45.1|ca}). Come tutti i buoni consigli, anche quello di Jacoby è destinato a rimanere inascoltato dal suo plurisecolare interlocutore.
     %%%%%%%%%%%%       
            \subsubsection{\textgreek{γράψας ἐπιστολὰς Ξέρξῃ Ἀργιλίῳ ἀγαπωμένῳ ἑαυτοῦ δίδωσι}}
            FGrHist 104 procede con il racconto dei fatti riguardanti ''Argilio\index[n]{Argilio} '' seguendo grosso modo la stessa linea di Tucidide. \cite[124]{Ogden2002} discute la differenza tra ''un uomo di Argilio\index[n]{Argilio} '' e ''Argilio\index[n]{Argilio} s'' che si trova nei soli  Nepote\index[n]{Cornelio Nepote} e FGrHist 104. Esso è un etnico soltanto per Tucidide. Aggiunge Ogden ''\emph{If Argilos is a proper name, a speaking one like Cleonice\index[n]{Cleonice} it means earth, clay}'' e richiamerebbe le cave sotterranee del \emph{nekumanteion} di Cuma 
            %(Strabo C244, Massimino di Tiro 8.2) 
            che sarebbe razionalizzata da Tucidide. \cite[264]{CulassoGastaldi1990}. Non so quanto possa essere ritenuto rilevante il diverso destinatario: se è vero ciò che dice Erodoto\index[n]{Erodoto} in \href{http://data.perseus.org/citations/urn:cts:greekLit:tlg0016.tlg001.perseus-grc1:8.98}{8.98}\indexp{Erodoto!8!00980000 @98|ca}, per arrivare da Serse\index[n]{Serse} una lettera sarebbe dovuta necessariamente partire dalla Sardi di Artabazo. Comunque, \textgreek{πρὸς Ξέρξην} di FGrHist 104 è consono a \textgreek{Πρὸς τὸν βασιλέα} di Diod. 11.45.1\indexp{Diodoro!11!00450001 @45.1|ca} mentre hanno \textgreek{πρὸς Ἀρτάβαζον} Thuc. 1.132.5\indexp{Tucidide!1!01320005 @132.5|ca} e Nep 4.1. Si ritrova nel nostro autore un dato molto interessante, e forse più significativo di quanto si sia ritenuto, in riferimento a Tucidide\index[n]{Tucidide} che dice di Argilio\index[n]{Argilio}  \textgreek{παιδικά ποτε ὢν αὐτοῦ καὶ πιστότατος ἐκείνῳ}. Anche in Nep. \emph{Paus.} 4.1:\emph{ Argilius quidam adulescentulus, quem puerum Pausanias amore uenerio dilexerat}. Si veda \cite[128]{Vattuone2004a} per la \emph{dike} non vincolante dell'erotica arcaica.  In FGrHist 104, Argilio\index[n]{Argilio}  è l'amante di Pausania\index[n]{Pausania il periegeta}  (\textgreek{ἀγαπωμένῳ ἑαυτοῦ}) e di nuovo, come per la reazione spartana all'inganno di Temistocle\index[n]{Temistocle}, la versione che abbiamo davanti legge e spiega, dà la soluzione digerita e non il testo dell'informazione. Diodoro, l'altra nostra fonte al riguardo lo dice semplicemente \textgreek{τις τῶν βιβλιαφόρων} e passa subito al problema dell'eliminazione consigliata nelle lettere tralasciando la relazione tra i due. Credo che questa notazione non possa essere sottovalutata nella lettura dell'episodio che, se ha un valore eziologico e si costruisce con le caratteristiche di un rito di evocazione, conserva un tratto evidentemente legato al \emph{nomos} spartano. Ogden (\cite*[123]{Ogden2002}) identifica il luogo di costruzione della ''capanna'' (\textgreek{καλύβην} Thuc. 1.133) al capo Tanaro col \emph{nekumantheion} che sarebbe stato la cava vicina al tempio di Poseidone (n.53 Strabo C636; Pausania\index[n]{Pausania il periegeta} 3.25\indexp{Pausania!000300250000 @3.25|ca}; Pomponius Mela 2.59\indexp{Pomponius Mela 2.59|ca} (cf. 1.103); \emph{Schol. Aristoph. Acharn.} 509\indexp{Scolia ad Aristoph.!\textit{Acharnenses} 509|ca}; Seneca \emph{Hercules Furens} 662-92\indexp{Seneca!\textit{Hercules Furens}!662-92|ca}; Statius \emph{Theb.} 2.32-57\indexp{Stazio!Tebaide 2.32-57|ca} ecc.). Esercita fascino su Ogden, soprattutto la notizia di  Nepote\index[n]{Cornelio Nepote} che fa scavare agli efori un buco. Dobbiamo ricordare innanzi tutto due delle caratteristiche dell'etica che regolamentava i rapporti tra \emph{erastes} ed \emph{eromenoi}. La prima la troviamo in Plutarco: \textgreek{ἔθος ἦν καὶ τοὺς νεωτέρους ὑπὸ τῶν πρεσβυτέρων ἐρωτᾶσθαι ποῦ πορεύονται καὶ ἐπὶ τί, καὶ τὸν μὴ ἀποκρινόμενον ἢ προφάσεις πλέκοντα ἐπιπλήττειν} [era consuetudine che gli adulti chiedessero ai più giovani dove essi fossero diretti e per quale motivo; e che percuotessero chi non rispondeva o adduceva pretesti] (Nom. Lacon. 237b-c \indexp{Plutarco!\textit{Instituta Laconica}!237b-c|ca}\label{bkm:plutnomlacon237bc79}). La seconda è quel ''biasimo sociale'' che colpiva a Sparta ''chiunque si fosse mostrato attratto dal corpo di un fanciullo (\textgreek{εἰδέτις παιδὸς σώματος ὀρεγόμενος φανείη} Xen. \emph{Lacaed. Resp.} 2.13\indexp{Senofonte!\textit{Costituzione degli Spartani}!2.13|ca}), ancor peggio se era colto sul fatto (\textgreek{Εἴ τις φωραθείη ἁμαρτάνων} Plut. Nom. Lacon. 237c). \cite[103 e 223-224]{Vattuone2004a}. Questi tre elementi caratteristici della normativa sull'erotica a Sparta si ritrovano tutti nel racconto di questi fatti fornito da FGrHist 104 che, abbiamo visto, pone l'intero problema di Pausania\index[n]{Pausania il periegeta}  in termini di \textgreek{δίαιταν/διετίθει}, cioè di comportamento abituale e sull'opposizione tra \textgreek{ἀναφανδὸν/διεξῄει}. Riguardo a quest'ultimo, l'intera tradizione ricorda gli efori nascosti per essere \textgreek{αὐτήκοοι} (diretti ascoltatori), fossero essi dietro una tenda costruita da loro o per loro, oppure in un buco (\cite[124]{Ogden2002}), ma solo in FGrHist 104, Argilio\index[n]{Argilio}  come Alessandro\index[n]{Alessandro I il filelleno} e poi Temistocle\index[n]{Temistocle} si vincola facendo una promessa. La promessa è una spiegazione semplice, una motivazione e una causa soddisfacente per un'azione, comoda per tutte le situazioni in cui è l'iniziativa personale l'elemento insondabile. Nell'ottica invece del primo punto, possiamo prendere in considerazione il \emph{nomos} sopra citato come filtro per osservare la tradizione sul dialogo intercorso tra Argilio\index[n]{Argilio}  e Pausania\index[n]{Pausania il periegeta} \label{ref:dialogoargiliopausania}. In Tucidide, che abbiamo visto marcare la relazione sociale tra i due, Pausania\index[n]{Pausania il periegeta}  tiene un atteggiamento che può suonare melenso nei confronti di chi ne ha tradito la fiducia, virtù ricercata seppur spesso disattesa, ma è in realtà chiaramente un rimprovero, inteso nelle due fasi del \textgreek{ἐρωτᾶσθαι} ed \textgreek{ἐπιπλήττειν} che abbiamo visto in Plutarco. \cite[130]{Vattuone2004a} per l'infedeltà ricorrente dei ragazzi come specchio di una crisi della polis. In un certo senso forse anche la disparità è infranta dato che Argilio\index[n]{Argilio}  era \textgreek{ποτε} l'eromenos. Tucidide\index[n]{Tucidide} riporta il discorso di Pausania\index[n]{Pausania il periegeta}  in modo indiretto ed in esso il reggente dice: \textgreek{περὶ τοῦ παρόντος οὐκ  ἐῶντος ὀργίζεσθαι, ἀλλὰ πίστιν ἐκ τοῦ ἱεροῦ διδόντος τῆς ἀναστάσεως καὶ ἀξιοῦντος ὡς τάχιστα πορεύεσθαι.} [non essere arrabbiato per il presente, ma datagli garanzia che  sarebbe uscito dal tempio, chiedeva che partisse al più presto. (1.133\indexp{Tucidide!1!01330000 @133|ca}) FGrHist 104 sinteticamente ci informa di come \textgreek{ἀπεμέμφετο ἐπὶ τῷ μὴ κομίσαι τὰς ἐπιστολὰς πρὸς Ξέρξην}, mentre Diodoro, che aveva tralasciato il dettaglio della relazione amorosa, trasforma l'intera interlocuzione in una serie di scuse e suppliche di Pausania\index[n]{Pausania il periegeta}  al portatore di lettere \textgreek{τοῦ δὲ Παυσανίου φήσαντος μεταμελεῖσθαι καὶ συγγνώμην αἰτουμένου τοῖς ἀγνοηθεῖσιν, ἔτι δὲ δεηθέντος ὅπως συγκρύψῃ }[Pausania si scusò e chiese perdono per gli errori, e implorò affinché conservasse il segreto] (11.45.5\indexp{Diodoro!11!00450005 @45.5|ca}). Resta solo l'elemento del non visibile e del segreto. Se la storia di Pausania\index[n]{Pausania il periegeta}  è frutto della narrazione per motivare un pretesto avanzato prima della guerra (Thuc. 1.128.2\indexp{Tucidide!1!01280002 @128.2|ca}), oppure per motivare un nomos come per il caso della legge licurgea contro l'allontanamento da Sparta, osserviamo qui invece un perdersi progressivo nella tradizione dell'originale contesto normativo, che da socio-etico diviene politico. Oltre ad essere più completo nel significato complessivo, e oltre a contenere peculiarità indipendenti (il nome del padre di Cleonice\index[n]{Cleonice} e l'\emph{hapax} \textgreek{περίυπνος}), FGrHist 104 in questo punto è coerente con la tradizione ricordata in Plutarco\index[n]{Plutarco} (\emph{Nom. Lacon.} 237b-c\indexp{Plutarco!\textit{Instituta Laconica}!237b-c|ca}), e Pausania\index[n]{Pausania il periegeta}  sarebbe coinvolto in un caso legato ai costumi, non solo estetici e di abitudini, ma di \textgreek{παιδικά}. L'indagine e il processo poggiano dunque su modi e termini dell'inchiesta su \textgreek{εἰδέτις παιδὸς σώματος ὀρεγόμενος φανείη} (Xen.\emph{ Lacaed. Resp.} 2.13\indexp{Senofonte!\textit{Costituzione degli Spartani}!2.13|ca}). Con questo abbiamo già due degli elementi fondativi di questo mito di Pausania\index[n]{Pausania il periegeta}  che serve molte funzioni, racconta storie in tanti contesti e riporta l'attenzione sulla complessità del meccanismo di trasmissione e memoria storica, che traspare nella razionalizzazione di FGrHist 104 proprio perché non è più complicato della semplice sintesi coerente. 
      %%%%%%%%%      
            \subsubsection{\textgreek{ἐν ἀπόρῳ ὄντων … ἡ μήτηρ τοῦ Παυσανίου βαστάσασα πλίνθον ἔθηκεν}}
            Anche gli eventi che portano alla morte di Pausania\index[n]{Pausania il periegeta}  vengono rivisti e rinarrati, forse anche in conseguenza di un lungo senso di colpa, come \emph{exemplum}. \cite[192]{Flower2002} ''\emph{evrytime spartans changed something they attributed it to Lycurgus}'' ma l'istituzionalizzazione dei costumi per lo più risale ad Agide III\index[n]{Agide III} e Cleomene IV\index[n]{Cleomene IV}. Nafissi (\cite*[168]{Nafissi2004}) porta ad esempio, tra i molti loci che rigurdano morti esemplari la legge di Demofanto, Andoc. \emph{De Myst.} 96\indexp{Andocide!\textit{De Myst.} 96|ca}, D. 20 159\indexp{Demostene!00200159 @20.159|ca}. La stessa madre di Pausania\index[n]{Pausania il periegeta}  (Alcitea), anziana signora spartana, compare e pone la prima pietra di quelle che mureranno vivo il figlio nel tempio di Atene Calcieca, ''dando inizio alla punizione del figlio''  (\textgreek{προκαταρχομένη τῆς κατὰ τοῦ παιδὸς κολάσεως}). Essa, come Cleonice\index[n]{Cleonice} e Argilio\index[n]{Argilio}  in dubbio tra la vita e la morte, e in perfetta coerenza strutturale con loro e con la storia del figlio, dà risposta alla proverbiale indecisione dei Lacedemoni che \textgreek{ἐν ἀπόρῳ ὄντων διὰ τὴν εἰς τὸν θεὸν θρησκείαν}: questa frase, singolare nella tradizione, pone i due temi in gioco nel racconto mitologico, l'etica e la superstizione, in un nuovo conflitto che genererà il grande \textgreek{ἄγος} Spartano. La madre che pone la pietra è ripresa direttamente da  Diodoro\index[n]{Diodoro} (\cite[115]{Ogden2002}), con le medesime parole di FGrHist 104 ma, al posto della motivazione di FGrHist 104, questo modello etico-civico materno, giunge al tempio \textgreek{μηδὲν μήτ'εἰπεῖν μήτε πρᾶξαι}.  Diodoro\index[n]{Diodoro} è successivo alla tradizione conservata in FGrHist 104 e, pur mantenendo elementi omessi da Tucidide, li rielabora funzionalmente ad un racconto di scorrettezza politica, eliminando i dettagli etico-paideutico-educativi.  Crisermo\index[n]{Crisermo} di Corinto dice invece che: \textgreek{Περσῶν τὴν Ἑλλάδα λεηλατούντων, Παυσανίας, ὁ τῶν Λακεδαιμονίων στρατηγὸς, πεντακόσια χρυσοῦ τάλαντα παρὰ Ξέρξου λαβὼν, ἔμελλε προδιδόναι τὴν Σπάρτην. Φωραθέντος δὲ τούτου, Ἀγησίλαος ὁ πατὴρ μέχρι τοῦ ναοῦ τῆς Χαλκιοίκου συνεδίωξεν Ἀθηνᾶς, καὶ τὰς θύρας τοῦ τεμένους πλίνθῳ φράξας, λιμῷ ἀπέκτεινεν· ἡ δὲ μήτηρ καὶ ἄταφον ἔρριψεν.}  [Fuggiti i Persiani dalla Grecia, Pausania\index[n]{Pausania il periegeta}, stratego dei Lacedemoni, presi cinquanta talenti d'oro da Serse\index[n]{Serse}, stava per tradire Sparta. Coltolo sul fatto, suo padre Agesilao gli corse dietro fino al tempio della Calcieca e ostruì l'ingresso del recinto sacro per ucciderlo con la fame. La madre lo lasciò insepolto.] (Plut. \emph{Parallela minora} 308b\indexp{Plutarco!\textit{Parallela minora}!00308 @308b|ca} = FGrHist 287 F4)
            Questa versione tenta di inserire nuovi personaggi e giustificare comportamenti, ma solo FGrHist 104 dà una spiegazione soddisfacente e logica all'\textgreek{ἀπόρῳ} spartano. Dobbiamo dunque riconoscervi una possibile tradizione di valore, per lo meno ''credibile'', che legava nel racconto dell'\emph{agos} la causa e il delitto, come il filo di Cilone\index[n]{Cilone}  (Cfr. p.\pageref{Ilsacrilegiociloniano}), Questa potrebbe davvero essere una versione più vicina alla fonte per la seconda parte, ma la mazzetta accettata da Pausania\index[n]{Pausania il periegeta}  la porta sulla strada di Diodoro. Pausania\index[n]{Pausania il periegeta} (3.17.8\indexp{Pausania!000300170008 @3.17.8|ca}), introducendo il racconto del reggente di Sparta e di Cleonice\index[n]{Cleonice} come causa della fine dello spartano, conferma di nuovo come Pausania\index[n]{Pausania il periegeta}  sia stato ''colto sul fatto'': \textgreek{ἐφ’οἷς ἐβουλεύετο, dice: ἤκουσα δὲ ἀνδρὸς Βυζαντίου Παυσανίαν φωραθῆναί τε ἐφ’οἷς ἐβουλεύετο καὶ μόνον τῶν ἱκετευσάντων τὴν Χαλκίοικον ἁμαρτεῖν ἀδείας κατ’ ἄλλο μὲν οὐδέν, φόνου δὲ ἄγος ἐκνίψασθαι μὴ δυνηθέντα.} [ho sentito da un uomo di Bisanzio che Pausania\index[n]{Pausania il periegeta}  fu colto sul fatto riguardo ciò che voleva e soltanto lui tra i supplici la licenza per aver sbagliato, non essendo riuscito ad espiare un omicidio sacrilego]. \cite[XXIVs]{ Musti1982}.
            Riguardo la morte di Pausania\index[n]{Pausania il periegeta}, infatti, gli elementi principali sono grossomodo condivisi. Muore di fame. Viene gettato nella Caeda per Plutarco/Crisermo\index[n]{Crisermo}, FGrHist 104, Eliano (VH 4.7) e Suda cfr. anche Licurgo\index[n]{Licurgo}  128\indexp{Licurgo!128|ca}.  Tucidide\index[n]{Tucidide} 1.134.4\indexp{Tucidide!1!01340004 @134.4|ca} conosce ma nega la tradizione stabilita, a conferma di quanto ipotizzato sopra riguardo alla trasmissione della tradizione. La pestilenza (\textgreek{λοιμός}) che avrebbe colpito gli Spartani per questo motivo non è presente in nessun'altra fonte, fatto salvo per il lessico Suda, dove si dice che  \textgreek{νοσησάσης δὲ τῆς πόλεως} [la città di ammalò]: come prospettato dalla punizione prevista nel giuramento della stele di Acarne (GHI n°88\index[pap]{GHI n°88}: \textgreek{ἡ πόλις ἡμὴ ἄνοσος εἴ/η, εἰ δὲ μή, νοσοίη}) cfr. p.\pageref{ref:stelediacarne}. Plat. \emph{Prot.} 322D\indexp{Platone!\textit{Protagora}!00322 @322d|ca}:\textgreek{νόμον γε θές παρ’ἐμοῦ τὸν μὴ δυνάμενον αἰδοῦς καὶ δίκης μετέχειν κτείνειν ὡς νόσον πόλεως.} Non è certo se \emph{Them. Ep.}  4.15-17\indexp{Lettere di Temistocle!00040015 @4.15-17|ca} si riferisca a ciò con quel  \textgreek{παλαμναῖον ἢ ἀλιτήριον προστρῖψαι τῇ πόλει} [spirito vendicatore e punitore implacabile contro la città]. Che invece sembrerebbe direttamente connesso a \textgreek{τοὺς δαίμονας τοῦ Παυσανίου}. La questione delle statue è facilmente frutto di un'oscillazione dovuta a problemi di auralità e memoria. Tucidide\index[n]{Tucidide} proponeva che fossero state date \textgreek{χαλκοῦς ἀνδρίαντας δύο}, FGrHist 104 ha \textgreek{ἀνδριάντα αὐτῷ} seguito da Suda con \textgreek{εἰκόνα ἔστησαν χαλκῆν Παυσανίου}  Diodoro\index[n]{Diodoro} \textgreek{δύο τοῦ Παυσανίου χαλκᾶς}. È molto importante per il ragionamento di Ogden, Plut. \emph{De Sera Num. Vind.} 560 e-f\indexp{Plutarco!\textit{De sera num. Vind.}!560 e-f|ca}: \textgreek{ὀμοίως δὲ καὶ Σπαρτιάταις χρησθὲν ἱλάσασθαι τὴν Παυσανίου ψυχὴν ἐξ Ἰταλίας μεταπεμφθέντες οἱ ψυχαγωγοὶ καὶ θύσαντες ἀπεσπάσαντο [da ἀποσπάω] τοῦ ἱεροῦ τὸ εἴδωλον}.
            Non potendo avanzare considerazioni su ciò che FGrHist 104 ha tagliato, e non potendo condividere di conseguenza le considerazioni di Jacoby al riguardo (Komm., 329.), ricordo  che Ogden diceva che stiamo ''\emph{witnessing an originally unitary tale in the process of diverging}'' (\cite{Carawan1989}; \cite{Westlake1977}; \cite[113]{Ogden2002})\label{ref:stesimbrotofontedithuc} e che Nafissi (\cite*[179]{Nafissi2004}) opponeva una versione spartana ufficiale ad una ateniese che sottolinea l'incuria e la lentezza, sulla base di una diffusione e autorevolezza, attestate anche per Stesimbroto, della \emph{hybris} di Pausania\index[n]{Pausania il periegeta}, che Erodoto\index[n]{Erodoto} non solo conosceva ma scartava. La divergenza da una fonte consapevolmente e spontaneamente accorta rispetto all'\emph{ethos} spartano ed alle implicazioni religiose e morali della morte di Pausania\index[n]{Pausania il periegeta}  di cui si trova traccia in FGrHist 104 e in Pausania\index[n]{Pausania il periegeta} (3.17.8-9\indexp{Pausania!000300170008 @3.17.8-9|ca}), avviene già nel V secolo quando Tucidide\index[n]{Tucidide} sceglie e ripulisce la versione della sua fonte ionica di dettagli come quello della mamma. Questa fonte continua ad essere utilizzata ed il filone razionalizzante si evidenzia soltanto nel processo di trasmissione che porta a Diodoro, forse tramite la critica di Eforo\index[n]{Eforo} a Tucidide\index[n]{Tucidide} ma non si ritrova in chi ha riutilizzato una fonte di V secolo come Crisermo\index[n]{Crisermo} o Pausania.
     %%%%%%%%%%       
            \subsection*{Il disco}\label{bkm:RefHeading696961501267828}
            L'idea dell'iscrizione circolare riportata da FGrHist 104 non è assurda. Jacoby FGrHist 104 Komm, \cite{Pownall2011}, \emph{ad loc}. Ho affrontato il problema relativo a questo passo in uno studio sul trofeo di Platea (Liuzzo 2012, 27-41) e mi limito a riportarne i principali risultati. La storia del monumento e le fonti ad esso relative (Hdt. \href{http://data.perseus.org/citations/urn:cts:greekLit:tlg0016.tlg001.perseus-grc1:9.81}{9.81}\indexp{Erodoto!9!00810000 @81|ca}, Pausania\index[n]{Pausania il periegeta} 10.13.9\indexp{Pausania!001000130009 @10.13.9|ca}, D. 59.16\indexp{Demostene!00590016 @59.16|ca}, Thuc. 1.132\indexp{Tucidide!1!01320000 @132|ca}) portano a pensare che un'iscrizione circolare si possa prendere in considerazione per un momento ben preciso della storia del testo, dopo la cancellazione dell'epigramma di Pausania\index[n]{Pausania il periegeta}. Il cratere, estratto dal tesoro persiano come la spada di Mardonio\index[n]{Mardonio} è un artefatto persiano ed esistono notevoli paralleli che attestano da un lato la diffusione di testi di questo tipo (iscritti attorno al bordo superiore di un cratere) e dall'altro la presenza di artefatti simili di provenienza persiana dell'accampamento persiano di Mardonio\index[n]{Mardonio}. Non ultimo il passo delle \emph{Lettere di Temistocle}\index[n]{Temistocle} (21.1\indexp{Lettere di Temistocle!00210001 @21.1|ca}). Più problematica è la collocazione di questo paragrafo in questo punto della narrazione. Non c'è nessun apparente motivo per interrompere le vicende di Temistocle\index[n]{Temistocle} e Pausania\index[n]{Pausania il periegeta}, né la coerenza cronologica interna a FGrHist 104 lo richiede. Due sono i possibili motivi, in primis la completezza narrativa che vuole preservare l'aneddoto e la peculiare ''idea''; in secondo luogo la narrazione che deve ritornare indietro a narrare la fine di Temistocle partendo da un punto già passato. Cronologia, sincronismi e diacronia sono problemi che lo storico, anche il più povero di ideali e intenzioni, non può che raggirare nella \emph{dieghesis} e il metodo impiegato da FGrHist 104 rientra in questa categoria e per risposta a questa necessità conserva una tradizione che aiuta a dirimere una domanda antica laddove se ne stimi il valore e l'origine.
     %%%%%%%%%       
            \subsection*{La fine di Temistocle}\label{bkm:RefHeading398631508721977}
           La centralità del personaggio nel testo è indiscutibile e significativa come si è già detto. Per l'incontro tra Artaserse\index[n]{Artaserse}  e Temistocle\index[n]{Temistocle}. \cite{Liuzzo2010}, \cite[35s]{Cagnazzi2001}. Per \cite[29]{Cagnazzi2001}, anche Ippia\index[n]{Ippia}  avrebbe scelto di imparare l'antico persiano nel tempo a disposizione. Anche per l'ateniese, come per Pausania\index[n]{Pausania il periegeta}  (5) si elencano miracoli (10.3) e gesta insieme ai detti famosi e alla morte. 
       %%%%%%%    
            \subsubsection{\textgreek{ὁ δὲ Θεμιστοκλῆς δεδοικς}}\label{bkm:RefHeading3451719231068}
            L'episodio della morte di Temistocle\index[n]{Temistocle} offre un ottimo punto di osservazione per rivedere l'intera tradizione di questo racconto anche perché, grazie al recente apporto alla tradizione indiretta di FGrHist 104, confermato dallo scolio sul papiro CLGP Aristoph. 5 (Oxford Bodl. Ms. Gr. Class. f.72\index[pap]{Oxford Bodl. Ms. Gr. Class. f.72|ca} + P. Acad. Inv. 3 d\index[pap]{P. Acad. Inv. 3 d|ca} \cite{Montana2006}), contribuisce a definire i tratti della tradizione indiretta del testo. Possiamo seguire questa tradizione dalla commedia fino ai lessici ed è uno degli elementi più chiaramente riconducibili a quell'interesse scolastico per la composizione di narrazioni sulle morti illustri di grandi uomini del passato (di cui un poetico esempio resta proprio nella ventunesima lettera di Temistocle\index[n]{Temistocle} che è forse tra le fonti di \textgreek{Θ}124\indexp{Suda!Theta124|ca}) e che dimostra come ''l'immagine dello statista è recepita in termini topici e aneddotici, intrinsecamente ambigua, deformata dall'uso strumentale nell'ambito politico e in quello letterario''. Montana 2002, 261. Per la deformazione delle tradizioni, per la tendenza filoateniese e per il parallelo tra Temistocle\index[n]{Temistocle} e il tipico \emph{Trickster} (\cite[26]{Vansina1985}) della tradizione orale africana. \cite[25-30]{Murray2001}. La topica e l'aneddotica diventano quasi filastrocche proverbiali anche per Magnesia, Miune, Lampsaco, che vengono concordemente indicate come il donativo di Artaserse\index[n]{Artaserse}  a Temistocle\index[n]{Temistocle}, per i benefici portati al re, come abbiamo visto per gli Scoli ad \emph{Equites} 84\indexp{Scolia ad Aristoph.!\textit{Equites} 84|ca}. Sin da Tucidide\index[n]{Tucidide} che, dopo aver detto della morte di Temistocle\index[n]{Temistocle}, ricorda che: \textgreek{μνημεῖον μὲν οὖν αὐτοῦ ἐν Μαγνησίᾳ ἐστὶ τῇ Ἀσιανῇ ἐν τῇ ἀγορᾷ· ταύτης γὰρ ἦρχε τῆς χώρας, δόντος βασιλέως αὐτῷ Μαγνησίαν μὲν ἄρτον, ἣ προσέφερε πεντήκοντα τάλαντα τοῦ ἐνιαυτοῦ, Λάμψακον δὲ οἶνον (ἐδόκει γὰρ πολυοινότατον τῶν τότε εἶναι), Μυοῦντα δὲ ὄψον} [il suo memoriale è a Magnesia d'Asia, nell'agorà; governava infatti la terra di questa poiché il Re gli aveva dato Magnesia come pane, che ogni anno gli forniva 50 Talenti, Lampsaco per il vino (che ai tempi si riteneva fosse ricca in vino), Miunte per l'\emph{opson}] (1.138.5\indexp{Tucidide!1!01380005 @138.5|ca}). \cite[66]{Ferretto1984}. Tuttavia, seguendo l'analisi di Marr (\cite*[562]{Marr1996}) dei versi 810-19 dei \emph{Cavalieri} \label{bkm:aristofeq8146} non è difficile vedere questi versi come una ''\emph{single sense unit}'' sui meriti e le azioni di Temistocle\index[n]{Temistocle}. Riporto il passo di Aristofane\index[n]{Aristofane}: \textgreek{Ὦ πόλις Ἄργους, κλύεθ’ οἷα λέγει. Σὺ Θεμιστοκλεῖ ἀντιφερίζεις; /ὃς ἐποίησεν τὴν πόλιν ἡμῶν μεστὴν εὑρὼν ἐπιχειλῆ, / καὶ πρὸς τούτοις ἀριστώσῃ τὸν Πειραιᾶ προσέμαξεν, / ἀφελών τ’ οὐδὲν τῶν ἀρχαίων ἰχθῦς καινοὺς παρέθηκεν· / σὺ δ’ Ἀθηναίους ἐζήτησας μικροπολίτας ἀποφῆναι / διατειχίζων καὶ χρησμῳδῶν, ὁ Θεμιστοκλεῖ ἀντιφερίζων.  / Κἀκεῖνος μὲν φεύγει τὴν γῆν, σὺ δ’ Ἀχιλλείων ἀπομάττει.} (Aristoph. \emph{Eq.} 814-816\indexp{Aristofane!\textit{Equites}!0814 @814-816|ca}). In questo frammento di commedia, se il campo semantico di riferimento del verso 815 (\textgreek{καὶ πρὸς τούτοις ἀριστώσῃ τὸν Πειραιᾶ προσέμαξεν} [per noi impastò il Pireo coi migliori ingredienti]) è il pane e quello del verso 816 (\textgreek{ἀφελών τ’ οὐδὲν τῶν ἀρχαίων ἰχθῦς καινοὺς παρέθηκεν} [togliendo nulla dal vecchio ci imbandì pesce fresco]) è il pesce, il verso 814 (\textgreek{ὃς ἐποίησεν τὴν πόλιν ἡμῶν μεστὴν εὑρὼν ἐπιχειλῆ} [che rese la nostra città trovata colma, zeppa]) riguarda il vino. Montana (\cite*[280]{Montana2002}) nota anche un'allusione colta nei commentatori antichi: al v.312 i tributi arrivano infatti come 'tonni' (\emph{Vesp.} 1087 \textgreek{θυννάζοντες}\indexp{Aristofane!\textit{Vespae}!1087 @1087|ca}). La posizione di Paflagone è assimilata a quella di Serse\index[n]{Serse} che assiste alla disfatta delle proprie navi sulla base di questo elemento, giacché la mattanza della flotta persiana è descritta come strage di navi e uomini \textgreek{ὥστε θύννους} [come tonni] (\emph{Pers.} 424\indexp{Eschilo!\textit{Persiani}!0424 @424|ca})). Il riferimento allusivo identificato dal Marr è la diadi narrativa di Thuc. 1.89.3\indexp{Tucidide!1!00890003 @89.3|ca} con la costruzione delle mura e l'ambasciata a Sparta, ma l'accostamento mi sembra anche rendere chiaro come  possa essere altrettanto presente un riferimento all'attribuzione delle città asiatiche, che peraltro non stupisce in una commedia che gioca con le tradizioni biografiche sullo stratego di Salamina.
            \indexp{Diodoro!11!00570007 @57.7|ca} Diodoro\index[n]{Diodoro} (11.57.7) concorda con Tucidide\index[n]{Tucidide} e Aristofane\index[n]{Aristofane}, ma, a differenza del primo, anziché attribuire un significato metaforico, usando \textgreek{Εἰς} con l'accusativo (\cite[537]{Marr1996}), firma per sempre la concretezza  del dato, che ritroviamo identica in FGrHist 104 e in Plutarco\index[n]{Plutarco} (\emph{Them.}  29.11). Anche nella \emph{Lettere di Temistocle}\index[n]{Temistocle}, \emph{Them. Ep.}  20.35-39\indexp{Lettere di Temistocle!00200035 @20.35-39|ca},  dove peraltro, Lampsaco è \textgreek{ἠλευθέρωσα καὶ πολλῷ φόρῳ βαρυνομένην} (in ATL però, paga dal 450). \cite[262 e n.20]{CulassoGastaldi1990} per bibliografia e diverse posizioni. La ''storicità''  del potentato di Lampsaco è attestata anche da I. aus Kleinasien n°6\index[pap]{I. aus Kleinasien n°6}: \textgreek{[ἐν δὲ τῇ ἑορτῇ] / τῇ Θεμιστοκλεῖ [ἀγομένῃ δι'ἐνιαυ] / τοῦ εἶναι πάντα α[ὐτῷ τἀγαθὰ ἃ ἐδόθη] / σαν Κλεοφάντῳ κ[αὶ τοῖς ἀπογόνοις]} del 200 a.C..  L'unico a discostarsi dalla tradizione che confluisce direttamente in FGrHist 104 sembrerebbe essere Teopompo\index[n]{Teopompo}  ma solo perché, in F87\indexp{Teopompo FGrHist 115!F!00870000 @87|ca} (Plut. \emph{Them.}  31,3\indexp{Plutarco!\textit{Temistocle}!00031003 @31.3|ca}) Temistocle\index[n]{Temistocle} \textgreek{οὐ γὰρ πλανώμενος περὶ τὴν Ἀσίαν, ὥς φησι Θεόπομπος, ἀλλ’ ἐν Μαγνησίαι μὲν οἰκῶν}. Teopompo\index[n]{Teopompo}  dunque affrontava il tema, ma in modo diverso, sebbene anche qui come in altri casi, la variazione si limiti ad una fase. Dopo qualche tempo passato in Asia (e comunque necessario per lo spostamento) Temistocle\index[n]{Temistocle} se ne va a Magnesia. Tucidide\index[n]{Tucidide} ci dice da subito, fin da Susa, che il Re sperava che Temistocle\index[n]{Temistocle} sottomettesse per lui la Grecia, un po' come Ippia\index[n]{Ippia}  (Hdt \href{http://data.perseus.org/citations/urn:cts:greekLit:tlg0016.tlg001.perseus-grc1:6.107}{6.107}\indexp{Erodoto!6!01070000 @107} e Hdt \href{http://data.perseus.org/citations/urn:cts:greekLit:tlg0016.tlg001.perseus-grc1:6.59}{6.59}\indexp{Erodoto!6!00590000 @59|ca}; \cite[25]{Cagnazzi2001}) aveva provato a fare conducendo Dati e Artaferne a Maratona (\textgreek{τοῦ Ἑλληνικοῦ ἐλπίδα ἣν ὑπετίθει αὐτῷ δουλώσειν} Thuc. 1.138.2\indexp{Tucidide!1!01380002 @138.2|ca}) e poi ricorda che, secondo alcuni, si sarebbe ucciso pensando di non poter mantenere la promessa fatta al Re (\textgreek{ἀδύνατον νομίσαντα εἶναι ἐπιτελέσαι βασιλεῖ ἃ ὑπέσχετο} 1.138.4\indexp{Tucidide!1!01380004 @138.4|ca}).\label{ref:promessaditemistocleeierone} Solo FGrHist 104 racconta di Temistocle\index[n]{Temistocle} che promette ad Artaserse\index[n]{Artaserse}  di sottomettere la Grecia con le medesime parole usate da Mardonio\index[n]{Mardonio} a 2.1. In  Diodoro\index[n]{Diodoro} (11.58\indexp{Diodoro!11!00580000 @58|ca}) troviamo soltanto l'invito di Serse\index[n]{Serse}. In Plutarco\index[n]{Plutarco} invece, Temistocle\index[n]{Temistocle} dichiara di essere divenuto nemico dei Greci (\textgreek{Ἑλλήνων πολέμιον γενόμενον}) ma, al capitolo 31.4\indexp{Plutarco!\textit{Temistocle}!00031004 @31.4|ca}, si trova a dover tener fede alle promesse fatte (\textgreek{βεβαιοῦν τὰς ὑποσχέσεις}). Poco prima tuttavia, riportando parole di Stesimbroto (\emph{Them.}   24.7\indexp{Plutarco!\textit{Temistocle}!00024007 @24.7|ca} = FgrHistCont 1002, F4\indexp{Stesimbroto di Taso FGrHistCont 1002!F!004 @4|ca}), Temistocle\index[n]{Temistocle} promette a Ierone di Siracusa di sottomettere la Grecia per lui (\textgreek{ὑπισχνούμενον αὐτῳ τοὺς Ἕλληνας ὑπηκόους ποιήσειν}). Ierone e Serse\index[n]{Serse}/Artaserse\index[n]{Artaserse}  vengono scambiati per le medesime serie di eventi nella fuga di Temistocle\index[n]{Temistocle} (Cfr. sezione \ref{104Stesimbroto}) ed è notevole che anche questo elemento possa essere stato duplicato nelle due vicende.
            La promessa di Temistocle\index[n]{Temistocle} e l'intesa con Artaserse\index[n]{Artaserse}  è dunque un punto di partenza, ed FGrHist 104 è la fonte più esplicita a riguardo. A ciò si aggiunge la certezza della provenienza dello scolio ad Aristofane\index[n]{Aristofane}  \emph{Eq.} 84 conservato da CLGP Aristoph. 5. In FGrHist 104 abbiamo infatti \textgreek{παραγενόμενος εἰς Μαγνησίαν, ἐγγὺς ἤδη γενόμενος τῆς Ἑλλάδος μετενόησεν} e nel papiro si trova proprio, \textgreek{ Περσῶν καὶ ἐστρατεύσατο ε [..8-10] / [3...]ακ ἐν τῇ …. μετανοήσας δ . . ηγεν} [mosse in armi verso... mutato avviso]. Anche se non avessimo la certezza del confronto nel testo conservato dal codice parigino, dovremmo comunque ipotizzare una volontaria presa di posizione per giustificare il cambiamento. La motivazione è ripresa molto da vicino dalle \emph{Lettere di Temistocle}\index[n]{Temistocle} dove troviamo rinarrati in forma letteraria quei dati che circolano tra l'\textgreek{ἠναγκάζετο μετὰ ταῦτα τοῖς Ἕλλησι πολεμεῖν} di Suda \textgreek{Θ}124 e l'\emph{unicum }del nostro testo ''\textgreek{οὐχ ἡγησάμενος δεῖν πολεμεῖν τοῖς ὁμοφύλοις}''. La ''coscienza tragica'' di Temistocle\index[n]{Temistocle} emerge laddove viene immaginato a scrivere a Polignoto: \textgreek{καὶ ἡμᾶς ἄρα τοῦ στρατοῦ προβαλεῖται ἡγεμόνας καὶ Μήδους ὑποτάξει Θεμιστοκλεῖ, καὶ στρατεύσομαι ἐπ’ Ἀθήνας ἐγὼ καὶ τῷ Ἀθηναίων ναυαρχήσοντι μαχοῦμαι; πολλὰ ἄλλα ἔσται, τοῦτο δὲ οὐδέποτε} [e mi proporrà come comandante dell'esercito e schiererà i Medi sotto Temistocle\index[n]{Temistocle}, e io porterò guerra contro Atene e combatterò  contro il navarca Ateniese? Farei molte altre cose, questa mai.] (\emph{Them. Ep.}  20.44\indexp{Lettere di Temistocle!00200044 @20.44|ca}) È evidente che il dettaglio delle mezze parole di Temistocle\index[n]{Temistocle}, fraintese dal Re, si mantiene costante in tutta la tradizione e non ha bisogno di un cambiamento di opinione di Temistocle\index[n]{Temistocle}, quanto invece la versione di FGrHist 104. 
      %%%%%%%%%%      
            \subsubsection{\textgreek{πληρώσας αἵματος ἔπιεν καὶ ἐτελεύτησεν}}\label{ref:morteditemistocle}
         La morte di Temistocle\index[n]{Temistocle} è avvenuta nel santuario di Artemide a Magnesia (Paus. 1.26.4\indexp{Pausania!000100260004 @1.26.4|ca}). Aristofane\index[n]{Aristofane}  commentato dalla letteratura scoliografica, rielaborato dalle lettere e reimpiegato in Suda, in \emph{Equites} 83-5\indexp{Aristofane!\textit{Equites}!83-5|ca} dice: \textgreek{Πῶς δῆτα, πῶς γένοιτ’ ἂν ἀνδρικώτατα; / Βέλτιστον ἡμῖν αἷμα ταύρειον πιεῖν· / ὁ Θεμιστοκλέους γὰρ θάνατος αἱρετώτερος} [come scusa? Come sarebbe la cosa più da tori? sarebbe meglio per noi bere sangue di toro scegliendo la morte di Temistocle\index[n]{Temistocle}] (nella traduzione, ''da tori'' è un'espressione dialettale che si adatta bene al gioco di parole rilevato da \cite[232]{Arnould1993} tra \textgreek{ἀνήρ / ταῦρος}). Il motivo dell'astuzia e della capacità di ideare espedienti in questi versi dove Aristofane\index[n]{Aristofane}  gioca sul rapporto analogico Sangue / Vino è stato notato da Montana (\cite[264-5]{Montana2002}), che mette un accento particolare anche sui versi 43-72\indexp{Aristofane!\textit{Equites}!43-72|ca} la cui la descrizione di Paflagone richiama la caratteristiche tipiche del figlio di Neocle (\textgreek{πανουργότατον καὶ διαβολώτατόν τινα; ᾔκαλλ’, ἐθώπευ’, ἐκολάκευ’, ἐξηπάτα; Ἄιδει δὲ χρησμούς; τέχνην πεπόηται;  ψευδῆ διαβάλλει}). Altri casi di decesso con ''sangue di toro'' si trovano (da \cite[231]{Arnould1993}) in Ctesia (F13.12\indexp{Ctesia FGrHist 688!F!00130012 @13.12|ca}) che racconta la morte di Tanyoxarkes \textgreek{αἵματι γὰρ ταύρου ὃ ἐξέπιεν ἀναιρεῖται Τανυοξάρκης}, in Sofocle Elena fr. 178 Radt\indexp{Sofocle!\textit{Elena}!fr. 178 Radt|ca} \textgreek{ἐμοὶ δὲ λῷστον αἷμα ταύρειον πιεῖν / καὶ μὴ ’πὶ πλεῖον τῶνδ’ ἔχειν δυσφημίας}, e Paus. 7.25.13\indexp{Pausania!000700250013 @7.25.13|ca}: a proposito della sacerdotessa di Gaios in Acaia: \textgreek{πίνουσα δὲ αἷμα ταύρου δοκιμάζονται}. La prima attestazione è il suicidio di Psammetino\index[n]{Psammetino} (Hdt. \href{http://data.perseus.org/citations/urn:cts:greekLit:tlg0016.tlg001.perseus-grc1:3.15}{3.15}\indexp{Erodoto!3!00150000 @15|ca}) che compie il gesto dopo il fallimento della sua rivolta contro Cambise\index[n]{Cambise}, nel 525. Nei commenti ad Erodoto\index[n]{Erodoto} se ne parla come di cosa nota, per la rapidità della morte, e la formula \textgreek{αἷμα πιών} simile alle formule come \textgreek{κώνεινων πιεῖν} o nelle ricette mediche \textgreek{γάλα πιεῖν}. In Egitto e in Asia pare fosse un medicinale prescritto e in alcuni casi fosse legato al sucidio. Vi sono molti esempi di farmaci vegetali con nomi animali; es. Dioscoride 4.51 \textgreek{τράγος}, 129 \textgreek{βούγλωσσον, κυνόγλωσσον, αἷμα δρακόντων}. Inoltre lo stesso nome in epoche diverse designa cose del tutto diverse, cosicché l'\textgreek{ἀνδρικώτατα} usato da Aristofane\index[n]{Aristofane}  può essere la chiave di lettura per l'equivalenza di \textgreek{ἀρσενικός} (vocalismo zero di questo termine si trova nel sanscrito rsa-bha = toro) con \textgreek{αἷμα ταύρειον}. Per Montana (\cite[291]{Montana2002}) Aristoph. \emph{Eq.} 83-84\indexp{Aristofane!\textit{Equites}!83-84|ca} è ''un preciso ammiccamento a un tema popolare e insieme storiografico, ben noto  e familiare all'opinione pubblica''. Tucidide\index[n]{Tucidide} e  Diodoro\index[n]{Diodoro} in modi simili nominano soltanto questa tradizione folcloristica che nasce, secondo Arnould (\cite*[230]{Arnould1993}), un vero e proprio ''romanzo psicologico'' sul sangue di toro. Per la velocità nell'uccidere Arist. \emph{HA} 3.19 (520b) : \textgreek{τάχιστα δὲ πήγνυται τὸ τοῦ ταύτου αἷμα πάντων}; per la velenosità Plinio \emph{HN} 11.90 (221), 28.53 (195): \emph{ideo pestifer potu}. Per la causa delle velenosità: \textgreek{καθαρώτατον παχύτατον}. Arist. \emph{HA} 3.19 (521a). La tradizione diventa vulgata approdando anche in FGrHist 104 senza difficoltà. Troviamo il racconto in Cicerone che dice di prenderlo da Tucidide\index[n]{Tucidide} e fornisce una buona chiave di lettura per il suo successivo uso e per la sua sopravvivenza così lunga e solida (Brutus §42-43\indexp{Cicerone!\textit{Bruto}!42-3}), in Valerio Massimo (5.6.3\indexp{Valerio Massimo!5.6.3|ca}) e in Plutarco\index[n]{Plutarco} che ne parla per Midas in \emph{De Superstitione} 168F\indexp{Plutarco!\textit{De Superstitione}!00168 @168F|ca} e nella \emph{Vita di Flamminino} 20.9\indexp{Plutarco!\textit{Flamminino}!00020009 @20.9|ca} per le diverse storie sulla morte di Annibale dicendo che \textgreek{ἔνιοι δὲ μιμησάμενον Θεμιστοκλέα καὶ Μίδαν αἷμα ταύρειον πιεῖν· Λίβιος δέ φησι φάρμακον ἔχοντα κεράσαι, καὶ τὴν κύλικα δεξάμενον εἰπεῖν}. Questo mito è dunque chiaramente reso solido dalla sua semplicità, dalla diffusione e dal suo contesto di formazione ed uso retorico, divenendo un \emph{mythos} per la spiegazione di qualcosa di mal conosciuto (\cite[230]{Arnould1993}, \cite[163]{Marr1996}). L'inversione del dubbio che si trova in Plutarco\index[n]{Plutarco} \emph{Them.}  31.6\indexp{Plutarco!\textit{Temistocle}!00031006 @31.6|ca}  è sintomatica: \textgreek{καὶ τοὺς φίλους συναγαγὼν καὶ δεξιωσάμενος, ὡς μὲν ὁ πολὺς λόγος αἷμα ταύρειον πιών, ὡς δ’ ἔνιοι φάρμακον ἐφήμερον προσενεγκάμενος}. Questo è dunque un ulteriore esempio di ''inganno per risolvere situazioni gravi o problematiche'': \cite[260]{Montana2002}. In questo caso dunque, di nuovo, troviamo un  Diodoro\index[n]{Diodoro} che segue e aggiunge a Tucidide, ma con elementi come l'inganno che afferiscono al filone apologetico. Invero, nessuna delle tradizioni in nostro possesso si può dire anti-temistoclea (Se si eccettua il frammento 126 K.-A.= \emph{Arist. }4.3 \textgreek{σοφὸς γὰρ ἁνήρ, τῆς δὲ χειρὸς οὐ κρατῶν}, peraltro assai moderato, e scherzoso più che invettivo): li corona tutti Plutarco\index[n]{Plutarco} in \emph{Them.}   31.6,\indexp{Plutarco!\textit{Temistocle}!00031006 @31.6|ca} aggiungendo alla narrazione tradita: \textgreek{ἄριστα βουλευσάμενος ἐπιθεῖναι τῷ βίῳ τὴν τελευτὴν πρέπουσαν}. Il ruolo dei figli di Temistocle\index[n]{Temistocle} ritornati ad Atene è innegabile a riguardo e si è dimostrato indubbiamente efficace. Thuc. 1.138.6\indexp{Tucidide!1!01380006 @138.6|ca}, Paus. 1.26.4\indexp{Pausania!000100260004 @1.26.4|ca}. Idomen. Lampsac. FGrHist 338 F1\indexp{Idomen. Lampsac. FGrHist 338!F!00010000 @1|ca} Cic. \emph{Brut.} 1.15.11\indexp{Cicerone!\textit{Bruto}!1.15.11|ca}, Suda \textgreek{Θ}126\indexp{Suda!Theta126|ca}. Per il loro rientro ad Atene: Stesimbroto di Taso, FGrHistCont 1002 F3\indexp{Stesimbroto di Taso FGrHistCont 1002!F!003 @3} (Plut. \emph{Them.}  24.6-25.1\indexp{Plutarco!\textit{Temistocle}!00024006 @24.6-25.1|ca}), Paus 1.1.2. Secondo Marr (\cite[161]{Marr1995}) ritornarono due dei figli di Temistocle\index[n]{Temistocle} (Plyeuctus\index[n]{Plyeuctus} e Cleophantus\index[n]{Cleophantus}), mentre Archeptolis\index[n]{Archeptolis} rimase a guardia dei possedimenti di Magnesia. Platone conserva una serie di testimonianze negative nei confronti di questi figli di Temistocle\index[n]{Temistocle} (93 d-e, 516d, 519a-c). Per \cite[296-8]{Montana2002} Eupoli\index[n]{Eupoli} immagina il ritorno dall'Ade di Solone\index[n]{Solone}, Milziade\index[n]{Milziade}, Aristide\index[n]{Aristide} e Pericle\index[n]{Pericle}  e un trimetro di Plutarco\index[n]{Plutarco} stigmatizza l'assenza di Temistocle\index[n]{Temistocle} dal novero delle personalità storiche tornate sulla terra (Arist 4.3\indexp{Plutarco!\textit{Aristide}!00004003 @4.3|ca}): \textgreek{σοφὸς γὰρ ἁνήρ, τῆς δὲ χειρὸς οὐ κρατῶν}. Isocrate\index[n]{Isocrate} accoglierà completamente, per canonizzarla, l'immagine di Temistocle\index[n]{Temistocle} rivitalizzandone il connotato nazionalistico antipersiano. 8.75\indexp{Isocrate!08075 @8.75|ca}; Lisia 2.42\indexp{Lisia!2.42|ca}. I figli portano per prima cosa le ossa del padre al Pireo e innalzano statue. Il momento del loro ritorno è discusso, ma probabilmente avvenne al momento del declino politico di  Cimone\index[n]{Cimone} (Thuc. 1.102.1-3\indexp{Tucidide!1!01020001 @102.1-3|ca}, Plut. \emph{Cim.} 16.6-8\indexp{Plutarco!\textit{Cimone}!00016006 @16.6-8|ca} e 17.1-2\indexp{Plutarco!\textit{Cimone}!00017001 @17.1-2|ca}), che è a capo dell'esercito che muove consapevolmente (solo in FGrHist 104) contro Temistocle\index[n]{Temistocle} altrimenti non noterebbe dopo la morte di questi che gli Ateniesi non ne erano al corrente, implicando che la presenza del generale fosse invece certa.
    %%%%%%%%%        
            \subsection*{L'Eurimedonte e l'Egitto}
            Questo paragrafo presupponendo un legame sincronico tra la spedizione di Temistocle\index[n]{Temistocle} contro la Grecia, noto anche in Plutarco\index[n]{Plutarco} (\emph{Cim.} 18.7) ed una presunta mossa difensiva dei Greci con il supporto ad Inaro\index[n]{Inaro}, narra della spedizione in Egitto e della sconfitta inferta agli Ateniesi da Megabizo\index[n]{Megabizo}. Si accenna soltanto a Tanagra ed Oenofita (12) per poi dedicare un lungo appunto alle origini del soprannome di Callia\index[n]{Callia} e alla pace da questi stipulata con Artaserse\index[n]{Artaserse}  (13). FGrHist 104 prosegue con l'inizio di una ''guerra greca'' che include la guerra sacra, gli scontri di  Tolmide\index[n]{Tolmide} con Beoti (14) e Peloponnesiaci, seguiti dalle punizioni inferte da Atene all'Eubea e a Samo per mano di Pericle\index[n]{Pericle}  e Sofocle\index[n]{Sofocle} (15).
    %%%%%%%%%%        
            \subsubsection{\textgreek{Οἱ δὲ Ἕλληνες  …  ἐξεδίωκον τὸν στρατὸν τὸν ἅμα τῷ Θεμιστοκλεῖ}}
            I Greci sono all'oscuro della morte di Temistocle\index[n]{Temistocle}, ma sono consapevoli che è lui a guidare l'esercito contro di loro. Si presenta uno scenario simile a quello di Ippia\index[n]{Ippia} a Maratona destinato a ripetersi almeno fino alla Cnido di Conone\index[n]{Conone}, con generali ateniesi al servizio del Re di Persia. Ma questo possibile scontro di Temistocle\index[n]{Temistocle} e Cimone, sventato dal suicidio del vincitore di Salamina è noto solo da FGrHist 104 e dalla Vita di  Cimone\index[n]{Cimone} (18.7\indexp{Plutarco!\textit{Cimone}!00018007 @18.7|ca}), seppure nel nostro racconto sia un elemento di raccordo degli eventi e invece in Plutarco\index[n]{Plutarco} un elemento inserito a colorire il personaggio principale. È solo l'inizio dell'opera di  Cimone\index[n]{Cimone} liberatore, mai ostracizzato, invitto. Senza interruzioni la storia continua con la battaglia dell'Eurimedonte e la spedizione in Egitto. Thuc. 1.100.1\indexp{Tucidide!1!01000001 @100.1|ca}; Licurgo\index[n]{Licurgo}  72\indexp{Licurgo!72|ca}; Diod. 11.60.6-61.7\indexp{Diodoro!11!00600006 @60.6-61.7|ca} \cite[219]{CulassoGastaldi1990} ricorda la recente riconsiderazione della datazione dell'Eurimedonte, svincolata dalle Dionisie del 468 e il 467 proposto dal Gomme e da \cite[75-86]{Meiggs1972}, \cite[125-7]{Green2006}. La cronologia interna di FGrHist 104 è riassunta nell'appendice \ref{cronofgrhist104}.
  %%%%%%%%%%%%          
            \subsubsection{\textgreek{κατὰ τὸν λεγόμενον Εὐρυμέδοντα ποταμόν}}\label{bkm:RefHeading3610319231068}
            Vengono riportati i risultati della spedizione e i trofei. Sui trofei nei principali storici greci si vedano le osservazioni di \cite{Hau2013}. La battaglia scomparsa di Badian c'è eccome, anzi, è  \textgreek{λαμπρὰ ἔργα }ed Elio Aristide pare servirsi del nostro testo per comporre parte della sua orazione (46.158,1\indexp{Elio Aristide!46.158|ca}). Plutarco\index[n]{Plutarco} dichiara tre fonti per la battaglia, Eforo\index[n]{Eforo},  Callistene\index[n]{Callistene}  e Fanodemo. Pare si possa trovare conferma in P.Oxy. 1610\index[pap]{P.Oxy.!1610} (Eforo FGrHist 70 F191\indexp{Eforo FGrHist 70!F!191|ca} frg. 9; 10; 53): \textgreek{Κίμων πυν/θανόμενος τὸ]\d{ν} τ[ῶν / Περσῶν στόλο]ν περὶ / [τὴν Κύπρον συ]ντετά/[χθαι … }(commentato da \cite[406]{Parmeggiani2011}) che diverge dal racconto di  Diodoro\index[n]{Diodoro} ma affronta il periodo comunque nell'ottica della politica ateniese nei confronti della Persia (\cite[399-415]{Parmeggiani2011}). Il colpo doppio, per mare e per terra, della vittoria di  Cimone\index[n]{Cimone} è l'elemento glorioso e distintivo della battaglia che viene conservato per distinguerla e ricordarla insieme ai numeri relativi ai contingenti. 
     %%%%%%%%%%       
            \subsubsection{\textgreek{ἐβασίλευσε δὲ τῆς Αἰγύπτου Ἴναρος υἱὸς Ψαμμητίχου}} \label{bkm:RefHeading3376319231068}
            La spedizione Ateniese in Egitto a supporto di Inaro\index[n]{Inaro}, capo della tribù libica di Bakalu (\cite[135–42]{Winnicki2006}.) è stata oggetto di diversi studi anche recenti che ne discutono la cronologia e i momenti salienti: \cite{Kahn2008}, \cite{Green2006}, \cite[129-131 (testo) e 266-7 (note)]{Lenfant2004}, \cite[591-4]{Briant1996}. FGrHist 104 segue la versione tucididea, con la rovinosa sconfitta ateniese. Gli Ateniesi arrivano in Egitto passando per Cipro: anche questa come altre è un'informazione dedotta. Importanti nella discussione sono anche ML\index[pap]{M-L!33} 33 e 34\index[pap]{M-L!34} e lo scolio al \emph{Pluto} di Aristofane\index[n]{Aristofane}. Si osserva in questo episodio una selezione che predilige la versione tucididea e la ripulisce mantenendo solo i dati principali (scompare per esempio Amirteo\index[n]{Amirteo}). La versione di Ctesia e quella di  Diodoro\index[n]{Diodoro} divergono molto ed escludono la possibilità di una comunanza di fonti con il nostro testo. È da ricordare che questo evento, nella stessa misura delle sorti di Pausania\index[n]{Pausania il periegeta}  è noto anche ad Erodoto\index[n]{Erodoto} (\href{http://data.perseus.org/citations/urn:cts:greekLit:tlg0016.tlg001.perseus-grc1:3.12}{3.12}.4\indexp{Erodoto!3!00120003 @12.3|ca}, \href{http://data.perseus.org/citations/urn:cts:greekLit:tlg0016.tlg001.perseus-grc1:3.15}{3.15}.3\indexp{Erodoto!3!00150003 @15.3|ca}, \href{http://data.perseus.org/citations/urn:cts:greekLit:tlg0016.tlg001.perseus-grc1:7.7}{7.7}\indexp{Erodoto!7!00070000 @7|ca}) che mostra peraltro di conoscere più diffusamente di Tucidide\index[n]{Tucidide} gli eventi che ritroviamo per esteso in Diodoro. Anche questi eventi facevano parte del repertorio retorico di IV a.C., come dimostra anche Isocrate\index[n]{Isocrate} (8.86\indexp{Isocrate!08086 @8.86|ca}).
      %%%%%%%%%%%      
            \subsection*{Tanagra ed Oenofita: \textgreek{Ἑλληνικὸς πόλεμος ἐγένετο}}
            In questi capitoli si osservano al massimo la sintesi e i principi di selezione. \cite[553s]{Sordi2002}. Le informazioni fornite sono: popolazioni coinvolte,  numeri,  esito. Una lista di eventi senza racconto, completa di dati ma priva di storia.  Su queste due battaglie i passi di riferimento sono  Tucidide\index[n]{Tucidide} 1.107 – 108.3\indexp{Tucidide!1!01070000 @107-108.3|ca} e  Diod. 11.79\indexp{Diodoro!11!00790000 @79|ca}, 83.1\indexp{Diodoro!11!00830001 @83.1|ca} ma FGrHist 104 offre ben poco da commentare in questo caso. A riportare il nostro testo nella tipologia che fa da riferimento per i componimenti epidittici abbiamo il Menesseno  platonico, discorso scritto da Aspasia\index[n]{Aspasia} maestra comune di Socrate\index[n]{Socrate} e Pericle\index[n]{Pericle}, ancora viva quando Platone immagina il discorso recitato da Socrate. \cite[140s]{Nouhaud1982}. per i legami tra questo testo e Polyb. IV 20, 4-7\indexp{Polibio!04020 @4.20.4-7|ca}; su Eforo\index[n]{Eforo}, \cite[87]{Parmeggiani2011}. Il dialogo riflette un pensiero posteriore alla Pace di Antalcida, ma soprattutto ci dice qualcosa di come il discorso storico e le strutture di sintesi e ricollegamento tra gli eventi fossero riproposte più o meno scolasticamente. Prima di arrivare a 241d\indexp{Platone!\textit{Menesseno}!00242 @242e|ca} dove inizia la carrellata di eventi di V secolo che arriva a nominare Tanagra ed Enofita, Socrate dichiara: \textgreek{ἑμάνθανόν γε τοι παρ’αὐτῆς , καὶ ὁλίγου πληγὰς ἔλαβον ὅτ’ἐπελανθανόμην}. [L'ho imparato proprio da lei e per poco non venivo preso a botte quando mi dimenticavo] (242e) Non importa neppure a FGrHist 104 quanto tempo sia intercorso tra le battaglie (Thuc. 1.108.2-3\indexp{Tucidide!1!01080002 @108.2-3|ca}, Diod. 11.81.1\indexp{Diodoro!11!00810001 @81.1|ca}, Pl. Men. 242b\indexp{Platone!\textit{Menesseno}!00242 @242b|ca}): accosta Tolmide\index[n]{Tolmide} a Mironide\index[n]{Mironide}, per nessun motivo specifico. \cite[50-51]{Ferretto1984}.
   %%%%%%%%%%         
            \subsection*{Cipro e la Pace di Callia}\label{bkm:RefHeading717011501267828}
             I punti principali dei problemi attorno alla Pace di Callia\index[n]{Callia} (data, proponente, termini, critica antica) si possono trovare in \cite[487-495]{Meiggs1972} e \cite[1-72]{Badian1993}. Un sunto delle varie posizioni si trova in \cite{Lachhein2008}. Questo è il capitolo più fortunato di FGrHist 104: ha sollevato una serie di problemi e discussioni in cui ha avuto un posto secondario ma importante, soprattutto grazie al ruolo accordatogli dal Badian in un celebre articolo del 1987. Badian in quella prima versione, leggendo solo i §13 e 14, argomentava in favore della derivazione da Eforo\index[n]{Eforo} del testo di FGrHist 104, data la comunanza di fonti con Suda. FGrHist 104 ha un posto di rilievo nella discussione sulla questione della Pace di Callia\index[n]{Callia} 
\begin{itemize}
\item per la posizione che dà nel racconto a questo evento;
\item per la definizione dei nomi e dei gruppi coinvolti;
\item per i termini del trattato. 
\end{itemize}
Inoltre, in questo paragrafo, che abbiamo visto tramandato anche da Suda e da Planude (Cfr. \ref{104Ermogene}: Suda, K1620\indexp{Suda!K1620|ca} e K214\indexp{Suda!K214|ca} nonché lo scolio ad Ermogene in Walz V 388\indexp{Scolia ad Hermog.!Waltz V 388|ca}), ci mostra come il sovrapporsi di semplificazioni e tentativi di sintesi sia guidato da una logica interna, che, se non altro è un tentativo di riordinare e sistemare. Se non possiamo prendere a piene mani la versione di questo testo come riferimento per ricostruire i fatti, dobbiamo ammettere che come manuale ha fatto delle scelte coerenti e consistenti. Quella che vi possiamo vedere è una storia del quinto secolo che pare elaborare le notizie che abbiamo, selezionarle e tenere un racconto lineare di base. Scompare Taso, per esempio. Nemmeno un accenno a Pericle\index[n]{Pericle}  in questa fase, né come difensore della politica ateniese di spostamento del tesoro, né come comandante della flottiglia oltre le Chelidonie. È una selezione senza incoerenze, che, paradossalmente, se questa fosse stata l'unica fonte a nostra disposizione avrebbe evitato secoli di discussioni. Alcune interessanti osservazioni relative alle mura delle città ioniche, che vanno a supportare la consapevolezza ateniese della fine della guerra dopo il 449, si trovano in \cite[121s]{Cawkwell1997}. Lo stesso autore sostiene l'impossibilità di una pace prima del 462/1 per incompatibilità con la lega ellenica di cui Atene faceva ancora parte prima dei fatti di Itome (Thuc. 1.102.4\indexp{Tucidide!1!01020004 @102.4|ca}).  La cronologia interna di FGrHist 104 prevede la morte di Serse\index[n]{Serse}, seguita dalla morte di Temistocle\index[n]{Temistocle} all'ipotesi dello scontro con Cimone\index[n]{Cimone}, che combatte comunque all'Eurimedonte contro l'esercito che sarebbe dovuto essere capitanato dal vincitore di Salamina; dopo l'Eurimedonte c'è una tappa a Cipro prima della spedizione in Egitto. Terminata l'impresa troviamo Cipro, la morte di  Cimone\index[n]{Cimone} e la pace stipulata dal cognato Callia. FGrHist 104 vota 449 a.C. senza dubbio, come  Diodoro\index[n]{Diodoro} e come Platone (Menex. 241e\indexp{Platone!\textit{Menesseno}!00241 @241e|ca}) che ricorda l'Eurimedonte, l'Egitto e Cipro \textgreek{βασιλέα ἐποίησαν δείσαντα} e Licurgo\index[n]{Licurgo}  (72-3\indexp{Licurgo!73|ca}). \cite[1]{Badian1987}.  Ormai tutte le prove convergono per una pace scritta in questo momento, come aveva sostenuto già il Wade-Gery argomentando intorno all'assenza di tributo per il 448. Giustamente il Cawkwell (\cite*[115]{Cawkwell1997}) sottolineava ripetutamente l'assenza di ostilità greco-persiane nella tradizione a noi pervenuta per il periodo successivo al 449 fino al 412 (tra le prove anche ML 44 \index[pap]{M-L!44|ca}
            % e ? = ? --&gt; IG I3 35 
\label{ref:dateeurimedonteepacedicallia} sul nuovo sacerdozio di Atena Nike, prima ancora dell'erezione del tempietto). Così anche \cite[143]{Accame1982}, riprendendo la controversia per la ricostruzione dei Templi in Plut. \emph{Per.} 17\indexp{Plutarco!\textit{Pericle}!000170000 @17|ca}; \cite[41s]{Bloedow1992} rivede la discussione di Badian su Suda K214\indexp{Suda!K214|ca} e K1620\indexp{Suda!K1620|ca} affermando che \textgreek{ἐπὶ Kίμωνος}  implica solo che  Cimone\index[n]{Cimone} era vivo quando si venne a definire un confine, e che ciò è vero sia per il 466 che per il 450. Badian a sua volta, nonostante la sua convinzione rispetto alle molteplici ''paci'', girava intorno alla pace del 449 per dire che già Eforo\index[n]{Eforo} aveva parlato della pace dopo l'Eurimedonte, basandosi sul riferimento poi approfondito dal Bloedow, che lega i trattati di Callia\index[n]{Callia} con il confine di  Cimone\index[n]{Cimone} (\textgreek{ἔταξε καὶ τοὺς ὅρους τοῖς βαρβάροις}).  Diodoro\index[n]{Diodoro} mette più in imbarazzo però rispetto alla datazione dell'Eurimedonte che risulterebbe essere qualche tempo dopo il 464, per ammettere la morte di Temistocle\index[n]{Temistocle} tra quella di Serse\index[n]{Serse} e la battaglia. Potrebbe essere un tentativo di strutturare un prima e un dopo. Con l'Eurimedonte si chiuderebbero le vicende di Temistocle\index[n]{Temistocle} e Pausania\index[n]{Pausania il periegeta}  per dare inizio a quelle di Cimone. La cronologia interna di FGrHist 104 non è tuttavia impossibile dati i nuovi risultati della ricerca sulla cronologia della pentecontaetia. Si può per esempio pensare che la rivolta di Taso sia finita \textgreek{μετὰ} la battaglia dell'Eurimedonte. Se infatti ammettiamo i tre anni di Tucidide\index[n]{Tucidide} a partire dal 467 per Taso, i due di Nasso a partire dal 466 tanto caro al Badian (già dal \cite*[7s]{Badian1987}), e la partenza per l'Egitto nel 463/2, nel 464/3 può essere avvenuto il colpo dell'Eurimedonte raccontato da  Diodoro\index[n]{Diodoro} ed FGrHist 104 seguito immediatamente appunto dalla partenza in soccorso ad Inaro\index[n]{Inaro}. Nel frattempo si avranno gli eventi che portano all'ostracismo di  Cimone\index[n]{Cimone} e le due ''regate'' ricordate da  Callistene\index[n]{Callistene}. La continuazione della spedizione in Egitto blocca la stipula del trattato ed anzi rinforza le speranze persiane, tanto che, anche dopo la vittoria a Cipro del 449, il confine verrà fissato in un punto che esclude l'isola. 
    %%%%%%%        
            \subsubsection{\textgreek{Εὐθὺς ἐστράτευσαν ἐπὶ Κύπρον … τελευτᾶι}}
            La rapida consequenzialità degli eventi è da attribuire alla forma del testo e sicuramente alla brevità di molti dei resoconti su questo periodo di storia. La morte di  Cimone\index[n]{Cimone} è unanimemente collocata durante questa seconda spedizione a Cipro, sebbene in momenti diversi. \cite[135 n.22]{Samons1998} enumera Thuc. 1.112.4\indexp{Tucidide!1!01120004 @112.4|ca}, Diod. 12.3-4\indexp{Diodoro!12!00030000 @3-4|ca}, Plut. \emph{Cim.} 19\indexp{Plutarco!\textit{Cimone}!000190000 @19|ca} (=Fanodemo FGrHist 325 F23\indexp{Fanodemo FGrHist 325!F!00230000 @23|ca}). La morte di  Cimone\index[n]{Cimone} non è causa della pace e la nuova duplice vittoria di Salamina Cipria (Diod. 12.3-4) ne è, al massimo, pretesto per la concretizzazione, ma probabilmente questa circostanza ha dato origine ad una ''confusione in buona fede'', per cui la stessa stipula del trattato, almeno limitatamente agli accordi sui confini, veniva attribuita al precedente, grandioso, risultato dell'Eurimedonte, al quale  Cimone\index[n]{Cimone} era sopravvissuto.  \cite[115-17]{Cawkwell1997}. Accame (1982, 126 e 141) riporta già a Stesimbroto gli effetti della pace, mettendo in evidenza Plut. \emph{Per.} 26.1\indexp{Plutarco!\textit{Pericle}!00026001 @26.1|ca} (FGrHistCont 1002 F8\index[p]{Stesimbroto di Taso FGrHistCont 1002!F!008 @8|ca}) dove Pericle\index[n]{Pericle}  naviga verso il mare aperto e si ferma prima della Caria per rispettare i termini della Pace di Callia\index[n]{Callia} che sarebbe stata operativa. Anche le richieste di Alcibiade per Tissaferne in Thuc. 8.56\indexp{Tucidide!8!00560000 @56|ca} sono un chiaro riferimento ad una pace in essere nella seconda metà del V secolo. Per \cite[404]{Parmeggiani2011} anche nel discorso di Pericle\index[n]{Pericle}  in Diod. 12.40.3\indexp{Diodoro!12!00400003 @40.3|ca} si fa riferimento a Callia\index[n]{Callia} perché il riferimento non può essere alla pace dei trent'anni. \cite[148-9]{Accame1982} pensa che questa circostanza e la perdita di interesse per la vittoria a Cipro (persa con il trattato) abbiano indotto allo spostamento sull'Eurimedonte anche dell'epigramma che infatti troviamo in  Diodoro\index[n]{Diodoro} 11.62\indexp{Diodoro!11!00620000 @62|ca} e che sarebbe più appropriato a 12.4\indexp{Diodoro!12!00040000 @4|ca}. Contro la confusione tra le due spedizioni \cite[408]{Parmeggiani2011}. Ulteriore discussione sull'argomento nelle note seguenti, soprattutto p.\pageref{pacedicallia}s.
     %%%%%%%%       
            \subsubsection{\textgreek{οἱ δὲ Πέρσαι ὁρῶντες κεκακωμένους τοὺς Ἀθηναίους}}
            FGrHist 104 concorda con Tucidide\index[n]{Tucidide} (1.112.2-4\indexp{Tucidide!1!01120002 @112.2-4|ca}) e  Diodoro\index[n]{Diodoro} (12.3-4\indexp{Diodoro!12!00030000 @3-4|ca}) ma toglie l'ultima vittoria a  Cimone\index[n]{Cimone} come Plutarco\index[n]{Plutarco} (\emph{Cim.} 19\indexp{Plutarco!\textit{Cimone}!000190000 @19|ca}) facendolo morire prima dello scontro navale a causa di una malattia poco gloriosa. Tuttavia, contrariamente al racconto di Plutarco, i Persiani lo sanno, ed è proprio per questo che attaccano. Perdono comunque e devono venire a patti concreti nonostante la vittoria in Egitto, ma la perdita di  Cimone\index[n]{Cimone} fa scegliere a Callia\index[n]{Callia} di lasciare Cipro, poco difendibile, ai Persiani. \cite[182 n. 17]{Green2006}.
    %%%%%%%%        
            \subsubsection{\textgreek{στρατηγὸν αἱροῦνται Καλλίαν τὸν ἐπίκλην Λακκόπλουτον}}
            %% FORSE QUESTO PASSO è DA SPOSTARE più decisamente nel discorso sulla tradizione di Erodoto?? 
            Callia\index[n]{Callia}, cognato di  Cimone\index[n]{Cimone} (Plut. \emph{Cim.} 4.8\indexp{Plutarco!\textit{Cimone}!00004008 @4.8|ca}), è coinvolto anche nelle trattative della pace trentennale nel 446 come \emph{proxenos} di Sparta (Diod. 12.7\indexp{Diodoro!12!00070000 @7|ca}) ed è ricordato anche da Erodoto\index[n]{Erodoto} (\href{http://data.perseus.org/citations/urn:cts:greekLit:tlg0016.tlg001.perseus-grc1:7.151}{7.151}\indexp{Erodoto!7!01510000 @151|ca}) per inciso, assieme ad una legazione argiva con tutta probabilità entrambe arrivate a Susa dopo l'Eurimedonte e prima della sconfitta in Egitto. La sua nomina ha fatto pensare ed è considerata straordinaria: qui infatti è in veste di stratego e non uno dei \textgreek{πρέσβεις αὐτοκράτορες} come in Diod. 12.4\indexp{Diodoro!12!00040000 @4|ca}. La tradizione rispetto al suo arricchimento (Plut. \emph{Arist. }25\indexp{Plutarco!\textit{Aristide}!000250000 @25|ca}; Lys. 19\indexp{Plutarco!\textit{Lisandro}!000190000 @19|ca}) con il tesoro trovato a Maratona (Plut. \emph{Arist. }5\indexp{Plutarco!\textit{Aristide}!000050000 @5|ca} = fr. 696 KA, adespoto) dove aveva combattuto è riportata anche da Suda, negli stessi termini  (\cite[132]{Samons1998}). Badian, criticato nell'articolo della Samons, aveva messo in dubbio anche la sua selezione come ambasciatore della pace per confermarla dopo aver argomentato il primo trattato nel 466. \cite[139]{Samons1998}) giustamente nota che, se ci fosse stato un primo trattato, non si sarebbe chiamato a stipularne un secondo proprio colui che aveva siglato quello disatteso pochi anni prima. Cfr. \emph{Them. Ep.}  9\indexp{Lettere di Temistocle!00090000 @9|ca}. e \cite[164-6]{CulassoGastaldi1990}. 
      %%%%%%%%%%      
            \subsubsection{\textgreek{ὁ Καλλίας ἐσπείσατο πρὸς Ἀρταξέρξην}}
            \label{pacedicallia}Nulla sappiamo dei  \textgreek{λοιποὺς Πέρσας}, dobbiamo però immaginare i satrapi e i dignitari di Artaserse\index[n]{Artaserse}. È una specificazione di certo non neutra, ma non meglio precisabile per ora. I patti stipulati da Callia\index[n]{Callia} con Artaserse\index[n]{Artaserse}  nel 449 hanno posto una serie di problemi legati anche alla costruzione di un luogo topico della retorica ateniese di IV secolo da Isocrate\index[n]{Isocrate} in poi: il confronto della Pace di Antalcida con quella di Callia. \cite[4-5]{Bosworth1990} sottolinea l'indipendenza dalla tradizione storica dei luoghi sviluppati dalla retorica. Il dubbio sulla veridicità o meno della pace riposa ora solo sull'omissione di Tucidide, alla quale sono state offerte numerosissime motivazioni, nessuna delle quali individua un vero dato nel silenzio dell'ateniese. Non solo grazie alla discussione sorta dalla proposta di una triplice pace da parte del Badian, il dibattito sulla Pace di Callia (che considero a partire da \cite{Eddy1970})\index[n]{Callia} ha preso una svolta negli ultimi anni, risvegliando critici, detrattori, sostenitori e revisori. Dobbiamo un passo importante soprattutto ad un lucido contributo del Krentz che elimina dalla controversia la pretesa critica di Teopompo\index[n]{Teopompo}  FGrHist 115 F153\indexp{Teopompo FGrHist 115!F!01530000 @153|ca} (e.g. \cite[48 e 80]{Ferretto1984}. Krentz (\cite*[231s]{Krentz2009}) slegando F153 da F154\indexp{Teopompo FGrHist 115!F!01540000 @154|ca} ipotizza in modo molto convincente che il riferimento ai trattati \textgreek{αἱ πρὸς βασιλέα Δαρεῖον Ἀθηναίων πρὸς Ἕλληνας συνθῆκαι} (Teone Progym. 2 II 67, 22 Sp = FGrHist 115 F 153) sia l'accordo degli Ateniesi del 507/6 (Hdt. \href{http://data.perseus.org/citations/urn:cts:greekLit:tlg0016.tlg001.perseus-grc1:5.73}{5.73}\indexp{Erodoto!5!00730000 @73|ca}). Teopompo\index[n]{Teopompo}  avrebbe criticato la falsificazione dell'autorizzazione che fu data, portando Atene dalla parte dei Persiani. Con questo cadono tutte le argomentazioni basate su questo frainteso di interpretazione del frammento teopompeo (e.g. \cite[395]{Mazzarino1966} e \cite[80]{Shrimpton1991}) e anche il ragionamento critico sul trattamento della Pace di Callia\index[n]{Callia} da parte di Eforo\index[n]{Eforo} in \cite[404-10]{Parmeggiani2011} pur non implicando una revisione del discorso sull'attitudine nei confronti della documentazione epigrafica (\cite[170-2]{Parmeggiani2011} su Eforo\index[n]{Eforo} FGrHist 70 F 106\indexp{Eforo FGrHist 70!F!01060000 @106|ca}). \label{Cratero}Anche la testimonianza plutarchea su Cratero (FGrHist 342 F13\indexp{Cratero!FGrHist 342!F!13|ca}, \cite[29]{Erdas2002}) resta dunque a sé stante. Circolava una versione scritta, probabilmente nel IV secolo, ma non era sottoposta a critica se non in modo generico dai commenti di Teopompo\index[n]{Teopompo}  ed Eforo\index[n]{Eforo} sulle iscrizioni e sull'atteggiamento della ''retorica da panegirico'' che di certo è responsabile per la lunga tradizione del confronto tra questa pace e quella di Antalcida (Isocr. 4.118\indexp{Isocrate!04118 @4.118|ca}; 7.80\indexp{Isocrate!07080 @7.80|ca}; 12.59\indexp{Isocrate!12059 @12.59|ca}).  I suoi risultati rispondono all'invito alla riduzione delle complessità argomentative del Bosworth che ha ridotto l'impatto del passo plutarcheo che 
% Callistene\index[n]{Callistene}  nel rigo seguente era Callistrato... sarebbe da controllare.            
discute  Callistene\index[n]{Callistene}  e Cratero. Bosworth (\cite[1]{Bosworth1990}) prova ad escludere il passo di  Callistene\index[n]{Callistene}  preso in considerazione da Plutarco\index[n]{Plutarco} (\emph{Cim.} 13\indexp{Plutarco!\textit{Cimone}!000130000 @13|ca} = FGrHist 124 F16\indexp{Callistene di Olinto FGrHist 124!F!00160000 @16|ca}). Se è del tutto condivisibile la collocazione del passo nell'opera di  Callistene\index[n]{Callistene}  e il conseguente disinteresse dell'autore per la pace stessa, nonché la lettura di \textgreek{οὑ φησι} in opposizione ad \textgreek{ἔργῳ δὲ} per indicare la pace ''di fatto'', non c'è bisogno di eliminare dalla ricostruzione queste informazioni, poiché non è contraddittorio con esse, ma solo con il passo di Plutarco\index[n]{Plutarco} in cui si trova ed al quale sarà da ricondurre l' ''errore'' (è critico rispetto a questa lettura \cite[410 n.66]{Parmeggiani2011}, ma le prove interne a Plutarco\index[n]{Plutarco} portate dal Bosworth sono decisive per la sua lettura del passo che non può essere pensato come \emph{ipsissima verba} di  Callistene\index[n]{Callistene} ). Nella fissazione del confine effettivo, protetto con spedizioni militari aggressive fino alla morte di  Cimone\index[n]{Cimone} e non difeso dal re avvilito (\textgreek{τοῦτο τὸ ἔργον} [scil. La battaglia dell'Eurimedonte] \textgreek{οὕτως ἐταπείνωσε τὴν γνώμην τοῦ βασιλέως}). Il nesso \textgreek{οὕτως … ὥστε...} stabilisce di certo una consequenzialità diretta tra questa condizione del re e la stipula della pace, ma è lo stesso timore che anche in Platone viene attribuito al re dopo gli scontri di Cipro del 449. Hdt. \href{http://data.perseus.org/citations/urn:cts:greekLit:tlg0016.tlg001.perseus-grc1:7.151}{7.151}\indexp{Erodoto!7!01510000 @151|ca} (\textgreek{ἑτέρου πρήγματος εἵνεκα ἀγγέλους Ἀθηναίων, Καλλίην τε τὸν Ἱππονίκου καὶ τοὺς μετὰ τούτου ἀναβάντας}), giustamente da collocare dopo la morte di Serse\index[n]{Serse}, è da pensare in un momento successivo anche alla battaglia dell'Eurimedonte, che, nella cronologia del nostro testo, è poco tempo dopo. Ma la presenza a corte non data il trattato, attesta il dialogo che ad esso avrebbe poi portato. Prendendo in considerazione assieme alle spedizioni oltre le Chelidonie e Cianee di Pericle\index[n]{Pericle}  ed Efialte citate da  Callistene\index[n]{Callistene}  (avvenute gioco forza prima della morte di quest'ultimo nel 462/1), anche la spedizione in Egitto secondo le nuove date proposte da Kahn (\cite*[440]{Kahn2008}: la chiamata degli Ateniesi è posta nel 463/2), una pace a cui attenersi non c'era al momento in cui Plutarco\index[n]{Plutarco} la ricorda, ma c'era appunto un confine che sarebbe poi stato confermato, perdendo Cipro, dalla Pace di Callia. La punizione che Demostene (19.273\indexp{Demostene!00190273 @19.273|ca}) riferisce per il direttore delle trattative è probabilmente dovuta all'aver rinunciato all'isola che era stata conquistata. L'altare e la statua di Callia\index[n]{Callia} (Paus 1.8.2\indexp{Pausania!000100080002 @1.8.2|ca}) sono chiaramente frutto del clima culturale successivo alla Pace di Antalcida, anche perché, come ricorda il Musti (\cite[286]{Musti1982}), era in vigore un divieto attestato da uno scolio a D. 21.534\indexp{Demostene!00210534 @21.534|ca}. \cite[175]{CulassoGastaldi1990}. D'altro canto è stata riaperta anche la questione sul decreto di Epilico e l'autenticità del \emph{De pace} andocideo. \cite{Blamire1975}. Harris (\cite[123s]{Harris1999}) dimostra in modo molto convincente che l'ambasceria proveniva dalla Persia e non vi era stata mandata. Se Andocide 3.29\indexp{Andocide!\textit{De Pace} 29|ca} (o comunque il testo di questa orazione se essa non è di Andocide) si riferisse ad una pace con il re del 424-3 si tratterebbe certo di un rinnovo della pace del 449. La cronologia interna di FGrHist 104 è riassunta nell'appendice \ref{cronofgrhist104}. 
          %%%%%%%%%%%%  
            \subsubsection{\textgreek{ἐγένοντο δὲ αἱ σπονδαὶ ἐπὶ τοῖσδε}} 
            È da notare l'introduzione e conclusione epigrafica dei termini del trattato (\textgreek{ἐγένοντο δὲ αἱ σπονδαὶ ἐπὶ τοῖσδε·  … καὶ σπονδαὶ οὖν ἐγένοντο τοιαῦται}) notata anche per le mura di Atene (5). La formula è solo leggermente diversa da quella in Suda K1620\indexp{Suda!K1620} dove si parla degli \textgreek{ὅροι} che si erano venuti a consolidare dopo la battaglia dell'Eurimedonte. Questi limiti non sono molto diversi da quelli che riporta  Diodoro\index[n]{Diodoro} (12.4\indexp{Diodoro!12!00040000 @4|ca}) che parla di Callia\index[n]{Callia} come funzionario di Atene; elementi stabili di ogni versione sono comunque Faselide e l'esclusione di Cipro (Cfr. p.\pageref{pacedicallia}). \cite[477-8]{Meiggs1972} che organizza le fonti a seconda dei confini di terra e per mare. Nessun indizio permette di identificare il \textgreek{Νέσσου ποταμοῦ} indicato da FGrHist 104.
          %%%%%%%  
            \subsection*{L'\textgreek{Ἑλληνικὸς πόλεμος}: guerra sacra e Coronea}\label{bkm:RefHeading3609919231068}
            Si torna, a ritmo di \textgreek{μετὰ δὲ ταῦτα} e \textgreek{καὶ μετὰ ταῦτα εὐθὺς} all'elenco del §12. Sulla guerra sacra tuttavia abbiamo un'altra interessante testimonianza di Filocoro (\emph{Scholia in Aves} 556b\indexp{Scholia ad Aristoph.!\textit{Aves} 556b|ca} = FGrHist 328 F 34a-c\indexp{Filocoro FGrHist 328!F!00340000 @34a-c|ca}, commento al frammento in \cite[247s]{Costa2007})  con discussione delle diverse versioni di Tucidide\index[n]{Tucidide} (Thuc. 1.112.5\indexp{Tucidide!1!01120005 @112.5|ca}), Eratostene e Teopompo\index[n]{Teopompo}  nel XXV dei \emph{Philippika}.  \cite{Hammond1937}; \cite{Pownall1998}.
            Per Filocoro ci sono due guerre a distanza di 3 anni, non un unico episodio che alterna gli interventi di Sparta e Atene come sembra da Tucidide\index[n]{Tucidide} e Plutarco\index[n]{Plutarco} (\textgreek{εὐθύς}). La tradizione è indecisa persino sui ''nemici'' dei Focesi che sono alternativamente Beoti, Delfi, Locresi, etc. La datazione di questi episodi è solitamente basata su Tucidide. Essendo dopo Cipro e la seconda spedizione in Egitto, che devono essere circa nel 449 per via dell'anno arcontale di Diodoro, allora la guerra sacra dev'essere datata al 448. La cronologia interna di FGrHist 104 è riassunta nell'appendice \ref{cronofgrhist104}.  
            L'agguato di Coronea (Thuc. 1.113\indexp{Tucidide!1!01130000 @113|ca}, Diod. 12.6\indexp{Diodoro!12!00060000 @6|ca}), causa la perdita della Beozia e probabilmente anche del santuario di Delfi se i Focesi in Thuc. 2.9\indexp{Tucidide!2!00090000 @9|ca} fanno parte dell'alleanza peloponnesiaca, rispetto alla situazione precedente in cui il santuario era indirettamente controllato dagli Ateniesi tramite la concessione (Thuc.: \textgreek{Ἀ. παρέδοσαν Φ.}; Fil. 34a \textgreek{Φωκεῦσι πάλιν ἀπέδωκαν}; Plut. \textgreek{πάλιν εἰσήγαγε τοὺς Φωκέας}) ai Focesi. In Thuc. 1.113.4 (\textgreek{οἱ ἄλλοι πάντες αὐτονόμοι πάλιν ἐγένοντο}) forse si intendono anche i Delfi.
            Doegnes, nella sua discussione del testo propone che la divergenza (singolo evento in FGrHist 104 e due distinte guerre sacre a distanza di due anni in Filocoro) sia comunque riconducibile ad una fonte comune. Il punto di contatto testuale non pare però del tutto probante. Filocoro, come per Fidia avrebbe modificato nel suo racconto quella medesima fonte che troviamo invece incontaminata in FGrHist 104. Si veda anche il commento ai tre passi che costituiscono il frammento filocoreo in \cite[247-254]{Costa2007}. Per l'analisi storica degli eventi qui coinvolti, è importante lo studio di \cite[225s]{Sordi2002}. 
            %%%%%%
            \subsection*{L'\textgreek{Ἑλληνικὸς πόλεμος}: Eubea e Samo}
            In Tucidide\index[n]{Tucidide} Tolmide cade a Cheronea, ed è qui inserito nel punto sbagliato probabilmente, ma senza che ciò crei alcun danno alla coerenza interna del racconto. Thuc. 1.108.5\indexp{Tucidide!1!01080005 @108.5|ca}; Diod. 11.84\indexp{Diodoro!11!00840000 @84|ca} riporta la morte di Tolmide ad Oinofita, e così Pausania\index[n]{Pausania il periegeta} 1.27.5\indexp{Pausania!000100270005 @1.27.5|ca}. Della spedizione di Tolmide attraverso il Peloponneso parla anche Eschine (2.75\indexp{Eschine!0075 @2.75|ca}): di nuovo è la tradizione retorica che offre i paralleli più vicini per modalità e contenuti al nostro testo. È possibile che in un momento non meglio precisabile della tradizione si sia avuta una confusione con la successiva spedizione di Pericle\index[n]{Pericle}  (Thuc. 1.111\indexp{Tucidide!1!01110000 @111|ca}; Diod. 11.85-88\indexp{Diodoro!11!00850000 @85-88|ca}; Plut. Per. 19\indexp{Plutarco!\textit{Pericle}!000190000 @19|ca}).
            L'uso di \textgreek{πάλιν} in questo testo è stato visto anche nel commento al paragrafo 5 a p.\pageref{bkm:RefHeading697021501267828} (\textgreek{πάλιν αὐξηθῆναι}). Dopo la pace dei trent'anni (Thuc. 1.115.1\indexp{Tucidide!1!01150001 @115.1|ca}; Diod. 12.7\indexp{Diodoro!12!00070000 @7|ca}) viene brevemente nominata la rivolta di Samo (Thuc. 1.115.2-117\indexp{Tucidide!1!01150002 @115.2-117|ca}, Diod. 12.27-28\indexp{Diodoro!12!00270000 @27-28|ca}), con alcuni riferimenti cronologici confusi, da ricollegare al testo tucidideo con tutta probabilità (Cfr. Thuc 1.87.6\indexp{Tucidide!1!00870006 @87.6|ca} e 2.2\indexp{Tucidide!2!00020000 @2|ca}). Non sappiamo se la risposta data da Pericle\index[n]{Pericle}  a Sofocle in Plut. \emph{Per.} 8 sia da attribuire come il seguito del testo a Stesimbroto (FGrHistCont 1002 F9\indexp{Stesimbroto di Taso FGrHistCont 1002!F!009 @9|ca}) e a questa spedizione comune ai due, ma è probabile.
            Viene infine preannunciato il \textgreek{Πελοποννησιακὸς πόλεμος}. Su questa denominazione si è brevemente soffermato de Ste. Croix, che la circoscrive al I a.C. (e.g. Diod. 12.37.2\indexp{Diodoro!12!00370002 @37.2|ca}, Strabone 13.1.39\indexp{Strabone!001300010039 @13.1.39|ca}, Cic. \emph{De Rep.} 3.44\indexp{Cicerone!\textit{Repubblica}!1.39|ca}). \cite[294-5]{DeSte.Croix1972}. Harpocraz. s.v. \textgreek{Ἀρχιδάμειος πόλεμος} riporta tra coloro che preferiscono questa denominazione Tucidide, ma anche Eforo\index[n]{Eforo} e Anassimene. \cite[79]{Schepens2007}. Il fatto che si trovi in  Diodoro\index[n]{Diodoro} proprio nel contesto in cui la troviamo in FGrHist 104 non può essere casuale. Già da questo paragrafo tuttavia si può notare una maggiore presenza di Eforo\index[n]{Eforo}. Lo storico di Cuma è ricordato da Plutarco\index[n]{Plutarco} come fonte proprio per i macchinari (\textgreek{μηχαναῖς θαυμασταῖς}) impiegati da Pericle\index[n]{Pericle}  durante la rivolta di Samo in \emph{Per.} 27\indexp{Plutarco!\textit{Pericle}!000270000 @27|ca}. \cite[420, n. 116 e 426]{Parmeggiani2011}: queste macchine sarebbero una critica alla pigrizia della società ateniese. La citazione immediatamente successiva, Plut. \emph{Per.} 28\indexp{Plutarco!\textit{Pericle}!000280000 @28|ca} (= Eforo\index[n]{Eforo} FGrHist 70 F 195\index[p]{Eforo FGrHist 70!F!01950000 @195|ca}) è negativa e ci dice che non aggiunse i dettagli che si trovano in Duride (FGrHist 76 F 67\indexp{Duride FGrHist 76!F!00670000 @67|ca}). Eforo\index[n]{Eforo} è in questo passo incluso tra Tucidide\index[n]{Tucidide} ed Aristotele: sembra che il modello di confronti sia rispettato e che venga controllata la fonte di quinto, la storiografia di scuola retorica, quella di scuola filosofica e Duride (Walbank 2011, 394). Ma se la denominazione della guerra fa riferimento ad un periodo di rielaborazione delle informazioni, ed è l'unico legame con  Diodoro\index[n]{Diodoro} per questo passo e, d'altro canto, in Plutarco\index[n]{Plutarco} Eforo\index[n]{Eforo} è nominato per altre informazioni e insieme ad altri tre storici, non è dato vedere il passaggio di Eforo\index[n]{Eforo} in questo passo, né c'è terreno per estendere il ragionamento che sul paragrafo seguente può invece essere fatto.     
            
            \subsection*{Le cause del \textgreek{Πελοποννησιακὸς πόλεμος}}
            \label{bkm:RefHeading697001501267828}
            La prima causa della guerra del Peloponneso è Pericle\index[n]{Pericle}. Prima in un elenco logico, forse in uno politico, certamente non cronologico, né necessariamente più importante delle altre né quella più facilmente attribuibile. La causa di politica interna FGrHist 104 ce la racconta in un modo che ci è noto da Filocoro,  Diodoro\index[n]{Diodoro} e Plutarco\index[n]{Plutarco} in quattro punti principali: 
            \begin{itemize}
            \item la questione di Fidia e dei rendiconti
            \item il decreto megarese 
            \item due passi di Aristofane\index[n]{Aristofane}  portati come ''prova'' (e fonte?) delle dichiarazioni su Fidia e sul decreto per i quali è necessario tenere presente la diversa impostazione eziologica di Eforo\index[n]{Eforo} (Parmeggiani 2011, 433)
            	\item la soluzione proposta da Alcibiade in esergo (piuttosto che a premessa come in Plutarco)
	\end{itemize} 
	È una collezione dei punti salienti e memorabili della vicenda, priva di complessità ed articolazione come siamo ormai abituati a leggere in FGrHist 104. Soprattutto la presenza dei passi di Aristofane\index[n]{Aristofane}  fa pensare immediatamente a  Diodoro\index[n]{Diodoro} e Plutarco\index[n]{Plutarco} ed è normale che questo passo del nostro, dove essi sono più estesi, abbia fatto pensare. La struttura stessa dell'argomentazione, come ho cercato di far notare, è indicatore di un passaggio intermedio nell'elaborazione del passo argomentativo, in cui la letteratura erudita di contorno al testo di Aristofane\index[n]{Aristofane}  viene usata come materia per la compilazione di opere che si servono del testo commentato come prova. Aggiungiamo a questa presenza problematica che la fonte dichiarata da  Diodoro\index[n]{Diodoro} alla fine di questo passo è Eforo\index[n]{Eforo} (\cite[88]{Schepens2007}; \cite[420]{Parmeggiani2011}) e capiamo subito quale sia il nodo, fortunatamente dipanato da Parmeggiani nel suo recente studio sullo storico di Cuma (\cite[417-458]{Parmeggiani2011}). Infatti, se Plutarco\index[n]{Plutarco} e  Diodoro\index[n]{Diodoro} hanno la stessa fonte, anche FGrHist 104 ''deve'' essere accostato ad Eforo\index[n]{Eforo} e, poiché è più rudimentale e scarno, è anche più pulito dalle interferenze successive e quindi potenzialmente più ''vicino'' al celebre autore disperso (\cite[420 n.114]{Parmeggiani2011}): congettura che penso sia stata sufficientemente sottoposta a critica. Anche Filocoro usa Eforo\index[n]{Eforo} perché racconta la storia di Fidia in due versioni, una delle quali, la seconda è simile al racconto di Diodoro. Tucidide\index[n]{Tucidide} è fuori gioco, almeno per questo paragrafo e ci lascia vedere Eforo\index[n]{Eforo} o una sua epitome. Questa la tesi corrente che è indubbiamente possibile, ma forse da ridimensionare leggermente tenendo l'attenzione su FGrHist 104 piuttosto che sullo storico di Cuma, o ancor meglio su Aristofane\index[n]{Aristofane}  che detiene il centro indiscusso dell'argomentazione. Diversamente da Plutarco\index[n]{Plutarco} e Diodoro, dove è il ricorrente concetto di \textgreek{λόγου δεινότης} a far da perno (\cite[419]{Parmeggiani2011}). Questa è una differenza fondamentale tra FGrHist 104 e  Diodoro\index[n]{Diodoro} 12.38s.\indexp{Diodoro!12!00380000 @38s.|ca} Non si può nemmeno intravedere quello sforzo di chiarimento ed elaborazione del discorso di Tucidide\index[n]{Tucidide} portato avanti da Eforo\index[n]{Eforo}. Il modo in cui i passi di Aristofane\index[n]{Aristofane} sono utilizzati da Diodoro e FGrHist 104 lo dimostra. A prescindere dai passaggi che hanno permesso di pervenirvi, il risultato dell'elaborazione è completo e documentato, conserva sia l'opinione popolare che attribuisce la colpa all'uomo in vista, sia l'approfondimento di politica inter-statale (Corcira e Potidea), sia l'analisi storica (la vera causa), sia l'aneddoto. Testimonia dunque una serie ormai cristallizzata di argomenti, digressioni, personaggi e temi attorno al topos delle cause della guerra del Peloponneso. 
            
            \subsubsection{\textgreek{ἁλόντος τοῦ Φειδίου ἐπὶ νοσφισμῷ, εὐλαβηθεὶς ὁ Περικλῆς}}
            Pericle\index[n]{Pericle}, nel racconto di  Diodoro\index[n]{Diodoro} tiene a cuore un vecchio consiglio datogli da Alcibiade quando era piccolo, ma qui è semplicemente un ragionamento di convenienza sul pericolo di essere trascinato nei problemi di Fidia. 
            Questo episodio non è da collocare per forza a ridosso dei vari decreti megaresi.  Diodoro\index[n]{Diodoro} 12.38\indexp{Diodoro!12!00380000 @38|ca} implica una serie di salti cronologici che lasciano ampio spazio alla cronologia di Filocoro (438/7). \cite[429 n.155]{Parmeggiani2011} argomenta invece a favore della datazione tradizionale c. 432 a.C.. Un'analisi alternativa basata sull'osservazione del metodo di Plutarco\index[n]{Plutarco} che fa confluire eventi di diversi momenti in un'unica narrazione si trova in \cite{Stadter1989}. Del resto non stupisce questo metodo, nemmeno vedendolo utilizzato sia da  Diodoro\index[n]{Diodoro} che da Plutarco, giacché altro non è che una delle necessità del raccontare la storia. In FGrHist 104 non c'è specifica connessione tra Fidia e Pericle\index[n]{Pericle}  se non dal punto di vista professionale. \cite[430]{Parmeggiani2011} con bibliografia per i capi d'accusa dei vari processi a Pericle\index[n]{Pericle}. Tanto meno da quello dei capi d'accusa: semplice furto. I titoli professionali attribuiti a Pericle\index[n]{Pericle}  e Fidia sono vari, ma estremamente generici nel nostro testo e credo non permettano di argomentare una data di composizione sulla base del loro utilizzo. \cite{Marr1998}.
            
            \subsubsection{\textgreek{ἐπολιτεύσατο τὸν πόλεμον τούτον}}
            Su queste parole è caduta l'attenzione di Schepens che vi riconosce un'indicazione dell'interpretazione del ruolo pericleo da parte di Eforo\index[n]{Eforo} ''Pericle fece politica facendo questa guerra''. Contro questa opinione \cite[455]{Parmeggiani2011}. Non credo tuttavia si possa attribuire tale peso a questa espressione che può essere molto più neutra, considerando il testo di FGrHist 104 in generale. Il decreto cui si fa riferimento è da identificare probabilmente come il primo di quelli ricordati da Plutarco, quello in cui Pericle\index[n]{Pericle}  si rifiuta di girare la stele dall'altra parte, ma che non è problematico quanto i successivi. Plut. \emph{Per.} 30\indexp{Plutarco!\textit{Pericle}!000300000 @30|ca}, prima e separatamente dalla questione di Fidia con il commento di \cite{Stadter1989}. e \cite{McDonald1994}. Il legame tra le varie cause è di primaria importanza in  Diodoro\index[n]{Diodoro} e Plutarco\index[n]{Plutarco} che strutturano il discorso con orientamenti e intenti diversi. 
            
            \subsubsection{\textgreek{ὁ τῆς ἀρχαίας κωμῳδίας ποιητὴς λέγων οὕτως}}
            Perché vengano selezionati passi diversi da  Diodoro\index[n]{Diodoro} \label{bkm:RefHeading469751508721977} è ben spiegato da Parmeggiani attraverso la centralità della descrizione dell'abilità oratoria di Pericle\index[n]{Pericle}. Su questo passo di Aristofane\index[n]{Aristofane}, si veda anche \cite[23s]{Cassio1982}. In  Diodoro\index[n]{Diodoro} i passi di Aristofane\index[n]{Aristofane}  sono separati dalla considerazione politica, qui invece fanno da prove per una delle cause. Un frainteso, che può avvenire solo ad opera di  Diodoro\index[n]{Diodoro} su FGrHist 104 e non viceversa apparentemente, ma che nella complessità della tradizione può essere avvenuto in diversi momenti.   L'intero verso 608 è omesso da  Diodoro\index[n]{Diodoro} e da Plutarco, mentre parrebbe proprio essere, per la versione di FGrHist 104 il motivo del  \textgreek{εὐλαβηθεὶς}, il verso al quale il commento viene accostato, la vera accusa personale diretta a Pericle\index[n]{Pericle}  che non c'è nelle altre fonti. FGrHist 104 come la seconda versione riportata dallo scolio lega i due eventi, Fidia e il decreto. I \textgreek{τινες} di Filocoro potrebbero forse essere già l'Eforo che riconosciamo tramite  Diodoro\index[n]{Diodoro} e Plutarco\index[n]{Plutarco} dietro alle parole di questo paragrafo, ma anche il Teopompo\index[n]{Teopompo}  che abbiamo visto in F34 di Filocoro. 
                        
            \subsubsection{\textgreek{καὶ πάλιν ὑποβάς}}
            Il passo dagli \emph{Acarnesi},\label{bkm:RefHeading3610119231068} non serve all'ottica presente in Diodoro. \cite[436]{Parmeggiani2011}. È fondamentale invece nel contesto delle contingenze a causa della guerra e Plutarco\index[n]{Plutarco} ne conserva i primi quattro versi soltanto, prima della parte effettivamente interessante per il discorso eziologico. Essi selezionano a seconda dell'impronta data alla loro opera e si possono apprezzare le differenze di impostazione e di intenti assenti in FGrHist 104, che riprende tutto il testo.
            
            \subsubsection{\textgreek{μὴ σκέπτου πῶς ἀποδῷς … ἀλλὰ πῶς μὴ ἀποδῷς}}
       Posto a questo punto del racconto perde il suo significato narrativo che conserva con grande efficacia in  Diodoro\index[n]{Diodoro} 12.38, con un vago erodoteo ricordo della giovane Gorgo all'incontro tra Cleomene ed Anassagora: è solo un aneddoto curioso ricordato per il piacere dell'arguzia ad alleggerire la lista e le citazioni. Plutarco\index[n]{Plutarco} conosce questo materiale ma lo usa nella ''successiva'' \emph{Vita}, quella di Alcibiade, nel contesto della dimostrazione della sua giovanile sagacia (Plut. \emph{Alc.} 7.2\indexp{Plutarco!\textit{Alcibiade}!00007002 @7.2|ca}). Questo è l'ultimo dei detti famosi di personaggi illustri conservato da FGrHist 104.
            
            \subsubsection{\textgreek{δευτέρα δὲ αἰτία φέρεται καὶ Κερκυραίων καὶ Ἐπιδαμνίων τοιαύτη}}
            C'è forse un po' di indecisione nel definire questo episodio nel doppio \textgreek{καί}, ma ciò che più importa è che l'anticipazione della sezione periclea alle cause tucididee della guerra, ridotte alle due principali e a quella ''vera'' non è certo casuale. FGrHist 104 torna al ritmo da elenco. Per i fatti di Corcira ed Epidamno si riduce dunque ad elencare pazientemente attori e alleanze, scontri e contingenti  (anche se non tutte coerenti con Tucidide, cfr. Thuc. 1.118\indexp{Tucidide!1!01180000 @118|ca} e 24-55\indexp{Tucidide!1!00240000 @24-55|ca}, Diod. 12.31-33\indexp{Diodoro!12!00310000 @31-33|ca}), per poi mettere l'accento sullo scioglimento dei patti, e quindi il nulla osta alla guerra non più impedita da vincoli di giuramento inter-statale. Una volta che gli Ateniesi hanno ufficialmente infranto i patti è questione di un pretesto. Le braci su cui soffia il Pericle\index[n]{Pericle} di Aristofane\index[n]{Aristofane} sono vive. Nella sintesi di nuovo vediamo l'organizzazione del discorso, semplice e precisa al punto della banalità formale, ma efficacemente strutturata. Se il riferimento è alla pace dei trent'anni, sciolta \textgreek{ἐν δὲ τῷ αὐτῷ ἔτει} della spedizione a Samo, allora la cronologia tradizionale vacilla di nuovo, ma abbiamo visto che i quattordici anni di Samo non sono del tutto affidabili e non è il caso di aggiungere calcoli ipotetici. È invece rilevante che la struttura di flashback della narrazione, come in Diodoro, utilizzata per la questione di Pericle\index[n]{Pericle}, qui sia usata all'inverso, anticipando l'episodio di Fidia e la questione megarese e aggiungendo Corcira e Potidea. Probabilmente l'ordine selezionato è di importanza ascendente. I patti sciolti sono quelli stipulati al paragrafo 15 e con quel periodo dobbiamo intendere la sincronia di questi eventi per FGrHist 104 che cerca così di ritornare sul percorso tracciato dopo l'elenco delle cause.
            
            \subsubsection{\textgreek{Ποτίδαια πόλις ἄποικος}}
           Probabilmente parte della sintesi estrema è dovuta anche al fatto che lo spazio è finito. Il conflitto di interessi e la scomoda posizione istituzionale che fanno di Potidea il \emph{casus belli} esemplare non emergono dal testo che lo riduce ad un semplice episodio di assedio. Thuc. 1.65\indexp{Tucidide!1!00650000 @65}; Diod. 12.34\indexp{Diodoro!12!00340000 @34} and 37\indexp{Diodoro!12!00370000 @37}.
            
            \subsubsection{\textgreek{ἡ καὶ ἀληθεστάτη...}}
            Quasi proverbiale la vera causa tucididea della guerra del Peloponneso (Thuc. 1.23.6\indexp{Tucidide!1!00230006 @23.6|ca}), che, sebbene qui sia persa, non è più \textgreek{πρόφασις}, ma \textgreek{αἰτία} (\cite[444]{Parmeggiani2011}): è Sparta che osserva e considera l'\emph{auxesis}  ateniese. Questa in FGrHist 104 è spezzettata, nelle ultime parole del manoscritto, come nella sintesi del discorso pericleo in Diod. 12.40\indexp{Diodoro!12!00400000 @40|ca}: navi, ricchezze e alleati. 
            }