\section{P.Oxy. 27.2469 (BL inv. 3052)}\label{bkm:RefHeading697951501267828}
Rinvenuto ad Oxyrynchos (Bahnasa), il papiro,\footnote{Altezza: 19,2 cm. Larghezza:
3,5-4,5 cm. Margine destro: 1,6 cm.}
composto di quattro frammenti combacianti, è datato al II secolo d.C. sulla base della
stretta somiglianza\footnote{''Ben oltre le pur accentuate
concordanze stilistiche'', secondo \cite[32]{Cavallo1975}.} con P. Fay. 87\index[pap]{P. Fay. 87}, del 155 d.C. (l.1-2:
\textgreek{ἔτους ιη' Αὐτοκράτορος  Καίσαρος  Τίτου  Αἰλί[ο]υ | Ἁδριανοῦ Ἀντωνείνου Σεβαστοῦ Εὐσεβοῦς}). Cavallo, nel suo prospetto cronologico
(\cite*[50]{Cavallo1975}) indica P.Oxy. 2469\index[pap]{P.Oxy.!00272469 @27.2469} prima di P. Fay. 87:
quest'ultimo sarà dunque da
ritenere come \textit{terminus ante
quem}.\footnote{Ed.:
Rea, P.Oxy. 27.2469 1962, 141-145; \cite[15-16]{Mette1978}; MP\textsuperscript{3} 137;
\cite[658-677]{Zuntz1938}, \cite[32s]{Cavallo1975}. LDAB n°4733.}


Cavallo considera il nostro papiro il più antico testimone della cosiddetta
onciale copta, la scrittura probabilmente utilizzata da Fozio\index[n]{Fozio} per dare al suo
codice un'austerità maggiore. Ecco come la descriveva:


\begin{quotation}
\small{La scrittura, rigorosamente verticale, mostra nel complesso
spiccata tendenza al disegno fluido e ricurvo e alle occhiellature corpose,
sostituite, in certe manifestazioni, da grossi ispessimenti; gli effetti
chiaroscurali sono dovuti, quindi, non tanto ad un particolare angolo di
scrittura [...], ma piuttosto al ripiegarsi e sovrapporsi delle linee nel
gioco delle occhiellature o all'artificioso innestarsi degli
ispessimenti. Peculiare è anche la tendenza dei tratti orizzontali (superiori
o inferiori) ma soprattutto degli obliqui discendenti da sinistra a destra
(di regola arcuati) ad allungarsi, sì da dar vita ad un vero e proprio
sistema di pseudolegature che creano l'illusione di teorie
di lettere allacciate l'una
all'altra. (\cite[30]{Cavallo1975})}
\end{quotation}





Si possono sicuramente osservare, anche nel nostro caso, il tratto orizzontale
lungo fino ad estendere il modulo al rettangolo, gli ''\textit{heavy bloblike
serifs}'' (Rea 1962, 141) con cui le lettere sono rifinite e la tendenza a
prolungare i tratti obliqui discendenti sul rigo di base. 

Per quel che riguarda l'aspetto del
supporto,\footnote{Ho preso visione del papiro alla British Library nel 2009.} un lungo e profondo solco percorre il papiro verticalmente e
parallelamente al suo margine destro, probabilmente segno di una piegatura.
Il testo è sul
\emph{recto}, il
\emph{verso} è bianco. Le righe dalla 25 alla 30 sono ben conservate per la quasi
totale interezza, quindi offrono un buon riferimento per la larghezza delle
precedenti e permettono scelte congetturali molto
calibrate. La
ricostruzione sulla base di FGrHist 104\index[n]{FGrHist 104} (2.2\indexp{FGrHist 104!00020002 @2.2} e
2.4\indexp{FGrHist 104!00020004 @2.4}) è legittima, come si può notare mettendo i
testi a fronte l'uno dell'altro.
Penso sia innegabile
l'appartenenza dei testi alla medesima tradizione, sebbene
forse in rami diversi:\footnote{Vedi
p.\pageref{bkm:RefHeading690021501267828}.}

\clearpage
\begin{multicols}{2}

\begin{quotation}
\footnotesize{\textgreek{λαβὼν δὲ ὁ Μαρδόνιος ἔπεμψε πρῶτον πρὸς Ἀθηναίους Ἀλέξανδρον τὸν Μακεδόνα, τὸν Φιλίππου πρόγονον, ὑπισχνούμενος δώσειν αὐτοῖς μύρια τάλαντα καὶ γῆν ὅσην αὐτοὶ βούλονται τῆς Ἑλλάδος τηρήσειν τε ὑποσχόμενος καὶ τὴν ἐλευθερίαν αὐτοῖς καὶ τὴν αὐτονομίαν, εἰ ἕλοιντο μένειν ἐφ’ ἑαυτῶν καὶ μὴ συμμαχεῖν τοῖς Ἕλλησιν. ἐπειδὴ δὲ ὁ Ἀλέξανδρος παρεγένετο εἰς τὰς Ἀθήνας καὶ ταῦτ’ ἐδήλωσεν, οἱ Ἀθηναῖοι οὔτε τοὺς λόγους προσεδέξαντο ὑβρίσαντές τε τὸν Ἀλέξανδρον ἀπεπέμψαντο. 
ὁ δὲ Μαρδόνιος ἀποτυχὼν ἐν τούτοις ἐπῆλθεν εἰς τὰς Ἀθήνας καὶ τὰ ἔτι περιλειπόμενα μέρη προσενέπρησεν, παραγενόμενός τε εἰς τὰς Ἀθήνας ἅμα τῷ στρατῷ ἐνταῦθα ἐστρατοπεδεύσατο, οἱ δὲ Ἕλληνες ἐστρατοπεδεύσαντο ἐν Πλαταιαῖς· τὰ δὲ μεταξὺ Θηβ\Ladd{αι}ῶν καὶ Πλαταιῶν στάδιά ἐστιν π. συνπαρετάσσοντο δὲ Μαρδονίῳ Βοιωτῶν μυριάδες δ. 
    εἶχον δὲ τὸ μὲν δεξιὸν κέρας Πέρσαι καὶ Μαρδόνιος, τὸ δὲ εὐώνυμον οἱ μηδίσαντες Ἕλληνες. τῶν δὲ Ἑλλήνων οἱ μὲν Ἀθηναῖοι εἶχον τὸ δεξιόν, τὸ δὲ εὐώνυμον Λακεδαιμόνιοι. μετέστησαν δὲ αὐτούς οἱ Λακεδαιμόνιοι, φήσαντες Ἀθηναίους ἐμπειροτέρους εἶναι πρὸς τὸ μάχεσθαι Πέρσαις. ἐν δὲ τούτῳ Μαρδόνιος
    }}
    \end{quotation}
    \columnbreak
    \begin{quotation}
    \footnotesize{\textgreek{
    \lbrk..........................\rbrkμψ\lbrk...\\
    \lbrk........................\rbrk Μακεδ\lbrk..\rbrk\\
    \lbrk........................\rbrkχνού\lbrk....\rbrk\\
    \lbrk...................\rbrkύρια τάλα\lbrk....\rbrk\\
    \lbrk....................\rbrkιαν αὐτ\lbrk....\rbrk\\
    \lbrk.........................\rbrk μένοιεν\\
    \lbrk.......................\rbrkὶ μὴ συμ\\
    \lbrk...................\rbrk Ἕλλησι\lbrk.\rbrk\\
    \lbrk.......................\rbrk ἐξέβαλον \\
    \lbrk.........................\rbrkερ Μαρδό\\
    \lbrk..............................\rbrkβαλεν\\
    \lbrk.........................\rbrkαὶ τὰ λε\lbrk..\rbrk\\
    \lbrk........................\rbrkνέπρησεν.\\
    \lbrk..........................\rbrkήβας ἐστρα\\
    \lbrk...........................\rbrkὲ Ἕλλη\\
    \lbrk....................\rbrkοπεδεύσαν\\
    \lbrk......................\rbrkάς. μεταξὺ \\
    \lbrk.........................\rbrk ἐν μέσω\\
    \lbrk.....................\rbrkήκοντα.\\
    \lbrk........................\rbrkτο δὲ Μαρ\\
    \lbrk......................\rbrkν καὶ τῶν\\
    \lbrk..................\rbrkῶν μηδισάν\\
    \lbrk............\rbrkιάδες τέσσαρες. \\
    \lbrk...........\rbrk τὸ μὲν δεξιὸν \\
    \lbrk..........\rbrkσαι, τὸ δὲ εὐώνυ\\
    \lbrk...........\rbrkηδ\Ladd{ε}ίσαντες Ἕλ\\
    \lbrk..............\rbrkν δὲ Ἑλλήνων \\
    \lbrk.................\rbrkιὸν κέρας Ἀθη\\
    \lbrk.............\rbrkτεῖχον, Λακε\\
    \lbrk............\rbrkι δὲ θάτερον, φή\\
    \lbrk.................\rbrkηναίους ἐμπε\lbrk..\rbrk\\
    \lbrk..............\rbrk εἶν ἤδη πρὸς\\
    \lbrk...............\rbrkι τοῖς Πέρσαις.\\
    ..................\rbrk Μαρδόνιος \lbrk...\rbrk 
    }
    }
    \end{quotation}
    \end{multicols}
    
    \clearpage
    Si può dunque proporre la seguente ricostruzione:
    
    \begin{quotation}
    \begin{verse}
    \poemlines{5}
    \small{
    \textgreek{\lbrkὁ Μαρδόνιος ἔπε\rbrkμψ\lbrkεν}\footnote{\textgreek{ἔπε\rbrkμψ\lbrkεν} Rea; \textgreek{...\lbrkλαβὼν δὲ ὁ Μαρδόνιος πρῶτον ἔπε\rbrkμψ\lbrkεν} Mette} \\
    \textgreek{\lbrkἈλέξανδρον τὸν\rbrk Μακεδ\lbrkό-\rbrk}\\
    \textgreek{\lbrkνα εἰς Ἀθήνας ὑπισ\rbrkχνού\lbrkμε-\rbrk}\\
    \textgreek{\lbrkνος δώσειν μ\rbrkύρια τάλα\lbrkν-\rbrk}\\
    \textgreek{\lbrkτα καὶ ἐλευθερ\rbrkίαν αὐτ\lbrkοις κα\rbrk}\footnote{\textgreek{\lbrkτα καὶ αὐτονομ\rbrkίαν αὐτ\lbrkοῖ-\rbrk\|\lbrkς τηρήσειν, εἰ\rbrk} Rea; \textgreek{\lbrkκαὶ\rbrk \| \lbrkἐλευθερίαν καὶ αὐτονομ\rbrkίαν αὐτ\lbrkοῖς\rbrk \| \lbrkτηρήσειν, εἰ\rbrk} Mette.}\\
    \textgreek{\lbrkὶ αὐτονομίαν, εἰ\rbrk μένοιεν}\footnote{Corretta da un \textgreek{η}.}\\
    \textgreek{\lbrkἐφ’ ἑαυτῶν κα\rbrkὶ μὴ συμ-}\\
    \textgreek{\lbrkμαχοῖεν τοῖς\rbrk Ἕλλησ\d{ι}\lbrkν\rbrk}\footnote{\textgreek{\lbrkμαχοῖεν τοῖς\rbrk Ἕλλησιν} Rea, \textgreek{\lbrkμαχ\rbrk\Ladd{οῖεν} \lbrkτοῖς\rbrk Ἕλλησιν} Mette.}\\
        \textgreek{\lbrkοἱ δὲ Ἀθηναῖοι\rbrk ἐξέβαλον} \\
        \textgreek{\lbrkτὸν ἄνδρα. διόπ\rbrkερ Μαρδό-}\footnote{\textgreek{διόπ\rbrkερ ὁ Μαρδό-} Rea, Mette}\\
        \textgreek{\lbrkνιος εἰς Ἀθήνας ἐνέ\rbrkβαλεν}\\
        \textgreek{\lbrkτὴν στρατιὰν κ\rbrkαὶ τὰ λε\lbrkι-\rbrk}\footnote{\textgreek{\lbrkτὴν στρατιὰν κ\rbrkαὶ τὰ λει-} Rea;  \textgreek{\lbrkτὴν στρατιὰν\rbrk \lbrkκ\rbrkαὶ τὰ \Ladd{περι}λει} Mette.}\\
        \textgreek{\lbrkπόμενα μέρη ἐ\rbrkνέπρησεν.}\footnote{\textgreek{\lbrkπόμενα προσενέ\rbrkπρησεν} Rea, Mette.}\\
        \textgreek{\lbrkἔπειτα δὲ περὶ Θ\rbrkήβας ἐστρα-}\\
        \textgreek{\lbrkτοπεδεύσατο, οἱ δ\rbrkὲ Ἕλλη-}\\
        \textgreek{\lbrkνες ἐστρατ\rbrkοπεδεύσαν}\\
        \textgreek{\lbrkτο περὶ Πλαται\rbrkάς. μεταξὺ}\\ 
        \textgreek{\lbrkδὲ τούτων ἔστιν\rbrk ἐν μέσω\Ladd{ι}}\footnote{Mette}\\
        \textgreek{\lbrkστάδια .......\rbrkήκοντα.}\footnote{\textgreek{\lbrkστάδια ὀγδο\rbrkήκοντα} Rea, Mette.}\\ 
        \textgreek{\lbrkσυνπαρετάσσον\rbrkτο δὲ Μαρ-}\\
        \textgreek{\lbrkδονίῳ Βοιωτῶ\rbrkν καὶ τῶν}\\
        \textgreek{\lbrkἙλλήνων τ\rbrkῶν μηδισάν-}\\
        \textgreek{\lbrkτων μυρ\rbrkιάδες τέσσαρες}\footnote{Rea; \textgreek{\lbrkτων\rbrk \lbrkμυρ\rbrkιάδες τέσσαρες} Mette.}. \\
        \textgreek{\lbrkεἶχον δὲ\rbrk τὸ μὲν δεξιὸν}\\
        \textgreek{\lbrkκέρας Πέρσ\rbrk\d{α}\d{ι}, τὸ δὲ εὐώνυ-}\\
        \textgreek{\lbrkμον οἱ μ\rbrkηδείσαντες}\footnote{Mette.} \textgreek{Ἕλ-}\\
        \textgreek{\lbrkληνες. τῶ\rbrkν δὲ Ἑλλήνων}\\
        \textgreek{\lbrkτὸ μὲν δε\rbrk\d{ξ}ιὸν κέρας Ἀθη-}\\
        \textgreek{\lbrkναῖοι κα\rbrkτεῖχον, Λακε-}\\
        \textgreek{\lbrkδαιμόνιο\rbrkι δὲ θάτερον, φή-}\footnote{Rea. \textgreek{\Ladd{μετέστησαν δὲ Ἀθηναίους οἱ Λακεδαιμόνιοι} φή-}}\\
        \textgreek{\lbrkσαντες Ἀθ\rbrkηναίους ἐμπε\lbrkί-\rbrk}\\
        \textgreek{\lbrkροτέρους\rbrk εἶν\Ladd{αι}}\footnote{Mette; \textgreek{ἐμπεί\lbrkρους ὑπάρχ\rbrkειν} Rea.} \textgreek{ἤδη πρὸς}\\
        \textgreek{\lbrkτὸ μάχεσθα\rbrkι τοῖς Πέρσαις.}\\
        \textgreek{ἐν δὲ τούτῳ\rbrk  Μαρδόνιος \lbrk\ ...\rbrk}\\
        }
        \end{verse}
        \end{quotation}
        
        
        
        
        Il papiro riporta dunque una versione del testo molto più ristretta del codice, mancando di
        interi sintagmi e frasi, come \textgreek{τὸν Φιλίππου πρόγονον} al primo
        rigo, \textgreek{τὴν ἐλευθερίαν αὐτοῖς καὶ τὴν αὐτονομίαν} al quinto
        \textgreek{μετέστησαν δὲ αὐτούς οἱ Λακεδαιμόνιοι}, in
        quest'ultimo caso anche a detrimento del senso.
        
        
        Vi sono poi due luoghi del papiro che forniscono una versione leggermente più
        dilatata, quando al rigo 22 si parla dei {\textquotedbl}Greci
        medizzanti{\textquotedbl} assieme ai Beoti, e con
        l'\textgreek{ἤδη} del trentaduesimo rigo.
        
        
        Al quinto rigo l'integrazione con \textgreek{αὐτονομίαν} o
        \textgreek{ἐλευθερίαν} resta arbitraria, ma non c'è posto comunque
        per il verbo. 
        
