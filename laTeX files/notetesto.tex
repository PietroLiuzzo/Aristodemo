
\section{Note al testo}\label{bkm:RefHeading690021501267828}
Schwartz,\footnote{RE II 926.} per cercare di definire il genere di
testo contenuto in FGrHist 104,
pensava ad una raccolta di \textgreek{κεφαλαία}, appunti o esercizi di
memorizzazione. Già Jacoby,\footnote{Jacoby FGrHist 104
\textit{Komm.}p. 320.} commentando Aristodemo scartava questa
definizione come esagerata. Pur nell'estrema sintesi FGrHist 104 è
un testo unitario, sebbene schematico nell'ultima parte. Non
si può nemmeno pensare ad un frammento degli
\textit{Excerpta historica de proeliis et
obsidionibus,} perché l'unità codicologica
contenente FGrHist 104 è paleograficamente precedente a Costantino
VII e risale all'inizio del X
secolo.\footnote{\cite[349]{Dain1967} con bibliografia
precedente. Cfr. \pageref{bkm:RefHeading707961501267828}.} 
I dispositivi peritestuali di
accesso alla riorganizzazione del
fascicolo\footnote{\cite[94-5]{Maniaci2002}. Cfr. \pageref{bkm:RefHeading703691501267828}.} sono
una buona chiave di interpretazione, insieme all'analisi del
manoscritto. \label{ref:maiuscolaminuscola} Il
copista che compone in minuscola su di un unico fascicolo, da due quaternioni
in maiuscola, facendo peraltro un po' confusione, è
probabilmente un intellettuale indipendente dagli interessi
vari.\footnote{Questo sarebbe dimostrato dalle caratteristiche strutturali
dei caratteri e della catena grafica. Laur. Plut. 59.9\index[pap]{Laur. Plut. 59.9}, non datato, ma
riferibile al X secolo e che Cavallo (\cite[222]{Cavallo2000} con tavole) inserisce in una
serie di esempi di libraria informale ''con una pretesa di
formalità''.} I fenomeni che ricorrono nel codice
dimostrano come la nota in maiuscoletto in cima alla prima pagina sia un
riferimento interno al fascicolo posto ad un determinato momento della sua
produzione per ovviare ad un errore.\footnote{Vedi
p.\pageref{bkm:RefHeading704631501267828}.} Data la
presenza di Aristodemo ''il traditore'' all'interno del testo
stesso, pensare ad un altro Aristodemo autore non è necessario. Nella
riorganizzazione il copista identifica il testo grazie al suo
contenuto.\footnote{Così anche Müller e Cfr.
p.\pageref{bkm:RefHeading703691501267828}.}
Resta certo, il fatto che questo Aristodemo non è il tema centrale del nostro testo, ma era una pratica comune, per i compilatori delle enciclopedie, copiare frammenti ampi intorno alla
porzione di testo interessante, copiare il contesto, e non credo debba sorprendere che anche nell'identificare il testo se ne prenda un elemento più o meno casuale.\footnote{\cite[XXV-XXXVI]{Cohen-Skalli2012} presenta in modo chiaro il metodo di composizione degli
\textit{Excerpta Constantiniana}, simile a questo.}
L'opzione più convincente, penso rimanga quella di un ampio
''appunto'' copiato e conservato per una compilazione successiva e identificato
ai fini della riorganizzazione del fascicolo, con uno dei suoi
contenuti\footnote{\cite[36]{Schepens2010}.} ed a conferma resta il fatto che FGrHist 104 è
stato ampiamente usato in commenti, scoli e lessici.\footnote{Vedi
p.\pageref{bkm:RefHeading704661501267828} e
\pageref{bkm:RefHeading701261501267828}.}
Già il testo da cui copia il nostro intellettuale
bizantino presenta ripetizioni, che risultano
{\textquotedbl}spontanee{\textquotedbl}, proprio come ci si aspetterebbe da una serie di
appunti quotidiani. 


Dove il Minas separasse i due autori, Carone\index[n]{Carone di Lampsaco} ed Eforo\index[n]{Eforo}, al
momento del ritrovamento, non lo sappiamo. Forse
all'interruzione di Filostrato? Sicuramente dopo
il paragrafo 10, dove Plut. \textit{Them.} 27\indexp{Plutarco!\textit{Temistocle}!000270000 @27}
permette di escludere Eforo\index[n]{Eforo} dalle possibili fonti, ma rispetto
al contenuto dovrebbe aver inteso
un'interruzione alla fine del terzo paragrafo.



La lingua impiegata in FGrHist 104\index[n]{FGrHist 104} mostra un'omogeneità che, se
può essere imputata al copista, resta comunque un fattore che non permette di
tracciare netti confini. Anche la segnalazione dello stacco, che i precedenti
editori leggevano nel manoscritto, non indica netti cambi di stile o
impostazione del racconto tra una sezione di ispirazione erodotea e una
invece di matrice tucididea. Vedremo peraltro nel commento, come riportare
tutto ai due storici sia parzialmente fuorviante per lo studio del testo. 
Ad
alcuni il greco di FGrHist 104\index[n]{FGrHist 104} è sembrato
''\textit{more
familiar with
Latin}''\footnote{Raphael
Sealey, citato in \cite[259 n.7]{Frost2005}.} ed indubbiamente esso si presenta generalmente lontano
dall'essere piacevole oltre a presentare errori che
potrebbero essere ricondotti ad una familiarità maggiore con questa lingua.
Già nelle parole iniziali di quella che dovrebbe
essere la sezione ''erodotea'' di FGrHist 104\index[n]{FGrHist 104} vediamo una terminologia molto
vicina a Tucidide\index[n]{Tucidide} (Thuc. 1.91), o l'impiego di termini
tardi come \textgreek{νεφωθέντα} (1.8\indexp{FGrHist 104!00010008 @1.8}) e accanto ad alcune assonanze forse casuali, vi sono termini usati da FGrHist 104 in un senso particolare, come
\textgreek{m'eroc (1.2\indexp{FGrHist 104!00010002 @1.2}, 2.3\indexp{FGrHist 104!00020003 @2.3}) }e  \textgreek{ὑπάρχων}, titolo generico
probabilmente che troviamo attribuito tanto ad Aristide (1.4) che a Pausania
(8\indexp{FGrHist 104!00080000 @8}). Anche l'\textit{hapax}
\textgreek{περίυπνος}, forse frutto di mala divisione (8\indexp{FGrHist 104!00080000 @8}) è probabilmente
riconducibile ad un precedente tucidideo (Thuc.
2.2.1\indexp{Tucidide!2!00020001 @2.1}: \textgreek{ἐσῆλθον περὶ πρῶτον ὕπνον}). È molto interessante
l'uso particolare di particelle e connettori. Troviamo per
esempio, lungo tutto il testo un uso di
\textgreek{δὲ} molto frequente a indicare i vari
eventi consecutivi uno dopo
l'altro, mentre 
\textgreek{ἐπειδή δὲ} (1.2\indexp{FGrHist 104!00010002 @1.2}, 2.2\indexp{FGrHist 104!00020002 @2.2},
3\indexp{FGrHist 104!00030000 @3}) sembra essere
sempre utilizzato per riprendere un elemento della
lista non immediatamente precedente. Sono inoltre utilizzati come marcatori
strutturali \textgreek{τε} e \textgreek{καὶ}: a
2.2\indexp{FGrHist 104!00020002 @2.2} e 3.2\indexp{FGrHist 104!00030002 @3.2} troviamo esempi molto chiari di questo
ruolo, che in  \textgreek{γιγνομένης τε τῆς μάχης τῆς ἐν Πλαταιαῖς καὶ νικώντων τῶν περὶ Μυκάλην Ἑλλήνων} diventano indispensabili per la
comprensione letterale del testo. In questo passo è evidente come si voglia
sottolineare, in linea con la tradizione, una relazione (anche se non
possiamo dire se di contemporaneità) tra le due battaglie. Se sia
invece un errore, dovuto alla sintesi effettuata dallo scriba, che
potrebbe aver tralasciato il verbo di questa frase
o scritto una seconda premessa dimenticando del tutto di inserire la
reggente, non possiamo provarlo in assenza di altri testimoni della
tradizione, ma è da rilevare come nella lingua parlata questo tipo di
strutture sia tutt'altro che infrequente o
scorretto. Nella
serializzazione dei vari elementi ''elencati'' dai vari \textgreek{δὲ} interviene un cambiamento al paragrafo 11\indexp{FGrHist 104!00110000 @11}, dove, in concomitanza con una
probabile presa di coscienza rispetto allo spazio a disposizione, la
narrazione diventa molto spedita e inizia una serie di
\textgreek{μετὰ δὲ ταῦτα} (11.4\indexp{FGrHist 104!00110004 @11.4},
12\indexp{FGrHist 104!00120000 @12}, 14\indexp{FGrHist 104!00140000 @14}), variati con  \textgreek{Εὐθὺς … ἐνταῦθα} (13\indexp{FGrHist 104!00130000 @13}) e \textgreek{Καὶ μετὰ ταῦτα εὐθὺς} (15\indexp{FGrHist 104!00150000 @15}). Verrebbe da pensare che ciascuno dei precedenti \textgreek{d`e
}non stia ad indicare il medesimo tipo di congiunzione, e se così fosse la
prospettiva strutturale sarebbe del tutto diversa: la parte che più sembra
ristretta, sarebbe invece quella copiata per esteso, di contro ad una sintesi
strutturale molto forte su tutta la prima parte. 
L'errore corretto dal copista a
2.4 è\indexp{FGrHist 104!00020004 @2.4} indicativo di una rinnovata sensibilità nell'utilizzo
dei deittici, anch'essa forse in parte riconducibile ad un
uso parlato. Sono presenti anche riprese formulari interne come il meccanismo
di richiesta dell'esercito e il
seguente \textgreek{λαβὼν δὲ} che troviamo, di
nuovo, sia per Aristide che per Mardonio. Infine, se
la sensibilità ai numeri di noi moderni rispetto
ad Erodoto\index[n]{Erodoto} è molto diversa, tanto si può dire della sensibilità di FGrHist 104\index[n]{FGrHist 104}
rispetto ad entrambi.

%
%Abbiamo altri esempi di testi compilati estraendo informazioni citate in testi
%non esplicitamente di quel genere, come P Berol. 9780\indexp{P.Berol. 9780}, un commentario a
%Demostene\index[n]{Demostene} che, secondo una recente ipotesi,\footnote{Gibson 2002, 68.}
%sarebbe una raccolta di informazioni storiche ad opera di uno storico
%ermopolitano del II-III secolo d.C., che le estraeva dai commenti a Demostene
%\index[n]{Demostene}
%di Didimo Calcentero\index[n]{Didimo Calcentero}.
%
%Anche questa è un'opzione valida,
%ma porterebbe a rovesciare il rapporto tra il testo del manoscritto e quello
%della tradizione confluita nella scoliografia. Non
%è necessario abbandonare questa
%possibilità che ''FGrHist 104\index[n]{[Aristodemo] FGrHist 104}'' sia, nella versione di Cod. Par. Suppl. Gr.
%607, frutto dell'escerptazione da raccolte di più scolii,
%commentari o altre opere, se pensiamo che questo può essere iniziato
%relativamente presto. Queste operazioni possono essere avvenute in diversi
%momenti della tradizione, anche se non lo possiamo provare. Ci dovremo
%limitare ad osservare che questo}}{ testo è anche un
%documento, con connotazioni decisamente forti e problematiche, di diversi
%importanti momenti di storia della storiografia,}\footnote{Frost 2005, 264;
%Schepens 2007, 96-8.}{{ che
%hanno prodotto solo in certi casi testimoni coerenti.}}

Il problema della relazione tra i manoscritti
conservati, il Cod. Par. Suppl. Gr. 607\index[pap]{Cod. Par. Suppl. Gr. 607} e il frammento
papiraceo P.Oxy. 2469\index[pap]{P.Oxy.!00272469 @27.2469}, si pone in realtà soltanto
in relazione alla possibile origine da una stessa opera, già in circolazione
alla metà del secondo secolo. I testi conservati dai manoscritti e dalla
tradizione scoliastica ad Aristofane\index[n]{Aristofane} sono afferenti a diversi rami della
tradizione del testo.  Zuntz\footnote{\cite[658-677]{Zuntz1938}.} aveva proposto, per la somiglianza con i
testi degli scoli ad Aristofane\index[n]{Aristofane}, la presenza di FGrHist 104\index[n]{FGrHist 104}
tra i materiali utilizzati
dall'autore degli scolii. In una tabulazione si riportavano
a confronto
FGrHist 104, 10.1\indexp{FGrHist 104!00100001 @10.1}, lo scolio ad Aristofane\index[n]{Aristofane}, \textit{Eq.} 84\indexp{Scolia ad Aristoph.!\textit{Equites} 84} (formato al tempo
dai soli frammenti marginali \textit{a} e \textit{b} del papiro
Oxford Bodl. Ms. Gr. Class. f.72\index[pap]{Oxford Bodl. Ms. Gr. Class. f.72}) e gli scolii medievali Schol. I 84b\indexp{Scolia ad Aristoph.!\textit{Equites} 84b I} e
Schol. II 84b\indexp{Scolia ad Aristoph.!\textit{Equites} 84b II}. Era già noto che un'opera come FGrHist 104\index[n]{FGrHist 104}
potesse essere alla base di Suda e degli scoliografi ad
Aristofane\index[n]{Aristofane},\footnote{Jacoby, FGrHist 104\index[n]{FGrHist 104} \textit{Komm},
319-20.} ma Rea escludeva, con diversi argomenti,
la possibilità che anche P.Oxy. 2469 fosse uno
scolio formato con materiali tratti da FGrHist 104\index[n]{FGrHist 104}:
l'elaborata scrittura, la minor corrispondenza di questi
scoli con FGrHist 104\index[n]{FGrHist 104} che non del nostro papiro con esso. Per Rea era persino
difficile pensare ad un commentario e proponeva di collocare questo papiro in
uno stadio recente della trasmissione del testo. Riproduco lo
schema di Zunzt, leggermente modificato secondo l'ipotesi di
Rea.\footnote{Rea ipotizzava anche X o B come posizioni possibili per P.Oxy.
2469, mentre l'ho spostato su A.}

\begin{figure}[htbp]
\begin{center}
\includegraphics[width=\columnwidth]{./images/Aristodemo/5BAristodemo5D20FGrHist10420v2-img1.png}
\caption{Manoscritti di FGrHist 104 secondo Zunzt e Rea}
\label{FGrHistZunztRea}
\end{center}
\end{figure}

Nel \cite*{Fournet2002} Fournet
pubblicò anche un frammento
papiraceo conservato a Parigi (P. Acad. Inv. 3 d\index[pap]{P. Acad. Inv. 3 d}), che fu riconosciuto come il pezzo mancante tra i frammenti a e b del papiro oxoniense. Il testo così
ricomposto concorda ulteriormente con la versione di
FGrHist 104\index[n]{FGrHist 104} per la presenza del verbo
\textgreek{μετανοήσας}, assente negli scoli più tardi e peculiare della
versione degli eventi narrata da FGrHist 104\index[n]{FGrHist 104}. Nell'edizione
completa del testo (\label{ref:CLGParistoph5}CLGP,
Aristophanes
n°5), Montana sostenne che il
commento ad Aristofane\index[n]{Aristofane} di Simmaco\index[n]{Simmaco} (II d.C.), che rigettava la tradizione sul
suicidio di Temistocle\index[n]{Temistocle}
testimoniata da FGrHist
104\footnote{\cite[54]{Montana2006}; \label{bkm:scholiaaristofeq84bII} Sch.
\textit{Eq} 84b II \textgreek{Σύμμαχος δέ φησι ψεύδεσθαι περὶ Θεμιστοκλέους. Οὔτε γὰρ Ἡρόδοτος οὔτε Θουκυδίδης ἱστορεῖ}.} fosse il
\textit{terminus ante
quem} per
l'uso di FGrHist 104\index[n]{FGrHist 104} come fonte per la spiegazione del passo
dei \textit{Cavalieri}. Tuttavia, la frase
\textgreek{Δια[βάλλει το]ὺς Ἀθηναίους ὡς κακοὺς πρὸς τοὺς εὐεργέτ[ας ...} che si trova
anche in Schol. 84b I\indexp{Scolia ad Aristoph.!\textit{Equites} 84b I}, non è presente in FGrHist 104.\footnote{Vedi p.\pageref{bkm:scholiaaristofeq84b}}.
È invece presente in Diodoro (Diod. 11.58\indexp{Diodoro!11!00580000 @58}). Questi due testi non
deriverebbero dunque direttamente né da FGrHist 104, né da P.Oxy. 2469, ''bensì
da una fonte ipomnematica comune, nella quale la fonte storica era citata''
secondo Montana.\footnote{\cite*[54]{Montana2006}} Ci troveremmo in una situazione di questo
tipo:\footnote{Forse possiamo identificare A con la ''tradizione, cui
l'annotatore ricorse per la spiegazione di lemmi difficili
dei cavallieri'' secondo \cite[96]{Montana2000}. \cite[48-9]{Montana2006} fornisce i criteri per
la datazione di CLGP Aristoph n°5.}

\begin{figure}[htbp]
\begin{center}
\includegraphics[width=\columnwidth]{./images/Aristodemo/5BAristodemo5D20FGrHist10420v2-img2.png}
\caption{Tradizione manoscritta di FGrHist 104}
\label{FGrHist104manos}
\end{center}
\end{figure}

considerati il discreto pregio stilistico della versione di P.Oxy. 2469\index[pap]{P.Oxy.!00272469 @27.2469}; il
fatto che essa stessa aggiunge alcuni termini rispetto ad FGrHist 104\index[n]{FGrHist 104}; e la
concomitanza cronologica attestata tra A e P.Oxy. 2469\index[pap]{P.Oxy.!00272469 @27.2469}, non penso si possa del tutto escludere che essi coincidano. Quest'ultimo appare
sicuramente compendiato, anche se ignoriamo con quale intento. Resta
difficile ''ricostruire una relazione ''genetica'' diretta fra questi due
diversi prodotti dall'esegesi antica.'' (\cite[96]{Montana2000})

Si può però ricavare da questa
tradizione, che il testo dell'autore di
FGrHist 104\index[n]{FGrHist 104} circolò nel Fayyum almeno dal II al V
secolo.

P.Oxy. 2469\index[pap]{P.Oxy.!00272469 @27.2469} invece è molto
probabilmente un elemento della medesima tradizione del testo nel
codice,\footnote{Vedi
p.\pageref{bkm:RefHeading697951501267828}.}
mentre gli altri papiri conservano un ramo della tradizione confluita in
quella della scoliografia ad Aristofane\index[n]{Aristofane}. Ecco un possibile schema:

\begin{figure}[htbp]
\begin{center}
\includegraphics[width=0.8\columnwidth]{./images/Aristodemo/5BAristodemo5D20FGrHist10420v2-img3}
\caption{Tradizione manoscritta di FGrHist 104}
\label{tradizionemanoscr104}
\end{center}
\end{figure}


\clearpage
Nel testo\footnote{\ref{testotraduzione104}} che di seguito riporto, ho scelto o
proposto alcune lezioni, che la visione del
microfilm del manoscritto alla Bibliothèque
Nationale de France mi pare permetta di sostenere.
Ho inserito nell'apparato critico una lezione di P.Oxy.
2469\index[pap]{P.Oxy.!00272469 @27.2469}. Per la direzione di marcia di
Mardonio dopo il secondo sacco di
Atene, infatti, sia Diodoro\index[n]{Diodoro} sia
FGrHist 104 riportano \textgreek{Ἀθήνας}, mentre
nel papiro troviamo l'agognato
\label{ref:cheDindorf}\textgreek{Θή\rbrkβας} che
Dindorf\footnote{\cite[XIX-XXI]{Cohen-Skalli2012}.} aveva corretto nell'edizione di
Diodoro\index[n]{Diodoro} sulla base della logica e di Erodoto.


I paragrafi da 13.2\indexp{FGrHist 104!00130002 @13.2} a 15.4\indexp{FGrHist 104!00150004 @15.4} del testo
del manoscritto corrispondono allo scolio di Massimo Planude\index[n]{Massimo Planude} ad Ermogene\index[n]{Ermogene}
(Walz V, 388\indexp{Scolia ad Hermog.!Waltz V 388}). Questa è la ragione per
cui Jacoby aggiunse nella sua edizione anche i
frammenti 2 (Waltz V, 387.4\indexp{Scolia ad Hermog.!Waltz V 387.4}) e 3 (Waltz V, 378.9\indexp{Scolia ad Hermog.!Waltz V 378.9}) in corpo minore.\footnote{Cfr. p.\pageref{bkm:waltzV388}.} Sebbene di questi due
si debba continuare a dubitare, Waltz V, 388\indexp{Scolia ad Hermog.!Waltz V 388}, conserva un testo innegabilmente afferente alla tradizione del testo di FGrHist 104\index[n]{FGrHist 104}, in una versione migliore del testo del
manoscritto che tramanda erroneamento Temistocle\index[n]{Temistocle} come collega di strategia di Pericle\index[n]{Pericle} contro Samo. Di questo scolio ho riportato dunque tutte le varianti,
adottando il nome di Sofocle\index[n]{Sofocle di Sofillo} (figlio di Sofillo del demo di Colono) da esso tradito poiché questi è attestato tra gli strateghi per il
441/0.\footnote{\cite[48]{Fornara1971} e Prosopographia 12834 per le
fonti.}

Il principio che ho seguito, per le poche scelte
editoriali del testo, è stato quello di privilegiare
l'edizione del codice in quanto
tale,\footnote{Seguendo le
considerazioni di \cite[21 e 30]{Cortassa1990}, per
l'edizione di testi con un unico
testimone.} eliminando ogni correzione o aggiunta
non strettamente necessaria alla comprensione.\footnote{\cite[25]{Cortassa1990}.} Ho corretto per lo più solo gli errori
sintattici o grammaticali dello scriba, già corretti fin dalle prime edizioni. Integrare là dove
il nostro occhio sente un forte bisogno di maggiore completezza, anche
soltanto sintattica finirebbe col mascherarne la natura.

Nel resto del manoscritto
parigino sono stati riscontrati errori fonetici, probabilmente scaturiti
dalla dettatura (effettiva o interiore), ''una prassi che trascende il
passaggio della traslitterazione, accomunando di fatto fase maiuscola e fase
minuscola'',\footnote{\cite[108]{Ronconi2003}.} tra i
quali  si possono annoverare per esempio \textgreek{τρεσσᾶς} per 
\textgreek{τρέσας} in 2.5\indexp{FGrHist 104!00020005 @2.5}, 
\textgreek{πειρευς} per
\textgreek{Πειραιεύς} (due volte) e
\textgreek{μουνουχία}
per \textgreek{Μουνυχία} in 5.4\indexp{FGrHist 104!00050004 @5.4},
\textgreek{γενομενομενος} per
\textgreek{γενόμενος} in 8.1\indexp{FGrHist 104!00080001 @8.1} (dovuto ad una semplice duplicazione) \textgreek{ίο} per \textgreek{τῷ} e \textgreek{διεξείη} per \textgreek{διεξῄει} in
8.3\indexp{FGrHist 104!00080003 @8.3}, \textgreek{ἀνελθόντες} per
\textgreek{ἀνελόντες} in 8.4\indexp{FGrHist 104!00080004 @8.4}, \textgreek{λευκοφρύνι} per
\textgreek{Λευκοφρύνῃ} in
10.5\indexp{FGrHist 104!00100005 @10.5}.

Lo scriba utilizza volentieri abbreviazioni, non
solo grafiche, come l'omicron sigma finale, ma anche di nomi
comuni come \textgreek{γαμ} per \textgreek{γάμον} in
4.2\indexp{FGrHist 104!00040002 @4.2}, \textgreek{μηρ} per \textgreek{μητὴρ} in
8.4\indexp{FGrHist 104!00080004 @8.4}, \textgreek{υν} e \textgreek{υσ} per \textgreek{υἱόν υἱός} in 10.4\indexp{FGrHist 104!00100004 @10.4}, \textgreek{πρα} per
\textgreek{πατέρα} e
\textgreek{σρια} per \textgreek{σωτηρίας} in
11.3\indexp{FGrHist 104!00110003 @11.3}.
Riporto anche alcuni errori ortografici presenti nel testo
\textgreek{οἱ}
per \textgreek{οὑ} in
3.2\indexp{FGrHist 104!00030002 @3.2},
\textgreek{ἀπεκατέστη} per \textgreek{ἀποκατέστη} in 8.1\indexp{FGrHist 104!00080001 @8.1};
scambi di vocale lunga e breve o dittongo
\textgreek{ὡς} per \textgreek{ὃς} in
4.2\indexp{FGrHist 104!00040002 @4.2},
\textgreek{αναφανδων} per \textgreek{ἀναφανδόν},
\textgreek{ἐπερώνησε} per \textgreek{ἐπερόνησε} in
8.1\indexp{FGrHist 104!00080001 @8.1},
\textgreek{κατάφορον} per \textgreek{κατάφωρον} in 8.2\indexp{FGrHist 104!00080002 @8.2},  \textgreek{προσχὼν} per \textgreek{προσσχὼν} in
10.5\indexp{FGrHist 104!00100005 @10.5}, \textgreek{ἐσπήσατο} per \textgreek{ἐσπείσατο} e
\textgreek{ἀνοίσῃ} per \textgreek{ἀνύσῃ}
in 14.2\indexp{FGrHist 104!00140002 @14.2},\textgreek{τειχνήτην} per 
 \textgreek{τεχνίτην} e
\textgreek{ἁλῶντος} per
\textgreek{ἁλόντος} in 16.1\indexp{FGrHist 104!00160001 @16.1}; errori legati al
raddoppiamento
\textgreek{γενήσεσθαι}per \textgreek{γεγενῆσθαι} in 10.4\indexp{FGrHist 104!00100004 @10.4} e  \textgreek{δέδωκεν} per \textgreek{ἔδωκεν} in 10.5\indexp{FGrHist 104!00100004 @10.4}. La lezione
\textgreek{πολιτιδαια} per \textgreek{Ποτίδαια} del manoscritto è facilmente
riconducibile ad un'anticipazione della scrittura della
parola seguente,
\textgreek{πόλις}. Sono invece probabilmente errori grammaticali
\textgreek{ἐπιστρεφόν\ladd{των}}
in 3.1\indexp{FGrHist 104!00030001 @3.1},
\textgreek{Ἐλευθέριαν} in 3.4\indexp{FGrHist 104!00030004 @3.4}, 
\textgreek{τὸ} in 4.2\indexp{FGrHist 104!00040002 @4.2},
\textgreek{παρατεθημένας} in 4.3\indexp{FGrHist 104!00040002 @4.2}
\textgreek{Θεμιστοκλέα} in 10.2\indexp{FGrHist 104!00040002 @4.2},
\textgreek{νύκταν} in 10.3\indexp{FGrHist 104!00100003 @10.3}
\textgreek{ἐπέμνησεν αὐτω} in 10.4\indexp{FGrHist 104!00100004 @10.4} e
\textgreek{διέφθαρον} in 11.4.\indexp{FGrHist 104!00100004 @10.4}

Tutte le letture alternative del manoscritto dei vari editori sono segnalate in apparato. Alcuni luoghi necessitano invece una discussione più precisa.
\begin{description}
\item[1.1 \textgreek{μίαν ἡμέραν μόνην}] La proposta di Jacoby è migliore rispetto a quella di Müller, perché questo è il testo che si legge sul manoscritto dove i \textgreek{ν} finali sono chiari. La necessità di Müller di correggere è tuttavia comprensibile poiché questo inizio è molto brusco e improvviso, ma la versione che ipotizza il verbo \textgreek{μένειν} mi pare più accettabile della lettura al genitivo con \textgreek{μιᾶς ἡμέρας}. La parte conservata è mutila della prima pagina, nella quale probabilmente si trovava la parte iniziale della frase, o almeno il nome di Temistocle, anche se questo non è strettamente necessario. 
\item[1.2 \textgreek{μέρος τι ἔχων}] Il sintagma suona incompleto e molto sintetico, così da portare Jacoby a proporre in apparato, con un punto interrogativo, di inserire \textgreek{τῶν νεῶν} dopo il verbo, sulla base di Giustino (2.12.22\indexp{Giustino!000200120022 @2.12.22} \textit{rex velut spectator pugnae cum parte navium in litore remanet}). In effetti il sintagma sembra non essere mai attestato senza un genitivo di specificazione per \textgreek{μέρος}.\textgreek{ Στρατιᾶς} e \textgreek{νεῶν} sono comuni negli storici.\footnote{Thuc. 7.53\indexp{Tucidide!7!00530000 @53}; Polyb 6.52.10\indexp{Polibio!06052 @6.52.10}; Diod. 15.34.5\indexp{Diodoro!15!00340005 @34.5}; 20.112.3\indexp{Diodoro!20!01120003 @112.3};  Arriano \emph{Anabasi} 2.1.2\indexp{Arriano!\textit{Anabasi}!2.1.2}; Cassio Dione 51.1.4.\indexp{Cassio Dione!51.1.4}.} Anche nelle iscrizioni, la formula si trova spesso impiegata per l'espressione della frazione (con l'ordinale al neutro a coprire la stessa posizione logica del nostro \textgreek{τι}, cfr. e.g.: SEG 12:100\index[pap]{SEG!XII 100} e Agora 19, Poletai P 5). La proposta di Jacoby è più che giustificata, ma non necessaria. Probabilmente \textgreek{στρατιᾶς} o \textgreek{πεζῆς} sarebbero ipotesi migliori per chi volesse integrare, essendo di queste truppe che si parla in precedenza nella frase.
\item[1.2 \textgreek{ἀδύνατον ἦν τὸ πᾶν γεφυρωθῆναι}] Anche in questo caso, con Jacoby, non inserisco \textgreek{Τὸ πᾶν \ladd{τὸν πόρον}} come suggeriva Müller, sebbene il sintagma risulti incompleto. Il problema riguarda il riferimento di \textgreek{τὸ πᾶν} a \textgreek{γεφυρωθῆναι} oppure ad \textgreek{ἀδύνατον ἦν}. Nel primo caso, la mancanza di un oggetto specificato rende necessaria l'integrazione di Müller.\footnote{Xen. Memor. 1.4.17\indexp{Senofonte!\textit{Memorabili}!1.4.17}, Menandro F 539 Koch\indexp{Menandro!F 539 Koch}.} Diversamente, leggere \textgreek{τὸ πᾶν} con riferimento ad \textgreek{ἀδύνατον ἦν} rende l'integrazione superflua. Alcuni \textit{loci similes} si trovano in Aristotele (Ph. 205a\indexp{Aristotele!\textit{Fisica}!205a}: \textgreek{ἀδύνατον τὸ πᾶν}), e possiamo prendere a confronto anche forme come \textgreek{δείσας μὴ πάνυ φωραθῇ ἀδύνατος ὤν} (Thuc. 8.56.4\indexp{Tucidide!8!00560004 @56.4}); \textgreek{παντάπασιν ἀδύνατον} (Xen. Anab. 5.6.10\indexp{Senofonte!\textit{Anabasi}!5.6.10}) e tantissimi altri esempi in Platone, Aristotele e Plutarco; \textgreek{οὐ κατὰ πᾶν δ’ ἀδύνατον κρίνοντα} (Diodoro 4.40\indexp{Diodoro!4!00400000 @40}); fino a sintagmi come il \textgreek{τοῦτο δὲ ἦν παντὸς μᾶλλον ἀδύνατον} (Elio Aristide, Panatenaico 46.144\indexp{Elio Aristide!46.144}), quindi nel senso di \textgreek{πάντως ἀδύνατον} (e.g. Plat. \textit{Leggi} 788a3\indexp{Platone!\textit{Leggi}!00788 @788a3}). Si veda anche 11.4, \textgreek{ὀλίγοι δὲ παντάπασιν ὑπέστρεψαν}.
\item[1.4 \textgreek{ἐκπληττομένος}] Questa è la lettura più probabile del codice, con l'omicron e il sigma finali in legatura. Bücheler correggeva con la forma attiva, per meglio rispondere al \textgreek{καὶ} che lega i due participi; Müller ne sottolineava il significato equivalente all'attivo, ma proponeva anche di riferire il participio futuro passivo agli Ateniesi aggiungendo un infinito, in tal modo intendendo tutto in dipendenza da \textgreek{βουλόμενος} (con due infiniti, quindi, ed i rispettivi participi plurali accusativi). Corretta formalmente, la proposta di Müller rende la frase sicuramente più chiara e completa, ma non è il testo che si trova nel codice e prevede almeno tre interventi (modificazione del verbo, anche se non del tutto necessaria nemmeno per lui; modificazione di caso e numero del participio; aggiunta dell'infinito \textgreek{καίνειν}). Conservando il testo del codice, penso si possa intendere il doppio intento, che resta comunque chiaro, e mantenere la biforcazione dove è, cioè a livello dei due participi, e non in dipendenza dal secondo. Dopo \textgreek{Ψυτταλείαν, ἐκπληττομένος} può essere inteso con valore finale in dipendenza da \textgreek{ἐπεβίβασεν}, mentre in \textgreek{βουλόμενος} prevale l'aspetto continuato del presente rispetto al pur persistente valore finale del participio (essendo comunque tutte queste azioni \textgreek{συνεστηκυίας τῆς μάχης}). L'accusativo prova il valore attivo del verbo e il significato dell'azione di Serse, che prepara l'effetto della reazione di Serse al secondo messaggio di  Temistocle (\textgreek{φοβηθεὶς ἔφευγεν}).
\item[1.5 \textgreek{ἠγωνίσα\Ladd{ν}το}] Jacoby ha espunto il \textgreek{ν} e riferito sia \textgreek{διασημότερον} sia il verbo al singolare ad \textgreek{Ἀμεινίας}, ritenendo la proposta di integrazione con \textgreek{ἠγωνίσαντο \ladd{Ἀθηναίους}} di Müller sbagliata. Quest'ultimo intendeva probabilmente il \textgreek{καὶ} coordinante due frasi i cui soggetti dovevano essere diversi, dati i diversi verbi, e inserire, seguendo la tradizione, gli Ateniesi in opposizione logica al seguente \textgreek{τῶν δὲ βαρβάρων}. La soluzione di Jacoby è più economica ed è sostenuta da una più lineare struttura della frase, con \textgreek{καὶ} a coordinare due verbi relativi entrambi al soggetto. Il  \textgreek{διασημότερον} va inteso in senso avverbiale, inoltre l'intero paragrafo è dedicato ad Aminia, e tutta la sezione è priva di verbi al plurale. Si potrebbe, in alternativa, lasciare il plurale e intendere \textgreek{ἠγωνίσαντο} come verbo di una premessa di valore concessivo, ma penso si possa pensare a questa eventualità più in quanto genesi dell'errore, che non viceversa. 
\item[2.1 \textgreek{\ladd{μά}γοις συμπ\ladd{επει}\d{κὼς}}] \textgreek{καὶ} dopo il participio è l'unico elemento riconoscibile. La ricostruzione del participio è fatta sul confronto con Erodoto (\href{http://data.perseus.org/citations/urn:cts:greekLit:tlg0016.tlg001.perseus-grc1:7.5}{7.5}\indexp{Erodoto!7!00050000 @5} e \href{http://data.perseus.org/citations/urn:cts:greekLit:tlg0016.tlg001.perseus-grc1:7.9}{7.9}\indexp{Erodoto!7!00090000 @9}) e visto l'infinito successivo.
\item[2.2 \textgreek{γῆν ὅσην αὐτοὶ βούλονται}] Jacoby riferisce nel pronome/aggettivo indeterminato il valore potenziale invece di correggere con Müller in \textgreek{ἄν αὐτοὶ βούλοιντο} per dare al verbo la caratteristica ottativa implicata nella proposta, in modo più vicino a Diodoro 11.28.1\indexp{Diodoro!11!00280001 @28.1} (\textgreek{ἣν ἃν βούλωνται}).
\item[2.2 \textgreek{τε ὑποσχόμενος}] concordo con la lettura di Jacoby piuttosto che con la proposta \textgreek{ὑποδεχόμενος} di Bücheler, che probabilmente era stata avanzata pensando alla possibilità di una confusione con il verbo nella riga superiore.
\item[2.3 \textgreek{παραγενόμενός τε εἰς τὰς Θήβας}] P.Oxy 2469\index[pap]{P.Oxy.!00272469 @27.2469}, proveniente probabilmente dal filone di una tradizione scoliografica che si è servita di FGrHist 104 riporta la lezione migliore. Il codice tramanda infatti \textgreek{εἰς τὰς Ἀθήνας} che veniva corretto secondo la logica e la tradizione, da tutti gli editori concordemente con Tebe. Questa correzione trova una conferma evidente nel papiro dove almeno \textgreek{β} è ancora ben visibile. La tradizione manoscritta di Diodoro riporta nel medesimo punto del racconto il medesimo errore, banale certo, ma fino al punto da far pensare che tale coincidenza non possa essere completamente casuale.
\item[2.4 \textgreek{μετέστησαν ... φήσαντες Ἀθηναίους}] Crea difficoltà, in primis allo scriba stesso, l'anticipazione del pronome, rispetto al termine a cui plausibilmente si riferisce (\textgreek{Ἀθηναίους}): lo dimostra la cancellazione, in corso di scrittura, del pronome nella seconda parte, dove sarebbe stato corretto, per sostituirlo appunto con \textgreek{Ἀθηναίους}.  Gli editori hanno risolto il problema nei modi più diversi. Müller intendeva così e ipotizzava che la frase procedesse normalmente come \textgreek{μετέστησαν δὲ τοὺς Ἀθηναίους οἱ Λακεδαιμόνιοι, φήσαντες αὐτοὺς}; Jacoby proponeva di sostituire il pronome “sbagliato” con il solo \textgreek{Ἀθηναίους} e di espungere non solo il cancellato dallo scriba ma anche il secondo, ridondante \textgreek{Ἀθηναίους}. 
\item[2.4 \textgreek{τὴν φάλαγγα}] Jacoby e Müller sentono il bisogno di completare con \textgreek{τὴν φάλαγγα \Ladd{καὶ αὐτός}}.  Questa aggiunta non è necessaria alla comprensione del testo. 
\item[2.4 \textgreek{Λακεδαιμονίοις καὶ ἀκουσίως}] Müller proponeva \textgreek{ἀκουσίοις} per motivare il \textgreek{καί}, inteso forse in senso avverbiale ''anche controvoglia''. Si può tuttavia tenere l'avverbio intendendo “anche svogliatamente”. \textgreek{Ἀκουσίως} con l'infinito è attestato fin da Tucidide (3.31.1\indexp{Tucidide!3!00310001 @31.1}), poi negli oratori e in Polibio (Polyb 3.25.5)\indexp{Polibio!03025 @3.25.5}. Si trova in Diodoro (mai con infinito) con una certa frequenza.
\item[3.1 \textgreek{ἴδιος πρεσβευσάμενος}] Jacoby leggeva \textgreek{ἰδίαι} ma metteva un punto di domanda davanti ad esso. Nel codice mi pare si legga chiaramente \textgreek{ἴδιος} data la presenza di spirito e accento sulla prima, marcata \textgreek{ι} e la finale con la precedente \textgreek{ο}. È interessante la lettura di Müller, che, laddove fosse confermata, offrirebbe un interessante parallelo con l'uso di Diodoro, che ha la medesima forma del participio medio aoristo in cinque passi (14.15.4\indexp{Diodoro!14!00150004 @15.4}; 15.9.4\indexp{Diodoro!15!00090004 @9.4}; 16.3.4\indexp{Diodoro!16!00030004 @3.4}; 19.75.2\indexp{Diodoro!19!00750002 @75.2}; 32.22.1\indexp{Diodoro!32!00220001 @22.1}).
\item[3.2 \textgreek{πλεύσαντές \ladd{τε} ... εἰς Μίλητον}] nonostante il foglio di pergamena sia rovinato su questo punto, la lettura del numero per esteso pare certa. Il problema è duplice, perché complicato da un'ipotesi di Schaefer,\footnote{Coeditore di Bücheler nel 1868.} che proponeva di sostituire come partenza  \textgreek{Σάμου}. L'osservazione di Jacoby, per cui questa ipotesi sarebbe \textit{unwahrscheinlich}, è condivisibile. Il problema della distanza resta comunque, anche aggiungendo con Müller il segnale delle migliaia alla lettera. Un'altra ipotesi potrebbe essere quella di una confusione tra \textgreek{A} e \textgreek{Δ}. Anche in questo caso dovremmo comunque supporre la caduta dell'indicazione delle migliaia. 
\item[3.2 \textgreek{τὰς δ μυριάδας}] l'aggiunta di  \textgreek{ὑπέρ} non è del tutto necessaria, data la negligenza rispetto alle preposizioni e la struttura della frase. In questo caso, peraltro, dove si usa come qualificatore, il risultato è quasi fuorviante,\footnote{Rubincam (\cite*[114]{Rubincam2008}) giunge, studiando a confronto il testo, traduzioni e rielaborazioni di Erodoto rispetto ai qualificatori numerali, ad una conclusione applicabile anche a questo tipo di congetture, e che ritengo valida anche per il commento ''\textit{Modern readers expect history to be written with a higher frequency of numerical information in it, specifying details of time, distance, military forces, and money to a much higher level of precision then their ancient sources were able to do'}'. } Cfr. anche \textgreek{τῆς Μυκάλης} (senza preposizione) in 3.3, dove Bücheler inserisce \textgreek{ἐν} e cambia al dativo.
\item[3.2 \textgreek{\ladd{κατὰ τὴν αὐτὴν ἡμέραν}} Jac, \textgreek{\ladd{ἡ αὐτὴ δὲ ἡμέρα ἦν}} Mül]  La frase è costituita da due genitivi assoluti senza correlazione strutturale e senza un verbo finito o un elemento di reggenza comune così da risultare intraducibile e sicuramente scorretta. Il senso, tuttavia, non è corrotto, solo molto sintetico. È chiaro che si vuole sottolineare, in linea con molta della tradizione, una relazione (anche se non possiamo dire se di contemporaneità) tra le due battaglie, sottolineata dai semplici elementi strutturali residui: \textgreek{τε} e \textgreek{καί}. I segni di interpunzione nel codice non permettono di legare i due genitivi al successivo\textgreek{ ἐστρατήγει}, ma ciò non risolverebbe il problema se non parzialmente. Le proposte di Jacoby e Müller quindi, rispondono alla carenza grammaticale/sintattica del testo, cercando di modificarlo il meno possibile. L'aggiunta di \textgreek{\ladd{κατὰ τὴν αὐτὴν ἡμέραν}} prima di \textgreek{γιγνομένης} proposta da Jacoby rende un po' più evidente il nesso, già chiaro, tra le due proposizioni. Müller ha inserito tutta la frase principale \textgreek{\ladd{ἡ αὐτὴ δὲ ἡμέρα ἦν}}, evidentemente ipotizzandone la caduta integrale nel momento della scrittura. Una proposta alternativa, ma molto invadente, potrebbe, seguendo la narrazione del testo, esplicitare il secondo elemento: \textgreek{γιγνομένης τε τῆς μάχης ἐν Πλαταιαῖς καὶ ἐνίκησαν οἱ περὶ Μυκάλην Ἕλληνες}. Anche in quest'ipotesi tuttavia, che prevede comunque troppe correzioni al testo, si leggerebbe la relazione tra i due eventi secondo la tradizione, scansando possibili alternative origini dell'errore, che di certo qui è presente. Ho scelto di lasciare la versione del codice, che rispecchia la natura di questo testo e di riferire entrambi i sintagmi alla frase precedente, se non all'intero paragrafo. L'origine di questo errore può essere rintracciata con pochi dubbi nella sintesi effettuata dallo scriba, che potrebbe aver tralasciato il verbo di questa frase o scritto una seconda premessa dimenticando del tutto di inserire la reggente, come spesso succede quando parliamo o prendiamo appunti.
\item[3.4 \textgreek{ἤγαγον Ἐλευθέρια}] trovandosi alla fine del foglio 84r, risulta di difficile lettura, poiché, man mano che si procede verso la fine della pagina, le righe vanno restringendo gli spazi tra di esse e la dimensione delle lettere insieme alla spaziatura. L'errore dello scriba è, in questo caso, meccanico: la concordanza di \textgreek{Ἐλευθερία} con \textgreek{ἑορτήν}, invece della corretta reggenza al nominativo di \textgreek{προσαγορεύσαντες}.
\item[4]  Si potrebbero considerare le varie proposte degli editori se solo si potesse leggere qualcosa sul codice. Nel microfilm che ho studiato nulla era visibile né alla fine del foglio 84r né all'inizio del foglio 84v, dove, peraltro, è caduta la seconda metà delle prime due righe circa, corrispondente a  venticinque o trenta caratteri per riga.
\item[4 \textgreek{προσπολεμοῦντες, καί}].  La proposta di Müller è ragionevole ma ''\textgreek{λε}'' è chiaro nel codice. Non vi è però traccia del \textgreek{καὶ} che tuttavia pare indispensabile per la costruzione della frase che deve appoggiare a \textgreek{προσέμενον} anche Pausania e non lo può fare paratatticamente. L'integrazione con \textgreek{ἀφίκετο δὲ} aiuterebbe decisamente la comprensione scorrevole del testo, che chiaramente manca di un verbo principale, ma, come nota lo stesso Jacoby nel commento, lo spazio non è sufficiente.  
\item[4 \textgreek{κατὰ... ἅμα διὰ...}] \textgreek{ἀλλὰ} è congettura di Müller, dove sul codice si trova \textgreek{ἅμα}. L'inserimento della negazione, scelto anche da Jacoby, risponde alla logica implicita di questo \textgreek{ἀλλὰ} congetturale, e si avvicina alla storia che conosciamo, appianando una sottile ma significativa divergenza, se non un tentativo del testo di sottolineare una contraddizione. È invece più interessante la proposta, dubbiosa, che fa Jacoby in apparato: \textgreek{κατὰ φιλοτιμίαν τὴν ὑπὲρ τῶν Ἑλλήνων, ἅμα \ladd{δὲ} διὰ προδοσίαν} che tenta di trovare un senso nell'opposizione tra le preposizioni (\textgreek{κατὰ} / \textgreek{διὰ}), che forse sarebbe chiara in presenza del verbo. Ho accettato sostanzialmente questa lettura, ma non vedo la necessità di inserire la particella \textgreek{\ladd{δὲ}} (che svolge una funzione simile all'\textgreek{ἀλλὰ} di Müller), nonostante la coerenza con l'uso del resto del testo. \textgreek{ Ἅμα} è usato anche nel parallelo passo tucidideo (1.91\indexp{Tucidide!1!00910000 @91} \textgreek{τε ἅμα καὶ τοὺς ξυμμάχους}).
\item[5.1 Jac. \textgreek{\ladd{Ἀκριβῶς γιγνώσκων}}] Il verbo è indispensabile, l'avverbio no.
\item[5.1 \textgreek{κτίζοιτο}] La proposta di Müller, rifiutata anche da Jacoby, è in linea con il testo ed è ciò che ci aspetteremmo, ma si può lasciare che il nostro testo ricordi con questo termine anche il fatto che Atene era stata saccheggiata e distrutta due volte. Anche in 5.4 è usato \textgreek{ἐκτίσθη} per la costruzione delle mura.
\item[5.4 \textgreek{ἔτι νῦν Δία}]  La lettura di Wescher mi sembra la più vicina al testo del codice, nonché la più problematica. C'è certo luogo anche per le proposte di correzione con \textgreek{Ἠετιώνεια}, sulla base di Tucidide (Thuc. 8.90.4\indexp{Tucidide!8!00900004 @90.4} \textgreek{χηλὴ γάρ ἐστι τοῦ Πειραιῶς ἡ Ἠετιωνεία, καὶ παρ’ αὐτὴν εὐθὺς ὁ ἔσπλους ἐστίν}) che è all'origine delle altre letture, e porta Schaefer ad inserire anche \textgreek{\ladd{ὁ εἴσπλους}}. Non ci sono paralleli per tale denominazione, che resta plausibile solo perché molto semplice.\footnote{Potrebbe esserci stato un errore in qualche fase che ha portato a confondere con Tucidide 1.126.3\indexp{Tucidide!1!01260003 @126.3} (\textgreek{ἔστι γὰρ καὶ Ἀθηναίοις Διάσια ἃ καλεῖται Διὸς ἑορτὴ}).}
\item[5.4 \textgreek{ἐφ’ ὅν}] Non c'è evidente necessità di sostituire il caso del pronome, dunque conservo l'accusativo del codice. 
\item[7 \textgreek{\ladd{καί ... χ}ρημάτων}] Dall'inizio delle prime due righe del foglio 85r sono cadute più o meno 22-24 lettere (18 circa per Wescher). La proposta di Müller accettata anche da Jacoby è più che credibile dato lo spazio a disposizione e \textgreek{θησαυροφυλάκιον ἐποιήσαντο}. 
\item[7 \textgreek{\ladd{ὕστερον δέ (?) ............ τάλ}αντα}] considerando il \textgreek{μετεκόμισαν} a cui l'indicazione si riferisce penso si possa ipotizzare che nello spazio vuoto fosse riportata la cifra complessiva dei tributi presenti all'epoca a Delo, come in Diodoro (12.38.2\indexp{Diodoro!12!00380002 @38.2} \textgreek{σχεδὸν ὀκτακισχίλια}). Anche la proposta \textgreek{\ladd{ὑστέρῳ δὲ χρόνῳ π}άντα} di Bücheler è interessante e non richiederebbe ipotesi sui numeri. Jacoby aveva proposto \textgreek{ὕστερον δὲ} come in 8.3, per integrare la prima parte, invece di seguire del tutto la proposta di Bücheler, di cui comunque riprende l'intuizione, ma si è astenuto dal proporre valori. Potremmo immaginarvi in lettere una cifra precisa di talenti, in conformità con altri luoghi del testo, che rivelano una certa passione per i numeri (e.g. 2.3, 5.4).
\item[8 \textgreek{θυγάτηρ Κορωνίδου ὄνομα}] La lettura è chiara così come stampata da Jacoby. Effettivamente qui è caduto il nome della giovane.\footnote{\textgreek{Κλεονίκη}, p. 164.} Anche Müller dice sul passo di Coronide ''\textit{filiae nomen exciderit}''.
\item[10 \textgreek{βασιλεύοντα}] Ho scelto di nuovo la lezione del codice che, sebbene di difficile lettura è comprensibile: il participio ha un preponderante valore di aggettivo. Si può ricondurre questo elemento sicuramente allo scriba. Legare un presunto participio al seguente aggettivo e riproporre l'apposizione \textgreek{βασιλέα} utilizzata da Diodoro (11.56.1\indexp{Diodoro!11!00560001 @56.1} \textgreek{Ἄδμητον τὸν Μολοττῶν βασιλέα}) e Tucidide (1.136.2\indexp{Tucidide!1!01360002 @136.2} \textgreek{Ἄδμητον τὸν Μολοττῶν βασιλέα, ὄντα αὐτῷ οὐ φίλον καταλῦσαι}) come fa Jacoby è comunque corretto e presuppone un semplice errore dello scriba o una diversa lettura.
\item[10 \textgreek{πολεμούντων}] Le lettere sono ben visibili nel codice e trovano un parallelo interno significativo per l'uso di questo termine in 4.2.
\item[11 \textgreek{\ladd{οὐ} γνόντες}] La negazione inserita da Müller non pesa sul senso del testo, e bilancia \textgreek{παραγενόμενοι δὲ ἔγνωσαν}, che altrimenti sarebbe mera ripetizione.
\item[11 \textgreek{δὲ εἰς Μαγνεσίαν ἀντεπεστράτευον}] Bücheler non avendo inserito \textgreek{\ladd{οὐ}} precedentemente si è trovato in difficoltà con la ripetizione del verbo, che di conseguenza sostituisce con la meta d'arrivo.
\item[11 \textgreek{ἐκτραπεισῶν}]  Bücheler e Jacoby propongono due verbi che significhino la dipartita delle 50 navi, ma i due participi restano problematici da riferire entrambi alle 50 navi ateniesi. Si potrebbe invece, come paiono suggerire la maggior parte degli editori (che lasciano la croce e il testo del codice), correggere \textgreek{προσπλεουσῶν} in un modo finito (\textgreek{προσέπλευσεν}), considerando \textgreek{ἐκτραπεισῶν} participio aoristo parte del genitivo assoluto con \textgreek{νεῶν}. Il soggetto anche di questo verbo risulterebbero \textgreek{οἱ περὶ  τὸν Μεγάβυξον}. Credo che, come nel caso del \textgreek{\ladd{κατὰ τὴν αὐτὴν ἡμέραν}} anche qui, i due participi si riferiscono sempre alle navi, legati dal \textgreek{δέ}. Al primo si può attribuire un valore concessivo e al secondo uno temporale.
\item[14 \textgreek{καὶ \ladd{Ἀθηναῖοι}}] Sembra difficile credere che il soggetto siano di nuovo i Lacedemoni. L'integrazione con un sostantivo almeno per il secondo \textgreek{ἀφελόμενοι} è necessaria. Ho scelto la proposta di Bücheler, che mi sembra il miglior compromesso nella necessità di integrare. \textgreek{ὕστερον} aggiunto prima e insieme ad  \textgreek{Ἀθηναῖοι} da Müller completa ancora meglio e secondo l'uso del testo.
\item[15.4 \textgreek{Περικλέους καὶ Σοφοκλέους}] Il \textgreek{Θεμιστοκλέους} del codice è probabilmente frutto di un lapsus.
\item[16.2-3] Ho lasciato le citazioni da Aristofane\footnote{Le edizioni di riferimento in apparato che ho usato per Aristofane sono quelle edite a Oxford dal prof. Olson. Per Pax 603-611, \cite{Olson1998} e per Acharn. 524-534, \cite{Olson2002}.} secondo la versione di FGrHist 104, piuttosto che riportandole all'Aristofane delle edizioni. La metrica è irrilevante per editare Aristodemo che non se ne preoccupa. Volendo restituire qui FGrHist 104 risultano invece interessanti le scelte conformi al resto del testo. La poca sicurezza, rispetto al tradito ma ormai desueto termine \textgreek{λιπερνῆτες} per il nostro copista, lo porta ad una confusione con \textgreek{πένητες} probabilmente. Oltre a perdere atticismi come \textgreek{ξυνίετε} sceglie \textgreek{ῥημάτια} e il verbo all'ottativo in Pax 603-4, rendendo la frase decisamente esplicita e semplificando la struttura. FGrHist 104 “risolve” la croce del verso 605 del testo aristofaneo (\textgreek{Πρῶτα μὲν γὰρ \crux αὐτῆς ἦρξε \crux  Φειδίας πράξας κακῶς}) usando \textgreek{πρῶτον} (come in 2.2, 2.5, 4.2, 16.1) in senso avverbiale e aggiustando il verbo per Fidia, \textgreek{ἤρξατ’αὐτῆς Φειδίας πράξας}, nel senso di “Fidia diede inizio ai fatti della città”. Nel verso 607 invece \textgreek{αὐτοδὰξ} è banalizzato in \textgreek{αὐθάδη} e il verso 608 (\textgreek{πρὶν παθεῖν τι δεινὸν αὐτός , ἐξέφλεξε τὴν πόλιν}) è omesso, forse per ridondanza con l'opinione fatta propria nell'introdurre il testo (\textgreek{εὐλαβηθεὶς ὁ Περικλῆς μὴ καὶ αὐτὸς εὐθύνας ἀπαιτηθῇ}). Anche il passo di Acarnesi (524-534) risente degli stessi meccanismi di citazione in prosa. Al verso 528 \textgreek{κἀντεῦθεν ἁρχὴ τοῦ πολέμου κατερράγη} diventa \textgreek{ἐνθένδ’ὁ πόλεμος ἐμφανῶς κατερράγη}, con \textgreek{ἐνθένδ’} a far la serie con il successivo \textgreek{ἐνθένδε μέντοι} e l'interessante aggiunta di \textgreek{ἐμφανῶς}. 
\item[18 \textgreek{ἐξεπολιόρκησαν \ladd{τὴν Ποτίδαιαν}}] Inserire l'accusativo è indispensabile per il verbo reggente della frase, ma anche in questo caso, in realtà si potrebbe lasciare come è.
\end{description}

\clearpage
La divisione in paragrafi e sezioni è quella di Jacoby.\footnote{Sebbene si trovino riferimenti diversi ad FGrHist 104, non di rado i paragrafi del manoscritto sono trattati come frammenti separati.} I segni diacritici adottati sia per il testo che per l'apparato sono quelli utilizzati nei FGrHist.
\begin{description}
%ATTENZIONE! INVERTITI I SEGNI! FARE DI CONSEGUENZA NEL TESTO!
\item[\Ladd{ }] testo superfluo espunto
\item[\ladd{ }] testo inserito dall'editore
\item[\lladd{ }] testo cancellato nel manoscritto
\end{description}





