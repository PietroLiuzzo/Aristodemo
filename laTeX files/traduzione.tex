\subsubsection*{1}... chiesto almeno un altro giorno, spedì di nascosto Sicinno, il suo pedagogo, da Serse consigliandogli di attaccare i Greci e di combattere per mare, rivelandogli l'imminente fuga da Salamina. Serse, credendo che Temistocle avesse mandato a dire queste cose poiché medizzava, inviò le navi verso Salamina e  fece circondare i Greci affinché restassero. (2) Si affrettò poi Serse a costruire un ponte per far passare l'esercito a Salamina così come aveva fatto per attraversare l'Ellesponto e assieme ad alcuni raggiunse l'Eracleion. Ma poiché era del tutto impossibile costruire il ponte, sedutosi sul Parnete, lì vicino, guardava la battaglia navale. (3) Diede inizio alla battaglia navale Amenia l'Ateniese, figlio di Euforione, fratello di Cinegiro e di Eschilo il tragediografo.  Vinsero tutti i Greci, ma si distinsero gli Ateniesi. (4) Iniziata la battaglia Serse fece sbarcare alcune decine di migliaia di uomini su di una vicina isoletta nei pressi di Salamina, chiamata Psittalea, per intimorire i Greci e volendo recuperare i relitti dei barbari che vi fossero trascinati. Aristide figlio di Lisimaco, detto il giusto, ostracizzato dagli Ateniesi e quella volta a capo degli Egineti, essendo alleato dei Greci, andò da Temistocle e gli chiese un esercito per soccorrere quelli di Psittalea. Pur essendo stato un suo nemico, ugualmente glielo diede. Ottenuto ciò, Aristide andò a Psittalea e uccise tutti i barbari che c'erano, e questa è la più grande impresa a difesa dei Greci che si ricordi. (5) Si distinse più di tutti e combatté nel modo migliore Amenia, mentre tra i barbari una donna di Alicarnasso, chiamata Artemisia, la quale, visto che la sua nave era inseguita e correva il pericolo di essere distrutta, affondò la nave davanti a sé, che pure era della sua parte. Amenia, pensando allora che si trattasse di un alleato dei Greci, abbandonò l'inseguimento. Serse visto l'accaduto disse: "Gli uomini mi son diventati donne e le donne uomini". (6)  I migliori a farsi valere tra i Greci, dopo gli Ateniesi, furono gli Egineti, i quali dispostisi presso lo stretto del passaggio affondavano mano a mano che arrivavano molte navi di barbari che scappavano verso quello stretto. (7) Poiché i barbari sconfitti fuggivano, i Greci volevano sciogliere il ponte all'Ellesponto per catturare Serse in Grecia, ma Temistocle,  non credendo che questo fosse sicuro, temendo che, se i barbari si fossero visti tolta la speranza della salvezza, tornando indietro avrebbero combattuto senza paura, si oppose. Non prevalendo alcuna decisione, mandò a Serse di nascosto un inviato che gli rendesse noto che i Greci volevano slegare il ponte. Serse, spaventato, fuggì. (8) Nella battaglia di Salamina combatterono anche gli Dei accanto ai Greci.  Diceo figlio di Teocide, Ateniese, disse di aver visto, nella pianura di Triasia un polverone come di 20.000 uomini che giungevano da Eleusi cantando il mistico Iacco. Il polverone annuvolatosi si posò sulle navi dei Greci. 

\subsubsection*{2} Fuggito Serse, Mardonio, figlio di quel Gobria che aveva combattuto i Magi, lo stesso che aveva convinto Serse a combattere contro la Grecia, addusse come causa della sconfitta avvenuta la gran moltitudine di Barbari e promise di vincere i Greci, se gli fosse stato dato un esercito di 300.000 uomini. (2) Ottenuto ciò Mardonio inviò per prima cosa agli Ateniesi, Alessandro il Macedone, l'antenato di Filippo, promettendo che avrebbe dato loro 10.000 talenti e la terra che avessero voluto prendere della Grecia e sostenendo la loro libertà e autonomia, se avessero preferito starsene tra di loro e non allearsi con i Greci. Quando Alessandro giunse ad Atene e dichiarò queste cose gli Ateniesi non accolsero i suoi discorsi e mandarono via Alessandro con supponenza. (3) Avendo fallito in ciò, Mardonio mosse verso Atene e diede alle fiamme anche la parte che era rimasta, giunto poi a Tebe assieme all'esercito vi dispose l'accampamento, mentre i Greci si disposero a Platea. Tra Tebe e Platea vi sono 80 stadi. Mardonio cooptò anche 40.000 Beoti. (4) Avevano la colonna destra i Persiani di Mardonio, la sinistra i Greci medizzanti. Tra i Greci gli Ateniesi avevano la destra, la sinistra gli Spartani. Gli Spartani cambiarono però posto, dicendo che gli Ateniesi erano più esperti nel combattere coi Persiani. In quel momento, Mardonio temendo di combattere con gli Ateniesi invertì lo schieramento e così capitò agli Spartani di combattere coi Persiani per di più involontariamente. Comandava gli Spartani Pausania figlio di Cleombroto, gli Ateniesi Aristide il giusto. Cominciato lo scontro, gli Ateniesi aiutarono gli Spartani e vinsero. (5) Durante lo scontro Mardonio cadde combattendo a testa scoperta, colpito da Aeimnesto, uomo spartano. Si distinse in quest'occasione Aristodemo il sopravvissuto delle Termopili e per questo chiamato il fuggitivo. Per questo motivo gli Spartani non attribuirono a lui il premio per il valore, ritenendo che il primo fatto, circa l'aver abbandonato lo schieramento, fosse questione di decisione, quest'ultimo, circa la sua virtù, fosse invece questione di fortuna.

\subsubsection*{3} Dopo che Mardonio cadde, i Persiani fuggirono verso Tebe, i Greci allora inseguendoli ne uccisero 120.000. 60.000 che stavano scappando verso casa invece, Alessandro il Macedone, lo stesso che aveva fatto da ambasciatore presso gli Ateniesi quando lo aveva inviato Mardonio, li uccise tutti quanti quando giunsero in Macedonia, giustificandosi dicendo che aveva medizzato sotto costrizione. (2) Anche i Greci sulle navi inseguirono la flotta di Serse: navigando 4000 stadi da Salamina verso Mileto raggiunsero le navi dei Persiani ed erano pronti alla battaglia navale. I Persiani non confidando sulle navi poiché avevano provato la perizia degli Ateniesi, sbarcarono e si accamparono presso Micale, che è il monte di Mileto. Sbarcati pure i Greci si scontrarono con loro e ne uccisero 40.000, e presero le navi lasciate vuote: avveniva la battaglia nelle piane di Platea mentre vincevano i Greci presso Micale. (3) Era comandante dei Lacedemoni a Micale il Re Leotichida, degli Ateniesi Santippo, figlio di Arifrone, padre di Pericle. (4) I Greci di Platea dopo la vittoria dedicarono dei trofei, e indissero una festa chiamandola Eleutheria, i Tebani invece, avendo giurato, pagarono la decima.

\subsubsection*{4} (1) Dopo le guerre persiane e prima della guerra del Peloponneso tra i Greci furono compiute queste cose. 
(2) Dopo che i Greci ebbero scacciato i Persiani fuori dall'Europa, e che i Persiani superstiti si furono rifugiati  a Sesto, gli Ateniesi erano rimasti, vogliosi di combattere e anche Pausania, figlio di Cleombroto, stratego dei Lacedemoni, per la bramosia di potere sui Greci e insieme per tradimento: era d'accordo con Serse di tradire i Greci prendendo in cambio una sua figlia in moglie. Come esaltato da questa speranza e dalla buona sorte ottenuta a Platea non controllava più i suoi sentimenti ma per prima cosa pose a Delfi un tripode ad Apollo con sopra scritto questo Epigramma: Pausania, comandante dei Greci, dopo aver distrutto l'esercito Medo, ha dedicato questo monumento a Febo. (3) Guidò coloro che gli erano sottoposti con violenza e tiranneggiando, dimentico dei modi Laconici, ormai volto a portare i costumi dei Persiani e a farsi preparare sontuosi banchetti, come loro abitudine.

\subsubsection*{5} In quel tempo gli Ateniesi, la cui città era stata data alle fiamme da Serse e da Mardonio, volevano munirla di mura, ma i Lacedemoni non lo permisero loro, portando come scusa che Atene sarebbe stata una roccaforte per i Barbari che fossero giunti navigando, ma in verità essendo invidiosi e non volendo che si innalzassero nuovamente; distinguendosi per perspicacia Temistocle li vinse con l'inganno, consapevole della loro invidia.  (2) Dopo aver incitato gli Ateniesi a circondare di mura la città, partì come ambasciatore verso Lacedemone, e quando gli Spartani vennero a sapere che gli Ateniesi circondavano di mura la città, Temistocle negò. Ma visto che i Lacedemoni non si fidavano, li convinse a mandare dei loro ambasciatori ad Atene per sapere se la città fosse in ricostruzione. Mentre i Lacedemoni sceglievano gli uomini e li inviavano, Temistocle mandò un messaggio segreto agli Ateniesi di trattenere presso di loro gli uomini mandati dai Lacedemoni, fino a che non fosse tornato ad Atene. (3) Avendo gli Ateniesi fatto ciò, i Lacedemoni capirono l'inganno di Temistocle, non gli fecero nulla di male, temendo per i loro ambasciatori, ma rimandatolo indietro, si ripresero i loro. (4) Nel frattempo gli Ateniesi avevano circondato con mura la città in siffatto modo. Costruirono un muro di 60 stadi attorno alla cittadella,  e anche un grande muro che portava al Pireo per entrambe le parti lungo 40 stadi, mentre quello intorno al Pireo è di 80 stadi e il porto era diviso in due, ed era chiamata una parte Munichia, mentre il promontorio a destra del Pireo è chiamato anche ora Divino: c'è infatti un'altura al Pireo, sulla quale è collocato il santuario di Artemide. Il muro al Falero fu costruito di 30 stadi, largo tanto quanto bastava per due carri che si incrociassero. In questo modo la città di Atene venne fortificata. 

\subsubsection*{6} Temistocle, invidiato per la sua superiore perspicacia e per il suo valore fu cacciato dagli Ateniesi e giunse ad Argo. (2) I Lacedemoni intanto, sentite le accuse riguardo all'imminente tradimento di Pausania, gli mandarono la scitale richiamandolo perché si spiegasse. (3) Pausania giunto a Sparta si difese, e ingannati i Lacedemoni, prosciolto dalle accuse, si allontanò e nuovamente tornò a mettere in atto il tradimento.

\subsubsection*{7} Nel frattempo i Greci si separarono dai Lacedemoni per il violento tiranneggiare subìto da Pausania, e si accostarono agli Ateniesi, così che gli Ateniesi ricominciarono, raccogliendo il tributo, ad esserne accresciuti in potere: preparavano navi e i beni comuni erano protetti a Delo, poi... i talenti raccolti a Delo furono trasportati ad Atene e riposti sull'acropoli. 

\subsubsection*{8} Pausania ora, come governatore di Bisanzio medizzava apertamente e trattava male i Greci. Fece anche questo. C'era la figlia di un indigeno, Coronide, di nome ... , che Pausania mandò a richiedere dal padre: Coronide timoroso della ferocia di Pausania gli mandò la figlia. Giunta essa di notte alla casa, mentre Pausania dormiva, e avvicinatasi, Pausania si svegliò e pensando ad un tentativo di ribellione imprecando trafisse con il pugnale la ragazza e la uccise. Per questo motivo divenne matto e delirando gracchiava, tormentato dalla ragazza; trascorso molto tempo placò i demoni della fanciulla e rinsavì. (2) Non cessò di tradire, ma scrisse lettere a Serse che affidò ad Argilio, suo amato, chiedendogli di portarle a Serse. Argilio temendo per se stesso, poiché coloro che erano stati inviati prima non erano tornati, non andò da Serse ma a Sparta dove denunciò il tradimento e si impegnò a cogliere Pausania sul fatto. Presi accordi circa queste cose andò sul Tanaro, nel santuario di Poseidone come supplice. Gli efori giunti anche essi allo stesso santuario prepararono una doppia parete e in quella si nascosero. (3) Non sapendo ciò Pausania, sentito che Argilio era supplice, lo raggiunse e lo rimproverò per non aver portato le lettere a Serse, peraltro fornendo ulteriori prove del tradimento. Gli efori sentendo i discorsi sul momento non lo catturarono, poiché il luogo era sacro, ma lo lasciarono andare, e decisero di catturarlo poi una volta arrivato a Sparta; egli, sospettando, corse al santuario di Atena Calcieca e si fece supplice. (4) Trovandosi i Lacedemoni in una situazione imbarazzante rispetto all'onorare la divinità, la madre di Pausania, sollevata una pietra, la pose all'uscita del recinto sacro, dando inizio alla punizione del figlio: i Lacedemoni seguendola racchiusero il recinto sacro, e Pausania si consumò con la fame, ma tolsero il tetto, e tirarono fuori Pausania che ancora respirava e lo gettarono giù. (5) Per questo avvenne una carestia. Interrogata la divinità su come liberarsi dei demoni di Pausania, per interrompere la carestia, dedicarono delle statue e fermarono la carestia. 

\subsubsection*{9} Interrogandosi poi i Greci su chi bisognasse scrivere per primo di coloro che avevano combattuto come alleati nella guerra contro i Medi, i Lacedemoni tirarono fuori l'idea del disco, sul quale a cerchio sarebbero state scritte le città che avevano combattuto, così che nessuno sarebbe stato scritto per primo e nessuno per ultimo.

\subsubsection*{10} I Lacedemoni, dopo che si furono occupati in modo riprovevole dei fatti di Pausania, convinsero gli Ateniesi, dicendo di aver trovato nelle lettere di Pausania la complicità nel tradimento da parte di Temistocle. Temistocle, temendo i Lacedemoni non restò ad Argo, ma si mosse verso Corcira e poi verso i Molossi, presso Admeto che era Re e già da prima gli era nemico. (2) Essendo giunti dei Lacedemoni presso Admeto e chiedendo di lui, la moglie di Admeto consigliò a Temistocle di rapire il figlio del Re e di rifugiarsi al focolare domestico come supplice. Avendo Temistocle fatto ciò, Admeto ebbe compassione di lui e non gli fece del male ma rispose ai Peloponnesiaci che non era corretto consegnare un supplice. (3) Temistocle, non avendo dove andare, navigò verso la Persia. Mentre navigava corse il pericolo di essere catturato e portato indietro. Infatti mentre gli Ateniesi stavano combattendo a Nasso, la nave di Temistocle colpita da una tempesta era stata portata a Nasso. Temistocle, temendo di essere catturato dagli Ateniesi minacciò il comandante di ucciderlo, se non avesse opposto resistenza ai venti. Il comandante, temendo la minaccia si diede da fare contro l'ondeggiamento per tutta la notte e il giorno successivo e procedette contro vento: così Temistocle giunse salvo in Persia. (4) Ma non trovò Serse in vita, bensì il figlio Artaserse; non gli si presentò subito ma, attendendo un anno, apprese la lingua Persiana, poi si presentò da Artaserse, e gli ricordò i suoi meriti, che presentò come compiuti per il padre di lui, Serse, dicendo anche di essere stato responsabile della sua salvezza, rivelando che i Greci andavano a sciogliere i ponti. Si impegnò anche, se avesse ottenuto un esercito da lui, a sottomettere i Greci. (5) Artaserse appoggiando le sue richieste gli diede un esercito e tre città in incarico, Magnesia per il grano, Lampsaco per il vino, Miunte per la carne.  Ottenuto ciò Temistocle, giunto a Magnesia essendo vicino alla Grecia mutò pensiero, non ritenendo giusto che bisognasse darsi guerra tra consanguinei. Sacrificando ad Artemide Leucofrone, sgozzando un toro mise sotto una coppa e riempitala di sangue, bevve e morì. 

\subsubsection*{11} I Greci <non> sapendo di tutto ciò volevano scacciare l'esercito insieme a Temistocle, e quando arrivarono e appresero tutto, portarono subito guerra ad Artaserse così che gli Ateniesi liberarono le città della Ionia e le restanti città greche. (2) Cimone figlio di Milziade essendo stratego, navigò verso la Pamfilia presso il fiume detto Eurimedonte e diede battaglia navale a Fenici e Persiani compiendo meravigliose gesta: 100 navi con tutti gli uomini furono distrutte quando combatterono a terra e due trofei furono innalzati, uno per la [vittoria] di terra e uno per quella sul mare. (3) Navigarono poi verso Cipro e l'Egitto. Regnava in Egitto Inaro, figlio di Psammetico, il quale, ribellatosi ad Artaserse, si era procurato in aiuto gli Ateniesi, i quali avevano 200 navi e combatterono per sei anni contro i Persiani. (4)  Poi però Megabizo figlio di Zopiro, inviato da Artaserse, mentre gli Ateniesi erano ormeggiati in un'isola chiamata Prosopitide, lungo il fiume, fece deviare la corrente del fiume e le navi restarono a secco. Le 50 navi attiche che pure erano state deviate mentre navigavano verso l'Egitto, gli uomini di Megabizo le intercettarono e alcune le distrussero, altre le trattennero. Degli uomini i più furono annientati, pochi in tutto invece tornarono a casa.

\subsubsection*{12} Dopo ciò vi fu la guerra Greca tra Ateniesi e Lacedemoni a Tanagra. I Lacedemoni erano 13.000 di numero, gli Ateniesi 16.000: vinsero gli Ateniesi. (2)  Schieratisi di nuovo a Oenofita, con gli strateghi Tolmide e Mironide, vinsero sui Beoti e sottomisero la Beozia. 

\subsubsection*{13} Subito portarono guerra verso Cipro, con lo stratega Cimone figlio di Milziade. In questa furono presi da una carestia, e Cimone ammalatosi a Chition di Cipro, lì morì. I Persiani vedendo gli Ateniesi messi male, pensarono di raggiungere le loro navi: ne venne uno scontro in mare, nel quale vinsero gli Ateniesi. (2) Fu scelto come stratego Callia, detto Laccoplutos, poiché avendo trovato un tesoro a Maratona, impossessatosene si era arricchito. Questo Callia fece un trattato con Artaserse e i rimanenti Persiani. Vennero dunque presi questi accordi, secondo cui i Persiani non potevano navigare con grandi navi oltre Kuaneon e il fiume Nesso e Faselis, che è una città della Pamfilia, e Calidoneo, nè potevano percorrere la strada verso l'interno che un cavallo correndo potesse effettuare in tre giorni senza sosta. Questi dunque gli accordi che vennero presi.  

\subsubsection*{14} Dopo di ciò ci fu una guerra greca, per questo motivo: i Lacedemoni tolsero ai Focesi il santuario di Delfi affidandolo ai Locresi, e gli Ateniesi tolto a questi lo ridiedero ai Focesi. (2)  Mentre gli Ateniesi rientravano dalla battaglia, con Tolmide stratego, giunti presso Coronea, i Beoti attaccandoli di sorpresa mentre non erano preparati li volsero in fuga e fecero alcuni prigionieri, che pur essendo stati richiesti dagli Ateniesi non furono restituiti prima che la Beozia fosse ripresa. 

\subsubsection*{15} Subito dopo ciò gli Ateniesi fecero il giro del Peloponneso e distrussero Gition. Tolmide, avendo 1.000 Ateniesi scelti percorse il Peloponneso. (2) Gli Ateniesi distrussero L'Eubea che si era nuovamente ribellata. (3) Nel frattempo, si fecero i patti di 30 anni tra i Greci. (4) Dopo 14 anni di assedio gli Ateniesi presero Samo, con gli strateghi Pericle e Sofocle. Nello stesso anno così si sciolsero i patti di 30 anni e la guerra del Peloponneso era alle porte.

\subsubsection*{16} Si possono portare molte cause per questa guerra. La prima riguarda Pericle. Si dice infatti che mentre gli Ateniesi costruivano l'Atena Crisoelefantina incaricarono Pericle sovrintendente, Fidia ingegnere. Preso Fidia con accusa di furto, Pericle prese precauzioni per non essere richiesto di rendiconti, e volendo far deviare l'opinione pubblica, rese questa guerra politica, scrivendo la legge contro i Megaresi. (2) Tali cose crede anche la commedia antica, visto che il poeta dice così: o sciocchi contadini, radunatevi qui insieme, se volete sentire qualche storiella su com'è stata distrutta questa [città]. In primo luogo Fidia ha dato inizio ai fatti malamente, poi Pericle temendo che mutasse la sorte, preoccupato per la vostra natura e i vostri modi testardi, lanciando una piccola scintilla con il decreto di Megara ha fatto nascere tutta questa guerra che per il fumo fa piangere tutta la Grecia sia di là che di qua. E di nuovo poco oltre: alcuni ragazzi rapiscono una prostituta che andava ad ubriacarsi a Megara; allora quelli di Megara furibondi per l'affronto rubano a loro volta due prostitute di Aspasia. Ed ecco che la guerra tra Greci è scoppiata apertamente per via di tre prostitute. Infatti proprio allora Pericle l'Olimpio lanciava fulmini, tuonava, metteva sottosopra la Grecia, stabiliva leggi come se stesse scrivendo canzonette su come bisognava che Megara non restasse né nei mercati né sul territorio.
(4) Si dice anche che mentre Pericle si interrogava riguardo a come spiegare la rendicontazione della sua sovrintendenza, Alcibiade figlio di Clinia, essendo sotto il suo tutorato disse: ''non cercare come si possa rendere conto agli Ateniesi, ma come non rendere conto”.

\subsubsection*{17} La seconda causa è quella di Corcira ed Epidamno. Epidamno era una città fondata dai Corciresi, mentre Corcira dai Corinzi. Comportandosi male fra di loro in quel momento, gli Epidamni trattati con arroganza dai Corciresi si fecero alleati i Corinzi, come madrepatria, combatterono contro Corcira e vi fu una guerra. (2) I Corciresi tormentati dalla guerra spedirono ambasciatori ad Atene per un'alleanza, poiché essi avevano molte navi. Allo stesso modo i Corinzi mandarono ambasciatori ad Atene chiedendo che essi non aiutassero i Corciresi. Gli Ateniesi credettero meglio aiutare i Corciresi: così combatterono in mare i Corinzi, pur essendo in vigore dei patti. In questo modo i patti erano sciolti.

\subsubsection*{18} Come terza causa si porta questa. Potidea era una città fondata dai Corinzi in Tracia. Gli Ateniesi le mandarono ambasciatori volendo prenderla con sé. I Potideati si associarono ai Corinzi e da questo ebbe origine una battaglia tra Ateniesi e Corinzi e gli Ateniesi assediarono Potidea.

\subsubsection*{19} La quarta causa di cui si parla è la più vera. I Lacedemoni vedendo innalzarsi gli Ateniesi e per navi e per ricchezze e per alleati ***