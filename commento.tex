
    \subsubsection{`Θηβαίους τε, καθὼς ὤμοσαν, ἐδεκάτευσαν`}
    Il giuramento e l'applicazione della decima sono riportati separatamente da Polibio, in una formulazione che tenta di rivalutare parzialmente il ruolo dei Tebani: `οἵ γε Θηβαίους τοὺς κατ’ἀνάγκην ἡσυχίαν ἄγειν βουλευσαμένους μόνους τῶν Ἑλλήνων κατὰ τὴν τῶν Περσῶν ἔφοδον ἐψηφίσαντο δεκατεύσειν τοῖς θεοῖς κρατήσαντες τῷ πολέμῳ τῶν βαρβάρων.` [quelli che avevano vinto nella guerra contro i Persiani votarono che i Tebani che, soli, avevano condotto il passaggio dei Persiani per essere lasciati in pace, pagassero una decima agli Dei] (Polyb 9.39.5). Erodoto ci ricorda un giuramento simile nel settimo libro, prima della battaglia delle Termopili: `οἱ Ἕλληνες ἔταμον ὅρκιον οἱ τῷ βαρβάρῳ πόλεμον ἀειρόμενοι· τὸ δὲ ὅρκιον ὧδε εἶχε· ὅσοι τῷ Πέρσῃ ἔδοσαν σφέας αὐτοὺς Ἕλληνες ἐόντες, μὴ ἀναγκασθέντες, καταστάντων σφι εὖ τῶν πρηγμάτων, τούτους δεκατεῦσαι τῷ ἐν Δελφοῖσι θεῷ` [i Greci che intrapresero la guerra contro il Persiano stabilirono un giuramento. Il giuramento era questo: quanti dei Greci si fossero dati ai Persiani, senza esservi costretti, se fossero andate bene le cose per loro, costoro avrebbero pagato una decima agli dei] ([7.132](http://data.perseus.org/citations/urn:cts:greekLit:tlg0016.tlg001.perseus-grc1:7.132)). L'`ἀνάγκην` lega questi due testi lasciando poco margine di dubbio rispetto al riferimento di Polibio  che pare essere direttamente ad Erodoto. I Tebani saranno infatti oggetto in Erodoto di un episodio di alcuni capitoli in cui i Greci attaccano la città per chiedere risarcimento ([9.86](http://data.perseus.org/citations/urn:cts:greekLit:tlg0016.tlg001.perseus-grc1:9.86)-88).  Evidentemente in Polibio i due fatti sono associati, uno come conseguenza dell'altro, ma il giuramento è ricordato come precedente l'invasione e non si era giurato di colpire Tebe, restando sul generico, non essendo ancora avvenuto lo schieramento dei Beoti con Mardonio e Serse. In  Diodoro troviamo un formulario più diretto che, al posto dell'`ἀνάγκην` e dell'eccezione, mette in primo piano la scelta compiuta: `ἐψηφίσαντο τοὺς ἐθελοντὶ τῶν Ἑλλήνων ἑλομένους τὰ Περσῶν δεκατεῦσαι τοῖς θεοῖς, ἐπὰν τῷ πολέμῳ κρατήσωσι.` [Votarono che avrebbero dedicato una decima agli dei dai Greci che avevano scelto la parte dei Persiani, se avessero vinto la guerra.] (11.3.3) Alla decima che poi, secondo questo giuramento, i Tebani dovettero pagare,  Diodoro fa probabilmente riferimento invece quando dice, che: `Θηβαῖοι ταπεινωμένοι διὰ τὴν πρὸς Ξέρξην γενομένην συμμαχίαν` [I Tebani essendo stati umiliati per l'alleanza intrattenuta con Serse] (11.81.1). Come in Erodoto quindi, nel racconto di  Diodoro questa decisione viene presa prima della battaglia delle Termopili, mentre, prima di Platea (11.29) è sottoscritto un altro giuramento formale che include un formulario molto simile a quello ricordato da Licurgo  come proprio della città di Atene e  precedente Platea, dove la decima da prendere dai medizzanti è inclusa nel giuramento stesso. Osserviamo in questo testo la fusione in una delle due diverse decisioni, una distorsione comprensibile, soprattutto nel contesto del discorso retorico ateniese nel IV secolo: `οὐ ποιήσομαι περὶ πλείονος τὸ ζῆν τῆς ἐλευθερίας , οὐδ’ἐγκαταλείψω τοὺς ἡγεμόνας οὔτε ζῶντας οὔτε ἀποθανόντας, ἀλλὰ τοὺς ἐν τῇ μάχῃ τελευτήσαντας τῶν συμμάχων ἅπαντας θάψω. καὶ κρατήσας τῷ πολέμῳ τοὺς βαρβάρους, τῶν μὲν μαχεσαμένων ὑπὲρ τῆς Ἑλλάδος πόλεων οὑδεμίαν ἀνάστατον ποιήσω, τὰς δὲ τὰ τοῦ βαρβάρου προελομένας ἁπάσας δεκατεύσω. καὶ τῶν ἱερῶν τῶν ἐμπρησθέντων καὶ καταβληθέντων ὑπὸ τῶν βαρβάρων οὐδὲν ἀνοικοδομήσω παντάπασιν, ἀλλ’ὑπόμνημα τοῖς ἐπιγιγνομένοις ἐάσω καταλείπεσθαι τῆς τῶν βαρβάρων ἀσεβείας` [Non considererò più la vita della libertà, non abbandonerò i capi né vivi né morenti, ma seppellirò tutti quelli tra gli alleati che dovessero cadere in battaglia. E se avrò la meglio nella guerra contro i Persiani, non distruggerò alcuna delle città che hanno combattuto per la Grecia, ma prenderò una decima da tutte quelle che hanno scelto la parte dei Persiani. E non ricostruirò proprio nessuno dei templi distrutti e bruciati dai Persiani, ma lascerò che restino come memoriale dell'empietà dei Persiani per le generazioni a venire] (Licurgo 81 ). La fonte di  Diodoro è molto vicina alla formulazione  di Licurgo, sia per il lessico della condizione di vittoria, sia per la scelta diretta. C'è un'altra importante fonte che lega il giuramento esplicitamente a Tebe, la famosa stele di Acarne, testo datato alla metà del IV secolo. Rhodes 2004, n 88. Si veda anche Pownall 2011, _ad loc._ e  Krentz 2007, 731s. Non riporto tutto il testo, ma, tra le altre cose, i due passi seguenti sono di grande interesse per arrivare alla sintetica formulazione del nostro: `καὶ ν/ικήσας μαχόμενος τοὺς βαρβάρους δεκατεύσω / τὴν Θηβαίων πόλιν` [e se vincerò combattendo i Barbari farò  pagare alla città dei Tebani una decima]  (ll.31s.) e poco oltre: `καὶ εἰ μὲν ἐμπεδορκοίην τὰ ἐν τῶι ὅ/ρκῶι γεγραμμένα, ἡ πόλις ἡμὴ ἄνοσος εἴ/η, εἰ δὲ μή, νοσοίη·` [e se sarò fedele alle cose scritte nel giuramento, la mia città sia sana, se no, appestata] (ll.39s.). L'iscrizione prevede un giuramento direttamente contro i Tebani e vi troviamo `ἄνοσος` detto della città, che ricordiamo in Tucidide 2.49, e in Suda (`νοσησάσης δὲ τῆς πόλεως`) nel passo in cui si tramandano le conseguenze della morte di Pausania  che in FGrHist 104 è descritta come `λοιμός` (8.5). Il primo dato è quello che qui maggiormente importa. Il testo dell'iscrizione è infatti stato inizialmente riconosciuto come precedente o appena successivo alla battaglia di Platea, ma con alcuni punti critici, che hanno portato Krentz a proporre che esso si riferisse alla battaglia di Maratona. Nessun dubbio sulla datazione del testo, sicuramente della metà del IV secolo. È probabilmente questo il tipo di documento criticato da Teopompo  secondo Teone (Progym. 2 II 67, 22 = FGrHist 115 F 153): `παρὰ δὲ Θεοπόμπου ἐκ τῆς πέμπτης καὶ εἰκοστῆς τῶν Φιλιππικῶν, ὅτι ὁ Ἑλληνικὸς ὅρκος καταψεύδεται, ὃν Ἀθηναῖοί φασιν ὀμόσαι τοὺς Ἕλληνας πρὸ τῆς μάχης τῆς ἐν Πλαταιαῖς πρὸς τοὺς βαρβάρους` [nel venticinquesimo libro dei _Philippika_, Teopompo dice che il giuramento dei Greci, che gli Ateniesi dicono essere stato giurato prima della battaglia di Platea contro i Persiani, è falso.] Vedi Shrimpton 1991, 9 80 e n.25. Il testo si riferisce all'iscrizione, che al suo tempo poteva essere stata re-iscritta con i caratteri ionici scelti da Archino (Teopompo FGrHist 115 F 155) e quindi era un ''falso'' l'iscrizione con il giuramento in termini di riproduzione, ma forse anche il giuramento in sé che così direttamente rivolgeva il suo intento di vendetta a Tebe, mescolando tra l'altro due momenti della storia. Non credo ci sia motivo di credere, con Jacoby, che FGrHist 104 segua volontariamente una tradizione respinta da Teopompo, anche perché non sappiamo a quale momento faccia riferimento ` καθὼς ὤμοσαν`. La collocazione è coerente con il racconto di Erodoto ma il giuramento di riferimento parrebbe proprio essere ricordato come è tramandato sulla stele, asciugato della complessità del processo storico, e riscritto _post eventum_. Il testo di FGrHist 104 si colloca quindi nella tradizione argomentata da Teopompo  nei _Philippika_, in quel filone di retorica atticista che elaborava il racconto del passato come vediamo nel testo di Licurgo  e nell'iscrizione. Non è questo l'unico esempio, come abbiamo visto per la stele di Trezene e il papiro P. Med. Inv. 71.76, 71.78. 71.79 (cfr. p.). Siewert 1972.
     %%%%%%%%%%%%   
        \subsection*{Pausania traditore}
        %%%%%%%%%
        \subsubsection{\textgreek{Ἀπὸ δὲ τῆς Περσικῆς … ἐπράχθη τάδε}}
        La divisione che il compilatore del codice o la sua fonte operano in questo punto ha un sapore molto tucidideo (1.118), ma non possiamo dire, dati i contenuti divergenti, e spesso più completi, sebbene meno dettagliati rispetto a Tucidide (Cfr. Plutarco, _De Gloria Atheniensium_ 347A), che il nostro autore si rifaccia in ultima analisi all'Ateniese da questo punto in poi. Dobbiamo comunque fare riferimento in generale a Thuc. 1.89-95, 1.128-138 per questa sezione relativa a Pausania, Temistocle e Cilone. L'ordine delle vicende esposte è diverso: in Tucidide troviamo innanzi tutto l'episodio delle mura di Atene, poi il periodo di Pausania  a Bisanzio e la fondazione della Lega e le prime missioni di Cimone: Itome, la spedizione in Egitto, Tanagra, Cheronea e tutti gli altri episodi della cosiddetta pentecontaetia (Meiggs 1972, 444s), che si concludono con la narrazione della morte di Pausania  e della conseguente persecuzione e morte di Temistocle in Asia. FGrHist 104 mostra in generale e nei dettagli, come vedremo, una riorganizzazione ''logica'' della narrazione, ma una sintesi più ridotta rispetto allo stesso Tucidide (per esempio conserva la favola della figlia di Coronide 8.1, e il capitoletto sulla lunghezza delle mura 5.4). Non può trattarsi, quindi di un'elaborazione diretta del solo Tucidide. Nella prima parte, parla di Pausania  a Bisanzio (4) seguendo il discorso lasciato con Leotichida e Santippo; nel capitolo successivo si osservano i movimenti di Temistocle riguardo l'episodio delle mura ad Atene (5), fornendo dettagli sulle misure che Tucidide dà in un contesto diverso (2.13.7). Troviamo Pausania  e gli Ioni, Temistocle e i Peloponnesiaci, con un breve paragrafo che lega i due personaggi nel partire dell'uno e tornare dell'altro (6), e prelude ad un paragrafo dedicato al rompersi dell'alleanza e all'istituzione della lega navale Ateniese, con il probabile riferimento allo spostamento del tesoro da Delo ad Atene (7). Alla collezione degli episodi che preludono e portano alla fine di Pausania  (8) segue quella che parrebbe una digressione o un recupero di un appunto sul trofeo delle guerre persiane (9) estranea a Tucidide ed al resto della tradizione. La conseguenza dei destini di Pausania, come in Tucidide è la fuga di Temistocle in Persia con tutte le sue peripezie (10). 
 %%%%%%%%%%%%       
        \subsubsection{`κατὰ φιλοτιμίαν τὴν ὑπὲρ τῶν Ἑλλήνων, ἅμα διὰ προδοσίαν`}
        FGrHist 104 è indeciso (Hdt. [8.3](http://data.perseus.org/citations/urn:cts:greekLit:tlg0016.tlg001.perseus-grc1:8.3) e Thuc. 1.95), quasi contraddittorio. Si dedica subito a Pausania, ma non prende parte, sorvola sul giudizio e accumula motivazioni e dati, sia personali che politici. Nonostante anche Erodoto nomini per inciso la vicenda ([5.32](http://data.perseus.org/citations/urn:cts:greekLit:tlg0016.tlg001.perseus-grc1:5.32)), è la linea di Tucidide e  Diodoro che FGrHist 104 conserva nella richiesta della mano della figlia di Serse  (Thuc. 1.128.7, Diod. 11.44, _Them. Ep._  14.4). Fortunatamente di questo pezzetto di tradizione possiamo notare il passaggio anche tramite Eforo e Teopompo. Plutarco (_De Herod. Mal._ 5 p. 855 F = Eforo FGrHist 70 F 189) ricorda che Eforo, parlando di Temistocle, criticava l'omissione tucididea favorendo invece la ''versione spartana'' degli inviati ad Atene. Parmeggiani 2011, 402-3. Non ci sono lettere però citate in Tucidide, mentre le abbiamo in FGrHist 104 10.1 su Pausania, come prova del fatto che lo stratego ateniese fosse in effetti a conoscenza dell'accaduto. Eforo, nel passo citato da Plutarco dice proprio `παρακαλοῦντος αὐτὸν ἐπὶ τὰς \Ladd{αὐτὰς` ἐλπίδας}, come troviamo poco oltre in questo passo di FGrHist 104 rispetto a Pausania: `ὃς ἐπηρμένος τε τῇ ἐλπίδι ταύτῃ.` 
            I banchetti di Pausania  e i suoi nuovi costumi sono il simbolo esteriore che fa da segnale evidente nonché da prova per i suoi detrattori contemporanei e posteri. In quanto elemento di ''evidenza'' ne troviamo memoria in tutte le fonti, dalle tavole imbandite di Erodoto ([9.82](http://data.perseus.org/citations/urn:cts:greekLit:tlg0016.tlg001.perseus-grc1:9.82)), a Tucidide (1.95.1, 1.130),  Diodoro (11.44.5; 11.46.1-3),  Nepote (_Paus._ 3), Plutarco (_Cim._ 6; _Arist._ 16) e Suda. La semplificazione dell'episodio Erodoteo è già in parte in atto in Tucidide.
            Pausania  `ἐμετριοπάθει` ricorda molte altre notazioni sullo stato d'animo dei personaggi in FGrHist 104 (1.7, 5, 10.3) che possono essere ricondotti alla descrizione dello stile di Teopompo  di Dionigi di Alicarnasso (D.H. Ad Pomp. 6). 
      %%%%%%%%%      
            \subsection*{Temistocle e le mura}
            \subsubsection{`φθονοῦντες καὶ μὴ βουλόμενοι πάλιν αὐξηθῆναι`}
           Nell'avverbio `πάλιν` c'è una prospettiva diversa, dimentica per un attimo del contesto della narrazione e forse coinvolta in altre ricostruzioni delle mura, come quella di Conone.  L'episodio è celeberrimo e celebrato. La sua posizione all'interno del racconto, speculare all'episodio del tradimento di Pausania, delinea la struttura parallela delle vite dei due illustri personaggi che si svolgerà nei seguenti capitoli. Rassegna delle fonti e analisi si trovano in CulassoGastaldi 1990, 82 e n.15. La centralità del personaggio è quasi biografica, tanto da far sparire gli altri ambasciatori, ma l'unità narrativa con il racconto delle vicende del generale spartano è una caratteristica ancora più pregnante. Quella dei tre sacrilegi di Cilone, Temistocle e Pausania, a partire da Pericle  testimoniata da Tucidide ha avuto ampia fortuna, fino ad oggi. Sull'_auxesis_ ateniese: Parmeggiani 2011, 447.
  %%%%%%%%%%%%%%          
            \subsubsection{\textgreek{οἱ Λακεδαιμόνιοι αἰσθόμενοι … δεδοικότες περὶ τῶν ἰδίων}}
            Il carattere tucidideo di questo appunto, necessario alla narrazione altrimenti spogliata di ogni dettaglio che renda l'inganno effettivamente comprensibile, è evidente: \textgreek{Οἱ δὲ Λακεδαιμόνιοι ἀκούσαντες ὀργὴν μὲν φανερὰν οὐκ ἐποιοῦντο τοῖς Ἀθηναίοις οὐδὲ γὰρ ἐπὶ κωλύμῃ, ἀλλὰ γνώμης παραινέσει δῆθεν τῷ κοινῷ ἐπρεσβεύσαντο […] τῆς μέντοι βουλήσεως ἁμαρτάνοντες ἀδήλως ἤχθοντο.} [Dopo aver ascoltato, gli Spartani non resero nota la loro ira agli Ateniesi, infatti non avevano mandato l'ambasceria alla loro assemblea per impedire ma per dare una raccomandazione rispetto ad un'opinione […] e perciò essendosi sbagliati nell'intento sotto sotto erano arrabbiati.] (Thuc. 1.92.1). Questa è una traccia dell'importanza storica dell'inciso più che del racconto. Il commento tucidideo deve restare, seppur banalizzato e razionalizzato e la `βούλησις` spartana è definitivamente fraintesa e così le sue conseguenze: `οἱ δὲ Λακεδαιμόνιοι οὐκ ἐπέτρεπον αὐτοῖς`. Se anche in questo filtro c'è traccia del secolo successivo,  è da dire che c'è forse più fortuna per Tucidide qui, che in qualsiasi altro passo citato alla lettera. La necessaria spiegazione della posizione spartana viene riletta alla luce della più scontata delle caratteristiche tucididee degli Spartani, il timore dell'ascesa di Atene. Vattuone 2007 in LHG\&L s.v. `Ἀφανής`.
  %%%%%%%%%%%          
            \subsubsection{`ἐτειχίσθησαν αἱ Ἀθῆναι τὸν τρόπον τοῦτον`}
            Ha ragione Pownall quando dice che è il desiderio di completezza che ha preservato in FGrHist 104  queste misure. Dopo il racconto sulla costruzione delle mura temistoclee, FGrHist 104 5.4 dà tutte le misure delle mura di Atene costruite `ἐν δὲ τῷ μεταξὺ χρόνῳ`, lunghe mura comprese, nonostante siano state iniziate da  Cimone e finite da Pericle. Per le cifre il nostro testo si allontana parzialmente da quelle date da Tucidide (2.13.7), oltre a fornire diverse informazioni topografiche. È probabilmente un'altra cinta muraria quella che ha in mente o rispetto alla quale le misure utilizzate sono state calcolate. Conwell (Conwell 2008, 37-54) discute ampiamente le fonti (Thuc. 1.105.1-106.2; Aeschin. 2.172-3; Plut. _Cim._ 13.5-7) relative a questa prima fase della costruzione delle mura e propone come punto di svolta per l'inizio della costruzione gli sviluppi dopo Itome (Conwell 2008, 52). Che i due testi siano in stretta relazione, è confermato dalla misura della larghezza delle mura, che in FGrHist 104 è definita come `πλατὺ δὲ ὥστε δύο ἅρματα ἀλλήλοις συναντᾶν` proprio come in Tucidide, dove troviamo (1.93.5) `δύο γὰρ ἅμαξαι ἐναντίαι ἀλλήλαις τοὺς λίθους ἐπῆγον`. La tradizione sul muro non risale però a Tucidide: si trova già in Aristofane, _Eq._ 814-6\indexp{Scolia ad Aristoph.!_Equites_ 814-6|ca}, dove il riferimento esplicito sembrerebbe essere proprio ai lavori delle mura ai tempi di Temistocle. Montana 2002, 272-3. I resti delle mura di Thuc. 1.89 (`περιβόλου βραχέα εἱστήκει`) potrebbero corrispondere all'`ἀρχαίων` di cui Aristofane  dice che \textgreek{ἀφελών τ’ οὐδέν}. Si parla poi anche di un allargamento del muro con un nuovo tratto (`καινοὺς παρέθηκεν` [ne pose di nuovi]), e di conseguenza la cifra che troviamo in Tucidide parrebbe essere stata presa da una fonte scritta giunta probabilmente fino a Dione Crisostomo (75.4 che da ''più di 90'') e precedente al completamento del nuovo tratto. Totaro 2004, 203 prende in considerazione anche _Pericle_  13.8 sulla lentezza nella costruzione del muro ''_dia mesou_''  confrontando con Plut. _De Gloria Athen_ 351A; Pl. _Gorgia_, 455E (= Cratino fr. 326 K.-A.) "da tempo, a parole Pericle  lo porta avanti; nei fatti, però, tutto è fermo". Il legame delle vicende di Temistocle e Pausania  con Pericle  sembra inscindibile fin proprio dalla commedia. 
            La ripetizione della formula finale, simile a quella che ritroveremo in 13.4 è riconducibile più alla tradizione scoliografica che agli stilemi di Erodoto. 
  %%%%%%%%%          
          \subsection*{Dalla tirannia di Pausania allo spostamento del tesoro}
            In questa sezione il testo mette in parallelo l'espulsione di Temistocle da Atene, probabilmente equivalente all'ostracismo di cui parlano le altre fonti, e il richiamo di Pausania  con la _scitale_. Fatto che lo accosta al secondo richiamo nel racconto di Tucidide (1.131.1), quando Pausania  si trovava per sua volontà e contro l'opinione degli Spartiati in Troade a curare i propri affari. La `σκυτάλη` è descritta nello scolio (_Schol. in Thuc._ 1.131.1). Anche in Nep. _Paus._ 2.6. Kelly 1985, 143s. È strano che Pausania, partito per conto suo  rientri al secondo richiamo tramite questa missiva.
            In Tucidide, l'ostracismo di Temistocle è ricordato per motivare la fuga da Argo al momento in cui gli Ateniesi mandano i Lacedemoni con alcuni dei loro a cercarlo, dopo la morte di Pausania  (Thuc. 1.135.3). Konishi 1970, 55; Ellis 1994, 174; Westlake 1977, 107-109. In  Diodoro (11.55), invece, l'ostracismo è conseguenza di un'accusa lacedemone, generata dall'umiliazione per il tradimento di Pausania  (Arist. _Costituzione degli Ateniesi_ 23.4) ed è associabile alla seconda accusa, che infatti segue immediatamente. In Plutarco (_Them._   22.1), come in FGrHist 104, la causa esterna  dell'ostracismo è l'invidia. Segue anche l'accenno alle grandi gesta, implicate dall'`ὑπερβάλλουσαν σύνεσιν καὶ ἀρετὴν` di FGrHist 104. Sarebbe intrigante ipotizzare che questi eventi lasciati intendere includessero l'operazione condotta con Efialte di cui ci informa la sola _Costituzione degli Ateniesi_ (25.3), testo nel quale mi pare si possa ravvisare un'indicazione cronologica che riporta a molto prima dell'arcontato di Conone  nell'inciso `ἔμελλε δὲ κρίνεσθαι μηδισμοῦ`. L'azione di Temistocle, nota Santoni (Santoni 1999, 188), è come quella di chi ha paura di essere punito (_Pol._ 5.1302b21) e dunque si confà alla fuga verso Argo. Se l'accusa è di medismo, l'unica fonte associabile ad Aristotele è Diodoro. La morte del reggente di Sparta stabilisce il momento, a seguito del quale gli Spartani decidono di pretendere vendetta anche contro Temistocle. In nessun caso comunque l'ostracismo è associato con un momento così preciso e probabilmente collocato più indietro nel tempo. Tucidide forse si riferiva a questo momento, ma non c'è modo di provarlo. Resta invece la consonanza con Plutarco.  Diodoro pone questo evento nel 471/0 e anche per White (White 1964, 146) di tutti gli eventi questo è quello più probabilmente databile in quell'anno. Cole 1978, 46; CulassoGastaldi 1990, 222. Tenendo conto di Giustino\emph{ haec urbs [Byzantium] condita primo a Pausania, rege Spartanorum, et per septem annos possessa fuit} (9.1.3) il periodo di Pausania  a Bisanzio andò dal 478/7 di Bisanzio (Thuc. 1.94) fino al 472/1. Fornara 1966, 267. Maggiori dettagli e bibliografia in Liuzzo 2010. Considerando questo dato si potrebbe presumere che il computo sia considerato dal ritorno di Pausania  dopo il primo processo, che, accettando il punto fermo del 471/0 offerto da  Diodoro (Giorgini 2004, 206), risalirebbe per sette anni al 477/6, che, mese più mese meno, coincide con le argomentazioni che portano a datare alla tarda primavera 477 la rottura tra la lega e Pausania, in perfetto accordo con Diodoro, che riporta appunto sotto questo anno il ''momento'' di Aristide (`τῷ καιρῷ χρώμενος ἐμφρόνως`). Dando un po' di tempo anche a Pausania, per ricevere la scitale e ritornare a Sparta tramite il Tanaro, anche le date del successivo periodo risultano adeguate. La morte di Pausania nel 468/7 lascerebbe tempo alla fuga di Temistocle, all'inseguimento e al passaggio per Nasso, senza troppe difficoltà.
            La seconda lettera di Temistocle (_Them. Ep._  2.3-5) dimostra nuovamente di utilizzare FGrHist 104 o la sua fonte, implicando la contemporaneità del dominio in Troade con la permanenza di Temistocle ad Argo. CulassoGastaldi 1990, 222. 
            % Nel papiro PBerol 5008 è riportata parte del commento di Didimo a Demostene (ἐκεῖνοι Θεμιστοκλέα λαβόντες μεῖζον ἑαυτῶν ἀξιοῦντα φρονεῖν ἐξήλασαν ἐκ τῆς πόλεως καὶ μηδισμὸν κατέγνωσαν 23.204-5), che si pensa usasse Eforo come sua fonte.
            FGrHist 104 riporta l'interpretazione tucididea della vicenda. Badian 1993, 130-132; Meiggs 1972, 42-67. La cifra del tributo non è conservata ma difficilmente, anche se lo fosse, potrebbe servire a dirimere il problema dei 460 Talenti. L'unica opzione possibile a riguardo pare essere quella dell'errore paleografico già in Tucidide, poi seguito dalle altre fonti,  Nepote (_Arist._ 2.3-3.1) e  Diodoro (11.47.3), che riprendono anche la notizia del trasferimento non presente nell'ateniese. Green 2006, 130 n240 e 150 n302. Meiggs 1972, 324-339.  Nepote soprattutto, pare ricalcare FGrHist 104, seppur con qualche modifica. Il ruolo di Aristide non è rimasto tuttavia nel nostro testo. Vedi Telo 2006, 275, Pap Cair. 4 3227, fr. 99 K.-A, scena in cui Aristide è un altro dei quattro `προστάται τοῦ δήμου` titolo utilizzato dall'_Athenaion Politeia_ per Aristide, anche se per Fornara 1966, 53 si tratta di un Aristide diverso. Qualche problema si presenta per la collocazione già a questo punto dello spostamento del tesoro da Delo ad Atene che è generalmente posto a conseguenza dalla sconfitta in Egitto e prima della vittoria di  Cimone a Cipro. Plutarco ne parla contestualmente all'ostracismo di  Cimone (Plut. _Per._ 12), lasciando pensare ad una datazione nella prima metà del V a.C. non meglio precisabile, nonostante la collocazione narrativa. Le parole di Pericle  sembrano avere già in mente il consiglio del giovane Alcibiade ma, facendo riferimento a qualcosa che è già avvenuto, cioè lo spostamento stesso, implicano che dopo l'ostracismo di  Cimone ciò venisse discusso, in relazione alla politica edilizia, ma non che il tesoro fosse stato allora spostato, e nemmeno che ciò fosse avvenuto dopo la spedizione in Egitto. Questo è uno dei dettagli omessi da Tucidide che però descrive il contributo a 1.99.3.  Diodoro 12.38 può aiutare a ricollocare l'evento e a valutare la collocazione data da FGrHist 104. Parmeggiani 2011, 422 n.123 calcola che lo spostamento del tesoro sia avvenuto nel 461. Sotto l'anno 431, iniziando la sua narrazione con un flashback, dice che: quando stavano ottenendo l'egemonia sul mare spostarono il tesoro, e lo diedero a Pericle  da custodire. Ma Pericle  arriva al potere quando l'egemonia c'era da un pezzo: `καὶ` implica un lasso di tempo tra il trasferimento e l'affidamento come c'è un lasso di tempo tra l'egemonia e lo spostamento, sottolineato dai modi verbali. `Μετὰ δὲ τινα χρόνον` vengono richiesti i rendiconti e qui infatti si ritrova in  Diodoro l'episodio di Alcibiade cui ''segue'' il discorso che ricorda Plutarco. Se la datazione al 454 sulla base di ATL (ML 39) non è utilizzabile perché basata appunto su queste fonti, anche quelle relative alla spedizione in Egitto, che abbiamo visto è possibile ricollocare tra il 463 e il 458, sono logiche ma non possono essere confermate. Per il trasferimento (Pritchett 1969) qualsiasi data a cavallo tra gli anni '60 e '50 può andare bene, anche prima dell'Eurimedonte. Certo, la spiegazione data da Pericle in Plutarco ha una collocazione cronologica diversa, che è quella della querelle sull'edilizia ateniese, rispetto al momento del trasporto `Ἀθηναῖοι τῆς κατὰ θάλατταν ἡγεμονίας ἀντεχόμενοι`, che fa invece proprio pensare ad un momento precedente persino all'Eurimedonte, sebbene di necessità successivo (ma non possiamo sapere di quanto) alla costituzione della lega. Credo che le lamentele sorte intorno a Pericle e al suo uso delle ricchezze, nonché i malumori degli alleati non avrebbero consentito un tale gesto, che si colloca invece meglio laddove gli alleati credevano ad Atene come nuova egemone sul mare invece di Sparta. Ed è  a questo punto peraltro che Plutarco colloca il sinodo di Samo (_Arist. _25.3). Non è in conflitto con le altri fonti, ed è anzi del tutto coerente e logica (sebbene non abbia pretese di verità in senso assoluto) la cronologia di FGrHist 104, per cui lo spostamento sarebbe stato di poco successivo alla fondazione della lega. Non sappiamo in che rapporto con Taso, sebbene si possa pensare che dopo una tale dimostrazione nessuno sarebbe stato ben disposto, ma sicuramente prima di Nasso, velocemente nominata poco oltre. 
     %%%%%%%%%%       
            \subsection*{Pausania,  Coronide, Arigilio e la mamma}
            Il principio di completezza proprio della narrazione presentata conserva sia il racconto di Cleonice che quello di Argilio  che il ruolo della madre di Pausania, pur non essendo testimone privilegiato di alcuna tradizione in particolare. Ogden 2002, 121 ''_Aristodemos is the exception that confirms the rule_''.
  %%%%%%%%          
            \subsubsection{`διεπράξατο δέ τι καὶ τοιοῦτον`}
            È un breve flash back che rincara la dose su Pausania  e ne riprende la storia prima del `πικρῶς τυραννεῖσθαι` del capitolo precedente, quasi a giustificazione. La legge con la quale Licurgo   \textgreek{οὑδ'ἀποδημεῖν ἔδωκε τοῖς βουλομένοις καὶ πλανᾶσθαι, ξενικὰ συνάγοντας ἤθη καὶ μιμήματα βίων ἀπαιδεύτων καὶ πολιτευμάτων διαφόρων} [non permise di viaggiare a chi volesse e anche di esplorare, imparando costumi stranieri e imitando vite senza istruzione e differenti politiche] (Plut. _Lyc._ 27.6)  è violata punto per punto da Pausania. Ma non perché vi fosse una legge da violare, quanto perché probabilmente in questo episodio la legge trovò la causa per la quale fu emanata (Flower 2002, 203): è la violazione che porta alla legislazione. Si potrebbe quasi dire che, nella ricostruzione degli eventi legati al secondo ritorno di Pausania, si cela in realtà la decostruzione di un mito fondativo, rimodellato su ogni successiva necessità di giustificazione. Un ragionamento giuridico non è adatto a questo testo, se posto in termini di medismo e tradimento e porta meno frutto della considerazione delle leggi sociali e religiose. Da tempo l'accento sulle vicende di Pausania  è posto su queste, piuttosto che sui primi aspetti. Già Erodoto e Tucidide non erano in accordo sul modo di definire il comportamento di Pausania  (Thuc. 1.130 e 138.3 `ἡγεμῶν`; Hdt. [5.32](http://data.perseus.org/citations/urn:cts:greekLit:tlg0016.tlg001.perseus-grc1:5.32) `τύραννος`), ed il nostro autore varia rispetto ad essi, utilizzando il titolo di `ὑπάρχων` (già usato per Aristide ad Egina in 1.4.) descrivendo poi il comportamento di Pausania, di cui già aveva parlato in 4.2 senza diretti paralleli nella tradizione, soprattutto per l'insistenza sulla  `δίαιταν` (4.3), come contenuto del suo comportamento `πικρῶς [...] καὶ τυραννικῶς`. Nafissi 2004, 153. Proprio quest'ultima notazione porta, tramite il `ζηλώσαντος γὰρ αὐτοῦ τὴν Περσικὴν τρυφὴν καὶ τυραννικῶς` di  Diodoro (11.44.5), senza necessità di interpretare, al passo da cui siamo partiti. 
    %%%%%%%%%%%%%        
            \subsubsection{`ἦν ἐπιχωρίου τινὸς θυγάτηρ Κορωνίδου ὄνομα`}
            La storia della figlia di Coronide non si trova in Tudicide ma è raccontata da  Plutarco (_Cim._ 6.4-7) come `ὑπὸ πολλῶν ἱστόρηται` e da Pausania il periegeta (3.17.8-9) in cui ritroviamo il maggior numero di dettagli concordi con FGrHist 104. Il nome Cleonice è caduto in fase di trascrizione, ma FGrHist 104 è l'unica fonte a conservare il nome dell' `ἐπιφανῶν γονέων` (_Cim._ 6.4), Coronide. Se il nome della figlia poteva parere ''parlante'' nel contesto della vicenda di Pausania, il nome del padre può riferirsi alla punizione che FGrHist 104 infligge a Pausania. Pur non potendo identificare quale dei molti storici indicati da Plutarco, sia alla base di FGrHist 104, resta il fatto che deve essere uno tra quelli che in modo diverso ha usato anche Pausania, e che si conclude per FGrHist 104 con una `μανία` che avrebbe richiesto tempo per guarire; similmente leggiamo in Plutarco che la giovane  `οὐκ ἐᾶν τὸν Παυσανίαν ἡσυχάζειν`, mentre Pausania punta direttamente all'`ἄγος` da cui non è possibile scampare. La tradizione che ha aggiunto questo elemento è sicuramente una tradizione interessata agli aspetti intimi e personali dell'affaire Pausania  ma lo scagiona anche, per via sacrale, dalle accuse a lui rivolte: un'attenzione ai `πάθη τῆς ψυχῆς` (DH _Ad Pomp._ 6) ''psicologica'' nota per Teopompo  e che troviamo qui, sulla scia di Tucidide, incentivata per la descrizione dei comportamenti degli Ateniesi, degli Spartani, ma anche di individui come Pausania  qui, Temistocle poco sopra e oltre, Aristide, Alessandro I, etc. Rebuffat 1993, 115. Plutarco, nella _Vita di Cimone_, inserisce l'episodio come causa dell'inasprirsi dei rapporti con gli Ioni (quindi nel 478/7) e conclude la persecuzione, con una visita al `νεκυμαντεῖον` di Eraclea, dove gli viene profetizzato dallo ''spettro'' della giovane che `παύσεσθαι τῶν κακῶν αὐτὸν ἐν Σπάρτῃ γενόμενον` [avrebbero avuto fine i suoi mali una volta tornato a Sparta], aggiunge Plutarco `αἰνιττομένη τὴν μέλλουσαν ὡς ἔοικεν αὐτῷ τελευτήν` [riferendosi, sembra, alla sua prossima fine]. 
            %Cfr. anche SEG 9.72.111-121, una legge della Cirenaica che ha gettato nuova luce sul significato di ''Supplice'' (Casalla 1997, 333s. Con bibliografia). 
            Il termine avrebbe un valore doppio, indicando sia il demone persecutore che il supplice. La _Legge Sacra_ di Selinunte (Giuliani 1998, 68s) identifica inoltre, alcune norme su come un individuo che deve purificarsi dall'influsso negativo di entità demoniche designate come `ἐλάστεροι` (`ἀλαστῶρ`) debba comportarsi. Ogden 2002, 114. In FGrHist 104 la specificazione degli effetti del _daimon_ della ragazza, specifico del nostro autore, penso vada riferita direttamente alla mania, razionalizzando o almeno completando, rispetto all'`εἴδωλον` persecutore e profeta di Plutarco. Se `Δαίμονας` davvero mostrasse come FGrHist 104 sia ''more familiar with latin'' perché usa questo termine nel senso di _Di Manes_ (Raphael Sealey, citato in Frost 2005, 259 n.7), l'ipotesi di Ogden (Ogden 2002, 121s) ne trarrebbe grande giovamento. Non credo sia così ma, dal confronto dei testi, si direbbe che  la versione di Plutarco semplifichi rispetto a questa.  
            Il suggerimento che Jacoby dà ad Aristodemo (FGrHist 104), che \emph{hätte besser getan, die geschichte 4,2 einzulegen} (Komm. 328) è ben giustificato dall'inizio del successivo periodo `τῆς δὲ προδοσίας οὐκ ἐπαύετο`, che di certo non richiama la storia di Cleonice, ma appunto si rifà al capitolo indicato da Jacoby (Diod. 11.45.1). Come tutti i buoni consigli, anche quello di Jacoby è destinato a rimanere inascoltato dal suo plurisecolare interlocutore.
     %%%%%%%%%%%%       
            \subsubsection{`γράψας ἐπιστολὰς Ξέρξῃ Ἀργιλίῳ ἀγαπωμένῳ ἑαυτοῦ δίδωσι`}
            FGrHist 104 procede con il racconto dei fatti riguardanti ''Argilio '' seguendo grosso modo la stessa linea di Tucidide. Ogden 2002, 124 discute la differenza tra ''un uomo di Argilio '' e ''Argilio s'' che si trova nei soli  Nepote e FGrHist 104. Esso è un etnico soltanto per Tucidide. Aggiunge Ogden ''_If Argilos is a proper name, a speaking one like Cleonice it means earth, clay_'' e richiamerebbe le cave sotterranee del _nekumanteion_ di Cuma 
            %(Strabo C244, Massimino di Tiro 8.2) 
            che sarebbe razionalizzata da Tucidide. CulassoGastaldi 1990, 264. Non so quanto possa essere ritenuto rilevante il diverso destinatario: se è vero ciò che dice Erodoto in [8.98](http://data.perseus.org/citations/urn:cts:greekLit:tlg0016.tlg001.perseus-grc1:8.98), per arrivare da Serse una lettera sarebbe dovuta necessariamente partire dalla Sardi di Artabazo. Comunque, `πρὸς Ξέρξην` di FGrHist 104 è consono a `Πρὸς τὸν βασιλέα` di Diod. 11.45.1 mentre hanno `πρὸς Ἀρτάβαζον` Thuc. 1.132.5 e Nep 4.1. Si ritrova nel nostro autore un dato molto interessante, e forse più significativo di quanto si sia ritenuto, in riferimento a Tucidide che dice di Argilio  `παιδικά ποτε ὢν αὐτοῦ καὶ πιστότατος ἐκείνῳ`. Anche in Nep. _Paus._ 4.1:_ Argilius quidam adulescentulus, quem puerum Pausanias amore uenerio dilexerat_. Si veda Vattuone 2004a, 128 per la _dike_ non vincolante dell'erotica arcaica.  In FGrHist 104, Argilio  è l'amante di Pausania  (`ἀγαπωμένῳ ἑαυτοῦ`) e di nuovo, come per la reazione spartana all'inganno di Temistocle, la versione che abbiamo davanti legge e spiega, dà la soluzione digerita e non il testo dell'informazione. Diodoro, l'altra nostra fonte al riguardo lo dice semplicemente `τις τῶν βιβλιαφόρων` e passa subito al problema dell'eliminazione consigliata nelle lettere tralasciando la relazione tra i due. Credo che questa notazione non possa essere sottovalutata nella lettura dell'episodio che, se ha un valore eziologico e si costruisce con le caratteristiche di un rito di evocazione, conserva un tratto evidentemente legato al _nomos_ spartano. Ogden (Ogden 2002, 123) identifica il luogo di costruzione della ''capanna'' (`καλύβην` Thuc. 1.133) al capo Tanaro col _nekumantheion_ che sarebbe stato la cava vicina al tempio di Poseidone (n.53 Strabo C636; Pausania 3.25; Pomponius Mela 2.59 (cf. 1.103); _Schol. Aristoph. Acharn._ 509\indexp{Scolia ad Aristoph.!_Acharnenses_ 509|ca}; Seneca _Hercules Furens_ 662-92; Statius _Theb._ 2.32-57 ecc.). Esercita fascino su Ogden, soprattutto la notizia di  Nepote che fa scavare agli efori un buco. Dobbiamo ricordare innanzi tutto due delle caratteristiche dell'etica che regolamentava i rapporti tra _erastes_ ed _eromenoi_. La prima la troviamo in Plutarco: `ἔθος ἦν καὶ τοὺς νεωτέρους ὑπὸ τῶν πρεσβυτέρων ἐρωτᾶσθαι ποῦ πορεύονται καὶ ἐπὶ τί, καὶ τὸν μὴ ἀποκρινόμενον ἢ προφάσεις πλέκοντα ἐπιπλήττειν` [era consuetudine che gli adulti chiedessero ai più giovani dove essi fossero diretti e per quale motivo; e che percuotessero chi non rispondeva o adduceva pretesti] (Nom. Lacon. 237b-c ). La seconda è quel ''biasimo sociale'' che colpiva a Sparta ''chiunque si fosse mostrato attratto dal corpo di un fanciullo (`εἰδέτις παιδὸς σώματος ὀρεγόμενος φανείη` Xen. _Lacaed. Resp._ 2.13), ancor peggio se era colto sul fatto (`Εἴ τις φωραθείη ἁμαρτάνων` Plut. Nom. Lacon. 237c). Vattuone 2004a, 103 e 223-224. Questi tre elementi caratteristici della normativa sull'erotica a Sparta si ritrovano tutti nel racconto di questi fatti fornito da FGrHist 104 che, abbiamo visto, pone l'intero problema di Pausania  in termini di `δίαιταν/διετίθει`, cioè di comportamento abituale e sull'opposizione tra `ἀναφανδὸν/διεξῄει`. Riguardo a quest'ultimo, l'intera tradizione ricorda gli efori nascosti per essere `αὐτήκοοι` (diretti ascoltatori), fossero essi dietro una tenda costruita da loro o per loro, oppure in un buco (Ogden 2002, 124), ma solo in FGrHist 104, Argilio  come Alessandro e poi Temistocle si vincola facendo una promessa. La promessa è una spiegazione semplice, una motivazione e una causa soddisfacente per un'azione, comoda per tutte le situazioni in cui è l'iniziativa personale l'elemento insondabile. Nell'ottica invece del primo punto, possiamo prendere in considerazione il _nomos_ sopra citato come filtro per osservare la tradizione sul dialogo intercorso tra Argilio  e Pausania . In Tucidide, che abbiamo visto marcare la relazione sociale tra i due, Pausania  tiene un atteggiamento che può suonare melenso nei confronti di chi ne ha tradito la fiducia, virtù ricercata seppur spesso disattesa, ma è in realtà chiaramente un rimprovero, inteso nelle due fasi del `ἐρωτᾶσθαι` ed `ἐπιπλήττειν` che abbiamo visto in Plutarco. Vattuone 2004a, 130 per l'infedeltà ricorrente dei ragazzi come specchio di una crisi della polis. In un certo senso forse anche la disparità è infranta dato che Argilio  era `ποτε` l'eromenos. Tucidide riporta il discorso di Pausania  in modo indiretto ed in esso il reggente dice: `περὶ τοῦ παρόντος οὐκ  ἐῶντος ὀργίζεσθαι, ἀλλὰ πίστιν ἐκ τοῦ ἱεροῦ διδόντος τῆς ἀναστάσεως καὶ ἀξιοῦντος ὡς τάχιστα πορεύεσθαι.` [non essere arrabbiato per il presente, ma datagli garanzia che  sarebbe uscito dal tempio, chiedeva che partisse al più presto. (1.133) FGrHist 104 sinteticamente ci informa di come `ἀπεμέμφετο ἐπὶ τῷ μὴ κομίσαι τὰς ἐπιστολὰς πρὸς Ξέρξην`, mentre Diodoro, che aveva tralasciato il dettaglio della relazione amorosa, trasforma l'intera interlocuzione in una serie di scuse e suppliche di Pausania  al portatore di lettere `τοῦ δὲ Παυσανίου φήσαντος μεταμελεῖσθαι καὶ συγγνώμην αἰτουμένου τοῖς ἀγνοηθεῖσιν, ἔτι δὲ δεηθέντος ὅπως συγκρύψῃ `[Pausania si scusò e chiese perdono per gli errori, e implorò affinché conservasse il segreto] (11.45.5). Resta solo l'elemento del non visibile e del segreto. Se la storia di Pausania  è frutto della narrazione per motivare un pretesto avanzato prima della guerra (Thuc. 1.128.2), oppure per motivare un nomos come per il caso della legge licurgea contro l'allontanamento da Sparta, osserviamo qui invece un perdersi progressivo nella tradizione dell'originale contesto normativo, che da socio-etico diviene politico. Oltre ad essere più completo nel significato complessivo, e oltre a contenere peculiarità indipendenti (il nome del padre di Cleonice e l'_hapax_ `περίυπνος`), FGrHist 104 in questo punto è coerente con la tradizione ricordata in Plutarco (_Nom. Lacon._ 237b-c), e Pausania  sarebbe coinvolto in un caso legato ai costumi, non solo estetici e di abitudini, ma di `παιδικά`. L'indagine e il processo poggiano dunque su modi e termini dell'inchiesta su `εἰδέτις παιδὸς σώματος ὀρεγόμενος φανείη` (Xen._ Lacaed. Resp._ 2.13). Con questo abbiamo già due degli elementi fondativi di questo mito di Pausania  che serve molte funzioni, racconta storie in tanti contesti e riporta l'attenzione sulla complessità del meccanismo di trasmissione e memoria storica, che traspare nella razionalizzazione di FGrHist 104 proprio perché non è più complicato della semplice sintesi coerente. 
      %%%%%%%%%      
            \subsubsection{\textgreek{ἐν ἀπόρῳ ὄντων … ἡ μήτηρ τοῦ Παυσανίου βαστάσασα πλίνθον ἔθηκεν}}
            Anche gli eventi che portano alla morte di Pausania  vengono rivisti e rinarrati, forse anche in conseguenza di un lungo senso di colpa, come _exemplum_. Flower 2002, 192 ''_evrytime spartans changed something they attributed it to Lycurgus_'' ma l'istituzionalizzazione dei costumi per lo più risale ad Agide III e Cleomene IV. Nafissi (Nafissi 2004, 168) porta ad esempio, tra i molti loci che rigurdano morti esemplari la legge di Demofanto, Andoc. _De Myst._ 96\indexp{Andocide!_De Myst._ 96|ca}, D. 20 159. La stessa madre di Pausania  (Alcitea), anziana signora spartana, compare e pone la prima pietra di quelle che mureranno vivo il figlio nel tempio di Atene Calcieca, ''dando inizio alla punizione del figlio''  (`προκαταρχομένη τῆς κατὰ τοῦ παιδὸς κολάσεως`). Essa, come Cleonice e Argilio  in dubbio tra la vita e la morte, e in perfetta coerenza strutturale con loro e con la storia del figlio, dà risposta alla proverbiale indecisione dei Lacedemoni che `ἐν ἀπόρῳ ὄντων διὰ τὴν εἰς τὸν θεὸν θρησκείαν`: questa frase, singolare nella tradizione, pone i due temi in gioco nel racconto mitologico, l'etica e la superstizione, in un nuovo conflitto che genererà il grande `ἄγος` Spartano. La madre che pone la pietra è ripresa direttamente da  Diodoro (Ogden 2002, 115), con le medesime parole di FGrHist 104 ma, al posto della motivazione di FGrHist 104, questo modello etico-civico materno, giunge al tempio \textgreek{μηδὲν μήτ'εἰπεῖν μήτε πρᾶξαι}.  Diodoro è successivo alla tradizione conservata in FGrHist 104 e, pur mantenendo elementi omessi da Tucidide, li rielabora funzionalmente ad un racconto di scorrettezza politica, eliminando i dettagli etico-paideutico-educativi.  Crisermo di Corinto dice invece che: \textgreek{Περσῶν τὴν Ἑλλάδα λεηλατούντων, Παυσανίας, ὁ τῶν Λακεδαιμονίων στρατηγὸς, πεντακόσια χρυσοῦ τάλαντα παρὰ Ξέρξου λαβὼν, ἔμελλε προδιδόναι τὴν Σπάρτην. Φωραθέντος δὲ τούτου, Ἀγησίλαος ὁ πατὴρ μέχρι τοῦ ναοῦ τῆς Χαλκιοίκου συνεδίωξεν Ἀθηνᾶς, καὶ τὰς θύρας τοῦ τεμένους πλίνθῳ φράξας, λιμῷ ἀπέκτεινεν· ἡ δὲ μήτηρ καὶ ἄταφον ἔρριψεν.}  [Fuggiti i Persiani dalla Grecia, Pausania, stratego dei Lacedemoni, presi cinquanta talenti d'oro da Serse, stava per tradire Sparta. Coltolo sul fatto, suo padre Agesilao gli corse dietro fino al tempio della Calcieca e ostruì l'ingresso del recinto sacro per ucciderlo con la fame. La madre lo lasciò insepolto.] (Plut. _Parallela minora_ 308b = FGrHist 287 F4)
            Questa versione tenta di inserire nuovi personaggi e giustificare comportamenti, ma solo FGrHist 104 dà una spiegazione soddisfacente e logica all'`ἀπόρῳ` spartano. Dobbiamo dunque riconoscervi una possibile tradizione di valore, per lo meno ''credibile'', che legava nel racconto dell'_agos_ la causa e il delitto, come il filo di Cilone  (Cfr. p.), Questa potrebbe davvero essere una versione più vicina alla fonte per la seconda parte, ma la mazzetta accettata da Pausania  la porta sulla strada di Diodoro. Pausania (3.17.8), introducendo il racconto del reggente di Sparta e di Cleonice come causa della fine dello spartano, conferma di nuovo come Pausania  sia stato ''colto sul fatto'': \textgreek{ἐφ’οἷς ἐβουλεύετο, dice: ἤκουσα δὲ ἀνδρὸς Βυζαντίου Παυσανίαν φωραθῆναί τε ἐφ’οἷς ἐβουλεύετο καὶ μόνον τῶν ἱκετευσάντων τὴν Χαλκίοικον ἁμαρτεῖν ἀδείας κατ’ ἄλλο μὲν οὐδέν, φόνου δὲ ἄγος ἐκνίψασθαι μὴ δυνηθέντα.} [ho sentito da un uomo di Bisanzio che Pausania  fu colto sul fatto riguardo ciò che voleva e soltanto lui tra i supplici la licenza per aver sbagliato, non essendo riuscito ad espiare un omicidio sacrilego]. Musti 1982, XXIVs.
            Riguardo la morte di Pausania, infatti, gli elementi principali sono grossomodo condivisi. Muore di fame. Viene gettato nella Caeda per Plutarco/Crisermo, FGrHist 104, Eliano (VH 4.7) e Suda cfr. anche Licurgo  128.  Tucidide 1.134.4 conosce ma nega la tradizione stabilita, a conferma di quanto ipotizzato sopra riguardo alla trasmissione della tradizione. La pestilenza (`λοιμός`) che avrebbe colpito gli Spartani per questo motivo non è presente in nessun'altra fonte, fatto salvo per il lessico Suda, dove si dice che  `νοσησάσης δὲ τῆς πόλεως` [la città di ammalò]: come prospettato dalla punizione prevista nel giuramento della stele di Acarne (GHI n°88\index[pap]{GHI n°88}: `ἡ πόλις ἡμὴ ἄνοσος εἴ/η, εἰ δὲ μή, νοσοίη`) cfr. p.. Plat. _Prot._ 322D:\textgreek{νόμον γε θές παρ’ἐμοῦ τὸν μὴ δυνάμενον αἰδοῦς καὶ δίκης μετέχειν κτείνειν ὡς νόσον πόλεως.} Non è certo se _Them. Ep._  4.15-17 si riferisca a ciò con quel  `παλαμναῖον ἢ ἀλιτήριον προστρῖψαι τῇ πόλει` [spirito vendicatore e punitore implacabile contro la città]. Che invece sembrerebbe direttamente connesso a `τοὺς δαίμονας τοῦ Παυσανίου`. La questione delle statue è facilmente frutto di un'oscillazione dovuta a problemi di auralità e memoria. Tucidide proponeva che fossero state date `χαλκοῦς ἀνδρίαντας δύο`, FGrHist 104 ha `ἀνδριάντα αὐτῷ` seguito da Suda con `εἰκόνα ἔστησαν χαλκῆν Παυσανίου`  Diodoro `δύο τοῦ Παυσανίου χαλκᾶς`. È molto importante per il ragionamento di Ogden, Plut. _De Sera Num. Vind._ 560 e-f\indexp{Plutarco!_De sera num. Vind._: `ὀμοίως δὲ καὶ Σπαρτιάταις χρησθὲν ἱλάσασθαι τὴν Παυσανίου ψυχὴν ἐξ Ἰταλίας μεταπεμφθέντες οἱ ψυχαγωγοὶ καὶ θύσαντες ἀπεσπάσαντο [da ἀποσπάω] τοῦ ἱεροῦ τὸ εἴδωλον`.
            Non potendo avanzare considerazioni su ciò che FGrHist 104 ha tagliato, e non potendo condividere di conseguenza le considerazioni di Jacoby al riguardo (Komm., 329.), ricordo  che Ogden diceva che stiamo ''_witnessing an originally unitary tale in the process of diverging_'' (Carawan 1989; Westlake 1977; Ogden 2002, 113) e che Nafissi (Nafissi 2004, 179) opponeva una versione spartana ufficiale ad una ateniese che sottolinea l'incuria e la lentezza, sulla base di una diffusione e autorevolezza, attestate anche per Stesimbroto, della _hybris_ di Pausania, che Erodoto non solo conosceva ma scartava. La divergenza da una fonte consapevolmente e spontaneamente accorta rispetto all'_ethos_ spartano ed alle implicazioni religiose e morali della morte di Pausania  di cui si trova traccia in FGrHist 104 e in Pausania (3.17.8-9), avviene già nel V secolo quando Tucidide sceglie e ripulisce la versione della sua fonte ionica di dettagli come quello della mamma. Questa fonte continua ad essere utilizzata ed il filone razionalizzante si evidenzia soltanto nel processo di trasmissione che porta a Diodoro, forse tramite la critica di Eforo a Tucidide ma non si ritrova in chi ha riutilizzato una fonte di V secolo come Crisermo o Pausania.
     %%%%%%%%%%       
            \subsection*{Il disco}
            L'idea dell'iscrizione circolare riportata da FGrHist 104 non è assurda. Jacoby FGrHist 104 Komm, Pownall 2011, _ad loc_. Ho affrontato il problema relativo a questo passo in uno studio sul trofeo di Platea (Liuzzo 2012, 27-41) e mi limito a riportarne i principali risultati. La storia del monumento e le fonti ad esso relative (Hdt. [9.81](http://data.perseus.org/citations/urn:cts:greekLit:tlg0016.tlg001.perseus-grc1:9.81), Pausania 10.13.9, D. 59.16, Thuc. 1.132) portano a pensare che un'iscrizione circolare si possa prendere in considerazione per un momento ben preciso della storia del testo, dopo la cancellazione dell'epigramma di Pausania. Il cratere, estratto dal tesoro persiano come la spada di Mardonio è un artefatto persiano ed esistono notevoli paralleli che attestano da un lato la diffusione di testi di questo tipo (iscritti attorno al bordo superiore di un cratere) e dall'altro la presenza di artefatti simili di provenienza persiana dell'accampamento persiano di Mardonio. Non ultimo il passo delle _Lettere di Temistocle_ (21.1). Più problematica è la collocazione di questo paragrafo in questo punto della narrazione. Non c'è nessun apparente motivo per interrompere le vicende di Temistocle e Pausania, né la coerenza cronologica interna a FGrHist 104 lo richiede. Due sono i possibili motivi, in primis la completezza narrativa che vuole preservare l'aneddoto e la peculiare ''idea''; in secondo luogo la narrazione che deve ritornare indietro a narrare la fine di Temistocle partendo da un punto già passato. Cronologia, sincronismi e diacronia sono problemi che lo storico, anche il più povero di ideali e intenzioni, non può che raggirare nella _dieghesis_ e il metodo impiegato da FGrHist 104 rientra in questa categoria e per risposta a questa necessità conserva una tradizione che aiuta a dirimere una domanda antica laddove se ne stimi il valore e l'origine.
     %%%%%%%%%       
            \subsection*{La fine di Temistocle}
           La centralità del personaggio nel testo è indiscutibile e significativa come si è già detto. Per l'incontro tra Artaserse  e Temistocle. Liuzzo 2010, Cagnazzi 2001, 35s. Per Cagnazzi 2001, 29, anche Ippia  avrebbe scelto di imparare l'antico persiano nel tempo a disposizione. Anche per l'ateniese, come per Pausania  (5) si elencano miracoli (10.3) e gesta insieme ai detti famosi e alla morte. 
       %%%%%%%    
            \subsubsection{`ὁ δὲ Θεμιστοκλῆς δεδοικς`}
            L'episodio della morte di Temistocle offre un ottimo punto di osservazione per rivedere l'intera tradizione di questo racconto anche perché, grazie al recente apporto alla tradizione indiretta di FGrHist 104, confermato dallo scolio sul papiro CLGP Aristoph. 5 (Oxford Bodl. Ms. Gr. Class. f.72 + P. Acad. Inv. 3 d Montana 2006), contribuisce a definire i tratti della tradizione indiretta del testo. Possiamo seguire questa tradizione dalla commedia fino ai lessici ed è uno degli elementi più chiaramente riconducibili a quell'interesse scolastico per la composizione di narrazioni sulle morti illustri di grandi uomini del passato (di cui un poetico esempio resta proprio nella ventunesima lettera di Temistocle che è forse tra le fonti di `Θ`124) e che dimostra come ''l'immagine dello statista è recepita in termini topici e aneddotici, intrinsecamente ambigua, deformata dall'uso strumentale nell'ambito politico e in quello letterario''. Montana 2002, 261. Per la deformazione delle tradizioni, per la tendenza filoateniese e per il parallelo tra Temistocle e il tipico _Trickster_ (Vansina 1985, 26) della tradizione orale africana. Murray 2001, 25-30. La topica e l'aneddotica diventano quasi filastrocche proverbiali anche per Magnesia, Miune, Lampsaco, che vengono concordemente indicate come il donativo di Artaserse  a Temistocle, per i benefici portati al re, come abbiamo visto per gli Scoli ad _Equites_ 84\indexp{Scolia ad Aristoph.!_Equites_ 84|ca}. Sin da Tucidide che, dopo aver detto della morte di Temistocle, ricorda che: \textgreek{μνημεῖον μὲν οὖν αὐτοῦ ἐν Μαγνησίᾳ ἐστὶ τῇ Ἀσιανῇ ἐν τῇ ἀγορᾷ· ταύτης γὰρ ἦρχε τῆς χώρας, δόντος βασιλέως αὐτῷ Μαγνησίαν μὲν ἄρτον, ἣ προσέφερε πεντήκοντα τάλαντα τοῦ ἐνιαυτοῦ, Λάμψακον δὲ οἶνον (ἐδόκει γὰρ πολυοινότατον τῶν τότε εἶναι), Μυοῦντα δὲ ὄψον} [il suo memoriale è a Magnesia d'Asia, nell'agorà; governava infatti la terra di questa poiché il Re gli aveva dato Magnesia come pane, che ogni anno gli forniva 50 Talenti, Lampsaco per il vino (che ai tempi si riteneva fosse ricca in vino), Miunte per l'_opson_] (1.138.5). Ferretto 1984, 66. Tuttavia, seguendo l'analisi di Marr (Marr 1996, 562) dei versi 810-19 dei _Cavalieri_  non è difficile vedere questi versi come una ''_single sense unit_'' sui meriti e le azioni di Temistocle. Riporto il passo di Aristofane: \textgreek{Ὦ πόλις Ἄργους, κλύεθ’ οἷα λέγει. Σὺ Θεμιστοκλεῖ ἀντιφερίζεις; /ὃς ἐποίησεν τὴν πόλιν ἡμῶν μεστὴν εὑρὼν ἐπιχειλῆ, / καὶ πρὸς τούτοις ἀριστώσῃ τὸν Πειραιᾶ προσέμαξεν, / ἀφελών τ’ οὐδὲν τῶν ἀρχαίων ἰχθῦς καινοὺς παρέθηκεν· / σὺ δ’ Ἀθηναίους ἐζήτησας μικροπολίτας ἀποφῆναι / διατειχίζων καὶ χρησμῳδῶν, ὁ Θεμιστοκλεῖ ἀντιφερίζων.  / Κἀκεῖνος μὲν φεύγει τὴν γῆν, σὺ δ’ Ἀχιλλείων ἀπομάττει.} (Aristoph. _Eq._ 814-816). In questo frammento di commedia, se il campo semantico di riferimento del verso 815 (`καὶ πρὸς τούτοις ἀριστώσῃ τὸν Πειραιᾶ προσέμαξεν` [per noi impastò il Pireo coi migliori ingredienti]) è il pane e quello del verso 816 (\textgreek{ἀφελών τ’ οὐδὲν τῶν ἀρχαίων ἰχθῦς καινοὺς παρέθηκεν} [togliendo nulla dal vecchio ci imbandì pesce fresco]) è il pesce, il verso 814 (`ὃς ἐποίησεν τὴν πόλιν ἡμῶν μεστὴν εὑρὼν ἐπιχειλῆ` [che rese la nostra città trovata colma, zeppa]) riguarda il vino. Montana (Montana 2002, 280) nota anche un'allusione colta nei commentatori antichi: al v.312 i tributi arrivano infatti come 'tonni' (_Vesp._ 1087 `θυννάζοντες`). La posizione di Paflagone è assimilata a quella di Serse che assiste alla disfatta delle proprie navi sulla base di questo elemento, giacché la mattanza della flotta persiana è descritta come strage di navi e uomini `ὥστε θύννους` [come tonni] (_Pers._ 424)). Il riferimento allusivo identificato dal Marr è la diadi narrativa di Thuc. 1.89.3 con la costruzione delle mura e l'ambasciata a Sparta, ma l'accostamento mi sembra anche rendere chiaro come  possa essere altrettanto presente un riferimento all'attribuzione delle città asiatiche, che peraltro non stupisce in una commedia che gioca con le tradizioni biografiche sullo stratego di Salamina.
             Diodoro (11.57.7) concorda con Tucidide e Aristofane, ma, a differenza del primo, anziché attribuire un significato metaforico, usando `Εἰς` con l'accusativo (Marr 1996, 537), firma per sempre la concretezza  del dato, che ritroviamo identica in FGrHist 104 e in Plutarco (_Them._  29.11). Anche nella _Lettere di Temistocle_, _Them. Ep._  20.35-39,  dove peraltro, Lampsaco è `ἠλευθέρωσα καὶ πολλῷ φόρῳ βαρυνομένην` (in ATL però, paga dal 450). CulassoGastaldi 1990, 262 e n.20 per bibliografia e diverse posizioni. La ''storicità''  del potentato di Lampsaco è attestata anche da I. aus Kleinasien n°6\index[pap]{I. aus Kleinasien n°6}: \textgreek{[ἐν δὲ τῇ ἑορτῇ] / τῇ Θεμιστοκλεῖ [ἀγομένῃ δι'ἐνιαυ] / τοῦ εἶναι πάντα α[ὐτῷ τἀγαθὰ ἃ ἐδόθη] / σαν Κλεοφάντῳ κ[αὶ τοῖς ἀπογόνοις]} del 200 a.C..  L'unico a discostarsi dalla tradizione che confluisce direttamente in FGrHist 104 sembrerebbe essere Teopompo  ma solo perché, in F87 (Plut. _Them._  31,3) Temistocle \textgreek{οὐ γὰρ πλανώμενος περὶ τὴν Ἀσίαν, ὥς φησι Θεόπομπος, ἀλλ’ ἐν Μαγνησίαι μὲν οἰκῶν}. Teopompo  dunque affrontava il tema, ma in modo diverso, sebbene anche qui come in altri casi, la variazione si limiti ad una fase. Dopo qualche tempo passato in Asia (e comunque necessario per lo spostamento) Temistocle se ne va a Magnesia. Tucidide ci dice da subito, fin da Susa, che il Re sperava che Temistocle sottomettesse per lui la Grecia, un po' come Ippia  (Hdt [6.107](http://data.perseus.org/citations/urn:cts:greekLit:tlg0016.tlg001.perseus-grc1:6.107) e Hdt [6.59](http://data.perseus.org/citations/urn:cts:greekLit:tlg0016.tlg001.perseus-grc1:6.59); Cagnazzi 2001, 25) aveva provato a fare conducendo Dati e Artaferne a Maratona (`τοῦ Ἑλληνικοῦ ἐλπίδα ἣν ὑπετίθει αὐτῷ δουλώσειν` Thuc. 1.138.2) e poi ricorda che, secondo alcuni, si sarebbe ucciso pensando di non poter mantenere la promessa fatta al Re (`ἀδύνατον νομίσαντα εἶναι ἐπιτελέσαι βασιλεῖ ἃ ὑπέσχετο` 1.138.4). Solo FGrHist 104 racconta di Temistocle che promette ad Artaserse  di sottomettere la Grecia con le medesime parole usate da Mardonio a 2.1. In  Diodoro (11.58) troviamo soltanto l'invito di Serse. In Plutarco invece, Temistocle dichiara di essere divenuto nemico dei Greci (`Ἑλλήνων πολέμιον γενόμενον`) ma, al capitolo 31.4, si trova a dover tener fede alle promesse fatte (`βεβαιοῦν τὰς ὑποσχέσεις`). Poco prima tuttavia, riportando parole di Stesimbroto (_Them._   24.7 = FgrHistCont 1002, F4), Temistocle promette a Ierone di Siracusa di sottomettere la Grecia per lui (`ὑπισχνούμενον αὐτῳ τοὺς Ἕλληνας ὑπηκόους ποιήσειν`). Ierone e Serse/Artaserse  vengono scambiati per le medesime serie di eventi nella fuga di Temistocle (Cfr. sezione \ref{104Stesimbroto}) ed è notevole che anche questo elemento possa essere stato duplicato nelle due vicende.
            La promessa di Temistocle e l'intesa con Artaserse  è dunque un punto di partenza, ed FGrHist 104 è la fonte più esplicita a riguardo. A ciò si aggiunge la certezza della provenienza dello scolio ad Aristofane  _Eq._ 84 conservato da CLGP Aristoph. 5. In FGrHist 104 abbiamo infatti `παραγενόμενος εἰς Μαγνησίαν, ἐγγὺς ἤδη γενόμενος τῆς Ἑλλάδος μετενόησεν` e nel papiro si trova proprio, \textgreek{ Περσῶν καὶ ἐστρατεύσατο ε [..8-10] / [3...]ακ ἐν τῇ …. μετανοήσας δ . . ηγεν} [mosse in armi verso... mutato avviso]. Anche se non avessimo la certezza del confronto nel testo conservato dal codice parigino, dovremmo comunque ipotizzare una volontaria presa di posizione per giustificare il cambiamento. La motivazione è ripresa molto da vicino dalle _Lettere di Temistocle_ dove troviamo rinarrati in forma letteraria quei dati che circolano tra l'`ἠναγκάζετο μετὰ ταῦτα τοῖς Ἕλλησι πολεμεῖν` di Suda `Θ`124 e l'_unicum _del nostro testo ''`οὐχ ἡγησάμενος δεῖν πολεμεῖν τοῖς ὁμοφύλοις`''. La ''coscienza tragica'' di Temistocle emerge laddove viene immaginato a scrivere a Polignoto: \textgreek{καὶ ἡμᾶς ἄρα τοῦ στρατοῦ προβαλεῖται ἡγεμόνας καὶ Μήδους ὑποτάξει Θεμιστοκλεῖ, καὶ στρατεύσομαι ἐπ’ Ἀθήνας ἐγὼ καὶ τῷ Ἀθηναίων ναυαρχήσοντι μαχοῦμαι; πολλὰ ἄλλα ἔσται, τοῦτο δὲ οὐδέποτε} [e mi proporrà come comandante dell'esercito e schiererà i Medi sotto Temistocle, e io porterò guerra contro Atene e combatterò  contro il navarca Ateniese? Farei molte altre cose, questa mai.] (_Them. Ep._  20.44) È evidente che il dettaglio delle mezze parole di Temistocle, fraintese dal Re, si mantiene costante in tutta la tradizione e non ha bisogno di un cambiamento di opinione di Temistocle, quanto invece la versione di FGrHist 104. 
      %%%%%%%%%%      
            \subsubsection{`πληρώσας αἵματος ἔπιεν καὶ ἐτελεύτησεν`}
         La morte di Temistocle è avvenuta nel santuario di Artemide a Magnesia (Paus. 1.26.4). Aristofane  commentato dalla letteratura scoliografica, rielaborato dalle lettere e reimpiegato in Suda, in _Equites_ 83-5 dice: \textgreek{Πῶς δῆτα, πῶς γένοιτ’ ἂν ἀνδρικώτατα; / Βέλτιστον ἡμῖν αἷμα ταύρειον πιεῖν· / ὁ Θεμιστοκλέους γὰρ θάνατος αἱρετώτερος} [come scusa? Come sarebbe la cosa più da tori? sarebbe meglio per noi bere sangue di toro scegliendo la morte di Temistocle] (nella traduzione, ''da tori'' è un'espressione dialettale che si adatta bene al gioco di parole rilevato da Arnould 1993, 232 tra `ἀνήρ / ταῦρος`). Il motivo dell'astuzia e della capacità di ideare espedienti in questi versi dove Aristofane  gioca sul rapporto analogico Sangue / Vino è stato notato da Montana (Montana 2002, 264-5), che mette un accento particolare anche sui versi 43-72 la cui la descrizione di Paflagone richiama la caratteristiche tipiche del figlio di Neocle (\textgreek{πανουργότατον καὶ διαβολώτατόν τινα; ᾔκαλλ’, ἐθώπευ’, ἐκολάκευ’, ἐξηπάτα; Ἄιδει δὲ χρησμούς; τέχνην πεπόηται;  ψευδῆ διαβάλλει}). Altri casi di decesso con ''sangue di toro'' si trovano (da Arnould 1993, 231) in Ctesia (F13.12) che racconta la morte di Tanyoxarkes `αἵματι γὰρ ταύρου ὃ ἐξέπιεν ἀναιρεῖται Τανυοξάρκης`, in Sofocle Elena fr. 178 Radt \textgreek{ἐμοὶ δὲ λῷστον αἷμα ταύρειον πιεῖν / καὶ μὴ ’πὶ πλεῖον τῶνδ’ ἔχειν δυσφημίας}, e Paus. 7.25.13: a proposito della sacerdotessa di Gaios in Acaia: `πίνουσα δὲ αἷμα ταύρου δοκιμάζονται`. La prima attestazione è il suicidio di Psammetino (Hdt. [3.15](http://data.perseus.org/citations/urn:cts:greekLit:tlg0016.tlg001.perseus-grc1:3.15)) che compie il gesto dopo il fallimento della sua rivolta contro Cambise, nel 525. Nei commenti ad Erodoto se ne parla come di cosa nota, per la rapidità della morte, e la formula `αἷμα πιών` simile alle formule come `κώνεινων πιεῖν` o nelle ricette mediche `γάλα πιεῖν`. In Egitto e in Asia pare fosse un medicinale prescritto e in alcuni casi fosse legato al sucidio. Vi sono molti esempi di farmaci vegetali con nomi animali; es. Dioscoride 4.51 `τράγος`, 129 `βούγλωσσον, κυνόγλωσσον, αἷμα δρακόντων`. Inoltre lo stesso nome in epoche diverse designa cose del tutto diverse, cosicché l'`ἀνδρικώτατα` usato da Aristofane  può essere la chiave di lettura per l'equivalenza di `ἀρσενικός` (vocalismo zero di questo termine si trova nel sanscrito rsa-bha = toro) con `αἷμα ταύρειον`. Per Montana (Montana 2002, 291) Aristoph. _Eq._ 83-84 è ''un preciso ammiccamento a un tema popolare e insieme storiografico, ben noto  e familiare all'opinione pubblica''. Tucidide e  Diodoro in modi simili nominano soltanto questa tradizione folcloristica che nasce, secondo Arnould (Arnould 1993, 230), un vero e proprio ''romanzo psicologico'' sul sangue di toro. Per la velocità nell'uccidere Arist. _HA_ 3.19 (520b) : `τάχιστα δὲ πήγνυται τὸ τοῦ ταύτου αἷμα πάντων`; per la velenosità Plinio _HN_ 11.90 (221), 28.53 (195): _ideo pestifer potu_. Per la causa delle velenosità: `καθαρώτατον παχύτατον`. Arist. _HA_ 3.19 (521a). La tradizione diventa vulgata approdando anche in FGrHist 104 senza difficoltà. Troviamo il racconto in Cicerone che dice di prenderlo da Tucidide e fornisce una buona chiave di lettura per il suo successivo uso e per la sua sopravvivenza così lunga e solida (Brutus §42-43), in Valerio Massimo (5.6.3) e in Plutarco che ne parla per Midas in _De Superstitione_ 168F 20.9 per le diverse storie sulla morte di Annibale dicendo che \textgreek{ἔνιοι δὲ μιμησάμενον Θεμιστοκλέα καὶ Μίδαν αἷμα ταύρειον πιεῖν· Λίβιος δέ φησι φάρμακον ἔχοντα κεράσαι, καὶ τὴν κύλικα δεξάμενον εἰπεῖν}. Questo mito è dunque chiaramente reso solido dalla sua semplicità, dalla diffusione e dal suo contesto di formazione ed uso retorico, divenendo un _mythos_ per la spiegazione di qualcosa di mal conosciuto (Arnould 1993, 230, Marr 1996, 163). L'inversione del dubbio che si trova in Plutarco _Them._  31.6  è sintomatica: \textgreek{καὶ τοὺς φίλους συναγαγὼν καὶ δεξιωσάμενος, ὡς μὲν ὁ πολὺς λόγος αἷμα ταύρειον πιών, ὡς δ’ ἔνιοι φάρμακον ἐφήμερον προσενεγκάμενος}. Questo è dunque un ulteriore esempio di ''inganno per risolvere situazioni gravi o problematiche'': Montana 2002, 260. In questo caso dunque, di nuovo, troviamo un  Diodoro che segue e aggiunge a Tucidide, ma con elementi come l'inganno che afferiscono al filone apologetico. Invero, nessuna delle tradizioni in nostro possesso si può dire anti-temistoclea (Se si eccettua il frammento 126 K.-A.= _Arist. _4.3 `σοφὸς γὰρ ἁνήρ, τῆς δὲ χειρὸς οὐ κρατῶν`, peraltro assai moderato, e scherzoso più che invettivo): li corona tutti Plutarco in _Them._   31.6, aggiungendo alla narrazione tradita: `ἄριστα βουλευσάμενος ἐπιθεῖναι τῷ βίῳ τὴν τελευτὴν πρέπουσαν`. Il ruolo dei figli di Temistocle ritornati ad Atene è innegabile a riguardo e si è dimostrato indubbiamente efficace. Thuc. 1.138.6, Paus. 1.26.4. Idomen. Lampsac. FGrHist 338 F1 Cic. _Brut._ 1.15.11, Suda `Θ`126. Per il loro rientro ad Atene: Stesimbroto di Taso, FGrHistCont 1002 F3 (Plut. _Them._  24.6-25.1), Paus 1.1.2. Secondo Marr (Marr 1995, 161) ritornarono due dei figli di Temistocle (Plyeuctus e Cleophantus), mentre Archeptolis rimase a guardia dei possedimenti di Magnesia. Platone conserva una serie di testimonianze negative nei confronti di questi figli di Temistocle (93 d-e, 516d, 519a-c). Per Montana 2002, 296-8 Eupoli immagina il ritorno dall'Ade di Solone, Milziade, Aristide e Pericle  e un trimetro di Plutarco stigmatizza l'assenza di Temistocle dal novero delle personalità storiche tornate sulla terra (Arist 4.3): `σοφὸς γὰρ ἁνήρ, τῆς δὲ χειρὸς οὐ κρατῶν`. Isocrate accoglierà completamente, per canonizzarla, l'immagine di Temistocle rivitalizzandone il connotato nazionalistico antipersiano. 8.75; Lisia 2.42. I figli portano per prima cosa le ossa del padre al Pireo e innalzano statue. Il momento del loro ritorno è discusso, ma probabilmente avvenne al momento del declino politico di  Cimone (Thuc. 1.102.1-3, Plut. _Cim._ 16.6-8 e 17.1-2), che è a capo dell'esercito che muove consapevolmente (solo in FGrHist 104) contro Temistocle altrimenti non noterebbe dopo la morte di questi che gli Ateniesi non ne erano al corrente, implicando che la presenza del generale fosse invece certa.
    %%%%%%%%%        
            \subsection*{L'Eurimedonte e l'Egitto}
            Questo paragrafo presupponendo un legame sincronico tra la spedizione di Temistocle contro la Grecia, noto anche in Plutarco (_Cim._ 18.7) ed una presunta mossa difensiva dei Greci con il supporto ad Inaro, narra della spedizione in Egitto e della sconfitta inferta agli Ateniesi da Megabizo. Si accenna soltanto a Tanagra ed Oenofita (12) per poi dedicare un lungo appunto alle origini del soprannome di Callia e alla pace da questi stipulata con Artaserse  (13). FGrHist 104 prosegue con l'inizio di una ''guerra greca'' che include la guerra sacra, gli scontri di  Tolmide con Beoti (14) e Peloponnesiaci, seguiti dalle punizioni inferte da Atene all'Eubea e a Samo per mano di Pericle  e Sofocle (15).
    %%%%%%%%%%        
            \subsubsection{\textgreek{Οἱ δὲ Ἕλληνες  …  ἐξεδίωκον τὸν στρατὸν τὸν ἅμα τῷ Θεμιστοκλεῖ}}
            I Greci sono all'oscuro della morte di Temistocle, ma sono consapevoli che è lui a guidare l'esercito contro di loro. Si presenta uno scenario simile a quello di Ippia a Maratona destinato a ripetersi almeno fino alla Cnido di Conone, con generali ateniesi al servizio del Re di Persia. Ma questo possibile scontro di Temistocle e Cimone, sventato dal suicidio del vincitore di Salamina è noto solo da FGrHist 104 e dalla Vita di  Cimone (18.7), seppure nel nostro racconto sia un elemento di raccordo degli eventi e invece in Plutarco un elemento inserito a colorire il personaggio principale. È solo l'inizio dell'opera di  Cimone liberatore, mai ostracizzato, invitto. Senza interruzioni la storia continua con la battaglia dell'Eurimedonte e la spedizione in Egitto. Thuc. 1.100.1; Licurgo  72; Diod. 11.60.6-61.7 CulassoGastaldi 1990, 219 ricorda la recente riconsiderazione della datazione dell'Eurimedonte, svincolata dalle Dionisie del 468 e il 467 proposto dal Gomme e da Meiggs 1972, 75-86, Green 2006, 125-7. La cronologia interna di FGrHist 104 è riassunta nell'appendice \ref{cronofgrhist104}.
  %%%%%%%%%%%%          
            \subsubsection{`κατὰ τὸν λεγόμενον Εὐρυμέδοντα ποταμόν`}
            Vengono riportati i risultati della spedizione e i trofei. Sui trofei nei principali storici greci si vedano le osservazioni di Hau 2013. La battaglia scomparsa di Badian c'è eccome, anzi, è  `λαμπρὰ ἔργα `ed Elio Aristide pare servirsi del nostro testo per comporre parte della sua orazione (46.158,1). Plutarco dichiara tre fonti per la battaglia, Eforo,  Callistene  e Fanodemo. Pare si possa trovare conferma in P.Oxy. 1610 (Eforo FGrHist 70 F191 frg. 9; 10; 53): `Κίμων πυν/θανόμενος τὸ]\d{ν` τ[ῶν / Περσῶν στόλο]ν περὶ / [τὴν Κύπρον συ]ντετά/[χθαι … }(commentato da Parmeggiani 2011, 406) che diverge dal racconto di  Diodoro ma affronta il periodo comunque nell'ottica della politica ateniese nei confronti della Persia (Parmeggiani 2011, 399-415). Il colpo doppio, per mare e per terra, della vittoria di  Cimone è l'elemento glorioso e distintivo della battaglia che viene conservato per distinguerla e ricordarla insieme ai numeri relativi ai contingenti. 
     %%%%%%%%%%       
            \subsubsection{`ἐβασίλευσε δὲ τῆς Αἰγύπτου Ἴναρος υἱὸς Ψαμμητίχου`} 
            La spedizione Ateniese in Egitto a supporto di Inaro, capo della tribù libica di Bakalu (Winnicki 2006, 135-42.) è stata oggetto di diversi studi anche recenti che ne discutono la cronologia e i momenti salienti: Kahn 2008, Green 2006, Lenfant 2004, 129-131 (testo) e 266-7 (note), Briant 1996, 591-4. FGrHist 104 segue la versione tucididea, con la rovinosa sconfitta ateniese. Gli Ateniesi arrivano in Egitto passando per Cipro: anche questa come altre è un'informazione dedotta. Importanti nella discussione sono anche ML 33 e 34 e lo scolio al _Pluto_ di Aristofane. Si osserva in questo episodio una selezione che predilige la versione tucididea e la ripulisce mantenendo solo i dati principali (scompare per esempio Amirteo). La versione di Ctesia e quella di  Diodoro divergono molto ed escludono la possibilità di una comunanza di fonti con il nostro testo. È da ricordare che questo evento, nella stessa misura delle sorti di Pausania  è noto anche ad Erodoto ([3.12](http://data.perseus.org/citations/urn:cts:greekLit:tlg0016.tlg001.perseus-grc1:3.12).4, [3.15](http://data.perseus.org/citations/urn:cts:greekLit:tlg0016.tlg001.perseus-grc1:3.15).3, [7.7](http://data.perseus.org/citations/urn:cts:greekLit:tlg0016.tlg001.perseus-grc1:7.7)) che mostra peraltro di conoscere più diffusamente di Tucidide gli eventi che ritroviamo per esteso in Diodoro. Anche questi eventi facevano parte del repertorio retorico di IV a.C., come dimostra anche Isocrate (8.86).
      %%%%%%%%%%%      
            \subsection*{Tanagra ed Oenofita: `Ἑλληνικὸς πόλεμος ἐγένετο`}
            In questi capitoli si osservano al massimo la sintesi e i principi di selezione. Sordi 2002, 553s. Le informazioni fornite sono: popolazioni coinvolte,  numeri,  esito. Una lista di eventi senza racconto, completa di dati ma priva di storia.  Su queste due battaglie i passi di riferimento sono  Tucidide 1.107 – 108.3 e  Diod. 11.79, 83.1 ma FGrHist 104 offre ben poco da commentare in questo caso. A riportare il nostro testo nella tipologia che fa da riferimento per i componimenti epidittici abbiamo il Menesseno  platonico, discorso scritto da Aspasia maestra comune di Socrate e Pericle, ancora viva quando Platone immagina il discorso recitato da Socrate. Nouhaud 1982, 140s. per i legami tra questo testo e Polyb. IV 20, 4-7; su Eforo, Parmeggiani 2011, 87. Il dialogo riflette un pensiero posteriore alla Pace di Antalcida, ma soprattutto ci dice qualcosa di come il discorso storico e le strutture di sintesi e ricollegamento tra gli eventi fossero riproposte più o meno scolasticamente. Prima di arrivare a 241d dove inizia la carrellata di eventi di V secolo che arriva a nominare Tanagra ed Enofita, Socrate dichiara: \textgreek{ἑμάνθανόν γε τοι παρ’αὐτῆς , καὶ ὁλίγου πληγὰς ἔλαβον ὅτ’ἐπελανθανόμην}. [L'ho imparato proprio da lei e per poco non venivo preso a botte quando mi dimenticavo] (242e) Non importa neppure a FGrHist 104 quanto tempo sia intercorso tra le battaglie (Thuc. 1.108.2-3, Diod. 11.81.1, Pl. Men. 242b): accosta Tolmide a Mironide, per nessun motivo specifico. Ferretto 1984, 50-51.
   %%%%%%%%%%         
            \subsection*{Cipro e la Pace di Callia}
             I punti principali dei problemi attorno alla Pace di Callia (data, proponente, termini, critica antica) si possono trovare in Meiggs 1972, 487-495 e Badian 1993, 1-72. Un sunto delle varie posizioni si trova in Lachhein 2008. Questo è il capitolo più fortunato di FGrHist 104: ha sollevato una serie di problemi e discussioni in cui ha avuto un posto secondario ma importante, soprattutto grazie al ruolo accordatogli dal Badian in un celebre articolo del 1987. Badian in quella prima versione, leggendo solo i §13 e 14, argomentava in favore della derivazione da Eforo del testo di FGrHist 104, data la comunanza di fonti con Suda. FGrHist 104 ha un posto di rilievo nella discussione sulla questione della Pace di Callia 

*  per la posizione che dà nel racconto a questo evento;
*  per la definizione dei nomi e dei gruppi coinvolti;
*  per i termini del trattato. 

Inoltre, in questo paragrafo, che abbiamo visto tramandato anche da Suda e da Planude (Cfr. \ref{104Ermogene}: Suda, K 1620 e K214 nonché lo scolio ad Ermogene in Walz V 388), ci mostra come il sovrapporsi di semplificazioni e tentativi di sintesi sia guidato da una logica interna, che, se non altro è un tentativo di riordinare e sistemare. Se non possiamo prendere a piene mani la versione di questo testo come riferimento per ricostruire i fatti, dobbiamo ammettere che come manuale ha fatto delle scelte coerenti e consistenti. Quella che vi possiamo vedere è una storia del quinto secolo che pare elaborare le notizie che abbiamo, selezionarle e tenere un racconto lineare di base. Scompare Taso, per esempio. Nemmeno un accenno a Pericle  in questa fase, né come difensore della politica ateniese di spostamento del tesoro, né come comandante della flottiglia oltre le Chelidonie. È una selezione senza incoerenze, che, paradossalmente, se questa fosse stata l'unica fonte a nostra disposizione avrebbe evitato secoli di discussioni. Alcune interessanti osservazioni relative alle mura delle città ioniche, che vanno a supportare la consapevolezza ateniese della fine della guerra dopo il 449, si trovano in Cawkwell 1997, 121s. Lo stesso autore sostiene l'impossibilità di una pace prima del 462/1 per incompatibilità con la lega ellenica di cui Atene faceva ancora parte prima dei fatti di Itome (Thuc. 1.102.4).  La cronologia interna di FGrHist 104 prevede la morte di Serse, seguita dalla morte di Temistocle all'ipotesi dello scontro con Cimone, che combatte comunque all'Eurimedonte contro l'esercito che sarebbe dovuto essere capitanato dal vincitore di Salamina; dopo l'Eurimedonte c'è una tappa a Cipro prima della spedizione in Egitto. Terminata l'impresa troviamo Cipro, la morte di  Cimone e la pace stipulata dal cognato Callia. FGrHist 104 vota 449 a.C. senza dubbio, come  Diodoro e come Platone (Menex. 241e) che ricorda l'Eurimedonte, l'Egitto e Cipro `βασιλέα ἐποίησαν δείσαντα` e Licurgo  (72-3). Badian 1987, 1.  Ormai tutte le prove convergono per una pace scritta in questo momento, come aveva sostenuto già il Wade-Gery argomentando intorno all'assenza di tributo per il 448. Giustamente il Cawkwell (Cawkwell 1997, 115) sottolineava ripetutamente l'assenza di ostilità greco-persiane nella tradizione a noi pervenuta per il periodo successivo al 449 fino al 412 (tra le prove anche ML 44 
            % e ? = ? --&gt; IG I3 35 
 sul nuovo sacerdozio di Atena Nike, prima ancora dell'erezione del tempietto). Così anche Accame 1982, 143, riprendendo la controversia per la ricostruzione dei Templi in Plut. _Per._ 17; Bloedow 1992, 41s rivede la discussione di Badian su Suda K214 e K 1620 affermando che `ἐπὶ Kίμωνος`  implica solo che  Cimone era vivo quando si venne a definire un confine, e che ciò è vero sia per il 466 che per il 450. Badian a sua volta, nonostante la sua convinzione rispetto alle molteplici ''paci'', girava intorno alla pace del 449 per dire che già Eforo aveva parlato della pace dopo l'Eurimedonte, basandosi sul riferimento poi approfondito dal Bloedow, che lega i trattati di Callia con il confine di  Cimone (`ἔταξε καὶ τοὺς ὅρους τοῖς βαρβάροις`).  Diodoro mette più in imbarazzo però rispetto alla datazione dell'Eurimedonte che risulterebbe essere qualche tempo dopo il 464, per ammettere la morte di Temistocle tra quella di Serse e la battaglia. Potrebbe essere un tentativo di strutturare un prima e un dopo. Con l'Eurimedonte si chiuderebbero le vicende di Temistocle e Pausania  per dare inizio a quelle di Cimone. La cronologia interna di FGrHist 104 non è tuttavia impossibile dati i nuovi risultati della ricerca sulla cronologia della pentecontaetia. Si può per esempio pensare che la rivolta di Taso sia finita `μετὰ` la battaglia dell'Eurimedonte. Se infatti ammettiamo i tre anni di Tucidide a partire dal 467 per Taso, i due di Nasso a partire dal 466 tanto caro al Badian (già dal Badian 1987, 7s), e la partenza per l'Egitto nel 463/2, nel 464/3 può essere avvenuto il colpo dell'Eurimedonte raccontato da  Diodoro ed FGrHist 104 seguito immediatamente appunto dalla partenza in soccorso ad Inaro. Nel frattempo si avranno gli eventi che portano all'ostracismo di  Cimone e le due ''regate'' ricordate da  Callistene. La continuazione della spedizione in Egitto blocca la stipula del trattato ed anzi rinforza le speranze persiane, tanto che, anche dopo la vittoria a Cipro del 449, il confine verrà fissato in un punto che esclude l'isola. 
    %%%%%%%        
            \subsubsection{\textgreek{Εὐθὺς ἐστράτευσαν ἐπὶ Κύπρον … τελευτᾶι}}
            La rapida consequenzialità degli eventi è da attribuire alla forma del testo e sicuramente alla brevità di molti dei resoconti su questo periodo di storia. La morte di  Cimone è unanimemente collocata durante questa seconda spedizione a Cipro, sebbene in momenti diversi. Samons 1998, 135 n.22 enumera Thuc. 1.112.4, Diod. 12.3-4, Plut. _Cim._ 19 (=Fanodemo FGrHist 325 F23). La morte di  Cimone non è causa della pace e la nuova duplice vittoria di Salamina Cipria (Diod. 12.3-4) ne è, al massimo, pretesto per la concretizzazione, ma probabilmente questa circostanza ha dato origine ad una ''confusione in buona fede'', per cui la stessa stipula del trattato, almeno limitatamente agli accordi sui confini, veniva attribuita al precedente, grandioso, risultato dell'Eurimedonte, al quale  Cimone era sopravvissuto.  Cawkwell 1997, 115-17. Accame (1982, 126 e 141) riporta già a Stesimbroto gli effetti della pace, mettendo in evidenza Plut. _Per._ 26.1 (FGrHistCont 1002 F8) dove Pericle  naviga verso il mare aperto e si ferma prima della Caria per rispettare i termini della Pace di Callia che sarebbe stata operativa. Anche le richieste di Alcibiade per Tissaferne in Thuc. 8.56 sono un chiaro riferimento ad una pace in essere nella seconda metà del V secolo. Per Parmeggiani 2011, 404 anche nel discorso di Pericle  in Diod. 12.40.3 si fa riferimento a Callia perché il riferimento non può essere alla pace dei trent'anni. Accame 1982, 148-9 pensa che questa circostanza e la perdita di interesse per la vittoria a Cipro (persa con il trattato) abbiano indotto allo spostamento sull'Eurimedonte anche dell'epigramma che infatti troviamo in  Diodoro 11.62 e che sarebbe più appropriato a 12.4. Contro la confusione tra le due spedizioni Parmeggiani 2011, 408. Ulteriore discussione sull'argomento nelle note seguenti, soprattutto p.s.
     %%%%%%%%       
            \subsubsection{`οἱ δὲ Πέρσαι ὁρῶντες κεκακωμένους τοὺς Ἀθηναίους`}
            FGrHist 104 concorda con Tucidide (1.112.2-4) e  Diodoro (12.3-4) ma toglie l'ultima vittoria a  Cimone come Plutarco (_Cim._ 19) facendolo morire prima dello scontro navale a causa di una malattia poco gloriosa. Tuttavia, contrariamente al racconto di Plutarco, i Persiani lo sanno, ed è proprio per questo che attaccano. Perdono comunque e devono venire a patti concreti nonostante la vittoria in Egitto, ma la perdita di  Cimone fa scegliere a Callia di lasciare Cipro, poco difendibile, ai Persiani. Green 2006, 182 n. 17.
    %%%%%%%%        
            \subsubsection{`στρατηγὸν αἱροῦνται Καλλίαν τὸν ἐπίκλην Λακκόπλουτον`}
            %% FORSE QUESTO PASSO è DA SPOSTARE più decisamente nel discorso sulla tradizione di Erodoto?? 
            Callia, cognato di  Cimone (Plut. _Cim._ 4.8), è coinvolto anche nelle trattative della pace trentennale nel 446 come _proxenos_ di Sparta (Diod. 12.7) ed è ricordato anche da Erodoto ([7.151](http://data.perseus.org/citations/urn:cts:greekLit:tlg0016.tlg001.perseus-grc1:7.151)) per inciso, assieme ad una legazione argiva con tutta probabilità entrambe arrivate a Susa dopo l'Eurimedonte e prima della sconfitta in Egitto. La sua nomina ha fatto pensare ed è considerata straordinaria: qui infatti è in veste di stratego e non uno dei `πρέσβεις αὐτοκράτορες` come in Diod. 12.4. La tradizione rispetto al suo arricchimento (Plut. _Arist. _25; Lys. 19) con il tesoro trovato a Maratona (Plut. _Arist. _5 = fr. 696 KA, adespoto) dove aveva combattuto è riportata anche da Suda, negli stessi termini  (Samons 1998, 132). Badian, criticato nell'articolo della Samons, aveva messo in dubbio anche la sua selezione come ambasciatore della pace per confermarla dopo aver argomentato il primo trattato nel 466. Samons 1998, 139) giustamente nota che, se ci fosse stato un primo trattato, non si sarebbe chiamato a stipularne un secondo proprio colui che aveva siglato quello disatteso pochi anni prima. Cfr. _Them. Ep._  9. e CulassoGastaldi 1990, 164-6. 
      %%%%%%%%%%      
            \subsubsection{`ὁ Καλλίας ἐσπείσατο πρὸς Ἀρταξέρξην`}
            Nulla sappiamo dei  `λοιποὺς Πέρσας`, dobbiamo però immaginare i satrapi e i dignitari di Artaserse. È una specificazione di certo non neutra, ma non meglio precisabile per ora. I patti stipulati da Callia con Artaserse  nel 449 hanno posto una serie di problemi legati anche alla costruzione di un luogo topico della retorica ateniese di IV secolo da Isocrate in poi: il confronto della Pace di Antalcida con quella di Callia. Bosworth 1990, 4-5 sottolinea l'indipendenza dalla tradizione storica dei luoghi sviluppati dalla retorica. Il dubbio sulla veridicità o meno della pace riposa ora solo sull'omissione di Tucidide, alla quale sono state offerte numerosissime motivazioni, nessuna delle quali individua un vero dato nel silenzio dell'ateniese. Non solo grazie alla discussione sorta dalla proposta di una triplice pace da parte del Badian, il dibattito sulla Pace di Callia (che considero a partire da Eddy 1970) ha preso una svolta negli ultimi anni, risvegliando critici, detrattori, sostenitori e revisori. Dobbiamo un passo importante soprattutto ad un lucido contributo del Krentz che elimina dalla controversia la pretesa critica di Teopompo  FGrHist 115 F153 (e.g. Ferretto 1984, 48 e 80. Krentz (Krentz 2009, 231s) slegando F153 da F154 ipotizza in modo molto convincente che il riferimento ai trattati `αἱ πρὸς βασιλέα Δαρεῖον Ἀθηναίων πρὸς Ἕλληνας συνθῆκαι` (Teone Progym. 2 II 67, 22 Sp = FGrHist 115 F 153) sia l'accordo degli Ateniesi del 507/6 (Hdt. [5.73](http://data.perseus.org/citations/urn:cts:greekLit:tlg0016.tlg001.perseus-grc1:5.73)). Teopompo  avrebbe criticato la falsificazione dell'autorizzazione che fu data, portando Atene dalla parte dei Persiani. Con questo cadono tutte le argomentazioni basate su questo frainteso di interpretazione del frammento teopompeo (e.g. Mazzarino 1966, 395 e Shrimpton 1991, 80) e anche il ragionamento critico sul trattamento della Pace di Callia da parte di Eforo in Parmeggiani 2011, 404-10 pur non implicando una revisione del discorso sull'attitudine nei confronti della documentazione epigrafica (Parmeggiani 2011, 170-2 su Eforo FGrHist 70 F 106). Anche la testimonianza plutarchea su Cratero (FGrHist 342 F13, Erdas 2002, 29) resta dunque a sé stante. Circolava una versione scritta, probabilmente nel IV secolo, ma non era sottoposta a critica se non in modo generico dai commenti di Teopompo  ed Eforo sulle iscrizioni e sull'atteggiamento della ''retorica da panegirico'' che di certo è responsabile per la lunga tradizione del confronto tra questa pace e quella di Antalcida (Isocr. 4.118; 7.80; 12.59).  I suoi risultati rispondono all'invito alla riduzione delle complessità argomentative del Bosworth che ha ridotto l'impatto del passo plutarcheo che 
% Callistene  nel rigo seguente era Callistrato... sarebbe da controllare.            
discute  Callistene  e Cratero. Bosworth (Bosworth 1990, 1) prova ad escludere il passo di  Callistene  preso in considerazione da Plutarco (_Cim._ 13 = FGrHist 124 F16). Se è del tutto condivisibile la collocazione del passo nell'opera di  Callistene  e il conseguente disinteresse dell'autore per la pace stessa, nonché la lettura di `οὑ φησι` in opposizione ad `ἔργῳ δὲ` per indicare la pace ''di fatto'', non c'è bisogno di eliminare dalla ricostruzione queste informazioni, poiché non è contraddittorio con esse, ma solo con il passo di Plutarco in cui si trova ed al quale sarà da ricondurre l' ''errore'' (è critico rispetto a questa lettura Parmeggiani 2011, 410 n.66, ma le prove interne a Plutarco portate dal Bosworth sono decisive per la sua lettura del passo che non può essere pensato come _ipsissima verba_ di  Callistene ). Nella fissazione del confine effettivo, protetto con spedizioni militari aggressive fino alla morte di  Cimone e non difeso dal re avvilito (`τοῦτο τὸ ἔργον` [scil. La battaglia dell'Eurimedonte] `οὕτως ἐταπείνωσε τὴν γνώμην τοῦ βασιλέως`). Il nesso \textgreek{οὕτως … ὥστε...} stabilisce di certo una consequenzialità diretta tra questa condizione del re e la stipula della pace, ma è lo stesso timore che anche in Platone viene attribuito al re dopo gli scontri di Cipro del 449. Hdt. [7.151](http://data.perseus.org/citations/urn:cts:greekLit:tlg0016.tlg001.perseus-grc1:7.151) (`ἑτέρου πρήγματος εἵνεκα ἀγγέλους Ἀθηναίων, Καλλίην τε τὸν Ἱππονίκου καὶ τοὺς μετὰ τούτου ἀναβάντας`), giustamente da collocare dopo la morte di Serse, è da pensare in un momento successivo anche alla battaglia dell'Eurimedonte, che, nella cronologia del nostro testo, è poco tempo dopo. Ma la presenza a corte non data il trattato, attesta il dialogo che ad esso avrebbe poi portato. Prendendo in considerazione assieme alle spedizioni oltre le Chelidonie e Cianee di Pericle  ed Efialte citate da  Callistene  (avvenute gioco forza prima della morte di quest'ultimo nel 462/1), anche la spedizione in Egitto secondo le nuove date proposte da Kahn (Kahn 2008, 440: la chiamata degli Ateniesi è posta nel 463/2), una pace a cui attenersi non c'era al momento in cui Plutarco la ricorda, ma c'era appunto un confine che sarebbe poi stato confermato, perdendo Cipro, dalla Pace di Callia. La punizione che Demostene (19.273) riferisce per il direttore delle trattative è probabilmente dovuta all'aver rinunciato all'isola che era stata conquistata. L'altare e la statua di Callia (Paus 1.8.2) sono chiaramente frutto del clima culturale successivo alla Pace di Antalcida, anche perché, come ricorda il Musti (Musti 1982, 286), era in vigore un divieto attestato da uno scolio a D. 21.534. CulassoGastaldi 1990, 175. D'altro canto è stata riaperta anche la questione sul decreto di Epilico e l'autenticità del _De pace_ andocideo. Blamire 1975. Harris (Harris 1999, 123s) dimostra in modo molto convincente che l'ambasceria proveniva dalla Persia e non vi era stata mandata. Se Andocide 3.29 (o comunque il testo di questa orazione se essa non è di Andocide) si riferisse ad una pace con il re del 424-3 si tratterebbe certo di un rinnovo della pace del 449. La cronologia interna di FGrHist 104 è riassunta nell'appendice \ref{cronofgrhist104}. 
          %%%%%%%%%%%%  
            \subsubsection{`ἐγένοντο δὲ αἱ σπονδαὶ ἐπὶ τοῖσδε`} 
            È da notare l'introduzione e conclusione epigrafica dei termini del trattato (\textgreek{ἐγένοντο δὲ αἱ σπονδαὶ ἐπὶ τοῖσδε·  … καὶ σπονδαὶ οὖν ἐγένοντο τοιαῦται}) notata anche per le mura di Atene (5). La formula è solo leggermente diversa da quella in Suda K 1620 dove si parla degli `ὅροι` che si erano venuti a consolidare dopo la battaglia dell'Eurimedonte. Questi limiti non sono molto diversi da quelli che riporta  Diodoro (12.4) che parla di Callia come funzionario di Atene; elementi stabili di ogni versione sono comunque Faselide e l'esclusione di Cipro (Cfr. p.). Meiggs 1972, 477-8 che organizza le fonti a seconda dei confini di terra e per mare. Nessun indizio permette di identificare il `Νέσσου ποταμοῦ` indicato da FGrHist 104.
          %%%%%%%  
            \subsection*{L'`Ἑλληνικὸς πόλεμος`: guerra sacra e Coronea}
            Si torna, a ritmo di `μετὰ δὲ ταῦτα` e `καὶ μετὰ ταῦτα εὐθὺς` all'elenco del §12. Sulla guerra sacra tuttavia abbiamo un'altra interessante testimonianza di Filocoro (_Scholia in Aves_ 556b\indexp{Scholia ad Aristoph.!_Aves_ 556b|ca} = FGrHist 328 F 34a-c, commento al frammento in Costa 2007, 247s)  con discussione delle diverse versioni di Tucidide (Thuc. 1.112.5), Eratostene e Teopompo  nel XXV dei _Philippika_.  Hammond 1937; Pownall 1998.
            Per Filocoro ci sono due guerre a distanza di 3 anni, non un unico episodio che alterna gli interventi di Sparta e Atene come sembra da Tucidide e Plutarco (`εὐθύς`). La tradizione è indecisa persino sui ''nemici'' dei Focesi che sono alternativamente Beoti, Delfi, Locresi, etc. La datazione di questi episodi è solitamente basata su Tucidide. Essendo dopo Cipro e la seconda spedizione in Egitto, che devono essere circa nel 449 per via dell'anno arcontale di Diodoro, allora la guerra sacra dev'essere datata al 448. La cronologia interna di FGrHist 104 è riassunta nell'appendice \ref{cronofgrhist104}.  
            L'agguato di Coronea (Thuc. 1.113, Diod. 12.6), causa la perdita della Beozia e probabilmente anche del santuario di Delfi se i Focesi in Thuc. 2.9 fanno parte dell'alleanza peloponnesiaca, rispetto alla situazione precedente in cui il santuario era indirettamente controllato dagli Ateniesi tramite la concessione (Thuc.: `Ἀ. παρέδοσαν Φ.`; Fil. 34a `Φωκεῦσι πάλιν ἀπέδωκαν`; Plut. `πάλιν εἰσήγαγε τοὺς Φωκέας`) ai Focesi. In Thuc. 1.113.4 (`οἱ ἄλλοι πάντες αὐτονόμοι πάλιν ἐγένοντο`) forse si intendono anche i Delfi.
            Doegnes, nella sua discussione del testo propone che la divergenza (singolo evento in FGrHist 104 e due distinte guerre sacre a distanza di due anni in Filocoro) sia comunque riconducibile ad una fonte comune. Il punto di contatto testuale non pare però del tutto probante. Filocoro, come per Fidia avrebbe modificato nel suo racconto quella medesima fonte che troviamo invece incontaminata in FGrHist 104. Si veda anche il commento ai tre passi che costituiscono il frammento filocoreo in Costa 2007, 247-254. Per l'analisi storica degli eventi qui coinvolti, è importante lo studio di Sordi 2002, 225s. 
            %%%%%%
            \subsection*{L'`Ἑλληνικὸς πόλεμος`: Eubea e Samo}
            In Tucidide Tolmide cade a Cheronea, ed è qui inserito nel punto sbagliato probabilmente, ma senza che ciò crei alcun danno alla coerenza interna del racconto. Thuc. 1.108.5; Diod. 11.84 riporta la morte di Tolmide ad Oinofita, e così Pausania 1.27.5. Della spedizione di Tolmide attraverso il Peloponneso parla anche Eschine (2.75): di nuovo è la tradizione retorica che offre i paralleli più vicini per modalità e contenuti al nostro testo. È possibile che in un momento non meglio precisabile della tradizione si sia avuta una confusione con la successiva spedizione di Pericle  (Thuc. 1.111; Diod. 11.85-88; Plut. Per. 19).
            L'uso di `πάλιν` in questo testo è stato visto anche nel commento al paragrafo 5 a p. (`πάλιν αὐξηθῆναι`). Dopo la pace dei trent'anni (Thuc. 1.115.1; Diod. 12.7) viene brevemente nominata la rivolta di Samo (Thuc. 1.115.2-117, Diod. 12.27-28), con alcuni riferimenti cronologici confusi, da ricollegare al testo tucidideo con tutta probabilità (Cfr. Thuc 1.87.6 e 2.2). Non sappiamo se la risposta data da Pericle  a Sofocle in Plut. _Per._ 8 sia da attribuire come il seguito del testo a Stesimbroto (FGrHistCont 1002 F9) e a questa spedizione comune ai due, ma è probabile.
            Viene infine preannunciato il `Πελοποννησιακὸς πόλεμος`. Su questa denominazione si è brevemente soffermato de Ste. Croix, che la circoscrive al I a.C. (e.g. Diod. 12.37.2, Strabone 13.1.39, Cic. _De Rep._ 3.44). DeSteCroix 1972, 294-5. Harpocraz. s.v. `Ἀρχιδάμειος πόλεμος` riporta tra coloro che preferiscono questa denominazione Tucidide, ma anche Eforo e Anassimene. Schepens 2007, 79. Il fatto che si trovi in  Diodoro proprio nel contesto in cui la troviamo in FGrHist 104 non può essere casuale. Già da questo paragrafo tuttavia si può notare una maggiore presenza di Eforo. Lo storico di Cuma è ricordato da Plutarco come fonte proprio per i macchinari (`μηχαναῖς θαυμασταῖς`) impiegati da Pericle  durante la rivolta di Samo in _Per._ 27. \cite[420, n. 116 e 426]{Parmeggiani2011}: queste macchine sarebbero una critica alla pigrizia della società ateniese. La citazione immediatamente successiva, Plut. _Per._ 28 (= Eforo FGrHist 70 F 195) è negativa e ci dice che non aggiunse i dettagli che si trovano in Duride (FGrHist 76 F 67). Eforo è in questo passo incluso tra Tucidide ed Aristotele: sembra che il modello di confronti sia rispettato e che venga controllata la fonte di quinto, la storiografia di scuola retorica, quella di scuola filosofica e Duride (Walbank 2011, 394). Ma se la denominazione della guerra fa riferimento ad un periodo di rielaborazione delle informazioni, ed è l'unico legame con  Diodoro per questo passo e, d'altro canto, in Plutarco Eforo è nominato per altre informazioni e insieme ad altri tre storici, non è dato vedere il passaggio di Eforo in questo passo, né c'è terreno per estendere il ragionamento che sul paragrafo seguente può invece essere fatto.     
            
            \subsection*{Le cause del `Πελοποννησιακὸς πόλεμος`}
            
            La prima causa della guerra del Peloponneso è Pericle. Prima in un elenco logico, forse in uno politico, certamente non cronologico, né necessariamente più importante delle altre né quella più facilmente attribuibile. La causa di politica interna FGrHist 104 ce la racconta in un modo che ci è noto da Filocoro,  Diodoro e Plutarco in quattro punti principali: 
            
            *  la questione di Fidia e dei rendiconti
            *  il decreto megarese 
            *  due passi di Aristofane  portati come ''prova'' (e fonte?) delle dichiarazioni su Fidia e sul decreto per i quali è necessario tenere presente la diversa impostazione eziologica di Eforo (Parmeggiani 2011, 433)
            	*  la soluzione proposta da Alcibiade in esergo (piuttosto che a premessa come in Plutarco)
	 
	È una collezione dei punti salienti e memorabili della vicenda, priva di complessità ed articolazione come siamo ormai abituati a leggere in FGrHist 104. Soprattutto la presenza dei passi di Aristofane  fa pensare immediatamente a  Diodoro e Plutarco ed è normale che questo passo del nostro, dove essi sono più estesi, abbia fatto pensare. La struttura stessa dell'argomentazione, come ho cercato di far notare, è indicatore di un passaggio intermedio nell'elaborazione del passo argomentativo, in cui la letteratura erudita di contorno al testo di Aristofane  viene usata come materia per la compilazione di opere che si servono del testo commentato come prova. Aggiungiamo a questa presenza problematica che la fonte dichiarata da  Diodoro alla fine di questo passo è Eforo (Schepens 2007, 88; Parmeggiani 2011, 420) e capiamo subito quale sia il nodo, fortunatamente dipanato da Parmeggiani nel suo recente studio sullo storico di Cuma (Parmeggiani 2011, 417-458). Infatti, se Plutarco e  Diodoro hanno la stessa fonte, anche FGrHist 104 ''deve'' essere accostato ad Eforo e, poiché è più rudimentale e scarno, è anche più pulito dalle interferenze successive e quindi potenzialmente più ''vicino'' al celebre autore disperso (Parmeggiani 2011, 420 n.114): congettura che penso sia stata sufficientemente sottoposta a critica. Anche Filocoro usa Eforo perché racconta la storia di Fidia in due versioni, una delle quali, la seconda è simile al racconto di Diodoro. Tucidide è fuori gioco, almeno per questo paragrafo e ci lascia vedere Eforo o una sua epitome. Questa la tesi corrente che è indubbiamente possibile, ma forse da ridimensionare leggermente tenendo l'attenzione su FGrHist 104 piuttosto che sullo storico di Cuma, o ancor meglio su Aristofane  che detiene il centro indiscusso dell'argomentazione. Diversamente da Plutarco e Diodoro, dove è il ricorrente concetto di `λόγου δεινότης` a far da perno (Parmeggiani 2011, 419). Questa è una differenza fondamentale tra FGrHist 104 e  Diodoro 12.38s. Non si può nemmeno intravedere quello sforzo di chiarimento ed elaborazione del discorso di Tucidide portato avanti da Eforo. Il modo in cui i passi di Aristofane sono utilizzati da Diodoro e FGrHist 104 lo dimostra. A prescindere dai passaggi che hanno permesso di pervenirvi, il risultato dell'elaborazione è completo e documentato, conserva sia l'opinione popolare che attribuisce la colpa all'uomo in vista, sia l'approfondimento di politica inter-statale (Corcira e Potidea), sia l'analisi storica (la vera causa), sia l'aneddoto. Testimonia dunque una serie ormai cristallizzata di argomenti, digressioni, personaggi e temi attorno al topos delle cause della guerra del Peloponneso. 
            
            \subsubsection{`ἁλόντος τοῦ Φειδίου ἐπὶ νοσφισμῷ, εὐλαβηθεὶς ὁ Περικλῆς`}
            Pericle, nel racconto di  Diodoro tiene a cuore un vecchio consiglio datogli da Alcibiade quando era piccolo, ma qui è semplicemente un ragionamento di convenienza sul pericolo di essere trascinato nei problemi di Fidia. 
            Questo episodio non è da collocare per forza a ridosso dei vari decreti megaresi.  Diodoro 12.38 implica una serie di salti cronologici che lasciano ampio spazio alla cronologia di Filocoro (438/7). Parmeggiani 2011, 429 n.155 argomenta invece a favore della datazione tradizionale c. 432 a.C.. Un'analisi alternativa basata sull'osservazione del metodo di Plutarco che fa confluire eventi di diversi momenti in un'unica narrazione si trova in Stadter 1989. Del resto non stupisce questo metodo, nemmeno vedendolo utilizzato sia da  Diodoro che da Plutarco, giacché altro non è che una delle necessità del raccontare la storia. In FGrHist 104 non c'è specifica connessione tra Fidia e Pericle  se non dal punto di vista professionale. Parmeggiani 2011, 430 con bibliografia per i capi d'accusa dei vari processi a Pericle. Tanto meno da quello dei capi d'accusa: semplice furto. I titoli professionali attribuiti a Pericle  e Fidia sono vari, ma estremamente generici nel nostro testo e credo non permettano di argomentare una data di composizione sulla base del loro utilizzo. Marr 1998.
            
            \subsubsection{`ἐπολιτεύσατο τὸν πόλεμον τούτον`}
            Su queste parole è caduta l'attenzione di Schepens che vi riconosce un'indicazione dell'interpretazione del ruolo pericleo da parte di Eforo ''Pericle fece politica facendo questa guerra''. Contro questa opinione Parmeggiani 2011, 455. Non credo tuttavia si possa attribuire tale peso a questa espressione che può essere molto più neutra, considerando il testo di FGrHist 104 in generale. Il decreto cui si fa riferimento è da identificare probabilmente come il primo di quelli ricordati da Plutarco, quello in cui Pericle  si rifiuta di girare la stele dall'altra parte, ma che non è problematico quanto i successivi. Plut. _Per._ 30, prima e separatamente dalla questione di Fidia con il commento di Stadter 1989. e McDonald 1994. Il legame tra le varie cause è di primaria importanza in  Diodoro e Plutarco che strutturano il discorso con orientamenti e intenti diversi. 
            
            \subsubsection{`ὁ τῆς ἀρχαίας κωμῳδίας ποιητὴς λέγων οὕτως`}
            Perché vengano selezionati passi diversi da  Diodoro  è ben spiegato da Parmeggiani attraverso la centralità della descrizione dell'abilità oratoria di Pericle. Su questo passo di Aristofane, si veda anche Cassio 1982, 23s. In  Diodoro i passi di Aristofane  sono separati dalla considerazione politica, qui invece fanno da prove per una delle cause. Un frainteso, che può avvenire solo ad opera di  Diodoro su FGrHist 104 e non viceversa apparentemente, ma che nella complessità della tradizione può essere avvenuto in diversi momenti.   L'intero verso 608 è omesso da  Diodoro e da Plutarco, mentre parrebbe proprio essere, per la versione di FGrHist 104 il motivo del  `εὐλαβηθεὶς`, il verso al quale il commento viene accostato, la vera accusa personale diretta a Pericle  che non c'è nelle altre fonti. FGrHist 104 come la seconda versione riportata dallo scolio lega i due eventi, Fidia e il decreto. I `τινες` di Filocoro potrebbero forse essere già l'Eforo che riconosciamo tramite  Diodoro e Plutarco dietro alle parole di questo paragrafo, ma anche il Teopompo  che abbiamo visto in F34 di Filocoro. 
                        
            \subsubsection{`καὶ πάλιν ὑποβάς`}
            Il passo dagli _Acarnesi_, non serve all'ottica presente in Diodoro. Parmeggiani 2011, 436. È fondamentale invece nel contesto delle contingenze a causa della guerra e Plutarco ne conserva i primi quattro versi soltanto, prima della parte effettivamente interessante per il discorso eziologico. Essi selezionano a seconda dell'impronta data alla loro opera e si possono apprezzare le differenze di impostazione e di intenti assenti in FGrHist 104, che riprende tutto il testo.
            
            \subsubsection{\textgreek{μὴ σκέπτου πῶς ἀποδῷς … ἀλλὰ πῶς μὴ ἀποδῷς}}
       Posto a questo punto del racconto perde il suo significato narrativo che conserva con grande efficacia in  Diodoro 12.38, con un vago erodoteo ricordo della giovane Gorgo all'incontro tra Cleomene ed Anassagora: è solo un aneddoto curioso ricordato per il piacere dell'arguzia ad alleggerire la lista e le citazioni. Plutarco conosce questo materiale ma lo usa nella ''successiva'' _Vita_, quella di Alcibiade, nel contesto della dimostrazione della sua giovanile sagacia (Plut. _Alc._ 7.2). Questo è l'ultimo dei detti famosi di personaggi illustri conservato da FGrHist 104.
            
            \subsubsection{`δευτέρα δὲ αἰτία φέρεται καὶ Κερκυραίων καὶ Ἐπιδαμνίων τοιαύτη`}
            C'è forse un po' di indecisione nel definire questo episodio nel doppio `καί`, ma ciò che più importa è che l'anticipazione della sezione periclea alle cause tucididee della guerra, ridotte alle due principali e a quella ''vera'' non è certo casuale. FGrHist 104 torna al ritmo da elenco. Per i fatti di Corcira ed Epidamno si riduce dunque ad elencare pazientemente attori e alleanze, scontri e contingenti  (anche se non tutte coerenti con Tucidide, cfr. Thuc. 1.118 e 24-55, Diod. 12.31-33), per poi mettere l'accento sullo scioglimento dei patti, e quindi il nulla osta alla guerra non più impedita da vincoli di giuramento inter-statale. Una volta che gli Ateniesi hanno ufficialmente infranto i patti è questione di un pretesto. Le braci su cui soffia il Pericle di Aristofane sono vive. Nella sintesi di nuovo vediamo l'organizzazione del discorso, semplice e precisa al punto della banalità formale, ma efficacemente strutturata. Se il riferimento è alla pace dei trent'anni, sciolta `ἐν δὲ τῷ αὐτῷ ἔτει` della spedizione a Samo, allora la cronologia tradizionale vacilla di nuovo, ma abbiamo visto che i quattordici anni di Samo non sono del tutto affidabili e non è il caso di aggiungere calcoli ipotetici. È invece rilevante che la struttura di flashback della narrazione, come in Diodoro, utilizzata per la questione di Pericle, qui sia usata all'inverso, anticipando l'episodio di Fidia e la questione megarese e aggiungendo Corcira e Potidea. Probabilmente l'ordine selezionato è di importanza ascendente. I patti sciolti sono quelli stipulati al paragrafo 15 e con quel periodo dobbiamo intendere la sincronia di questi eventi per FGrHist 104 che cerca così di ritornare sul percorso tracciato dopo l'elenco delle cause.
            
            \subsubsection{`Ποτίδαια πόλις ἄποικος`}
           Probabilmente parte della sintesi estrema è dovuta anche al fatto che lo spazio è finito. Il conflitto di interessi e la scomoda posizione istituzionale che fanno di Potidea il _casus belli_ esemplare non emergono dal testo che lo riduce ad un semplice episodio di assedio. Thuc. 1.65; Diod. 12.34 and 37.
            
            \subsubsection{`ἡ καὶ ἀληθεστάτη...`}
            Quasi proverbiale la vera causa tucididea della guerra del Peloponneso (Thuc. 1.23.6), che, sebbene qui sia persa, non è più `πρόφασις`, ma `αἰτία` (Parmeggiani 2011, 444): è Sparta che osserva e considera l'_auxesis_  ateniese. Questa in FGrHist 104 è spezzettata, nelle ultime parole del manoscritto, come nella sintesi del discorso pericleo in Diod. 12.40: navi, ricchezze e alleati. 
            }