\subsubsection*{1} … \textgreek{αἰτησάμενος γὰρ μίαν ἡμέραν μόνην}\footnote{\textgreek{μίαν ἡμέραν μόνην} \textbf{Jac} \textgreek{μιᾶς ἡμέρας μονήν} \textit{vel} \textgreek{μίαν ἡμέραν μένειν} \textbf{Mül}} \textgreek{ἔπεμψε κρύφα Σίκινον τὸν ἑαυτοῦ παιδαγωγὸν πρὸς Ξέρξην, ἐγκελευσάμενος αὐτῷ ἐπιτίθεσθαι τοῖς Ἕλλησιν καὶ ναυμαχεῖν, δηλῶν τὸν μέλλοντα δρασμὸν ἀπὸ Σαλαμῖνος. } \textgreek{Ὁ δὲ Ξέρξης νομίσας τὸν Θεμιστοκλέα μηδίζοντα ταῦτα ἀπεσταλκέναι, ἔπεμψε τὰς ναῦς ἐπὶ Σαλαμῖνα καὶ ἐκυκλώσατο τοὺς Ἕλληνας εἰς τὸ μένειν αὐτούς.} (2) \textgreek{ἐσπούδασεν δὲ ὁ Ξέρξης ζεῦγμα κατασκευάσας πεζῇ ἐπιβῆναι ἐπὶ τὴν Σαλαμῖνα ὃν τρόπον διῆλθε \Ladd{ἐπὶ}}\footnote{\textgreek{\ladd{ἐπὶ}} \textbf{Mül}}  \textgreek{  τὸν Ἑλλήσποντον, καὶ μέρος τι ἔχων}\footnote{\textgreek{ἔχων} \textbf{Jac} \textgreek{Ἔχων \ladd{τῶν νεῶν}}? \textbf{Jac}} \textgreek{ἧκεν κατὰ τὸ Ἡράκλειον· ἐπειδὴ δὲ ἀδύνατον ἦν τὸ πᾶν}\footnote{\textgreek{τὸ πᾶν} \textbf{Jac} \textgreek{τὸ πᾶν \ladd{τὸν πόρον}}\textbf{Mül}} \textgreek{γεφυρωθῆναι, καθεζόμενος ἐπὶ τοῦ Πάρνηθος ὄρους (ἐγγὺς δὲ ἦν τοῦτο) ἑώρα τὴν ναυμαχίαν.}  (3) \textgreek{ἤρξαντο δὲ τοῦ ναυμαχεῖν Ἀμεινίας Ἀθηναῖος, υἱὸς μὲν Εὐφορίωνος, ἀδελφὸς δὲ Κυνεγείρου καὶ Αἰσχύλου τοῦ τραγῳδοποιοῦ.  ἐνίκων μὲν οὖν πάντες οἱ Ἕλληνες, ἐκπρεπέστερον δὲ οἱ Ἀθηναῖοι.} (4) \textgreek{συνεστηκυίας}\footnote{\textgreek{συνεστηκυίας} \textbf{Jac} \textgreek{ἐνεστεκυίας}? \textbf{Mül}} \textgreek{δὲ τῆς μάχης ὁ Ξέρξης ἱκανὰς μυριάδας ἐπεβίβασεν εἰς τὴν πλησίον νησῖδα παρακειμένην τῇ Σαλαμῖνι, ὀνομαζομένην Ψυτ\ladd{τ}άλειαν, ἐκπληττόμενός}\footnote{\textgreek{ἐκπληττόμενός} \textbf{Jac} \textgreek{Ἐκπλήττων}  \textbf{Büch} \textgreek{ἐκπληξόμενος} \textit{vel} \textgreek{ἐκβληθησομένους τε τοὺς Ἕλληνες καίνειν βουλόμενος καὶ}? \textbf{Mül}}\textgreek{τε τοὺς Ἕλληνας καὶ βουλόμενος τὰ προσφερόμενα ναυάγια τῶν βαρβάρων ἀνασώζεσθαι.  Ἀριστείδης δὲ Ἀθηναῖος, υἱὸς Λυσιμάχου, καλούμενος δίκαιος, ἐξωστρακισμένος ἐκ τῶν Ἀθηνῶν καὶ ὑπάρχων ἐν Αἰγίνῃ τότε, συμμαχῶν καὶ αὐτὸς τοῖς Ἕλλησιν παρεγένετο πρὸς Θεμιστοκλέα καὶ στρατὸν αὐτὸν ᾔτησεν}\footnote{\textgreek{ᾔτησεν}  \textbf{We} \textbf{Jac} \textgreek{ἢτησαν}  \textbf{P}} \textgreek{εἰς τὸ ἀμύνασθαι τοὺς ἐν τῇ Ψυτ\ladd{τ}αλείᾳ· ὁ δὲ καίπερ ἐχθρὸς αὐτῷ γεγονὼς ὅμως ἔδωκε. Λαβὼν δὲ Ἀριστείδης ἐπέβη εἰς τὴν Ψυτ\ladd{τ}άλειαν καὶ πάντας τοὺς βαρβάρους ἐφόνευσε, καὶ μέγιστον τοῦτο τὸ ἔργον ἐπεδείξατο ὑπὲρ τῶν Ἑλλήνων. (5) διασημότερον δὲ ἠγωνίσα\Ladd{ν}το}\footnote{\textgreek{ἠγωνίσα\Ladd{ν}το} \textbf{Jac} \textgreek{ἠγωνίσαντο \ladd{Ἀθηναίους}} \textbf{Mül}} \textgreek{τῇ ναυμαχίᾳ καὶ ἠρίστευσεν Ἀμεινίας, τῶν δὲ βαρβάρων γυνὴ Ἁλικαρνασὶς τὸ γένος, ὄνομα δὲ Ἀρτεμισία, ἥτις διωκομένης τῆς νεὼς αὐτῆς καὶ κινδυνεύουσα ἀπολέσθαι τὴν ἔμπροσθεν ναῦν ἰδίαν οὖσαν ἐβύθισεν· ὁ δὲ Ἀμεινίας δόξας σύμμαχον εἶναι τῶν Ἑλλήνων, ἀπετράπη τοῦ διώκειν. ὁ δὲ Ξέρξης θεασάμενος τὸ γενόμενον εἶπεν· οἱ μὲν ἄνδρες μοι γυναῖκες γεγόνασιν, αἱ δὲ γυναῖκες ἄνδρες. (6) ἠρίστευσαν δὲ τῶν Ἑλλήνων ἐκπρεπέστερον μετὰ Ἀθηναίους Αἰγινῆται, οἵτινες κατὰ τὸ στενὸν τοῦ πορθμοῦ κατατάξαντες ἑαυτοὺς πολλὰς τῶν βαρβάρων νῆας φευγούσας εἰς τὸ στενὸν παραδεχόμενοι ἐβύθιζον.  (7) ἡττηθέντων δὲ τῶν βαρβάρων καὶ φυγόντων οἱ Ἕλληνες ἐβούλοντο λύειν τὸ ἐπὶ τοῦ Ἑλλησπόντου ζεῦγμα καὶ καταλαμβάνεσθαι Ξέρξην ἐν  τῇ Ἑλλάδι, Θεμιστοκλῆς δὲ οὐκ οἰόμενος ἀσφαλὲς εἶναι οὐδὲ τοῦτο, δεδοικὼς μήποτε, ἐὰν ἀπογνῶσι τὴν σωτηρίαν οἱ βάρβαροι, φιλοκινδυνώτερον  ἀγωνίσονται ἐξ ὑποστροφῆς, ἀντέπρασσε. κεκυρωμένων δὲ οὐδὲν ἰσχύων  ἔπεμψε κρύφα Ξέρξῃ δηλῶν ὅτι μέλλουσιν οἱ Ἕλληνες λύειν τὸ ζεῦγμα·  ὁ δὲ φοβηθεὶς ἔφευγεν. (8) ἐν δὲ τῇ ναυμαχίᾳ τῇ περὶ Σαλαμῖνα καὶ οἱ θεοὶ συνεμάχησαν τοῖς Ἕλλησιν. Δίκαιος}\footnote{\textgreek{Δίκαιος} \textbf{Jac} \textgreek{ϊνεος} \textbf{P}} \textgreek{γὰρ ὁ Θεοκύδους, ἀνὴρ  Ἀθηναῖος, ἔφη θεάσασθαι ἐν τῷ Θριασίῳ πεδίῳ κονιορτὸν ὡς δισμυρίων ἀνδρῶν ἀναφερόμενον ἀπ’ Ἐλευσῖνος, βοώντων τὸν μυστικὸν Ἴακχον· τὸν δὲ κονιορτὸν νεφωθέντα ἐμπεσεῖν ἐς τὰς ναῦς τῶν Ἑλλήνων.} 

\subsubsection*{2} \textgreek{Φεύγοντος δὲ τοῦ Ξέρξου Μαρδόνιος, υἱὸς Γωβρύου τοῦ καὶ αὐτοῦ ἐπιθεμένου τοῖς \ladd{μά}γοις, συμπ\ladd{επει}κὼς}\footnote{\textgreek{\ladd{μά}γοις, συμπ\ladd{επει}κὼς} \textbf{Jac} \textgreek{συμ\ladd{πεπει}κὼς} \textbf{Mül}} \textgreek{καὶ αὐτὸς Ξέρξην στρατεῦσαι  ἐπὶ τὴν Ἑλλάδα, ᾐτιᾶτο τὸ πολὺ πλῆθος τῶν βαρβάρων ὡς αἴτιον γεγονὸς τῆς ἥττης ὑπέσχετό τε νικήσειν τοὺς Ἕλληνας, εἰ λάβοι στρατοῦ μυριάδας λ. (2) λαβὼν δὲ ὁ Μαρδόνιος ἔπεμψε πρῶτον πρὸς Ἀθηναίους Ἀλέξανδρον τὸν Μακεδόνα, τὸν Φιλίππου πρόγονον, ὑπισχνούμενος δώσειν αὐτοῖς μύρια τάλαντα καὶ γῆν ὅσην αὐτοὶ βούλονται}\footnote{\textgreek{βούλονται} \textbf{Jac} \textgreek{ἄν αὐτοι βούλοιντο} \textbf{Mül}} \textgreek{τῆς Ἑλλάδος τηρήσειν τε ὑποσχόμενος}\footnote{\textgreek{Ὑποσχόμενος} \textbf{Jac} \textgreek{ὑποδεχόμενος}  \textbf{Büch}} \textgreek{καὶ τὴν ἐλευθερίαν αὐτοῖς καὶ τὴν αὐτονομίαν, εἰ ἕλοιντο μένειν ἐφ’ ἑαυτῶν καὶ μὴ συμμαχεῖν τοῖς Ἕλλησιν. ἐπειδὴ δὲ ὁ Ἀλέξανδρος παρεγένετο εἰς τὰς Ἀθήνας καὶ ταῦτ’ ἐδήλωσεν, οἱ Ἀθηναῖοι οὔτε τοὺς λόγους προσεδέξαντο ὑβρίσαντές τε τὸν Ἀλέξανδρον ἀπεπέμψαντο. (3) ὁ δὲ Μαρδόνιος ἀποτυχὼν ἐν τούτοις ἐπῆλθεν εἰς τὰς Ἀθήνας καὶ τὰ ἔτι περιλειπόμενα μέρη προσενέπρησεν, παραγενόμενός τε εἰς τὰς Θήβας}\footnote{\textgreek{Θήβας} \textbf{Jac}  \textbf{Büch} \textbf{Mül} \textgreek{\ladd{ἔπειτα δὲ περὶ Θή}βας ἐστρα\|\ladd{τοπεδεύσατο}} \textgreek{Π εἰς τὰς Ἀθήνας}  \textbf{P}} \textgreek{ἅμα τῷ στρατῷ ἐνταῦθα ἐστρατοπεδεύσατο, οἱ δὲ Ἕλληνες ἐστρατοπεδεύσαντο ἐν Πλαταιαῖς· τὰ δὲ μεταξὺ Θηβ\ladd{αι}ῶν}\footnote{\textgreek{Θηβ\ladd{αι}ῶν} \textbf{Jac} \textgreek{Θηβῶν} \textbf{Mül}} \textgreek{καὶ Πλαταιῶν στάδιά ἐστιν π. συνπαρετάσσοντο δὲ Μαρδονίῳ Βοιωτῶν μυριάδες δ. (4) εἶχον δὲ τὸ μὲν δεξιὸν κέρας Πέρσαι καὶ Μαρδόνιος, τὸ δὲ εὐώνυμον οἱ μηδίσαντες Ἕλληνες. τῶν δὲ Ἑλλήνων οἱ μὲν Ἀθηναῖοι εἶχον τὸ δεξιόν, τὸ δὲ εὐώνυμον Λακεδαιμόνιοι. μετέστησαν δὲ αὐτούς οἱ Λακεδαιμόνιοι, φήσαντες Ἀθηναίους}\footnote{\textgreek{μετέστησαν δὲ αὐτούς οἱ Λακεδαιμόνιοι, φήσαντες Ἀθηναίους}  \textbf{Mül} \textgreek{μετέσσαν δὲ αὐτούς οἱ Λακεδαιμόνιοι, 	φήσαντες \lladd{αὐτοὺς} Ἀθηναίους}  \textbf{P} \textgreek{δὲ \Ladd{αὐτοὺς} \ladd{Ἀθηναίους} οἱ Λακεδαιμόνιοι, φήσαντες \Ladd{αὐτοὺς Ἀθηναίους}} \textbf{Jac} \textgreek{μετέστησαν δὲ τοὺς Ἀθηναίους οἱ Λακεδαιμόνιοι, φήσαντες αὐτοὺς} \textbf{Mül}; \textgreek{ἐμπειροτέρος} \textbf{Jac} \textgreek{ἐμπ\Ladd{ει}ροτέρος} \textbf{Müe}} \textgreek{ἐμπειροτέρους εἶναι πρὸς τὸ μάχεσθαι Πέρσαις. ἐν δὲ τούτῳ Μαρδόνιος δεδοικὼς μάχεσθαι Ἀθηναίοις μετέστησέ τε τὴν φάλαγγα},\footnote{\textgreek{φάλαγγα}  \textbf{P} \textgreek{φάλαγγα \ladd{καὶ αὐτός}} \textbf{Jac} \textbf{Mül}} \textgreek{, καὶ οὕτως συνέβη τοῖς Λακεδαιμονίοις καὶ ἀκουσίως}\footnote{\textgreek{Ἀκουσίως}  \textbf{P} \textgreek{ἀκουσίοις} \textbf{Mül} \textbf{Jac}} \textgreek{μάχεσθαι τοῖς Πέρσαις. Ἐστρατήγει δὲ Λακεδαιμονίων μὲν Παυσανίας ὁ Κλεομβρότου, Ἀθηναίων δὲ Ἀριστείδης ὁ δίκαιος. Γενομένης δὲ τῆς συμβολῆς τῶν Περσῶν, Ἀθηναῖοι ἐπεβοήθησαν τοῖς Λακεδαιμονίοις καὶ ἐνίκησαν. (5) ἐνταῦθα Μαρδόνιος ἔπεσεν γυμνῇ τῇ κεφαλῇ μαχόμενος, ἀναιρεθεὶς ὑπὸ Ἀειμνήστου  ἀνδρὸς Λακεδαιμονίου. ἠρίστευσε δὲ ἐνταῦθα καὶ Ἀριστόδημος ὁ ὑποστρέψας  ἀπὸ Θερμοπυλῶν καὶ κληθεὶς διὰ τοῦτο ὁ τρέσας}\footnote{\textgreek{τρέσας} \textbf{Jac} \textgreek{τρεσσᾶς}  \textbf{P}} \textgreek{διὸ Λακεδαιμόνιοι οὐκ ἔδωκαν αὐτῷ τὸ γέρας τῆς ἀριστείας, ἡγησάμενοι τὸ μὲν πρῶτον γενόμενον περὶ τὴν λειποταξίαν γνώμης εἶναι, τὸ τελευταῖον δὲ περὶ τὴν ἀριστείαν τύχης.}

\subsubsection*{3} \textgreek{Ἐπειδὴ δὲ ἔπεσεν ὁ Μαρδόνιος, οἱ Πέρσαι ἔφυγον εἰς τὰς Θήβας, οἱ δὲ Ἕλληνες ἐπελθόντες δώδεκα μυριάδας αὐτῶν ἐφόνευσαν. Ἑξάκις δὲ μυρίων ἐπιστρεφόν\ladd{των}}\footnote{\textgreek{ἐπιστρεφόν\ladd{των}}  \textbf{We}} \textgreek{ἐπὶ τὴν οἰκείαν, Ἀλέξανδρος ὁ Μακεδών, ἴδιος πρεσβευσάμενος}\footnote{\textgreek{Ἰδίαι} (?) \textgreek{πρεσβευσάμενος} \textbf{Jac} \textgreek{\ladd{ὁ} ἰδίαι} \textit{vel}, \textgreek{ὀ διαπρεσβευσάμενος} \textbf{Mül}} \textgreek{πρὸς τοὺς Ἀθηναίους περὶ ὧν ἀπεστάλη ὑπὸ  Μαρδονίου, πάντας αὐτοὺς γενομένους κατὰ Μακεδονίαν ἐφόνευσεν, ἀπολογούμενος ὅτι ἄκων ἐμήδισεν.  (2) καὶ οἱ ἐν ταῖς ναυσὶν δὲ Ἕλληνες ἐδίωκον τὸ ναυτικὸν τὸ Ξέρξου, πλεύσαντές \ladd{τε} σταδίους τέσσαρας}\footnote{\textgreek{τέσσαρας}  \textbf{P}  \textgreek{δ\textsubscript{\ladd{,}}} \textbf{Mül}} \textgreek{τοὺς ἀπὸ Σαλαμῖνος εἰς Μίλητον κατέλαβον τὰς ναῦς τῶν βαρβάρων καὶ ἕτοιμοι ἦσαν ναυμαχεῖν. οἱ δὲ βάρβαροι οὐ}\footnote{\textgreek{οὐ}  \textbf{We} \textbf{Jac} \textgreek{οἱ}  \textbf{P}} \textgreek{πιστεύοντες ταῖς ναυσὶ διὰ τὸ πεπειρᾶσθαι τῆς Ἀθηναίων ἐμπειρίας ἐξέβησαν καὶ ἐστρατοπεδεύσαντο περὶ Μυκάλην, ὅπερ ἐστὶν ὄρος τῆς  Μιλησίας. καὶ οἱ Ἕλληνες δὲ ἀποβάντες συνέβαλον αὐτοῖς καὶ τὰς δ μυριάδας}\footnote{\textgreek{τὰς δ μυριάσας}   \textbf{P} \textgreek{\ladd{ὑπὲρ} τὰς δ μυριάσας} \textbf{Jac}}\textgreek{ἐφόνευσαν τάς τε ναῦς ἐρήμους παρέλαβον, γιγνομένης}\footnote{\textgreek{γιγνομένης} \textbf{P}\textgreek{\ladd{κατὰ τὴν αὐτὴν ἡμέραν} γιγνομένης} \textbf{Jac} \textgreek{\ladd{ἡ αὐτὴ δὲ ἡμέρα ἠν} γιγνομένης} \textbf{Mül}} \textgreek{τε τῆς μάχης τῆς ἐν Πλαταιαῖς καὶ νικώντων τῶν περὶ Μυκάλην Ἑλλήνων. (3) ἐστρατήγει δὲ \ladd{ἐπὶ}}\footnote{\textgreek{\ladd{ἐπὶ} τῆς} \textbf{Mül} \textgreek{\ladd{ἐν} τῇ Μυκάλῃ}  \textbf{Büch} \textgreek{\ladd{ἐπὶ} Μυκάλης} ? \textbf{Jac}} \textgreek{τῆς Μυκάλης Λακεδαιμονίων μὲν Λεωτυχίδας ὁ βασιλεύς, Ἀθηναίων δὲ Ξάνθιππος ὁ Ἀρίφρονος,  ὁ Περικλέους πατήρ. (4) οἱ δὲ ἐν ταῖς Πλαταιαῖς Ἕλληνες μετὰ τὸ νικῆσαι ἔστησαν τρόπαια, καὶ ἑορτὴν ἤγαγον Ἐλευθέρια}\footnote{\textgreek{Ἐλευθέρια}  \textbf{Büch} \textbf{Mül} \textbf{Jac} \textgreek{-ίαν}  \textbf{P}} \textgreek{ προσαγορεύσαντες, Θηβαίους τε, καθὼς ὤμοσαν, ἐδεκάτευσαν.}\footnote{\textgreek{τέλος τοῦ \star\star τὸ} \textbf{Jac} \textgreek{Τέλος τοῦ Δ'} + \textgreek{ἀρχὴ}  \textbf{We} \textgreek{τοῦ Α'} o \textgreek{Δ'} o \textgreek{Λ'} + \textgreek{ἀρ ........ ο} (oppure \textgreek{υ}) \textgreek{Ἀριστοδήμου}   \textbf{Prinz} \textgreek{τοῦ Σ'} \textbf{Mül}}

\subsubsection*{4} \textgreek{(1) Ἀπὸ δὲ τῆς Περσικῆς στρατείας ἐπὶ τὸν Πελοποννησ\ladd{ιακὸν πόλεμον ὑπὸ τῶν Ἑλλήνων} (?)}\footnote{\textgreek{Πελοποννησ\ladd{ιακὸν πόλεμον ὑπὸ τῶν Ἑλλήνων}}(?) \textbf{Mül} \textgreek{κατὰ τὴν Ἑλλάδα ἔργα}  \textbf{Büch}} \textgreek{ἐπράχθη τάδε. (2) ἐπειδὴ ἐξήλασαν τοὺς Πέρσας οἱ Ἕλληνες \ladd{ἐκ τῆς Εὐρώπης, ἀπο}φυγόν\ladd{των} τῶν ἀπολει\ladd{φθέντων β}α\ladd{ρβά}ρων}\footnote{\textgreek{\ladd{ἐκ τῆς Εὐρώπης, ἀπο}φυγόν\ladd{των} τῶν ἀπολει\ladd{φθέντων β}α\ladd{ρβά}ρων}  \textbf{Büch}   \textbf{Prinz} \textgreek{\ladd{τῆς Ιωνίας πλεύσαντες στόλῳ ξ' τριή}ρων} \textbf{Mül}; \textgreek{\ladd{ἀφίκετο δὲ} καὶ} ? \textbf{Jac}}\textgreek{εἰς Σηστόν, οἱ Ἀθηναῖοι προσέμενον προσπολεμοῦντες,}\footnote{ \textgreek{προσπολεμοῦντες}  \textbf{Jac} \textgreek{προσπολιορκοῦντες} \textbf{Mül}} ..\textgreek{καὶ Παυσανίας ὁ Κλεομβρότου, ὁ τῶν Λακεδαιμονίων στρατηγός, κατὰ φιλοτιμίαν τὴν ὑπὲρ τῶν Ἑλλήνων, ἅμα διὰ προδοσίαν}·\footnote{\textgreek{\ladd{οὐ} κατὰ φιλοτιμίαν τὴν ὑπὲρ τῶν Ἑλλήνων, ἀλλὰ διὰ προδοσίαν·} \textit{vel} \textgreek{κατὰ φιλοτιμίαν τὴν ὑπὲρ τῶν Ἑλλήνων, ἅμα \ladd{δὲ} διὰ προδοσίαν }? \textbf{Jac} \textgreek{\ladd{οὐ} ἀλλὰ διὰ προδώσειν} \textbf{Mül} \textgreek{κατὰ φιλοτιμίαν τὴν ὑπὲρ τῶν ἔργων}  \textbf{Büch}} \textgreek{συντεθειμένος γὰρ ἦν Ξέρξῃ προδώσεσθαι}\footnote{\textgreek{προδώσεσθαι}  \textbf{P} \textgreek{προδώσειν} ? \textbf{Mül}} \textgreek{αὐτῷ τοὺς Ἕλληνας ἐπὶ τῷ λαβεῖν θυγατέρα παρ’ αὐτοῦ πρὸς γάμον. Ὡς}\footnote{\textgreek{Ὡς}  \textbf{P} \textgreek{ὃς} \textbf{Mül}} \textgreek{ἐπηρμένος τε τῇ}\footnote{\textgreek{τε τῇ}  \textbf{P} \textgreek{τῇ τε}  \textbf{Büch} \textbf{Mül}} \textgreek{ἐλπίδι ταύτῃ καὶ τῷ εὐτυχήματι τῷ ἐν}\footnote{\textgreek{τῷ ἐν}  \textbf{We} \textgreek{τὸ ἐν}  \textbf{P}} \textgreek{Πλαταιαῖς οὐκ ἐμετρ\ladd{ι}οπάθει}\footnote{\textgreek{ἐμετρ\ladd{ι}οπάθει}  \textbf{We}} \textgreek{ἀλλὰ πρῶτον μὲν τρίποδα ἀναθεὶς τῷ ἐν Δελφοῖς  Ἀπόλλωνι ἐπίγραμμα ἔγραψε πρὸς αὑτὸν τοιοῦτον· Ἑλλήνων ἀρχηγὸς ἐπεὶ στρατὸν ὤλεσε Μήδων Παυσανίας Φοίβῳ μνῆμ’ ἀνέθηκε τόδε. (3) τῶν δὲ ὑποτεταγμένων αὐτῷ πικρῶς ἦρχε καὶ τυραννικῶς, τὴν μὲν Λακωνικὴν δίαιταν ἀποτεθειμένος, ἐπιτετηδευκὼς δὲ τὰς τῶν Περσῶν ἐσθῆτας φορεῖν καὶ Περσικὰς τραπέζας παρατεθειμένος}\footnote{\textgreek{παρατεθειμένος}  \textbf{Büch} \textbf{Mül} \textgreek{παρατεθημένας}  \textbf{P}} \textgreek{πολυτελεῖς, ὡς ἔθος ἐκείνοις.}

\subsubsection*{5} \textgreek{Κατὰ δὲ τοῦτον τὸν χρόνον Ἀθηναῖοι, ἐμπεπρησμένης αὐτῶν τῆς πόλεως ὑπὸ Ξέρξου καὶ Μαρδονίου, ἐβουλεύοντο τειχίζειν αὐτήν, οἱ δὲ Λακεδαιμόνιοι}\footnote{\textgreek{οἱ δὲ Λακεδαιμόνιοι} \textbf{Mül} \textgreek{Λακεδαιμόνιοι· οἱ δὲ}  \textbf{P}} \textgreek{οὐκ ἐπέτρεπον αὐτοῖς, πρόφασιν μὲν ποιούμενοι ὁρμητήριον εἶναι τὰς Ἀθήνας τῶν ἐπιπλεόντων βαρβάρων, τὸ δὲ ἀληθὲς φθονοῦντες καὶ μὴ βουλόμενοι πάλιν αὐξηθῆναι· οὓς}\footnote{\textgreek{οὓς} \textbf{Jac}\textgreek{ ὁ δὲ}  \textbf{Büch} \textbf{Mül}} \textgreek{Θεμιστοκλῆς συνέσει διαφέρων  κατεστρατήγησεν, \ladd{γιγνώσκων}}\footnote{\textgreek{\ladd{Ἀκριβῶς γιγνώσκων}} \textbf{Jac} } \textgreek{αὐτῶν τὸν φθόνον. (2) ἐγκελευσάμενος γὰρ τοῖς Ἀθηναίοις τειχίζειν τὴν πόλιν ᾤχετο εἰς Λακεδαίμονα   ὡς πρεσβεύων, λόγων τε γιγνομένων παρὰ τοῖς Λακεδαιμονίοις ὅτι Ἀθηναῖοι  τειχίζουσι τὴν πόλιν, ἀντέλεγεν Θεμιστοκλῆς. ὡς δὲ}\footnote{\textgreek{ὡς δὲ} \textbf{Jac} \textgreek{ὡς τε}  \textbf{P}}\textgreek{οὐκ ἐπίστευον οἱ Λακεδαιμόνιοι, ἔπεισεν αὐτοὺς πρέσβεις πέμψαι τινὰς ἐξ αὐτῶν εἰς τὰς Ἀθήνας τοὺς γνωσομένους εἰ κτίζοιτο}\footnote{\textgreek{κτίζοιτο}  \textbf{P} \textgreek{τειχίζοιτο} ? \textbf{Mül}} \textgreek{ἡ πόλις. τῶν δὲ Λακεδαιμονίων ἑλομένων ἄνδρας καὶ πεμψάντων, Θεμιστοκλῆς κρύφα ὑπέπεμπε τοῖς Ἀθηναίοις κατέχειν παρ’ ἑαυτοῖς τοὺς ἀπεσταλμένους τῶν Λακεδαιμονίων}\footnote{\textgreek{τῶν Λακεδαιμονίων}  \textbf{P} \textgreek{\ladd{παρὰ} τῶν Λ. }? \textbf{Jac}} \textgreek{ἄνδρας, ἕως ἂν αὐτὸς ὑποστρέψῃ εἰς τὰς Ἀθήνας. (3) πραξάντων δὲ τοῦτο τῶν Ἀθηναίων, οἱ Λακεδαιμόνιοι αἰσθόμενοι τὴν ἀπάτην Θεμιστοκλέους,}\footnote{\textgreek{Θεμιστοκλέους}  \textbf{P} \textgreek{τὴν Θεμιστοκλέους} \textbf{Jac}} \textgreek{οὐδὲν διέθεσαν αὐτὸν δεινόν, δεδοικότες περὶ τῶν ἰδίων, ἀλλ’ἀποδόντες αὐτὸν ἐκομίσαντο τοὺς ἰδίους. (4) ἐν δὲ τῷ μεταξὺ χρόνῳ ἐτειχίσθησαν αἱ Ἀθῆναι τὸν τρόπον τοῦτον. ὁ μὲν τοῦ ἄστεως περίβολος  ἑξήκοντα σταδίων ἐτειχίσθη, τὰ δὲ μακρὰ τείχη φέροντα ἐπὶ τὸν Πειραιᾶ ἐξ ἑκατέρου μέρους σταδίων μ, ὁ δὲ τοῦ Πειραιῶς περίβολος σταδίων π (ἔστι δὲ ὁ Πειραιεὺς}\footnote{\textgreek{Πειραιεὺς} \textbf{Jac} \textgreek{πειρευς}  \textbf{P}} \textgreek{λιμὴν εἰς δύο διῃρημένος, κέκληται δὲ αὐτοῦ τὸ μέν τι μέρος}\footnote{\textgreek{τὸ μέν τι μέρος} \textbf{Jac} \textgreek{τὸ μὲν λαιὸν μέρος} ? \textbf{Schäfer}} \textgreek{Μουνυχία,}\footnote{\textgreek{Μουνυχία} \textbf{Jac} \textgreek{μουνουχία}  \textbf{P}} \textgreek{τὰ δεξιὰ δὲ ἄκρα τοῦ Πειραιῶς ᾑ ἐστιν ἔτι νῦν Δία}\footnote{\textgreek{ἔτι νῦν Δία}  \textbf{We} \textgreek{Ἠετιώνεια} \textbf{Jac}  \textbf{Büch} \textbf{Mül} \textgreek{ἡ δεξιὰ δὲ ἄκρα τοῦ Πειραιῶς, ᾗ ἑστιν \ladd{ὁ εἴσπλους}, Ἠετιωνεία καλεῖται}  \textbf{Schäfer} \textgreek{ηετινετινυνδια} \textbf{P}} \textgreek{καλεῖται· ὄχθος δέ ἐστιν ἐν Πειραιεῖ,}\footnote{\textgreek{Πειραιεῖ} \textbf{Jac} \textgreek{πειραεῖ}  \textbf{P}} \textgreek{ἐφ’ ὅν}\footnote{\textgreek{ἐφ’ ὅν}  \textbf{P} \textgreek{οὗ} \textbf{Jac} \textbf{Mül} \textgreek{ᾧ} \textbf{Schäfer}} \textgreek{τὸ τῆς Ἀρτέμιδος ἱερὸν ἵδρυται). τὸ  δὲ Φαληρικὸν τεῖχος ἐκτίσθη σταδίων λ· πλατὺ δὲ ὥστε δύο ἅρματα  ἀλλήλοις συναντᾶν. καὶ ἡ μὲν τῶν Ἀθηναίων πόλις οὕτως ἐτειχίσθη.}

\subsubsection*{6} \textgreek{ Ὁ δὲ Θεμιστοκλῆς διὰ τὴν ὑπερβάλλουσαν σύνεσιν καὶ ἀρετὴν φθονηθεὶς ἐξεδιώχθη ὑπὸ τῶν Ἀθηναίων καὶ παρεγένετο εἰς Ἄργος. (2) Λακεδαιμόνιοι δὲ ἀκούσαντες τὰ περὶ τῆς ἐγκεχειρισμένης προδοσίας Παυσανίαι, πέμψαντες αὐτῷ τὴν σκυτάλην μετεκαλοῦντο αὐτὸν ὡς ἀπο λογησόμενον. (3) ὁ δὲ Παυσανίας ἐλθὼν εἰς τὴν Σπάρτην ἀπελογήσατο, καὶ ἀπατήσας τοὺς Λακεδαιμονίους, ἀπολυθεὶς τῆς αἰτίας ὑπεξῆλθεν καὶ πάλιν ἐνήργει τὴν προδοσίαν.}

\subsubsection*{7} \textgreek{Ἐν δὲ τούτῳ οἱ Ἕλληνες ἀφιστάμενοι ἀπὸ τῶν Λακεδαιμονίων διὰ τὸ πικρῶς τυραννεῖσθαι ὑπὸ τοῦ Παυσανίου προσετίθεντο τοῖς Ἀθηναίοις, καὶ οὕτως ἤρξαντο πάλιν οἱ Ἀθηναῖοι φόρους λαμβάνοντες αὔξεσθαι· ναῦς τε γὰρ κατεσκεύαζον \ladd{καὶ  κοινὸν τῶν Ἑλληνικῶν χ}ρημάτων}\footnote{\textgreek{\ladd{καὶ  κοινὸν τῶν Ἑλληνικῶν χ}ρημάτων} \textbf{Jac} \textbf{Mül} \textgreek{\ladd{καὶ στρατὸν συνέλεγον καὶ}}  \textbf{Büch}} \textgreek{θησαυροφυλάκιον ἐποιήσαντο ἐν Δήλῳ,} \ladd{\textgreek{ὕστερον δὲ} (?) ............ \textgreek{τάλ}}\footnote{\ladd{\textgreek{ὕστερον δὲ} (?) ............ \textgreek{τάλ}}\textgreek{αντα} \textbf{Jac} \textgreek{\ladd{ὑστέρῳ δὲ χρόνῳ π}άντα}  \textbf{Büch} \textgreek{\ladd{εἰς ὃ κατ'ἔτος συνῆγον υξ' τάλ}αντα \ladd{εἷτα}} (\textit{vel} \textgreek{ὕστερον δὲ}) \textbf{Mül} } \textgreek{αντα ἐκ τῆς Δήλου τὰ συναχθέντα  μετεκόμισαν εἰς τὰς Ἀθήνας καὶ κατέθεντο ἐντὸς ἐν ἀκροπόλει.} 

\subsubsection*{8} \textgreek{῾Ο δὲ Παυσανίας ὑπάρχων ἐν Βυζαντίῳ ἀναφανδὸν}\footnote{\textgreek{ἀναφανδὸν} \textbf{Jac} \textgreek{αναφανδων}  \textbf{P}} \textgreek{ἐμήδιζεν καὶ κακὰ διετίθει τοὺς Ἕλληνας. διεπράξατο δέ τι καὶ τοιοῦτον. ἦν ἐπιχωρίου τινὸς θυγάτηρ Κορωνίδου ὄνομα}\footnote{\textgreek{Κορωνίδου ὄνομα} \textbf{Mül} \textgreek{\ladd{ἀνδρὸς} Κορωνίδου ὄνομα} ? \textbf{Jac}}, \textgreek{ἐφ’ ἣν ἔπεμψεν ὁ Παυσανίας ἐξαιτῶν τὸν πατέρα· ὁ δὲ Κορωνίδης δεδοικὼς τὴν ὠμότητα τοῦ Παυσανίου ἔπεμψεν αὐτῷ τὴν παῖδα. ἧς καὶ παραγενομένης νυκτὸς ἐς τὸ οἴκημα, κοιμωμένου τοῦ Παυσανίου, καὶ παραστάσης, περίυπνος γενόμενος}\footnote{\textgreek{γενόμενος} \textbf{Jac} \textgreek{γενομενομενος}  \textbf{P}} \textgreek{ὁ Παυσανίας δόξας τε κατ’ ἐπιβουλήν τινα εἰσεληλυθέναι ἐπαράμενος τὸ ξιφίδιον ἐπερόνησε}\footnote{\textgreek{ἐπερόνησε} \textbf{Jac} \textgreek{ἐπερώνησε}  \textbf{P}} \textgreek{τὴν κόρην καὶ ἀπέκτεινε. καὶ διὰ τοῦτο εἰς μανίαν περιέστη, καὶ γενόμενος φρενομανὴς ἐκεκράγει, ὡς δὴ μαστιγούμενος ὑπὸ τῆς κόρης· πολλοῦ δὲ χρόνου διαγενομένου ἐξιλάσατο τοὺς δαίμονας τῆς παιδὸς καὶ οὕτως ἀποκατέστη.}\footnote{\textgreek{ἀποκατέστη} \textbf{Jac} \textgreek{ἀπεκατέστη}  \textbf{P}} \textgreek{(2) τῆς δὲ προδοσίας οὐκ ἐπαύετο ἀλλὰ γράψας ἐπιστολὰς Ξέρξῃ Ἀργιλίῳ ἀγαπωμένῳ ἑαυτοῦ δίδωσι ταύτας, ἐγκελευσάμενος κομίζειν πρὸς Ξέρξην. ὁ δὲ Ἀργίλιος δεδοικὼς περὶ αὑτοῦ, ἐπειδὴ γὰρ}\footnote{\textgreek{ἐπειδὴ γὰρ}  \textbf{P} \textgreek{\Ladd{γὰρ}}  \textbf{Büch} \textbf{Mül}} \textgreek{οὐδὲ οἱ πρότεροι}\footnote{\textgreek{πρότεροι} \textbf{Jac} \textgreek{πρότερον}  \textbf{Büch} \textbf{Mül}}  \textgreek{πεμφθέντες ἀπενόστησαν, πρὸς Ξέρξην οὐ παρεγένετο, ἐλθὼν δὲ εἰς Σπάρτην τοῖς ἐφόροις ἐμήνυσε τὴν προδοσίαν, ὑπέσχετο δὲ κατάφωρον}\footnote{\textgreek{Κατάφωρον}  \textbf{Büch} \textbf{Mül} \textgreek{κατάφορον}  \textbf{P}} \textgreek{δείξειν τὸν Παυσανίαν. καὶ συνθέμενος περὶ  τούτων ἦλθεν εἰς Ταίναρον ἔν τε τῷ τοῦ Ποσειδῶνος τεμένει ἱκέτευεν· οἱ δὲ ἔφοροι παραγενόμενοι καὶ αὐτοὶ εἰς τὸ αὐτὸ τέμενος καὶ διπλῆν σκηνὴν κατασκευάσαντες ἐν αὐτῇ ἔκρυψαν ἑαυτούς. (3) οὐκ ἐπιστάμενος δὲ ὁ Παυσανίας ταῦτα, ἀκούσας δὲ τὸν Ἀργίλιον ἱκετεύοντα, παρεγένετο πρὸς αὐτὸν καὶ ἀπεμέμφετο ἐπὶ τῷ}\footnote{\textgreek{τῷ}  \textbf{Büch} \textbf{Mül} \textgreek{ιὸ}  \textbf{P}} \textgreek{μὴ κομίσαι τὰς ἐπιστολὰς πρὸς Ξέρξην, ἄλλα τέ τινα τεκμήρια διεξῄει}\footnote{\textgreek{διεξήιει}  \textbf{We} \textgreek{διεξείη}  \textbf{P}} \textgreek{τῆς προδοσίας. οἱ δὲ ἔφοροι ἀκούσαντες  τῶν ῥηθέντων παραχρῆμα μὲν οὐ συνελάβοντο αὐτὸν διὰ τὸ εἶναι ἅγιον τὸ τέμενος ἀλλ’ εἴασαν ἀπελθεῖν, ὕστερον δὲ αὐτὸν ἐλθόντα εἰς Σπάρτην ἐβούλοντο συλλαμβάνεσθαι· ὁ δὲ ὑπονοήσας εἰσέδραμεν εἰς τὸ τῆς Χαλκιοίκου Ἀθηνᾶς τέμενος καὶ ἱκέτευεν. (4) τῶν δὲ Λακεδαιμονίων ἐν ἀπόρῳ ὄντων διὰ τὴν εἰς τὸν θεὸν}\footnote{\textgreek{τὸν θεὸν}  \textbf{P}  \textgreek{τὴν θεὸν}  \textbf{We}} \textgreek{θρησκείαν, ἡ μήτηρ τοῦ Παυσανίου βαστάσασα πλίνθον ἔθηκεν ἐπὶ τῆς εἰσόδου τοῦ τεμένους, προκαταρχομένη τῆς κατὰ τοῦ παιδὸς κολάσεως· οἱ δὲ Λακεδαιμόνιοι κατακολουθήσαντες αὐτῇ ἐνῳκοδόμησαν τὸ τέμενος, καὶ λιμῷ διαφθαρέντος τοῦ Παυσανίου, ἀνελόντες}\footnote{\textgreek{ἀνελόντες}  \textbf{Büch} \textbf{Mül} \textgreek{ἀνελθόντες}  \textbf{P}} \textgreek{τὴν στέγην, ἐξείλκυσαν τοῦ ναοῦ ἔτι ἐμπνέοντα τὸν Παυσανίαν καὶ ἐξέρριψαν. (5) διὰ δὲ τοῦτο λοιμὸς αὐτοὺς κατέσχεν· θεοῦ δὲ χρήσαντος, ἐπὰν ἐξιλάσωνται τοὺς δαίμονας τοῦ Παυσανίου, παύσασθαι}\footnote{\textgreek{Παύσασθαι} \textbf{Jac} \textgreek{παύσεσθαι}  \textbf{Büch}} \textgreek{τὸν λοιμόν, ἀνδριάντα αὐτῷ ἀνέστησαν, καὶ ἐπαύσατο ὁ λοιμός}

\subsubsection*{9} \textgreek{Ζητήσεως δὲ οὔσης παρὰ τοῖς Ἕλλησι τίνας δεῖ προγραφῆναι αὐτῶν τῶν συμμεμαχηκότων ἐν τῷ  Μηδικῷ πολέμῳ, ἐξεῦρον οἱ Λακεδαιμόνιοι τὸν δίσκον, ἐφ’ οὗ κυκλοτερῶς ἐπέγραψαν τὰς ἠγωνισμένας πόλεις, ὡς μήτε πρώτους τινὰς γεγράφθαι μήθ’ ὑστέρους.}

\subsubsection*{10} \textgreek{Λακεδαιμόνιοι δέ, ἐπειδὴ τὰ τοῦ Παυσανίου ἐπονειδίστως ἐκεχωρήκει, τοὺς Ἀθηναίους ἔπειθον λέγοντες ἐν ταῖς Παυσανίου ἐπιστολαῖς κοινωνὸν εὑρηκέναι τῆς προδοσίας Θεμιστοκλέα. ὁ δὲ Θεμιστοκλῆς δεδοικὼς τοὺς Λακεδαιμονίους οὐκ ἔμεινεν ἐν τῷ Ἄργει ἀλλὰ παρεγένετο εἰς Κέρκυραν κἀκεῖθεν εἰς Μολοσσοὺς πρὸς Ἄδμητον βασιλεύοντα}\footnote{\textgreek{βασιλεύοντα}  \textbf{P} \textgreek{βασιλέα, ὄντα} \textbf{Jac} \textgreek{βασιλεύοντα, καὶ \ladd{τοι} ἐχθρὸν \ladd{ὄντα}} \textbf{Mül}} \textgreek{καὶ ἐχθρὸν αὐτῷ πρότερον. (2) τῶν δὲ Λακεδαιμονίων παραγενομένων πρὸς τὸν Ἄδμητον καὶ ἐξαιτούντων αὐτόν, ἡ γυνὴ τοῦ Ἀδμήτου ὑπέθετο Θεμιστοκλεῖ}\footnote{\textgreek{Θεμιστοκλεῖ}  \textbf{Büch} \textbf{Mül} \textgreek{-εα}  \textbf{P}} \textgreek{ἁρπάσαι τὸν τοῦ βασιλέως παῖδα καὶ καθεσθῆναι ἐπὶ τῆς ἑστίας ἱκετεύοντα. πράξαντος δὲ τοῦ Θεμιστοκλέους, ὁ Ἄδμητος κατελεήσας αὐτὸν οὐκ ἐξέδωκεν ἀλλ’ ἀπεκρίθη τοῖς Πελοποννησίοις μὴ ὅσιον εἶναι ἐκδοῦναι τὸν ἱκέτην. (3) ὁ δὲ Θεμιστοκλῆς οὐκ ἔχων ὅπου ὑποστρέψει ἐπὶ τὴν Περσίδα ἔπλει. ἐκινδύνευσε δὲ καὶ πλέων ἁλῶναι καὶ παραληφθῆναι. Νάξον γὰρ  πολεμούντων}\footnote{\textgreek{Πολεμούντων}  \textbf{P} \textgreek{πολιορκούντων}  \textbf{Büch} \textbf{Mül} \textbf{Jac}} \textgreek{τῶν Ἀθηναίων ἡ ναῦς τοῦ Θεμιστοκλέους χειμῶνος ἐπιγενομένου προσήγετο τῇ Νάξῳ· ὁ δὲ Θεμιστοκλῆς δεδοικὼς μήποτε συλληφθῇ ὑπὸ τῶν Ἀθηναίων ἠπείλησε τῷ κυβερνήτῃ ἀναιρήσειν αὐτόν, εἰ μὴ ἀντέχοι τοῖς πνεύμασιν. ὁ δὲ κυβερνήτης δείσας τὴν ἀπειλὴν ὥρμησεν  ἐπὶ σάλου νύκτα}\footnote{\textgreek{νύκτα}  \textbf{We} \textgreek{νύκταν}  \textbf{P}} \textgreek{καὶ ἡμέραν καὶ ἀντέσχε τοῖς ἀνέμοις· καὶ οὕτω Θεμιστοκλῆς διασωθεὶς παρεγένετο εἰς τὴν Περσίδα. (4) καὶ Ξέρξην μὲν οὐ κατέλαβε ζῶντα, Ἀρταξέρξην δὲ τὸν υἱὸν αὐτοῦ· ᾧ οὐκ}\footnote{\textgreek{οὐκ ἐνεφανίσθη}  \textbf{P} \textgreek{οὐκ \ladd{τότε μὲν} ἐνεφανίσθη} \textit{vel} \textgreek{οὐκ \ladd{παραχρῆμα} ἐνεφανίσθη} \textbf{Mül}} \textgreek{ἐνεφανίσθη ἀλλὰ διατρίψας ἐνιαυτὸν καὶ μαθὼν τὴν Περσικὴν γλῶσσαν, τότε παρεγένετο πρὸς τὸν Ἀρταξέρξην, καὶ ὑπέμνησεν αὐτὸν}\footnote{\textgreek{ὑπέμνησεν αὐτὸν}  \textbf{Büch} \textbf{Mül} \textgreek{ἐπέμνησεν αὐτῶ}  \textbf{P}} \textgreek{τῶν εὐεργεσιῶν, ἃς ἐδόκει  κατατεθεῖσθαι εἰς τὸν πατέρα αὐτοῦ Ξέρξην, λέγων καὶ τῆς σωτηρίας αὐτῷ  γεγενῆσθαι}\footnote{\textgreek{γεγενῆσθαι}  \textbf{Büch} \textbf{Mül} \textgreek{γενήσεσθαι}  \textbf{P}} \textgreek{αἴτιος, \d{ὑπο}δείξας λύ\ladd{σο}ντας}\footnote{\textgreek{\d{ὑπο}δείξας λύ\ladd{σο}ντας τοὺς \ladd{Ἕλλ}ηνας}   \textbf{Prinz}} \textgreek{τοὺς \ladd{Ἕλλ}ηνας τὸ ζεῦγμα. ὑπέσχετο δέ, εἰ λάβοι στρατὸν παρ’ αὐτοῦ, χειρώσασθαι}\footnote{\textgreek{χειρώσασθαι}  \textbf{Büch}} \textgreek{τοὺς Ἕλληνας. (5) ὁ δὲ Ἀρταξέρξης προσσχὼν}\footnote{\textgreek{προσσχὼν} \textbf{Jac} \textgreek{προσχὼν}  \textbf{P}} \textgreek{τοῖς εἰρημένοις ἔδωκεν}\footnote{\textgreek{ἔδωκεν}  \textbf{Büch} \textbf{Mül} \textgreek{δέδωκεν}  \textbf{P}} \textgreek{αὐτῷ στρατὸν καὶ τρεῖς πόλεις εἰς χορηγίαν, Μαγνησίαν μὲν εἰς σῖτον, Λάμψακον δὲ εἰς οἶνον, Μυοῦντα δὲ εἰς ὄψον. λαβὼν δὲ Θεμιστοκλῆς καὶ παραγενόμενος εἰς Μαγνησίαν, ἐγγὺς ἤδη γενόμενος τῆς Ἑλλάδος μετενόησεν, οὐχ ἡγησάμενος δεῖν πολεμεῖν τοῖς ὁμοφύλοις· θύων δὲ τῇ Λευκοφρύνῃ}\footnote{\textgreek{Λευκοφρύνῃ}  \textbf{Büch} \textbf{Mül} \textgreek{λευκοφρύνι}  \textbf{P}} \textgreek{Ἀρτέμιδι, σφαττομένου ταύρου ὑποσχὼν φιάλην καὶ πληρώσας αἵματος ἔπιεν καὶ ἐτελεύτησεν.}

\subsubsection*{11} \textgreek{Οἱ δὲ Ἕλληνες \ladd{οὐ}}\footnote{\textgreek{\ladd{οὐ}} \textbf{Mül}} \textgreek{γνόντες ταῦτα ἐξεδίωκον τὸν στρατὸν τὸν ἅμα τῷ Θεμιστοκλεῖ, καὶ}\footnote{\textgreek{καὶ}  \textbf{P} \textgreek{\Ladd{καὶ}} \textbf{Mül}} \textgreek{παραγενόμενοι δὲ ἔγνωσαν καὶ ἀντεπεστράτευον}\footnote{\textgreek{ἀντεπεστράτευον} \textbf{Jac} \textgreek{παραγενόμενοι δὲ εἰς Μαγνεσίαν ἀντεπεστράτευον}  \textbf{Büch}} \textgreek{τῷ Ἀρταξέρξῃ εὐθέως τε τὰς Ἰωνικὰς καὶ τὰς λοιπὰς πόλεις Ἑλληνίδας ἠλευθέρουν Ἀθηναῖοι. (2) Κίμωνος δὲ τοῦ Μιλτιάδου στρατηγοῦντος ἀνέπλευσαν ἐπὶ τὴν Παμφυλίαν κατὰ τὸν λεγόμενον Εὐρυμέδοντα ποταμὸν καὶ ἐναυμάχησαν Φοίνιξι καὶ Πέρσαις καὶ λαμπρὰ ἔργα ἐπεδείξαντο, ἑκατόν τε ναῦς ἑλόντες αὐτάνδρους ἐπεζομάχησαν· καὶ δύο τρόπαια ἔστησαν, τὸ μὲν κατὰ γῆν τὸ δὲ κατὰ θάλατταν. (3) ἔπλευσαν δὲ καὶ κατὰ Κύπρον καὶ ἐπ’ Αἴγυπτον. ἐβασίλευσε δὲ τῆς Αἰγύπτου Ἴναρος}\footnote{\textgreek{Ἴναρος}  \textbf{We} \textgreek{Ἰνάρως} \textbf{Mül}} \textgreek{υἱὸς Ψαμμητίχου,}\footnote{\textgreek{Ψαμμητίχου}  \textbf{We} \textgreek{Ψαμμιτείχου}  \textbf{P}} \textgreek{ὃς ἀποστὰς Ἀρταξέρξου βοηθοὺς ἐπηγάγετο αὑτῷ τοὺς Ἀθηναίους, οἵτινες ἔχοντες ς ναῦς ἐπολέμησαν ἐπὶ ἔτη ἓξ τοῖς βαρβάροις. (4) μετὰ δὲ ταῦτα Μεγάβυξος ὁ Ζωπύρου καταπεμφθεὶς ὑπὸ Ἀρταξέρξου, ὡρμημένων}\footnote{\textgreek{Ὡρμημένων} \textbf{Jac} \textgreek{ὡρμηκότων} (cfr. 10.3) \textgreek{ὡρμισμένων}  \textbf{Büch}} \textgreek{τῶν Ἀθηναίων ἐν τῇ καλουμένῃ Προσωπίτιδι νήσῳ ἐπί τινος ποταμοῦ, ἐκτρέπει τὸ ῥεῖθρον τοῦ ποταμοῦ ἐποίησέ τε τὰς ναῦς ἐπὶ τῆς γῆς ἀπολειφθῆναι. ἐκτραπεισῶν}\footnote{\textgreek{ἐκτραπεισῶν}  \textbf{P} \textgreek{οὐκ ἐντραπεισῶν δὲ ν νεῶν} ?  \textbf{Büch} \crux \textgreek{ἐκτραπεισῶν, ἐκπεμφθεισῶν} ? \textbf{Jac} } \textgreek{δὲ ν νεῶν Ἀττικῶν προσπλεουσῶν τῇ Αἰγύπτῳ οἱ περὶ  τὸν Μεγάβυξον καὶ ταύτας παρέλαβον καὶ ἃς μὲν διέφθειραν, ἃς δὲ κατέσχον. τῶν δὲ ἀνδρῶν οἱ μὲν πλείους διεφθάρησαν}\footnote{\textgreek{διεφθάρησαν} \textbf{Jac} \textgreek{διέφθειραν}  \textbf{We} \textgreek{διέφθαρον}  \textbf{P}}, \textgreek{, ὀλίγοι δὲ παντάπασιν ὑπέστρεψαν εἰς τὴν οἰκείαν.}

\subsubsection*{12} \textgreek{Μετὰ δὲ ταῦτα Ἑλληνικὸς πόλεμος ἐγένετο Ἀθηναίων καὶ Λακεδαιμονίων ἐν Τανάγραι. καὶ οἱ μὲν Λακεδαιμόνιοι ἦσαν τὸν ἀριθμὸν μύριοι τρισχίλιοι, οἱ δὲ Ἀθηναῖοι μύριοι ἑξακισχίλιοι· καὶ νικῶσιν Ἀθηναῖοι. (2) παραταξάμενοι δὲ πάλιν ἐν Οἰνοφύτοις, στρατηγοῦντος αὐτῶν Τολμίδου καὶ Μυρωνίδου, ἐνίκησαν Βοιωτοὺς καὶ κατέσχον Βοιωτίαν.}

\subsubsection*{13} \textgreek{Εὐθὺς}\footnote{\textgreek{Εὐθὺς}  \textbf{P} \textgreek{Εὐθὺς \ladd{Μετὰ δὲ ταῦτα}} \textbf{Jac}} \textgreek{ἐστράτευσαν ἐπὶ Κύπρον, στρατηγοῦντος αὐτῶν Κίμωνος τοῦ Μιλτιάδου. ἐνταῦθα λιμῷ συνεσχέθησαν, καὶ Κίμων νοσήσας ἐν Κιτίῳ πόλει τῆς Κύπρου τελευτᾶι. οἱ δὲ Πέρσαι ὁρῶντες κεκακωμένους τοὺς Ἀθηναίους, περιφρονήσαντες αὐτῶν ἐπῆλθον ταῖς ναυσίν· καὶ ἀγὼν γίνεται κατὰ θάλατταν, ἐν ᾧ νικῶσιν Ἀθηναῖοι. (2) καὶ στρατηγὸν αἱροῦνται Καλλίαν τὸν ἐπίκλην}\footnote{\textgreek{ἐπίκλην}  \textbf{Büch} \textbf{Mül} \textgreek{τὸν ἐπίκλησιν}  \textbf{We} \textgreek{τὸν ἐπίκλιν}  \textbf{P}} \textgreek{Λακκόπλουτον, ἐπεὶ θησαυρὸν εὑρὼν ἐν Μαραθῶνι ἀνελόμενος αὐτὸν ἐπλούτησεν. οὗτος ὁ Καλλίας ἐσπείσατο}\footnote{\textgreek{ἐσπείσατο} \textbf{Jac} \textgreek{ἑσπήσατο}  \textbf{P}} \textgreek{πρὸς Ἀρταξέρξην καὶ τοὺς λοιποὺς Πέρσας. ἐγένοντο δὲ αἱ σπονδαὶ ἐπὶ τοῖσδε·}\footnote{\textgreek{ἐπὶ τοῖσδε}  \textbf{P} \textgreek{\Ladd{ἐπὶ τοῖσδε}} \textbf{Mül}} \textgreek{ἐφ’ ᾧ ἐντὸς Κυανέων καὶ Νέσσου} (?) \textgreek{ποταμοῦ καὶ Φασηλίδος, ἥτις ἐστὶν πόλις Παμφυλίας, καὶ Χελιδονέων μὴ μακροῖς πλοίοις καταπλέωσι Πέρσαι, καὶ ἐντὸς τριῶν ἡμερῶν ὁδόν, ἣν ἂν ἵππος ἀνύσῃ}\footnote{\textgreek{Ἀνύσῃ}  \textbf{Büch} \textbf{Mül} \textgreek{ἀνοισῃ}  \textbf{P}} \textgreek{διωκόμενος, μὴ κατιῶσιν. καὶ σπονδαὶ οὖν ἐγένοντο τοιαῦται.}

\subsubsection*{14} \textgreek{Μετὰ δὲ ταῦτα Ἑλληνικὸς πόλεμος ἐγένετο ἐξ αἰτίας τοιαύτης Λακεδαιμόνιοι ἀφελόμενοι Φωκέων τὸ ἐν Δελφοῖς ἱερὸν παρέδοσαν Λοκροῖς, καὶ \ladd{Ἀθηναῖοι} ἀφελόμενοι αὐτοὺς ἀπέδοσαν πάλιν τοῖς Φωκεῦσιν.}\footnote{\textgreek{καὶ \ladd{Ἀθηναῖοι} ἀφελόμενοι αὐτοὺς ἀπέδοσαν πάλιν τοῖς Φωκεῦσιν}  \textbf{Büch} \textgreek{\ladd{ὕστερον Ἀθηναῖοι}} \textbf{Mül} \textgreek{εἷτα πάλιν Λοχροὺς ἀφελόμενοι παρέσχον Φωκεῦσιν}  \textbf{SH}} \textgreek{(2) ὑποστρεφόντων δὲ τῶν Ἀθηναίων ἀπὸ τῆς μάχης,}\footnote{\textgreek{ἀπὸ τῆς μάχης}  \textbf{P} \textgreek{ἀπὸ τῶν πρὸς Ἀρταξέρξην 	σπονδῶν}  \textbf{SH}} \textgreek{στρατηγοῦντος αὐτῶν Τολμίδου, καὶ γενομένων κατὰ Κορώνειαν, ἐπιθέμενοι αὐτοῖς ἄφνω Βοιωτοὶ οὖσιν ἀπαρασκεύοις ἐτρέψαντο}\footnote{\textgreek{ἐτρέψαντο}  \textbf{P} \textgreek{ἔτρεψάν τε}  \textbf{SH}} \textgreek{αὐτοὺς καί τινας ἐξ αὐτῶν ἐζώγρησαν,}\footnote{\textgreek{ἐζώγρησαν}  \textbf{P} \textgreek{καὶ ἐζώγρησαν}  \textbf{SH}} \textgreek{οὕστινας ἀπαιτούντων Ἀθηναίων οὐ πρότερον ἀπέδοσαν ἢ τὴν Βοιωτίαν ἀπολαβεῖν.}

\subsubsection*{15} \textgreek{Καὶ μετὰ ταῦτα εὐθὺς Ἀθηναῖοι περιπλεύσαντες τὴν Πελοπόννησον Γύθειον}\footnote{\textgreek{Γύθειον} \textbf{Jac} \textgreek{Γύγιον}  \textbf{SH} \textgreek{θύγιον}  \textbf{P}} \textgreek{εἷλον· καὶ Τολμίδης χιλίους ἔχων Ἀθηναίους ἐπιλέκτους διῆλθε τὴν Πελοπόννησον. (2) καὶ πάλιν Εὔβοιαν ἀποστᾶσαν εἷλον Ἀθηναῖοι. (3) ἐν δὲ τούτῳ τοῖς Ἕλλησι}\footnote{\textgreek{τοῖς Ἕλλησι}  \textbf{P} \textgreek{καὶ ἐπὶ τούτοις Ἀθηναίοις καὶ Πελοποννησίοις}  \textbf{SH}} \textgreek{σπονδαὶ τριακοντούτεις ἐγένοντο. (4) τῷ τεσσαρεσκαιδεκάτῳ}\footnote{\textgreek{τῷ τεσσαρεσκαιδεκάτῳ}  \textbf{P} \textgreek{Ἃς τεσσαρεσκαιδεκάτῳ κτλ.}  \textbf{SH}} \textgreek{δὲ ἔτει Ἀθηναῖοι Σάμον πολιορκήσαντες εἷλον, στρατηγοῦντος αὐτῶν Περικλέους καὶ Σοφοκλέους.}\footnote{\textgreek{Σοφοκλέους}  \textbf{SH} \textgreek{Θεμιστοκλέους}  \textbf{P}} \textgreek{Ἐν δὲ τῷ αὐτῷ ἔτει οὕτω}\footnote{\textgreek{οὕτω}  \textbf{P} \textgreek{\Ladd{οὕτω}} \textbf{Mül}} \textgreek{λύονται αἱ τῶν λ ἐτῶν σπονδαί, καὶ ἐνίσταται ὁ Πελοποννησιακὸς πόλεμος.}

\subsubsection*{16} \textgreek{Αἰτίαι δὲ καὶ πλείονες φέρονται περὶ τοῦ πολέμου· πρώτη δὲ ἡ κατὰ Περικλέα. φασὶ γὰρ ὅτι τῶν Ἀθηναίων κατασκευαζόντων τὴν ἐλεφαντίνην Ἀθηνᾶν καὶ ἀποδειξάντων ἐργεπιστάτην τὸν Περικλέα, τεχνίτην}\footnote{\textgreek{τεχνίτην} \textbf{Jac} \textgreek{τειχνήτην}  \textbf{P}}\textgreek{δὲ Φειδίαν, ἁλόντος}\footnote{ \textgreek{Ἁλόντος} \textbf{Jac} \textgreek{ἁλῶντος}  \textbf{P}} \textgreek{τοῦ Φειδίου ἐπὶ νοσφισμῷ, εὐλαβηθεὶς ὁ Περικλῆς μὴ καὶ αὐτὸς εὐθύνας ἀπαιτηθῇ, βουλόμενος ἐκκλῖναι τὰς κρίσεις ἐπολιτεύσατο τὸν πόλεμον τοῦτον, γράψας τὸ κατὰ Μεγαρέων ψήφισμα.
(2) διαπιστοῦται δὲ ταῦτα καὶ ὁ τῆς ἀρχαίας κωμῳδίας ποιητὴς λέγων οὕτως· ὦ \ladd{λι}περνῆτες}\footnote{\textgreek{ὦ λιπερνῆτες} Diod \textgreek{ὦ σοφώτατοι} Ar \textgreek{ῶπερνθητες}  \textbf{P}} \textgreek{γεωργοί, τἀμὰ δὴ συνίετε}\footnote{\textgreek{δὴ συνίετε}  \textbf{P} \textgreek{Δὴ ξυνίετε} Ar \textgreek{τις ξυνιέτω }Diod}\textgreek{ῥημάτια εἰ βούλοισθ’}\footnote{\textgreek{ρηματια βούλοισθ'}  \textbf{P} \textgreek{ῥήματ’, εἰ βούλεσθ’} Ar Diod} \textgreek{ἀκοῦσαι τήνδ’ ὅπως ἀπώλετο. πρῶτον}\footnote{\textgreek{πρῶτον}  \textbf{P}  \textgreek{Πρῶτα} Ar Diod } \textgreek{μὲν γὰρ ἤρξατ’αὐτῆς}\footnote{\textgreek{ἤρξατ’αὐτῆς}  \textbf{P} \crux \textgreek{αὐτῆς ἦρξε}\crux Ar \textgreek{αὐτῆς ἦρχε} Diod} \textgreek{Φειδίας πράξας κακῶς· εἶτα Περικλέης φοβηθεὶς μὴ μετάσχοι τῆς τύχης, τὰς φύσεις ὑμῶν δεδοικὼς καὶ τὸν αὐθάδη  τρόπον,}\footnote{\textgreek{αὐθάδη}  \textbf{P} \textgreek{Αὐτοδὰξ} Ar v. 607 om Diod; v. 608 om  \textbf{P} Diod \textgreek{πρὶν παθεῖν τι δεινὸν αὐτός, ἐξέφλεξε τὴν πόλιν} Ar.} \textgreek{ἐμβαλὼν σπινθῆρα μικρὸν Μεγαρικοῦ ψηφίσματος ἐξεφύσησεν τοσοῦτον πόλεμον ὥστ’ ἐκ τοῦ καπνοῦ}\footnote{\textgreek{ἐκ τοῦ καπνοῦ}  \textbf{P} \textgreek{Ὥστε τῷ καπνῷ} Ar Diod} \textgreek{πάντας Ἕλληνας δακρῦσαι τούς τ’ ἐκεῖ τούς τ’ ἐνθάδε.  (3) καὶ πάλιν ὑποβάς· πόρνην εἰς μέθην ἰοῦσαν Μεγαρίδα}\footnote{\textgreek{καὶ πάλιν ὑποβάς}  \textbf{P} \textgreek{Καὶ πάλιν ἐν ἄλλοις} Diod; \textgreek{εἰς μέθην ἰοῦσαν Μεγαρίδα}  \textbf{P} \textgreek{δὲ Σιμαίθαν ἰόντες Μέγαράδε} Ar} \textgreek{νεανίαι κλέπτουσι μεθυσοκότταβοι· κἄιπειθ’}\footnote{\textgreek{κἄιπειθ’}  \textbf{P} \textgreek{κᾆθ’} Ar}\textgreek{οἱ Μεγαρῆς ὀδύναις πεφυσιγγωμένοι ἀντ<ἐξ>έκλεψαν Ἀσπασίας πόρνας δύο. ἐνθένδ’}\footnote{\textgreek{ἐνθένδ’ ὁ πόλεμος ἐμφανῶς} \textbf{P} \textgreek{κἀντεῦθεν ἀρχὴ τοῦ πολέμου} Ar}\textgreek{ὁ πόλεμος ἐμφανῶς κατερράγη Ἕλλησι πᾶσιν ἐκ τριῶν λαικαστριῶν.}\footnote{\textgreek{λαικαστριῶν} Ar \textgreek{Δεκαστριων}  \textbf{P}} \textgreek{ἐνθένδε}\footnote{\textgreek{ἐνθένδε μέντοι}  \textbf{P} \textgreek{Ἐντεῦθεν ὀργῇ} Ar}\textgreek{μέντοι Περικλέης Ὀλύμπιος}\footnote{\textgreek{Ὀλύμπιος}  \textbf{P} \textgreek{Οὑλύμπιος} Ar} \textgreek{ἤστραπτ’,}\footnote{\textgreek{ἤστραπτ’}  \textbf{P} Ar \textgreek{Ἤστραπτεν} Diod} \textgreek{ἐβρόντα, συνεκύκα τὴν Ἑλλάδα, ἐτίθει νόμους ὥσπερ σκόλια γεγραμμένους, ὡς χρὴ Μεγαρέας μήτ’ ἐν ἀγορᾷ}\footnote{\textgreek{μήτ’ ἐν ἀγορᾷ μήτ’ ἐν ἠπείρῳ}  \textbf{P} \textgreek{μήτε γῇ μήτ’ἐν ἀγορᾷ μήτ’ἐν θαλάττῃ μήτ’ἐν ἠπείρῳ} Ar \textgreek{μήτ’ ἐν ἀγορᾷ \ladd{μήτε γῇ μήτ’ ἐν θαλάττῃ} μήτ’ἐν ἠπείρῳ} \textbf{Jac} \textgreek{μήτ’ἐν ἀγορᾷ \ladd{μήτε γῇ} μήτ’ἐν ἠπείρῳ}  \textbf{Büch}} \textgreek{μήτ’ ἐν ἠπείρῳ μένειν. (4) φασὶ δὲ ὅτι τοῦ Περικλέους σκεπτομένου περὶ τῆς ἀποδόσεως τῶν λόγων ὑπὲρ τῆς ἐργεπιστασίας Ἀλκιβιάδης ὁ Κλεινίου, ἐπιτροπευόμενος ὑπ’ αὐτοῦ εἶπεν· ‘μὴ σκέπτου πῶς ἀποδῷς τοὺς λόγους Ἀθηναίοις, ἀλλὰ πῶς μὴ ἀποδῷς.’}

\subsubsection*{17} \textgreek{Δευτέρα δὲ αἰτία φέρεται καὶ Κερκυραίων καὶ Ἐπιδαμνίων τοιαύτη. Ἐπίδαμνος ἦν πόλις Κερκυραίων ἄποικος, ἡ δὲ}\footnote{\textgreek{Κερκυραίων ἄποικος, ἡ δὲ  Κέρκυρα} \textbf{Jac} \textgreek{Κερκυραίων ἄποικος, δὲ ἡ}  \textbf{P} \textgreek{\ladd{ἄποικος} κερκυραίων}   \textbf{Büch}} \textgreek{Κέρκυρα Κορινθίων. πλημμελούμενοι οὖν κατ’ ἐκεῖνον τὸν καιρὸν καὶ ὑπερηφανευόμενοι ὑπὸ τῶν Κερκυραίων οἱ Ἐπιδάμνιοι, προσποιησάμενοι συμμάχους τοὺς Κορινθίους ὡς μητροπολίτας, ἐστράτευσαν ἐπὶ Κέρκυραν καὶ ἐπολέμουν. (2) πιεζόμενοι δὲ Κερκυραῖοι τῷ πολέμῳ ἔπεμψαν περὶ συμμαχίας πρὸς Ἀθηναίους, ἔχοντες πολὺ ναυτικόν. ὁμοίως δὲ καὶ οἱ Κορίνθιοι ἔπεμψαν πρὸς Ἀθηναίους, ἀξιοῦντες ἑαυτοῖς καὶ μὴ τοῖς Κερκυραίοις βοηθεῖν αὐτούς. οἱ δὲ Ἀθηναῖοι εἵλοντο μᾶλλον βοηθεῖν τοῖς Κερκυραίοις· καὶ ἐναυμάχησαν τοῖς Κορινθίοις οὖσιν ἐνσπόνδοις. καὶ διὰ τοῦτο αἱ σπονδαὶ ἐλύθησαν.}

\subsubsection*{18} \textgreek{Τρίτη αἰτία φέρεται τοιαύτη. Ποτίδαια}\footnote{\textgreek{Ποτίδαια}  \textbf{We} \textgreek{πολιτιδαια}  \textbf{P}} \textgreek{πόλις ἄποικος Κορινθίων ἦν ἐπὶ Θρᾴκης. ἐπὶ ταύτην ἔπεμψαν Ἀθηναῖοι, βουλόμενοι παραλαβεῖν αὐτήν. οἱ δὲ Ποτιδαιᾶται προσέθεντο τοῖς Κορινθίοις, καὶ διὰ τοῦτο μάχη ἐγένετο Ἀθηναίων καὶ Κορινθίων, καὶ ἐξεπολιόρκησαν \ladd{τὴν Ποτίδαιαν}}\footnote{\textgreek{\ladd{τὴν Ποτίδαιαν} o οἱ Ἀθηναῖοι \ladd{τὴν πόλιν}}  \textbf{Büch}} \textgreek{οἱ Ἀθηναῖοι.}

\subsubsection*{19} \textgreek{Τετάρτη αἰτία φέρεται ἡ καὶ ἀληθεστάτη. οἱ Λακεδαιμόνιοι ὁρῶντες αὐξανομένους τοὺς Ἀθηναίους καὶ ναυσὶ καὶ χρήμασι καὶ ξυμμάχοις} ...

