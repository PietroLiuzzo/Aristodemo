\small{\subsection*{Salamina}
FGrHist 104\index[n]{FGrHist 104} non si interessa minimamente alla
battaglia di Salamina come scontro navale, si occupa di tutto il suo
contorno, dei personaggi e delle azioni legate al \textit{racconto} di
questa battaglia. Come è noto, Erodoto\index[n]{Erodoto} (\href{http://data.perseus.org/citations/urn:cts:greekLit:tlg0016.tlg001.perseus-grc1:8.56}{8.56}-98\indexp{Erodoto!8!00560000 @56-98|ca}) ed
Eschilo (\textit{Pers.}
302-514\indexp{Eschilo!\textit{Persiani}!0302 @302-514|ca}) sono considerati le fonti principali a
riguardo, ma la narrazione  di Diodoro\index[n]{Diodoro} (11.15-19\indexp{Diodoro!11!00150000 @15-19|ca}) fornisce i dettagli più
completi per la battaglia. \cite[67-73]{Roux1974}; \cite[13s]{Parker2007}.
Sono importanti anche alcuni dati riportati da Plutarco\index[n]{Plutarco}, soprattutto nella\textit{Vita di Temistocle} (10-17\indexp{Plutarco!\textit{Temistocle}!00100000 @10-17|ca}) e nella \textit{Vita di Aristide} (8-9\indexp{Plutarco!\textit{Aristide}!00080000 @8-9|ca}). 
Elementi di queste ed altre narrazioni vengono enfatizzati anche in opere come quella di
Timoteo di Mileto\index[n]{Timoteo} che nel suo nomo lirico
sulla battaglia,  fornisce interessanti particolari come (vv.
101-104\indexp{Timoteo!\textit  {Persiani}!101 @101-4|ca}): \textgreek{χειρῶν δ'ἔγβαλλον ὀρεί/ους πόδας ναός, στόματος / δ'ἐξήλλοντο μαρμαροφε / γεῖς παῖδες συγκρουόμενοι.} \cite[60]{Janssen1984}; \cite[155]{ Kuch1995} (trad.); \cite[48s]{Gambetti2001}.